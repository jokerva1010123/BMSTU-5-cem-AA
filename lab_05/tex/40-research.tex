\chapter{Исследовательская часть}

В данном разделе приводятся результаты замеров затрат реализаций алгоритмов по процессорному времени.

\section{Технические характеристики}

Технические характеристики устройства, на котором выполнялось тестирование:

\begin{itemize}
	\item[---] операционная система Window 10 Home Single Language;
	\item[---] память 8 Гб;
	\item[---] процессор 11th Gen Intel(R) Core(TM) i5-1135G7 2.42 ГГц, 4 ядра.
\end{itemize}

Во время замера устройство было подключено к сети электропитания, нагружено приложениями окружения и самой системой замера.

\section{Время выполнения реализаций алгоритмов}

Замеры времени для каждой длины входного массива полигонов проводились 100 раз. В качестве результата взято среднее время работы алгоритма на данной длине.

В таблице \ref{tbl:cnt_time1} приведены результаты замеров времени работы реализаций линейного и конвейерного алгоритмов c одном размером матрицы.
\newpage
\begin{table}[h]
	\begin{center}
		\captionsetup{justification=raggedright,singlelinecheck=off}
		\caption{\label{tbl:cnt_time1} Результат замеров времени (с)}
		\begin{tabular}{|c@{\hspace{7mm}}|c@{\hspace{7mm}}|c@{\hspace{7mm}}|c@{\hspace{7mm}}|}
			\hline
		  Размер  & Количество матрицы & Линейный & Конвейерный\\ 
			\hline
			100x100&50&0.238&0.008\\
            \hline
            100x100&100&0.426&0.011\\
            \hline
            100x100&200&0.833&0.029\\
            \hline
            100x100&400&1.638&0.077\\
            \hline
            100x100&800&3.354&0.138\\
            \hline
            100x100&1600&6.909&0.276\\
            \hline
		\end{tabular}
	\end{center}
\end{table}
В таблице \ref{tbl:cnt_time2} приведены результаты замеров времени работы реализаций линейного и конвейерного алгоритмов c одном количеством матрицы.
\begin{table}[h]
	\begin{center}
		\captionsetup{justification=raggedright,singlelinecheck=off}
		\caption{\label{tbl:cnt_time2} Результат замеров времени (с)}
		\begin{tabular}{|c@{\hspace{7mm}}|c@{\hspace{7mm}}|c@{\hspace{7mm}}|c@{\hspace{7mm}}|}
			\hline
		  Размер  & Количество матрицы & Линейный & Конвейерный\\ 
			\hline
			20x20&100&0.266&0.002\\
            \hline
            40x40&100&0.062&0.005\\
            \hline
            80x80&100&0.305&0.011\\
            \hline
            160x160&100&1.900&0.032\\
            \hline
            320x320&100&11.791&0.127\\
            \hline
		\end{tabular}
	\end{center}
\end{table}

\section*{Вывод}

В этом разделе были указаны технические характеристики машины, на которой происходило сравнение времени работы алгоритмов обработки матриц для конвейерной и ленейной реализаций.

В результате замеров времени было установлено, что конвейерная реализация обработки лучше линейной
при большом кол-ве матриц (в 21 раза при 400 матрицах, в 25 раза при 800 и в 25 при 1600). Так же конвейерная обработка показала себя лучше при увеличении размеров обрабатываемых матриц (в 60 раза при размере матриц 160х160 и в 86 раза при размере 320х320). Значит при большом количестве обрабатываемых матриц, а так же при матрицах большого размера стоит использовать конвейерную реализацию обработки, а не линейную.





