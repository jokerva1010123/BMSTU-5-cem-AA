\chapter{Конструкторская часть}
 В данном разделе будут приведены схемы конвейерной и линейной реализаций алгоритмов обработки матриц.

\section{Алгоритмы обработки матриц}

На рис. \ref{img:s1} -- \ref{img:s6} приведены схемы линейной и конвейерной реализаций алгоритмов обработки матрицы, схема трёх лент для конвейерной обработки матрицы, а также схемы реализаций этапов обработки матроицы.

\newpage
\img{220mm}{s1}{Схема алгоритма линейной обработки матрицы}

\newpage
\img{220mm}{s2}{Схема алгоритма конвейерной обработки матрицы}

\newpage
\img{220mm}{s3}{Схема 1-ой ленты конвейерной обработки матрицы}

\newpage
\img{210mm}{s4}{Схема 2-ой ленты конвейерной обработки матрицы}

\newpage
\img{220mm}{s5}{Схема 3-ей ленты конвейерной обработки матрицы}

\newpage
\img{215mm}{s6}{Схема реализаций этапов обработки мматрицы}

\section{Классы эквивалентности}

Выделенные классы эквивалентности для тестирования:

\begin{itemize}[label=---]
	\item количество строк матрицы <= 0;
	\item количество столбцов матрицы <= 0;
	\item количество строк матрицы не является целым числом;
	\item количество столбцов матрицы не является целым числом;
	\item количество обрабатываемых матриц <= 0;
	\item количество обрабатываемых матриц не является целым числом;
	\item номер команды < 0 или > 3;
	\item номер команды не является целым числом;
	\item корректный ввод всех параметров;
\end{itemize}


\section*{Вывод}

В данном разделе на основе теоретических данных были построены схемы требуемых методов обработки матриц (конвейерного и линейного), выделены классы эквивалентности для тестирования.
