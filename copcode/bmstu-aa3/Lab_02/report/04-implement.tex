\chapter{Технологический раздел}

В данном разделе будут приведены требования к ПО, средства его реализации и листинга кода алгоритмов, а также рассмотрены тестовые случаи.

\section{Требования к ПО}

Программное обеспечение должно удовлетворять следующим требованиям:
\begin{itemize}
	\item ПО принимает две целочисленные матрицы размерностью не больше 700;
	\item ПО возвращает результирующую матрицу - произведение двух входных матриц.
\end{itemize}

\section{Средства реализации} 

Для реализации ПО был выбран компилируемый язык C. В качестве среды разработки - QtCreator. Оба средства были выбраны из тех соображений, что навыки работы с ними были получены в более ранних курсах.

\section{Листинги кода}

Листинги 3.1 - 3.3 демонстрируют реализацию алгоритмов умножения матриц.

\captionsetup{singlelinecheck = false, justification=raggedright}
\begin{lstlisting}[label=multStand, caption=Стандартный алгоритм умножения матриц]
int **multStand(int **m1, int **m2, int l, int m, int n)
{
    int **res = allocateMatrix(l, n);
    if (res)
    {
        for (int i = 0; i < l; i++)
        {
            for (int j = 0; j < n; j++)
            {
                res[i][j] = 0;
                for (int r = 0; r < m; r++)
                    res[i][j] = res[i][j] m1[i][r] * m2[r][j];
            }
        }
    }
    else
        printf("Error!\n");
    return res;
}
\end{lstlisting}

\begin{lstlisting}[label=multVin, caption=Алгоритм умножения матриц Винограда]
int **multVin(int **m1, int **m2, int l, int m, int n)
{
    int **res = allocateMatrix(l, n);
    int *mulH = calloc(l, sizeof(int));
    int *mulV = calloc(n, sizeof(int));
    if (res && mulH && mulV)
    {
        for (int i = 0; i < l; i++)
            for (int k = 0; k < m / 2; k++)
                mulH[i] = mulH[i] + m1[i][2 * k] * m1[i][2 * k + 1];

        for (int i = 0; i < n; i++)
            for (int k = 0; k < m / 2; k++)
                mulV[i] = mulV[i] + m2[2 * k][i] * m2[2 * k + 1][i];

        for (int i = 0; i < l; i++)
            for (int j = 0; j < n; j++)
            {
                res[i][j] = -mulH[i] - mulV[j];
                for (int k = 0; k < m / 2; k++)
                    res[i][j] = res[i][j] + (m1[i][2 * k]\
                            + m2[2 * k + 1][j]) *\
                            (m1[i][2 * k + 1] + m2[2 * k][j]);
            }

        if (m % 2 == 1)
            for (int i = 0; i < l; i++)
                for (int j = 0; j < n; j++)
                    res[i][j] += m1[i][m - 1] * m2[m - 1][j];

        free(mulH);
        free(mulV);
    }
    else
        printf("Error!\n");
    return res;
}
\end{lstlisting}

\begin{lstlisting}[label=multVinOptimize, caption=Алгоритм умножения матриц Винограда (оптимизированный)]
int **multVinOptimize(int **m1, int **m2, int l, int m, int n)
{
    int **res = allocateMatrix(l, n);
    int *mulH = calloc(l, sizeof(int));
    int *mulV = calloc(n, sizeof(int));
    int buf;
    if (res && mulH && mulV)
    {
        for (int i = 0; i < l; i++)
            for (int k = 0; k < m / 2; k++)
                mulH[i] += m1[i][2 * k] * m1[i][2 * k + 1];

        for (int i = 0; i < n; i++)
            for (int k = 0; k < m / 2; k++)
                mulV[i] += m2[2 * k][i] * m2[2 * k + 1][i];

        for (int i = 0; i < l; i++)
            for (int j = 0; j < n; j++)
            {
                buf = 0;
                buf = -mulH[i] - mulV[j];
                for (int k = 0; k < m / 2; k++)
                    buf += (m1[i][2 * k] + m2[2 * k + 1][j])\
                            * (m1[i][2 * k + 1] + m2[2 * k][j]);
                if (m % 2 == 1)
                    buf += m1[i][m - 1] * m2[m - 1][j];
                res[i][j] = buf;
            }

        free(mulH);
        free(mulV);
    }
    else
        printf("Error!\n");
    return res;
}
\end{lstlisting}
\captionsetup{singlelinecheck = false, justification=centering}

\section{Тестирование ПО}

В таблице 3.1 приведены тестовые случаи для алгоритмов умножения матриц. Случаи 1 - 2 - умножение однострочных/одностолбцовых матриц, 3 - 5 - матрицы одного размера, 6 - некорректный размер.

\begin{table}[H]
	\begin{center}
		\captionsetup{singlelinecheck = false, justification=centering}
		\caption{Тестовые случаи}
		\begin{tabular}{l|l|l|l}
			N & Матрица 1 & Матрица 2 & Результат \\ 
			\hline
			\vspace{2mm}
			\vspace{2mm}
			1 & $\begin{pmatrix}
				1 & 1 & 1\\
				5 & 5 & 5\\
				2 & 2 & 2
			\end{pmatrix}$ &
			$\begin{pmatrix}
				1 \\
				1 \\
				1 
			\end{pmatrix}$ &
			$\begin{pmatrix}
				3 \\
				15 \\
				6
			\end{pmatrix}$ \\
			\vspace{2mm}
			\vspace{2mm}
			2 & $\begin{pmatrix}
				1 & 1 & 1
			\end{pmatrix}$ &
			$\begin{pmatrix}
				1 \\
				1 \\
				1
			\end{pmatrix}$ &
			$\begin{pmatrix}
				1 & 1 & 1\\
				1 & 1 & 1 \\
				1 & 1 & 1
			\end{pmatrix}$ \\
			\vspace{2mm}
			\vspace{2mm}
			3 & $\begin{pmatrix}
				1 & 1 \\
				1 & 1
			\end{pmatrix}$ &
			$\begin{pmatrix}
				1 & 1\\
				1 & 1
			\end{pmatrix}$ &
			$\begin{pmatrix}
				2 & 2\\
				2 & 2
			\end{pmatrix}$ \\
			\vspace{2mm}
			\vspace{2mm}
			4 & $\begin{pmatrix}
				2
			\end{pmatrix}$ &
			$\begin{pmatrix}
				2
			\end{pmatrix}$ &
			$\begin{pmatrix}
				4
			\end{pmatrix}$ \\
			\vspace{2mm}
			\vspace{2mm}
			5 & $\begin{pmatrix}
				1 & -2 & 3\\
				1 & 2 & 3\\
				1 & 2 & 3
			\end{pmatrix}$ &
			$\begin{pmatrix}
				-1 & 2 & 3\\
				1 & 2 & 3\\
				1 & 2 & 3
			\end{pmatrix}$ &
			$\begin{pmatrix}
				0 & 4 & 6\\
				4 & 12 & 18\\
				4 & 12 & 18
			\end{pmatrix}$\\
			\vspace{2mm}
			\vspace{2mm}
			6 & $\begin{pmatrix}
				1 & 2
			\end{pmatrix}$ &
			$\begin{pmatrix}
				1 & 2
			\end{pmatrix}$ &
			Некорректный размер\\
		\end{tabular}
	\end{center}
\end{table}

\section{Вывод}

На основе схем из конструкторского раздела были написаны реализации требуемых алгоритмов, а также проведено их тестирвание.