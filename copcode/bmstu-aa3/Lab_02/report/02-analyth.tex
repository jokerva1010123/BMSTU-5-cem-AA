\chapter{Аналитический раздел}

В данном разделе будут представлены описания алгоритмов умножения матриц стандартным способом, методом Винограда и его оптимизации.

\section{Стандартный алгоритм}

Пусть даны две прямоугольные матрицы:

\begin{equation}
	A_{lm} = \begin{pmatrix}
		a_{11} & a_{12} & \ldots & a_{1m}\\
		a_{21} & a_{22} & \ldots & a_{2m}\\
		\vdots & \vdots & \ddots & \vdots\\
		a_{l1} & a_{l2} & \ldots & a_{lm}
	\end{pmatrix},
	\quad
	B_{mn} = \begin{pmatrix}
		b_{11} & b_{12} & \ldots & b_{1n}\\
		b_{21} & b_{22} & \ldots & b_{2n}\\
		\vdots & \vdots & \ddots & \vdots\\
		b_{m1} & b_{m2} & \ldots & b_{mn}
	\end{pmatrix},
\end{equation}

Индексы у матриц обозначают их размерность - матрица $A$ размером $l \cdot m$, матрица $B$ - $m \cdot n$ соответственно. Перемножать матрицы можно только когда вторая размерность $A$ и первая размерность $B$ совпадают. Тогда результатом умножения данных матриц будет являться матрица $C$ размером $l \cdot n$:

\begin{equation}
	C_{ln} = \begin{pmatrix}
		c_{11} & c_{12} & \ldots & c_{1n}\\
		c_{21} & c_{22} & \ldots & c_{2n}\\
		\vdots & \vdots & \ddots & \vdots\\
		c_{l1} & c_{l2} & \ldots & c_{ln}
	\end{pmatrix},
\end{equation}

где каждый элемент матрицы вычисляется следующим образом:

\begin{equation}
	c_{ij} =
	\sum_{r=1}^{m} a_{ir}b_{rj} \quad (i=\overline{1,l}; j=\overline{1,n})
\end{equation}

\section{Алгоритм Копперсмита - Винограда}

Ассимптотика данного алгоритма является лучшей среди существующих, она составляет $O(N^{2,3755})$.

Рассмотрим два вектора: $V = (v_1, v_2, v_3, v_4)$ и $W = (w_1, w_2, w_3, w_4)$. Здесь веткор $V$ играет роль строки первой матрицы-множителя, вектор $W$ - столбца второй матрицы-множителя. Их скалярное произведение равно:

\begin{equation}
	V \cdot W = (v_1 + w_2)(v_2 + w_1) + (v_3 + w_4)(v_4 + w_3) - v_1v_2 - v_3v_4 - w_1w_2 - w_3w_4.
\end{equation}

Это выражение эквивалентно:

\begin{equation}
	V \cdot W = (u_1 + w_2)(u_2 + w_1) + (u_3 + w_4)(u_4 + w_3) - u_1u_2 - u_3u_4 - w_1w_2 - w_3w_4
\end{equation}

В случае, если размерность умножаемых векторов нечетна (например, равна 5), добавляется следующее слагаемое:

\begin{equation}
	V \cdot W += v_5w_5 
\end{equation}

Преимуществом данного метода вычисления является то, что слагаемые $v_1v_2, v_3v_4, w_1w_2, w_3w_4$ можно рассчитать для каждых строки/столбца заранее, тем самым сократить количество операций умножения и привести к менее затратной операции сложения.

Также как варианты оптимизации можно использовать побитовые сдвиги, += , использовать буферы и другое.

\section{Вывод}

В данном разделе были рассмотренны принципы работы алгоритмов умножения матриц. Полученных знаний достаточно для разработки выбранных алгоритмов. На вход алгоритмам будут подаваться две матрицы элементов и их размерность. Реализуемое ПО будет работать в пользовательском режиме (вывод отсортированного массива тремя методами), а также в экспериментальном (проведение замеров времени выполнения алгоритмов).