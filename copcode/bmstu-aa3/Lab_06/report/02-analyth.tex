\chapter{Аналитический раздел}

В данном разделе будут представлены теоретические сведения о рассматриваемых алгоритмах.

\section{Полный перебор}

Пронумеруем все города от 1 до n. Базовому городу присвоим номер n. Каждый тур по
городам однозначно соответствует перестановке целых чисел от 1 до n - 1.

Задачу коммивояжера можно решить образуя все перестановки первых n - 1 целых положительных чисел. Для каждой перестановки строится соответствующий тур и вычисляется
его стоимость. Обрабатывая таким образом все перестановки, запоминается тур, который к
текущему моменту имеет наименьшую стоимость. Если находится тур с более низкой стоимостью, то дальнейшие сравнения производятся с ним. Сложность алгоритма полного перебора составляет O(n!). \cite{enum}

\section{Муравьиный алгоритм}

Моделирование поведения муравьев связано с распределением феромона на тропе —
ребре графа в задаче коммивояжера. При этом вероятность включения ребра в маршрут отдельного муравья пропорциональна количеству феромона на этом ребре, а количество откладываемого феромона пропорционально длине маршрута. Чем короче маршрут, тем больше
феромона будет отложено на его ребрах, следовательно, большее количество муравьев будет включать его в синтез собственных маршрутов. Моделирование такого подхода, использующего только положительную обратную связь, приводит к преждевременной сходимости
— большинство муравьев двигается по локально оптимальному маршруту. Избежать, этого
можно, моделируя отрицательную обратную связь в виде испарения феромона. При этом если
феромон испаряется быстро, то это приводит к потере памяти колонии и забыванию хороших
решений, с другой стороны, большое время испарения может привести к получению устойчивого локально оптимального решения. Теперь, с учетом особенностей задачи коммивояжера,
мы можем описать локальные правила поведения муравьев при выборе пути.

\begin{itemize}
	\item Муравьи имеют собственную «память». Поскольку каждый город может быть посещен
	только один раз, у каждого муравья есть список уже посещенных городов — список
	запретов. Обозначим через $J_{ik}$ список городов, которые необходимо посетить муравью k, находящемуся в городе i;
	\item Муравьи обладают «зрением» — видимость есть эвристическое желание посетить город j, если муравей находится в городе i. Будем считать, что видимость обратно пропорциональна расстоянию между городами i и j — $D_{ij}$;
	\begin{equation}
		\eta_{ij} = 1 / D_{ij}
	\end{equation}
	\item Муравьи обладают «обонянием» — они могут улавливать след феромона, подтверждающий желание посетить город j из города i, на основании опыта других муравьев. Количество феромона на ребре (i, j) в момент времени t обозначим через $\tau_{ij}(t)$.
\end{itemize}

Исходя из этого можем сформировать правило, по которому k-ый муравей может перейти из города i в город j:
 \begin{equation}
 P_{kij} = \begin{cases}
 \frac{\tau_{ij}^a\eta_{ij}^b}{\sum_{q=1}^m \tau^a_{iq}\eta^b_{iq}}, j \in J_{ik} \\
 0, \textrm{иначе}
 \end{cases}
 \end{equation}
где a, b — параметры, задающие веса следа феромона, при a = 0 алгоритм вырождается до жадного алгоритма (будет выбран ближайший город). Заметим, что выбор города
является вероятностным, правило 2 лишь определяет ширину зоны города j; в общую зону
всех городов $J_{ik}$ ; бросается случайное число, которое и определяет выбор муравья. 

Правило 2 не изменяется в ходе алгоритма, но у двух разных муравьев значение вероятности перехода будут отличаться, т. к. они имеют разный список разрешенных городов.
Пройдя ребро (i, j), муравей откладывает на нем некоторое количество феромона, которое должно быть связано с оптимальностью сделанного выбора. Пусть $T_{k}(t)$ есть маршрут,
пройденный муравьем k к моменту времени t, а $L_{k}(t)$ — длина этого маршрута. Пусть также
Q — параметр, имеющий значение порядка длины оптимального пути. Тогда откладываемое
количество феромона может быть задано в виде:
\begin{equation}
	P_{kij} = \begin{cases}
	Q/L_{k}(t), (i,j) \in T_{k}(t)\\
	0, \textrm{иначе}
	\end{cases}	
\end{equation}

Правила внешней среды определяют, в первую очередь, испарение феромона. Пусть
$\rho \in [0, 1]$ есть коэффициент испарения, тогда правило испарения имеет вид
\begin{equation}
\tau_{ij}(t) = (1 - \rho) \cdot \tau_{ij}(t) + \Delta \tau_{ij}(t); \Delta \tau_{ij}(t) = \sum\limits_{k=1}^m \Delta \tau_{ijk}(t)
\end{equation}
где m - количество муравьев в колонии. 

В начале алгоритма количество феромона на ребрах принимается равным небольшому
положительному числу. Общее количество муравьев остается постоянным и равным количеству городов, каждый муравей начинает маршрут из своего города. 
Сложность алгоритма: $O(t_{max} \cdot max(m, n^2))$, где $t_{max}$ — время жизни колонии, m — количество муравьев в колонии, n — размер графа.

\newpage
\section{Вывод}

В данном разделе были рассмотрены алгоритмы поиска кратчайшего расстояния между городами. Полученных знаний достаточно для разработки. На вход алгоритмам будет подаваться количество городов $n$ и матрица расстояний между городами размера $n \cdot n$. Затем вводятся коэффициенты $a$ от 0 до 1, коэффициент испарения $p$ от 0 до 1, количество итераций $t$ и количество "элитных" муравьев $e$.

Реализуемое ПО будет работать в пользовательском режиме (вывод результатов), а также в экспериментальном (проведение замеров времени выполнения алгоритмов).