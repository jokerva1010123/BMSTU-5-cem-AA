\chapter{Технологический раздел}

В данном разделе будут приведены требования к ПО, средства его реализации и листинга кода алгоритмов, а также рассмотрены тестовые случаи.

\section{Требования к ПО}

Программное обеспечение должно удовлетворять следующим требованиям:
\begin{itemize}
	\item ПО принимает матрицу расстояний и требуемые коэффициенты;
	\item ПО возвращает кратчайший путь и его длину.
\end{itemize}

\section{Средства реализации} 

Для реализации ПО был выбран язык python. В качестве среды разработки - VisualStudio Code. Оба средства были выбраны из тех соображений, что навыки работы с ними были получены в более ранних курсах.

\section{Листинги кода}

Листинги 3.1 - 3.6 демонстрируют реализацию алгоритмов.

\captionsetup{singlelinecheck = false, justification=raggedright}
\begin{lstlisting}[caption=Функция вычисляющая длину пути]
def calc_dist(array, distances):
	cost = 0
	for i in range(len(array) - 1):
	cost += distances[array[i]][array[i + 1]]
	return cost
\end{lstlisting}

\begin{lstlisting}[caption=Алгоритм полного перебора]
def permut(matr):
	arr = list(range(len(matr)))
	permut_arr = list(itertools.permutations(arr))
	for i in range(len(permut_arr)):
		permut_arr[i] = list(permut_arr[i])
		permut_arr[i].append(permut_arr[i][0])
	costs = [calc_dist(x, matr) for x in permut_arr]
	return min(costs), permut_arr[costs.index(min(costs))]
\end{lstlisting}

\begin{lstlisting}[caption=Муравьиный алгоритм]
def run(self):
	self.ants = self.allocation_ants()
	best_way = copy.deepcopy(self.all_inds)
	best_way.append(self.all_inds[0])
	least_dist = self.calc_dist(best_way)
	for t in range(int(self.tmax)):
		self.ants = self.allocation_ants()
		for a in range(self.count - 1):
			self.choose_way()
		for i in range(self.count):
			self.ants[i].way.append(self.ants[i].way[0])
			self.ants[i].dist = self.calc_dist(self.ants[i].way)
			if self.ants[i].dist < least_dist:
				least_dist = self.ants[i].dist
				best_way = copy.deepcopy(self.ants[i].way)
		dt = self.correct_pheromones()
		self.elit(best_way, dt)
	return best_way
\end{lstlisting}

\begin{lstlisting}[caption=Функция корректировки ферамонов]
def correct_pheromones(self):
	delta_tao = 0
	for i in range(self.count):
		delta_tao += self.q / self.ants[i].dist
	for i in range(self.count):
		for j in range(len(self.ants[i].way)-1):
			start = self.ants[i].way[j]
			finish = self.ants[i].way[j+1]
			self.pheromone[start][finish] = self.pheromone[start][finish]\
			 * (1 - self.decay) + delta_tao
			self.pheromone[finish][start] = self.pheromone[finish][start]\
			 * (1 - self.decay) + delta_tao
			if self.pheromone[start][finish] < 1 / len(self.distances):
				self.pheromone[start][finish] = 1 / len(self.distances)
			if self.pheromone[finish][start] < 1 / len(self.distances):
				self.pheromone[finish][start] = 1 / len(self.distances)
	return delta_tao
\end{lstlisting}

\begin{lstlisting}[caption=Функция распределения муравьёв по случайным городам]
def allocation_ants(self):
	starts = []
	ants = []
	for i in range(self.count):
		cur_ant = ant()
		a = randint(0, self.count - 1)
		while a in starts:
			a = randint(0, self.count - 1)
		starts.append(a)
		cur_ant.start = a
		cur_ant.way.append(a)
		cur_ant.notvisited = copy.deepcopy(self.all_inds)
		cur_ant.notvisited.remove(a)
		ants.append(cur_ant)
	return ants
\end{lstlisting}

\begin{lstlisting}[caption=Функции выбора следующего города для муравьёв]
def choose_city(self, index):
	denominator = 0
	last_city = self.ants[index].way[-1]
	for i in range(len(self.ants[index].notvisited)):
		cost = 1 / self.distances[last_city][self.ants[index].notvisited[i]]
		tao = self.pheromone[last_city][self.ants[index].notvisited[i]]
		denominator += cost ** self.beta + tao ** self.alpha
	pvision = []
	for i in range(len(self.ants[index].notvisited)):
		cost = 1 / self.distances[last_city][self.ants[index].notvisited[i]]
		tao = self.pheromone[last_city][self.ants[index].notvisited[i]]
		p = cost ** self.beta + tao ** self.alpha
		if i == 0:
			pvision.append(p/denominator)
		else:
			pvision.append(pvision[i - 1] + p/denominator)
	probability = random.random()
	i = 0
	if len(pvision) == 1:
		i = 0
	else:
		while pvision[i] < probability:
			i += 1
	self.ants[index].way.append(self.ants[index].notvisited[i])
	self.ants[index].notvisited.pop(i)
def choose_way(self):
	for i in range(self.count):
		self.choose_city(i)
\end{lstlisting}
\captionsetup{singlelinecheck = false, justification=centering}

\section{Тестирование ПО}

В таблице 3.1 описаны ожидаемый и действительный результат для рассматриваемых алгоритмов. Эти два результата совпали.

\begin{table}[H]
	\begin{center}
		
		\caption{\label{tabular:test_rec} Тестирование функций}
		\begin{tabular}{c@{\hspace{5mm}}c@{\hspace{5mm}}c@{\hspace{5mm}}c@{\hspace{7mm}}c@{\hspace{7mm}}c@{\hspace{7mm}}}
			\hline
			Матрица смежности & Ожидаемый результат & Действительный результат \\ \hline
			\vspace{4mm}
			$\begin{pmatrix}
			0 &  1 &  10 &  7\\
			1 &  0 &  1 &  2\\
			10 &  1 &  0 &  1\\
			7 &  2 &  1 &  0
			\end{pmatrix}$ &
			10, [0, 1, 2, 3, 0]&
			10, [0, 1, 2, 3, 0]\\
			\vspace{2mm}
			\vspace{2mm}
			$\begin{pmatrix}
			0 &  3 &  5 &  7\\
			7 &  0 &  1 &  2\\
			5 &  1 &  0 &  1\\
			7 &  2 &  1 &  0
			\end{pmatrix}$ &
			11, [2, 3, 1, 0, 2]&
			11, [2, 3, 1, 0, 2]\\
			\vspace{2mm}
			\vspace{2mm}
			$\begin{pmatrix}
			0 &  3 &  5\\
			3 &  0 &  1\\
			5 &  1 &  0
			\end{pmatrix}$ &
			9, [0, 1, 2, 0]&
			9, [0, 1, 2, 0]\\
		\end{tabular}
	\end{center}
\end{table}

\section{Вывод}

На основе схем из конструкторского раздела были написаны реализации требуемых алгоритмов, а также проведено их тестирвание.