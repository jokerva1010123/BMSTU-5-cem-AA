\setcounter{page}{4}
\addchap{Введение}

Одна из самых известных и важных задач транспортной логистики (и комбинаторной оптимизации) – задача коммивояжера или «задача о странствующем торговце» (англ.
«Travelling Salesman Problem», TSP). \cite{travelProblem} Суть задачи сводится к поиску оптимального
(кратчайшего, быстрейшего или самого дешевого) пути, проходящего через промежуточный
пункты по одному разу и возвращающегося в исходную точку. К примеру, нахождение наиболее выгодного маршрута, позволяющего коммивояжеру посетить со своим товаром определенные города по одному разу и вернуться обратно. Мерой выгодности маршрута может быть минимальное время поездки, минимальные расходы на дорогу или минимальная длина пути. В наше время, когда стоимость доставки часто бывает сопоставима со стоимостью самого товара, а скорость доставки — один из главных приоритетов, задача нахождения оптимального маршрута приобретает огромное значение.

Муравьиные алгоритмы \cite{ant} представляют собой новый перспективный метод решения задач оптимизации, в основе которого лежит моделирование поведения колонии муравьев. Колония представляет собой систему с очень простыми правилами автономного поведения особей. Однако, не смотря на примитивность поведения каждого отдельного муравья, поведение всей колонии оказывается достаточно разумным. Эти принципы проверены временем — удачная адаптация к окружающему миру на протяжении миллионов
лет означает, что природа выработала очень удачный механизм поведения.

Целью данной лабораторной работы является является изучение муравьиного алгоритма. Для достижения поставленной цели необходимо выполнить следующие задачи:
\begin{itemize}
	\item изучить понятие муравьиного алгоритма на примере решения задачи коммивояжёра;
	\item изучить решение этой задачи с помощью метода полного перебора;
	\item составить схемы алгоритмов;
	\item реализовать разработанные алгоритмы;
	\item провести параметризацию муравьиного алгоритма для выбранных классов задач;
	\item провести сравнительный анализ скорости работы реализованных алгоритмов;
	\item описать и обосновать полученные результаты.
\end{itemize}