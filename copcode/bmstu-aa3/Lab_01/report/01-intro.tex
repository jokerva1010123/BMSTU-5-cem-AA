\setcounter{page}{4}
\addchap{Введение}

В настоящее время перед компьютерной лингвистикой ставится множество задач. Одна из них - поиск редакционного расстояния между строками. Это определение минимального количества редакционных операций, необходимых для превращения одной строки в другую. Впервые эту задачу обозначил В. И. Левенштейн, имя которого закрепилось за ней.

При вычислении расстояния Левенштейна редакционные операции ограничиваются вставкой, удалением и заменой. В случае расстояния Дамерау - Левенштейна к операциям добавляется транспозиция - перестановка двух соседних символов.

Данные алгоритмы находят применение не только в компьютерной лингвистике для исправления ошибок или автозамены слов, но также в биоинформатике для определения разных участков ДНК и РНК.

Существует множество модификаций упомянутых алгоритмов. В данной работе будут рассмотрены лишь те, которые используют парадигмы динамического программирования.

Цель работы - получение навыков динамического программирования.

Задачи лабораторной работы:
\begin{itemize}
	\item Изучение различных реализаций алгоритмов Левенштейна и Дамерау - Левенштейна;
	\item Использование парадигм динамического программирования при разработке ПО;
	\item Сравнительный анализ алгоритмов, основанный на полученных экспериментальных данных (по времени и памяти).
\end{itemize}