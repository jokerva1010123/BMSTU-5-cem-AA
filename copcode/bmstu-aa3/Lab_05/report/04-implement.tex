\chapter{Технологический раздел}

В данном разделе будут приведены требования к ПО, средства его реализации и листинга кода алгоритмов, а также рассмотрены тестовые случаи.

\section{Требования к ПО}

Программное обеспечение должно удовлетворять следующим требованиям:
\begin{itemize}
	\item ПО принимает количество заявок на конвейере;
	\item ПО возвращает целое число - 1 если определитель матрицы произведения двух исходных матриц - простое число, 0 иначе.
\end{itemize}

\section{Средства реализации} 

Для реализации ПО был выбран компилируемый язык C. В качестве среды разработки - QtCreator. Оба средства были выбраны из тех соображений, что навыки работы с ними были получены в более ранних курсах.

\section{Листинги кода}

Листинги 3.1 - 3.4 демонстрируют реализацию работы конвейера.

Листинг 3.5 демонстрирует работу с очередями заявок для каждой линии конвейера.

Листинги 3.6 - 3.8 демонстрируют функции, реализуемые на конкретных лентах конвейера.

\captionsetup{singlelinecheck = false, justification=raggedright}
\begin{lstlisting}[caption=Функция общего запуска конвейера]
void startPipeline(int n, int size)
{
	int tNum = 3;
	pthread_t threads[tNum];
	struct queue data;
	data.fstart = NULL;
	data.fend = NULL;
	data.sstart = NULL;
	data.send = NULL;
	data.tstart = NULL;
	data.tend = NULL;
	data.resstart = NULL;
	data.resend = NULL;
	data.num = n;
	pthread_mutex_init(&(data.m1), NULL);
	pthread_mutex_init(&(data.m2), NULL);
	pthread_mutex_init(&(data.m3), NULL);
	pthread_mutex_init(&(data.resm), NULL);
	pthread_create(&(threads[0]), NULL, startFirst, &data);
	pthread_create(&(threads[1]), NULL, startSecond, &data);
	pthread_create(&(threads[2]), NULL, startThird, &data);
	
	for (int i = 0; i < n; i++)
	{
		pthread_mutex_lock(&(data.m1));
		firstPush(&data, size);
		printf("[%d]: Start matrix:\n", i + 1);
		outputMatrix(data.fend->n, data.fend->m1);
		printf("\n");
		outputMatrix(data.fend->n, data.fend->m2);
		pthread_mutex_unlock(&(data.m1));
	}
	data.startTime = tick();
	
	for (int i = 0; i < tNum; i++)
	pthread_join(threads[i], NULL);
	
	for (int i = 0; i < n; i++)
	{
		printf("[%d]: Determinator is prime? %d\n", i + 1,\
		 data.resstart->res);
		resPop(&data);
	}
	
	FILE *f = fopen("texRes.txt", "w");
	for (int i = 0; i < n; i++)
	{
		fprintf(f, "Tape 1 & Task %d & %d & %d\\\\\n", i + 1,\
		 data.fileMatrix[i][0][0], data.fileMatrix[i][0][1]);
		fprintf(f, "Tape 2 & Task %d & %d & %d\\\\\n", i + 1,\
		 data.fileMatrix[i][1][0], data.fileMatrix[i][1][1]);
		fprintf(f, "Tape 3 & Task %d & %d & %d\\\\\n", i + 1,\
		 data.fileMatrix[i][2][0], data.fileMatrix[i][2][1]);
		fprintf(f, "\\hline\n");
	}
	fclose(f);
	
	pthread_mutex_destroy(&(data.m1));
	pthread_mutex_destroy(&(data.m2));
	pthread_mutex_destroy(&(data.m3));
	pthread_mutex_destroy(&(data.resm));
}
\end{lstlisting}

\begin{lstlisting}[caption=Первая лента конвейера]
void *startFirst(void *firstData)
{
	struct queue *data = (struct queue *) firstData;
	int done = 0;
	while (1)
	{
		if (done == data->num)
			break;
		pthread_mutex_lock(&(data->m1));
		if (data->fstart == NULL)
		{
			pthread_mutex_unlock(&(data->m1));
			continue;
		}
		
		data->fileMatrix[done][0][0] = tick() - data->startTime;
		pthread_mutex_unlock(&(data->m1));
		multMatrix(data->fstart->n, data->fstart->m1, data->fstart->m2,\
		 data->fstart->res);
		pthread_mutex_lock(&(data->m2));
		data->fileMatrix[done][0][1] = tick() - data->startTime;
		secondPush(data);
		pthread_mutex_unlock(&(data->m2));
		
		done++;
	}
	return NULL;
}
\end{lstlisting}

\begin{lstlisting}[caption=Вторая лента конвейера]
void *startSecond(void *secondData)
{
	struct queue *data = (struct queue *) secondData;
	int done = 0;
	while (1)
	{
		if (done == data->num)
			break;
		pthread_mutex_lock(&(data->m2));
		if (data->sstart == NULL)
		{
			pthread_mutex_unlock(&(data->m2));
			continue;
		}
		
		data->fileMatrix[done][1][0] = tick() - data->startTime;
		pthread_mutex_unlock(&(data->m2));
		data->sstart->res = determinator(data->sstart->n, data->sstart->m);
		pthread_mutex_lock(&(data->m3));
		data->fileMatrix[done][1][1] = tick() - data->startTime;
		thirdPush(data);
		pthread_mutex_unlock(&(data->m3));
		
		done++;
	}
	return NULL;
}
\end{lstlisting}

\begin{lstlisting}[caption=Третья лента конвейера]
void *startThird(void *thirdData)
{
	struct queue *data = (struct queue *) thirdData;
	int done = 0;
	while (1)
	{
		if (done == data->num)
			break;
		pthread_mutex_lock(&(data->m3));
		if (data->tstart == NULL)
		{
			pthread_mutex_unlock(&(data->m3));
			continue;
		}
		
		data->fileMatrix[done][2][0] = tick() - data->startTime;
		pthread_mutex_unlock(&(data->m3));
		data->tstart->res = prime(data->tstart->n);
		pthread_mutex_lock(&(data->resm));
		data->fileMatrix[done][2][1] = tick() - data->startTime;
		resPush(data);
		pthread_mutex_unlock(&(data->resm));
		
		done++;
	}
	return NULL;
}
\end{lstlisting}

\begin{lstlisting}[caption=Функции работы с очередями заявок]
void firstPush(struct queue *data, int n)
{
	struct flItem *new = malloc(sizeof(struct flItem));
	new->n = n;
	new->m1 = allocMatrix(n);
	fillMatrix(n, new->m1);
	new->m2 = allocMatrix(n);
	fillMatrix(n, new->m2);
	new->res = allocMatrix(n);
	new->next = NULL;
	if (data->fstart == NULL)
	{
		data->fstart = new;
		data->fend = new;
	}
	else
	{
		data->fend->next = new;
		data->fend = new;
	}
}

void firstPop(struct queue *data)
{
	if (data->fstart != data->fend)
	{
		struct flItem *buf = data->fstart;
		data->fstart = buf->next;
		free(buf->m1);
		free(buf->m2);
		free(buf);
	}
	else
	{
		free(data->fstart->m1);
		free(data->fstart->m2);
		free(data->fstart);
		data->fstart = data->fend = NULL;
	}
}

void secondPush(struct queue *data)
{
	struct slItem *new = malloc(sizeof(struct slItem));
	struct flItem *old = data->fstart;
	new->n = old->n;
	new->m = old->res;
	if (data->sstart == NULL)
	{
		data->sstart = new;
		data->send = new;
	}
	else
	{
		data->send->next = new;
		data->send = new;
	}
	firstPop(data);
}

void secondPop(struct queue *data)
{
	if (data->sstart != data->send)
	{
		struct slItem *buf = data->sstart;
		data->sstart = buf->next;
		free(buf->m);
		free(buf);
	}
	else
	{
		free(data->sstart->m);
		free(data->sstart);
		data->sstart = data->send = NULL;
	}
}

void thirdPush(struct queue *data)
{
	struct tlItem *new = malloc(sizeof(struct tlItem));
	struct slItem *old = data->sstart;
	new->n = old->res;
	if (data->tstart == NULL)
	{
		data->tstart = new;
		data->tend = new;
	}
	else
	{
		data->tend->next = new;
		data->tend = new;
	}
	secondPop(data);
}

void thirdPop(struct queue *data)
{
if (data->tstart != data->tend)
	{
		struct tlItem *buf = data->tstart;
		data->tstart = buf->next;
		free(buf);
	}
	else
	{
		free(data->tstart);
		data->tstart = data->tend = NULL;
	}
}

void resPush(struct queue *data)
{
	struct reslItem *new = malloc(sizeof(struct reslItem));
	struct tlItem *old = data->tstart;
	new->res = old->res;
	if (data->resstart == NULL)
	{
		data->resstart = new;
		data->resend = new;
	}
	else
	{
		data->resend->next = new;
		data->resend = new;
	}
	thirdPop(data);
}

int resPop(struct queue *data)
{
	int res;
	if (data->resstart != data->resend)
	{
		struct reslItem *buf = data->resstart;
		res = buf->res;
		data->resstart = buf->next;
		free(buf);
	}
	else
	{
		res = data->resstart->res;
		free(data->resstart);
		data->resstart = data->resend = NULL;
	}
	return res;
}
\end{lstlisting}

\begin{lstlisting}[caption=Функция вычисления произведения матриц (первая лента)]
void multMatrix(int n, int **m1, int **m2, int **res)
{
	for (int i = 0; i < n; i++)
	{
		for (int j = 0; j < n; j++)
		{
			res[i][j] = 0;
			for (int k = 0; k < n; k++)
				res[i][j] += m1[i][k] * m2[k][j];
		}
	}
}
\end{lstlisting}

\begin{lstlisting}[caption=Функция вычисления определителя матрицы (вторая лента)]
int determinator(int n, int **m)
{
	int sum = 0, proizv = 0;
	for(int i= 0; i< n; i++)
	{
		proizv = 1;
		for(int j= 0, k= i; j< n; j++, k= (k==n-1)?0: k+1 )
		proizv *= m[j][k];
		sum += proizv;
	}
	for(int i= 0; i< n; i++)
	{
		proizv= 1;
		for(int j= 0, k= i; j< n; j++, k= (k==0)?n - 1: k-1 )
		proizv *= m[j][k];
		sum -= proizv;
	}
	return sum;
}
\end{lstlisting}

\begin{lstlisting}[caption=Функция проверки числа на простоту]
int prime(int n)
{
	int res = 1;
	for (int i = 2; i < n && res; i++)
		if (n % i == 0)
			res = 0;
	return res;
}
\end{lstlisting}
\captionsetup{singlelinecheck = false, justification=centering}

\section{Тестирование ПО}

В таблице 3.1 приведены тестовые случаи для конвейерной обработки.
Случаи 1-2 - результат простое число, случаи 3-4 - результат не простое число.

\begin{table}[H]
	\begin{center}
		\caption{Тестовые случаи}
		\begin{tabular}{l|l|l|c}
			№ & Матрица 1 & Матрица 2 & Простое число\\
			\hline
			1 & $\begin{pmatrix}
			2 & 6 & 1\\ 
			8 & 7 & 9\\
			2 & 0 & 2 
			\end{pmatrix}$ & $\begin{pmatrix}
			3 & 7 & 5\\ 
			9 & 2 & 2\\
			8 & 9 & 7
			\end{pmatrix}$ & 1\\ 
			\hline
			2 & $\begin{pmatrix}
			6 & 3 & 2\\ 
			0 & 6 & 1\\ 
			5 & 5 & 4 
			\end{pmatrix}$ & $\begin{pmatrix}
			7 & 6 & 5\\ 
			6 & 9 & 3\\ 
			7 & 4 & 5 
			\end{pmatrix}$ & 1\\
			\hline
			3 & $\begin{pmatrix}
			3 & 6 & 1\\ 
			2 & 9 & 3\\ 
			1 & 9 & 4  
			\end{pmatrix}$ & $\begin{pmatrix}
			7 & 8 & 4\\ 
			5 & 0 & 3\\ 
			6 & 1 & 0 
			\end{pmatrix}$  & 0\\
			\hline
			4 & $\begin{pmatrix}
			3 & 6 & 7\\ 
			5 & 3 & 5\\ 
			6 & 2 & 9
			\end{pmatrix}$ & $\begin{pmatrix}
			1 & 2 & 7\\ 
			0 & 9 & 3\\ 
			6 & 0 & 6  
			\end{pmatrix}$ & 0\
		\end{tabular}
	\end{center}
\end{table}

\section{Вывод}

На основе схем из конструкторского раздела были написаны реализации требуемых алгоритмов, а также проведено их тестирвание.