\chapter{Аналитический раздел}

В данном разделе будут представлены описания конвейерной обработки данных, а также используемых на линиях алгоритмов.

\section{Конвейерная обработка данных}

Конвейер \cite{pipeline} — способ организации вычислений, используемый в современных процессорах и контроллерах с целью повышения их производительности (увеличения числа инструкций, выполняемых в единицу времени — эксплуатация параллелизма на уровне инструкций), технология, используемая при разработке компьютеров и других цифровых электронных устройств. 

Идея заключается в параллельном выполнении нескольких инструкций процессора. Сложные инструкции процессора представляются в виде последовательности более простых стадий. Вместо выполнения инструкций последовательно (ожидания завершения конца одной инструкции и перехода к следующей), следующая инструкция может выполняться через несколько стадий выполнения первой инструкции. Это позволяет управляющим цепям процессора получать инструкции со скоростью самой медленной стадии обработки, однако при этом намного быстрее, чем при выполнении эксклюзивной полной обработки каждой инструкции от начала до конца.

\section{Этапы конвейерной обработки}

В данной лабораторной работе работа конвейера будет осуществлена на 3 линиях. Каждая из них выполняет следующие этапы:
\begin{enumerate}
	\item перемножение матриц (стандартный алгоритм);
	\item вычисление определителя матрицы;
	\item определение, является ли модуль определителя простым числом.
\end{enumerate}

Данный алгоритм будет реализован с помощью многопоточности - под каждую линию конвейера будет выделено по потоку \cite{threads}.

\section{Вывод}

В данном разделе были рассмотренны принципы конвейерной обработки данных, алгоритмы на линиях самого конвейера. Полученных знаний достаточно для разработки. На вход конвейеру будет подаваться односвязный список, каждый элемент которого содержит в себе следующие поля: указатели на исходные матрицы, указатель на результирующую матрицу, указатель на числовую переменную (под определитель), указатель на булевую переменную (определитель - простое число или нет) и указатель на следующий элемент списка.

Реализуемое ПО будет работать в пользовательском режиме (вывод результатов конвейерной обработки входных данных), а также в экспериментальном (проведение замеров времени выполнения алгоритмов).