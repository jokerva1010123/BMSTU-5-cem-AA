\chapter{Технологический раздел}

В данном разделе будут приведены требования к ПО, средства его реализации и листинга кода алгоритмов, а также рассмотрены тестовые случаи.

\section{Требования к ПО}

Программное обеспечение должно удовлетворять следующим требованиям:
\begin{itemize}
	\item ПО принимает целочисленный массив размером не больше 1000;
	\item ПО возвращает отсортированный массив.
\end{itemize}

\section{Средства реализации} 

Для реализации ПО был выбран компилируемый язык C. В качестве среды разработки - QtCreator. Оба средства были выбраны из тех соображений, что навыки работы с ними были получены в более ранних курсах.

\section{Листинги кода}

Листинги 1 - 3 демонстрируют реализацию алгоритмов сортировки.

\begin{lstlisting}[label=bubbleSort, caption=Алгоритм сортировки пузырьком]
int *bubbleSort(int *array, int n)
{
    for (int i = n - 1; i > 0; i--)
        for (int j = 0; j < i; j++)
        {
            if (array[j] > array[j + 1])
               swap(&array[j], &array[j + 1]);
        }
    return array;
}
\end{lstlisting}

\begin{lstlisting}[label=insertionSort, caption=Алгоритм сортировки вставками]
int *insertionSort(int *array, int n)
{
    int j, buf;
    for (int i = 2; i < n; i++)
    {
        if (array[i] < array[i - 1])
        {
            buf = array[i];
            j = i - 1;
            while (j >= 0 && array[j] > buf)
            {
                array[j + 1] = array[j];
                j--;
            }
            array[j + 1] = buf;
        }
    }
    return array;
}
\end{lstlisting}

\begin{lstlisting}[label=selectionSort, caption=Алгоритм сортировки выбором]
int *selectionSort(int *array, int n)
{
    for (int i = 0; i < n - 1; i++)
    {
        int min = array[i], ind = i;
        for (int j = i + 1; j < n; j++)
            if (array[j] < min)
            {
                min = array[j];
                ind = j;
            }
        swap(array + i, array + ind);
    }
    return array;
}
\end{lstlisting}

\section{Тестирование ПО}

В таблице 3.1 приведены тестовые случаи для алгоритмов сортировки. Случай 1 описывает пустой массив, случай 2 - неотсортированный массив, случай 3 - отсортированный массив, случай 4 - обратно отсортированный массив, случай 5 - массив из равных между собой элементов.

\begin{table}[H]
	\begin{center}
		\caption{Тестовые случаи}
		\begin{tabular}{l|l|l|l}
			№ & Входной массив & Ожидаемый результат & Результат\\
			\hline
			1 &  &  & \\ 
			\hline
			2 & -8 5 1 25 7 -44 & -44 -8 1 5 7 25 & -44 -8 1 5 7 25\\
			\hline
			3 & -116 3 11 28 47 89 & -116 3 11 28 47 89 & -116 3 11 28 47 89\\
			\hline
			4 & 283 115 97 42 13 0 & 0 13 42 97 115 283 & 0 13 42 97 115 283\\
			\hline
			5 & 1 1 1 1 1 & 1 1 1 1 1 & 1 1 1 1 1\\
		\end{tabular}
	\end{center}
\end{table}

\section{Вывод}

На основе схем из конструкторского раздела были написаны реализации требуемых алгоритмов, а также проведено их тестирвание.