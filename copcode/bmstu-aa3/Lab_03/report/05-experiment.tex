\chapter{Исследовательский раздел}

В данном разделе будет проведен сравнительный анализ алгоритмов по
времени и затрачиваемой памяти.

\section{Технические характеристики}

Тестирование выполнялось на устройстве со следующими характеристиками: 

\begin{itemize}
	\item Операционная система Windows 10 \cite{windows10}
	\item Память 8 Гб
	\item Процессор Intel Core i3 7020U, 2.3 ГГц \cite{intel}
\end{itemize}

Во время проведения эксперимента устройство не было нагружено сторонними задачами, а также было подключено к блоку питания.

Замеры процессорного времени проводились с помощью ассемблерной вставки, вычисляющей затраченное процессорное время в тиках.

\begin{lstlisting}[label=tick, caption=Ассемблерная вставка замера процессорного времени в тиках]
unsigned long long tick(void)
{
    unsigned long long d;
    __asm__ __volatile__ ("rdtsc": "=A" (d));
    return d;
}
\end{lstlisting}

При проведении эксперимента была отключена оптимизация командой компилятору -o0 \cite{std=c99}.

\section{Оценка времени работы алгоритмов}

В таблицах 4.1 - 4.3 представлены замеры процессорного времени работы алгоритмов на массивах размером от 50 до 650 элементов. Такая размерность выбрана, потому что на больших размерах массива происходит переполнение типа счетчика тиков. Каждое значение было получено усреднением по 1000 замерам. 


\begin{table}[H]
	\centering
	\caption{Временные замеры работы алгоритмов на случайных значениях}
	\begin{tabular}{l|r|r|r|r|r|r|r|r|r}
		\text{N} & \text{Bubble} & \text{Insert} & \text{Selection}\\
		\hline
		50 & 30705 & 8844 & 17135\\
100 & 109026 & 27299 & 63087\\
150 & 233112 & 63269 & 124151\\
200 & 378871 & 102549 & 209498\\
250 & 568232 & 156701 & 315817\\
300 & 828940 & 231688 & 451743\\
350 & 1065136 & 296117 & 589815\\
400 & 1367519 & 392281 & 774546\\
450 & 1755752 & 497132 & 952358\\
500 & 2131493 & 626607 & 1182272\\
550 & 2577364 & 735687 & 1407664\\
600 & 2966913 & 862352 & 1703104\\
650 & 3972662 & 1131503 & 1988339\\
	\end{tabular}
\end{table}

\begin{table}[H]
	\centering
	\caption{Временные замеры работы алгоритмов на отсортированных данных}
	\begin{tabular}{l|r|r|r|r|r|r|r|r|r}
		\text{N} & \text{Bubble} & \text{Insert} & \text{Selection}\\
		\hline
		50 & 11582 & 625 & 12805\\
100 & 45791 & 1312 & 51898\\
150 & 102348 & 1737 & 104398\\
200 & 177895 & 3238 & 180238\\
250 & 288828 & 2874 & 279981\\
300 & 404472 & 3624 & 401549\\
350 & 548377 & 4775 & 536613\\
400 & 718761 & 4758 & 709811\\
450 & 899666 & 5272 & 890204\\
500 & 1107400 & 6437 & 1092161\\
550 & 1324354 & 6395 & 1323357\\
600 & 1577523 & 9432 & 1563866\\
650 & 1964788 & 8014 & 1871389\\
	\end{tabular}
\end{table}

\begin{table}[H]
	\centering
	\caption{Временные замеры работы алгоритмов на обратно отсортированных данных}
	\begin{tabular}{l|r|r|r|r|r|r|r|r|r}
		\text{N} & \text{Bubble} & \text{Insert} & \text{Selection}\\
		\hline
		50 & 25466 & 12921 & 14846\\
100 & 101150 & 49445 & 55077\\
150 & 224551 & 111362 & 125920\\
200 & 393320 & 192321 & 213532\\
250 & 609787 & 304684 & 320888\\
300 & 874043 & 426119 & 455669\\
350 & 1189252 & 588355 & 620395\\
400 & 1552689 & 762852 & 813592\\
450 & 1948077 & 970662 & 1015567\\
500 & 2410845 & 1173846 & 1265244\\
550 & 2935642 & 1466136 & 1523410\\
600 & 3589197 & 1761747 & 1826844\\
650 & 4225705 & 2065891 & 2126514\\
	\end{tabular}
\end{table}

На рисунке 4.1 представлен график сравнения времени работы алгоритмов на случайных данных. В результате эксперимента было получено, что на размерах массивов до 650 элементов сортировка пузырьком работает в 2 раза медленнее сортировки выбором и в 4 раза медленнее сортировки вставками.

\begin{figure}[H]
	\captionsetup{singlelinecheck = false, justification=centering}
	\centering
	\begin{tikzpicture}
			\begin{axis}[
				xlabel={размер массива},
				ylabel={время, ms},
				width=0.95\textwidth,
				height=0.3\textheight,
				xmin=0, xmax=700,
				legend pos=north west,
				xmajorgrids=true,
				grid style=dashed,
				]
				\addplot[
				color=blue,
				mark=asterisk
				]
				table [x=N, y=time]{
					N time
					50 30705
100 109026
150 233112
200 378871
250 568232
300 828940
350 1065136
400 1367519
450 1755752
500 2131493
550 2577364
600 2966913
650 3972662
				};
				%table {assets/randBubble.csv};
				\addplot[
				color=red,
				mark=asterisk
				]
				table [x=N, y=time]{
					N time
					50 8844
100 27299
150 63269
200 102549
250 156701
300 231688
350 296117
400 392281
450 497132
500 626607
550 735687
600 862352
650 1131503
				};
				\addplot[
				color=yellow,
				mark=asterisk
				]
				table [x=N, y=time]{
					N time
					50 17135
100 63087
150 124151
200 209498
250 315817
300 451743
350 589815
400 774546
450 952358
500 1182272
550 1407664
600 1703104
650 1988339
				};
				\legend{Пузырек, Вставки, Выбором}
			\end{axis}
		\end{tikzpicture}
		\caption{Сравнение времени работы алгоритмов на случайных данных}
\end{figure}

На рисунке 4.2 представлен график сравнения времени работы алгоритмов на отсортированных данных. В результате эксперимента было получено, что на размерах массивов до 650 элементов сортировка пузырьком работает так же как сортировка выбором, имея квадратичную зависимость, в то время как сортировка вставками линейно зависит от размера массива. Ее поведение представленно на рисунке 4.3.

\begin{figure}[H]
	\captionsetup{singlelinecheck = false, justification=centering}
	\centering
	\begin{tikzpicture}
			\begin{axis}[
				xlabel={размер массива},
				ylabel={время, ms},
				width=0.95\textwidth,
				height=0.3\textheight,
				xmin=0, xmax=700,
				legend pos=north west,
				xmajorgrids=true,
				grid style=dashed,
				]
				\addplot[
				color=blue,
				mark=asterisk
				]
				table [x=N, y=time]{
					N time
					50 11582
100 45791
150 102348
200 177895
250 288828
300 404472
350 548377
400 718761
450 899666
500 1107400
550 1324354
600 1577523
650 1964788
				};
				%table {assets/randBubble.csv};
				\addplot[
				color=red,
				mark=asterisk
				]
				table [x=N, y=time]{
					N time
					50 625
100 1312
150 1737
200 3238
250 2874
300 3624
350 4775
400 4758
450 5272
500 6437
550 6395
600 9432
650 8014
				};
				\addplot[
				color=yellow,
				mark=asterisk
				]
				table [x=N, y=time]{
					N time
					50 12805
100 51898
150 104398
200 180238
250 279981
300 401549
350 536613
400 709811
450 890204
500 1092161
550 1323357
600 1563866
650 1871389
				};
				\legend{Пузырек, Вставки, Выбором}
			\end{axis}
		\end{tikzpicture}
		\caption{Сравнение времени работы алгоритмов на отсортированных данных}
\end{figure}

\begin{figure}[H]
	\captionsetup{singlelinecheck = false, justification=centering}
	\centering
	\begin{tikzpicture}
			\begin{axis}[
				xlabel={размер массива},
				ylabel={время, ms},
				width=0.95\textwidth,
				height=0.3\textheight,
				xmin=0, xmax=700,
				legend pos=north west,
				xmajorgrids=true,
				grid style=dashed,
				]
				\addplot[
				color=red,
				mark=asterisk
				]
				table [x=N, y=time]{
					N time
					50 625
100 1312
150 1737
200 3238
250 2874
300 3624
350 4775
400 4758
450 5272
500 6437
550 6395
600 9432
650 8014
				};
				\legend{Вставки}
			\end{axis}
		\end{tikzpicture}
		\caption{Время работы алгоритма сортировки вставками на отсортированных данных}
\end{figure}

На рисунке 4.4 представлен график сравнения времени работы алгоритмов на обратно отсортированных данных. В результате эксперимента было получено, что на размерах массивов до 650 элементов сортировка пузырьком работает в 2 раза медленнее сортировки выбором и вставками, в то время как между собой их графики почти идентичны.

\begin{figure}[H]
	\captionsetup{singlelinecheck = false, justification=centering}
	\centering
	\begin{tikzpicture}
			\begin{axis}[
				xlabel={размер массива},
				ylabel={время, ms},
				width=0.95\textwidth,
				height=0.3\textheight,
				xmin=0, xmax=700,
				legend pos=north west,
				xmajorgrids=true,
				grid style=dashed,
				]
				\addplot[
				color=blue,
				mark=asterisk
				]
				table [x=N, y=time]{
					N time
					50 25466
100 101150
150 224551
200 393320
250 609787
300 874043
350 1189252
400 1552689
450 1948077
500 2410845
550 2935642
600 3589197
650 4225705
				};
				%table {assets/randBubble.csv};
				\addplot[
				color=red,
				mark=asterisk
				]
				table [x=N, y=time]{
					N time
					50 12921
100 49445
150 111362
200 192321
250 304684
300 426119
350 588355
400 762852
450 970662
500 1173846
550 1466136
600 1761747
650 2065891
				};
				\addplot[
				color=yellow,
				mark=asterisk
				]
				table [x=N, y=time]{
					N time
					50 14846
100 55077
150 125920
200 213532
250 320888
300 455669
350 620395
400 813592
450 1015567
500 1265244
550 1523410
600 1826844
650 2126514
				};
				\legend{Пузырек, Вставки, Выбором}
			\end{axis}
		\end{tikzpicture}
		\caption{Сравнение времени работы алгоритмов на обратно отсортированных данных}
\end{figure}

\section{Вывод}

Реализиция алгоритма сортировки пузырьком по сравнению с другими алгоритмами:
\begin{itemize}
	\item На случайных данных - в 2 раза медленнее сортировки выбором, в 4 раза медленнее сортировки вставками;
	\item На отсортированных данных - на 5\% медленнее сортировки выбором, в 2.5 $\cdot 10^2$ медленнее сортировки вставками;
	\item На обратно отсортированных данных - в 2 раза медленнее обеих сортировок.
\end{itemize}

Это связано с большим количеством проверок соседних элементов и простых обменов.

Реализиция алгоритма сортировки вставками по сравнению с другими алгоритмами:
\begin{itemize}
	\item На случайных данных - в 2 раза быстрее сортировки выбором, в 4 раза быстрее сортировки пузырьком;
	\item На отсортированных данных - в 4 $\cdot 10^3$ быстрее сортировки пузырьком и выбором;
	\item На обратно отсортированных данных - в 2 раза быстрее сортировки пузырьком, на 5\% быстрее сортировки выбором.
\end{itemize}

Реализиция алгоритма сортировки выбором по сравнению с другими алгоритмами:
\begin{itemize}
	\item На случайных данных - в 2 раза медленнее сортировки вставками, в 2 раза быстрее сортировки пузырьком;
	\item На отсортированных данных - на 5\% быстрее сортировки пузырьком, в 2.5 $\cdot 10^2$ медленнее сортировки вставками;
	\item На обратно отсортированных данных - в 2 раза быстрее сортировки пузырьком, на 5\% медленнее сортировки вставками.
\end{itemize}