\addchap{Заключение}
Как видно из графиков, частотный анализ требует в среднем тре­бует в 1.5-2 раза меньше сравнений. Как и ожидалось, алгоритм полного перебора требует наибольшего количества сравнений из всех представ­ленных алгоритмов.
Также видно, что гистограмма типа 1 для полного перебора - воз­растающая последовательность, в то время как для алгоритма бинарно­го поиска и, в частности, алгоритма частотного анализа, последователь­ность не является монотонно возрастающей в виду идеи работы алгорит­ма бинарного поиска.

На основании проделанной работы можно сделать следующий вы­вод: чем больше размер словаря, тем более рационально использовать эффективные алгоритмы поиска. Однако стоит помнить, что бинарный поиск применим только к отсортированным данным, поэтому может воз­никнуть ситуация, когда поддержание упорядоченной структуры занима­ет нерационально большое время и преимущества от бинарного поиска нивелируются.