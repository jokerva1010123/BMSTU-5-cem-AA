\setcounter{page}{2}
\addchap{Введение}

С непрерывным ростом количества доступной текстовой информации появляется потребность определенной организации ее хранения, удобной для поиска. Если текстовая информация представляет собой некоторое количество пар, то ее удобно хранить в словаре. 

Словарь(ассоциативный массив) -- это абстрактный тип данных, состоящий из коллекции элементов вида "ключ -- значение".  

Словарь с точки зрения интерфейса удобно рассматривать как обычный массив, в котором в качестве индексов можно использовать не только целые числа, но и значения других типов — например, строки.

Цель лабораторной работы -- анализ предложенных алгоритмов поиска в словаре. Для достижения поставленной цели необходимо выполнить следующие задачи:
\begin{itemize}
	\item провести анализ алгоритмов полного перебора, бинарного поиска и максимально эффективного поиска с разбиением словаря ключей на сегменты;
	\item оценить объем памяти для хранения данных;
	\item разработать и протестировать ПО, реализующее три предложенных алгоритма;
	\item исследовать зависимость количества сравнений при поиске от способа реализации поиска.
\end{itemize}
Результаты сравнительного анализа будут приведены в виде гистограмм, из которых можно будет сделать вывод об эффективности каждого из предложенных алгоритмов.