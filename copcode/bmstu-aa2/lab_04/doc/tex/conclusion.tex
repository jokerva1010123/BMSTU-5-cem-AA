\chapter*{Заключение}
\addcontentsline{toc}{chapter}{Заключение}

В результате исследования было получено, что на устройстве, использованном для тестирования, распараллеливание алгоритма Флойда показывает лучшие результаты при работе 4 потоков - быстрее последовательного алгоритма в 1.7 раза для порядка графа, равного 150. При этом многопоточность следует использовать на больших значениях порядка графа, так как параллельный алгоритм на порядке, равном 100, работает быстрее последовательного в 1.4 раза, а на порядке, равном 200 - в 1.6 раза.

Цель, поставленная перед началом работы, была достигнута. В ходе лабораторной работы были решены следующие задачи:

\begin{itemize}
	\item были изучены последовательный и параллельный варианты алгоритма Флойда поиска кратчайших путей между двумя любыми вершинами в графе;
	\item были разработаны изученные алгоритмы;
	\item был проведен сравнительный анализ реализованных алгоритмов;
	\item был подготовлен отчет о выполненной лабораторной работе.
\end{itemize}
