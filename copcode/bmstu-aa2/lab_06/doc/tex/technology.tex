\chapter{Технологическая часть}

В данном разделе будут указаны средства реализации, будут представлены листинги кода, а также функциональные тесты.

\section{Средства реализации}

Реализация данной лабораторной работы выполнялась при помощи языка программирования Python \cite{python}. Выбор ЯП обусловлен простотой синтаксиса, большим числом библиотек и эффективностью визуализации данных.

Замеры времени проводились при помощи функции process\_time из библиотеки time \cite{python-time}.

\section{Сведения о модулях программы}

Программа состоит из следующих модулей:

\begin{itemize}
	\item main.py - главный файл программы, предоставляющий пользователю меню для выполнения основных функций;
	\item matrix.py - файл, содержащий функции работы с матрицами;
	\item salesman.py - файл, содержащий метды решения задачи коммивояжера;
	\item time\_test.py - файл, содержащий функции замеров времени работы указанных алгоритмов;
	\item graph\_result.py - файл, содержащий функции визуализации временных характеристик описанных алгоритмов.
\end{itemize}

\section{Листинги кода}

Полный перебор приведен в листинге \ref{lst:brute-force}, муравьиный алгоритм - в листнге \ref{lst:ant}. Алгоритмы, используемые в муравьином алгоритмы представлены в листингах \ref{lst:proximity}-\ref{lst:update}.

\begin{center}
\captionsetup{justification=raggedright,singlelinecheck=off}
\begin{lstlisting}[label=lst:brute-force,caption=Полный перебор]
def get_cost(matrix, route):
    now_cost = 0

    for num in range(len(route) - 1):
        start_city = route[num]
        end_city = route[num + 1]

        now_cost += matrix[start_city][end_city]
    
    return now_cost


def get_all_routes(cities):
    routes = []

    for route in itertools.permutations(cities, len(cities)):
        route = list(route)
        route.append(route[0])
        routes.append(route)
    
    return routes


def brute_force(count, matrix):
    cities = [city for city in range(count)]
    routes = get_all_routes(cities)

    min_cost = inf

    for route in routes:
        now_cost = get_cost(matrix, route)

        if now_cost < min_cost:
            min_cost = now_cost
            min_rout = route
    
    return min_rout, min_cost
\end{lstlisting} 
\end{center}

\begin{center}
\captionsetup{justification=raggedright,singlelinecheck=off}
\begin{lstlisting}[label=lst:ant,caption=Муравьиный алгоритм]
def ant_algorithm(matrix, size, a, b, p, count_days):
    min_cost = inf

    q = get_q(matrix, size)

    proximity = get_proximity(matrix, size)
    pheromones = [[START_PHEROMON for i in range(size)] for j in range(size)]
    count_ants = size

    for _ in range(count_days):
        cities = [city for city in range(size)]
        visited = get_visited_cities(cities, count_ants)

        for ant in range(count_ants):
            while len(visited[ant]) != count_ants:
                pk = get_probability(pheromones, proximity,
                                     visited, a, b, ant, size)
                next_city = choose_city(pk)
                visited[ant].append(next_city)
            visited[ant].append(visited[ant][0])

            now_cost = get_cost(matrix, visited[ant])

            if now_cost < min_cost:
                min_cost = now_cost
                min_route = visited[ant]

        pheromones = update_pheromon(visited, matrix, pheromones, size, p, q)

    return min_route, min_cost

\end{lstlisting}
\end{center}
\clearpage
\begin{center}
\captionsetup{justification=raggedright,singlelinecheck=off}
\begin{lstlisting}[label=lst:proximity,caption=Алгоритм получения матрицы видимостей городов]
def get_proximity(matrix, size):
    proximity = [[0 for i in range(size)] for j in range(size)]

    for i in range(size):
        for j in range(i):
            proximity[i][j] = 1 / matrix[i][j]
            proximity[j][i] = proximity[i][j]
    
    return proximity
\end{lstlisting}
\end{center}

\begin{center}
\captionsetup{justification=raggedright,singlelinecheck=off}
\begin{lstlisting}[label=lst:prob,caption=Алгоритм нахождения вероятностных переходов]
def get_probability(pheromones, proximity, visited, a, b, ant, city_count):
    probabilities = [0] * city_count

    now_city = visited[ant][-1]

    for city in range(city_count):
        if city not in visited[ant]:

            p = ( pheromones[now_city][city] ** a ) * \
                ( proximity[now_city][city] ** b )
            
            probabilities[city] = p
        else:
            probabilities[city] = 0

    sum_p = sum(probabilities)

    for city in range(city_count):
        probabilities[city] /= sum_p
    
    return probabilities
\end{lstlisting}
\end{center}
\clearpage
\begin{center}
\captionsetup{justification=raggedright,singlelinecheck=off}
\begin{lstlisting}[label=lst:update,caption=Алгоритм обновления феромона на путях]
def get_q(matrix, size):
    q = 0
    count = 0

    for i in range(size):
        for j in range(i + 1, size):
            q += matrix[i][j] * 2
            count += 2

    return q / count


def check_route(i, j, route):
    if i in route:
        index_i = route.index(i)

        if j == route[index_i + 1]:
            return True
    
    return False


def update_pheromon(visited, matrix, pheromones, city_count, p, q):
    for i in range(city_count):
        for j in range(city_count):
            add = 0

            for ant in range(city_count):
                if check_route(i, j, visited[ant]):
                    cost = get_cost(matrix, visited[ant])
                    add += q / cost
            
            pheromones[i][j] *= (1 - p) 
            pheromones[i][j] += add

            if pheromones[i][j] < MIN_PHEROMON:
                pheromones[i][j] = MIN_PHEROMON

    return pheromones
\end{lstlisting}
\end{center}

\begin{center}
\captionsetup{justification=raggedright,singlelinecheck=off}
\begin{lstlisting}[label=lst:choose,caption=Алгоритм выбора города случайным образом]
def choose_city(probabilities):
    num = random()
    value = 0

    for i in range(len(probabilities)):
        value += probabilities[i]

        if value > num:
            return i
\end{lstlisting}
\end{center}

\section{Функциональные тесты}

В таблице \ref{tbl:func_test_brute} приведены функциональные тесты для функций, реализующих полный перебор, в таблице \ref{tbl:func_test_ant} - для функций, реализующих муравьиный алгоритм. Все тесты пройдены успешно.

\begin{center}
    \captionsetup{justification=raggedright,singlelinecheck=off}
    \begin{longtable}[c]{|c|c|c|c|c|}
    \caption{Функциональные тесты (полный перебор)\label{tbl:func_test_brute}} \\ \hline
		Матрица & Число городов & Ожидаемый результат \\
		\hline
		$ \begin{pmatrix}
			0 & 5 & 2 & 5 \\
			5 & 0 & 1 & 1 \\
			2 & 1 & 0 & 1 \\
			5 & 1 & 1 & 0 \\
		\end{pmatrix}$ &
		4 &
		[1 2 4 3 1], 9 \\
		\hline
		- &
		-2 &
		Ошибка \\
		\hline
	\end{longtable}
\end{center}
\clearpage
\begin{center}
    \captionsetup{justification=raggedright,singlelinecheck=off}
    \begin{longtable}[c]{|c|c|c|c|c|c|c|c|c|c|}
    \caption{Функциональные тесты (муравьиный алгоритм)\label{tbl:func_test_ant}} \\ \hline
		Матрица & Число городов & $\alpha$ & $\rho$ & Число дней & Результат \\
		\hline
		$ \begin{pmatrix}
			0 & 5 & 2 & 5 \\
			5 & 0 & 1 & 1 \\
			2 & 1 & 0 & 1 \\
			5 & 1 & 1 & 0 \\
		\end{pmatrix}$ &
		4 &
		0.1 &
		0.3 &
		200 &
		[1 3 2 4 1], 9 \\
		\hline
		- &
		-2 &
		0.1 &
		0.3 &
		200 &
		Ошибка \\
		\hline
		$ \begin{pmatrix}
			0 & 5 & 2 & 5 \\
			5 & 0 & 1 & 1 \\
			2 & 1 & 0 & 1 \\
			5 & 1 & 1 & 0 \\
		\end{pmatrix}$ &
		4 &
		-2 &
		0.3 &
		200 &
		Ошибка \\
		\hline
		$ \begin{pmatrix}
			0 & 5 & 2 & 5 \\
			5 & 0 & 1 & 1 \\
			2 & 1 & 0 & 1 \\
			5 & 1 & 1 & 0 \\
		\end{pmatrix}$ &
		4 &
		0.1 &
		-5 &
		200 &
		Ошибка \\
		\hline
		$ \begin{pmatrix}
			0 & 5 & 2 & 5 \\
			5 & 0 & 1 & 1 \\
			2 & 1 & 0 & 1 \\
			5 & 1 & 1 & 0 \\
		\end{pmatrix}$ &
		4 &
		0.1 &
		0.3 &
		0 &
		Ошибка \\
		\hline
	\end{longtable}
\end{center}

\section{Вывод}

Были реализованы функции алгоритмов решения задачи коммивояжера. Было проведено функциональное тестирование указанных функций.