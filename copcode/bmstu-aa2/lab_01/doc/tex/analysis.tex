\chapter{Аналитическая часть}

В данном разделе будут описаны алгоритмы нахождения расстояния Левенштейна и Дамерау-Левенштейна.

\section{Рекурсивный алгоритм нахождения расстояния Левенштейна}

Рассмотрим перевод строки $A'= A[:i]$ в строку $B' = B[:j]$, последние символы которых соответственно $a_{i-1}$ и $b_{j-1}$, при помощи следующих операций, каждая из которых имеет свою стоимость ($x$, $y$ - символы, $\lambda$ - строка):

\begin{itemize}
	\item вставка символа в произвольное место - $I$, стоимостью $w(x,\lambda)=1$;
	\item удаление символа с произвольной позиции - $D$, $w(\lambda,y)=1$;
	\item замена символа на другой - $R$, $w(x,y)=1, \medspace x \neq y$.
\end{itemize}

Расстояние Левенштейна - минимальное количество операций $I$, $D$, $R$ для перевода $A'= A[:i]$ в $B' = B[:j]$ \cite{foxford}. Его можно подсчитать по следующей реккурентной формуле:

\begin{equation}
	\label{eq:rL}
	D(i, j) = \begin{cases}
		0 &\text{i = 0, j = 0}\\
		i &\text{j = 0, i > 0}\\
		j &\text{i = 0, j > 0}\\
		\min \lbrace \\
		\qquad D(i, j-1) + 1\\
		\qquad D(i-1, j) + 1 &\text{i > 0, j > 0}\\
		\qquad D(i-1, j-1) + m(a_{i-1}, b_{j-1}) &\text(\ref{eq:rm})\\
		\rbrace
	\end{cases}
\end{equation}

Функция \ref{eq:rm} определена как:
\begin{equation}
	\label{eq:rm}
	m(x, y) = \begin{cases}
		0 &\text{если x = y,}\\
		1 &\text{иначе}
	\end{cases}.
\end{equation}

Рекурсия используется для реализации формулы \ref{eq:rL}. Пусть $D(i,j)$ - есть расстояние Левенштейна между строками $A'= A[:i]$ и $B' = B[:j]$. $B' = B[:j]$ может быть получен из $A'= A[:i]$ следующими способами:

\begin{itemize}
	\item если символ $a_{j-1}$ был добавлен при редактировании, то необходимо $A[:i]$ превратить в $B[:j-1]$, на что необходимо $D(i,j-1)$ операций редактирования. В этом случае $D(i,j) = D(i,j-1) + 1$;
	\item если символ $a_{i-1}$ был удален при редактировании, тогда необходимо $A[:i-1]$ превратить в $B[:j]$, на что необходимо $D(i-1,j)$ операций редактирования. В этом случае $D(i,j) = D(i-1,j) + 1$;
	\item если последние символы префиксов совпадают, т. е. $a_{i-1} = b_{j-1}$, то в этом случае можно не менять эти последние символы. Тогда $D(i,j) = D(i-1,j-1)$;
	\item если $\medspace a_{i-1} \neq b_{j-1}$, то тогда можно потратить 1 операцию на замену символа $a_{i-1}$ на $b_{j-1}$ и также потратить  операцию $D(i-1,j-1)$ на превращение $A[:i-1]$ в $B[:j-1]$. Тогда $D(i,j) = D(i-1,j-1) + 1$.
\end{itemize}

Далее при вычислении  необходимо взять минимум из всех перечисленных возможностей (при этом из случаев 3 или 4 рассматривается только один, в зависимости от условия $a_{i-1} = b_{j-1}$).

Начальные значения: $D(i,0) = i$, $D(0,j) = j$.

\section{Матричный алгоритм нахождения расстояния Левенштейна}

Промежуточные значения $D(i,j)$ вычисляются несколько раз, поэтому при больших значениях $i$, $j$ временная эффективность алгоритма снижается. Для решения этой проблемы используют матрицу для хранения значений расстояний.

Матрицу $M$ размером $N+1$ и $L+1$, где $N$ и $L$ - длины строк $A'= A[:i]$ и $B' = B[:j]$, заполняют по следующим правилам:

\begin{itemize}
	\item $M[0][0] = 0$;
	\item $M[i][j]$ вычисляется по формуле \ref{eq:mL}.
\end{itemize}

\begin{equation}
	\label{eq:mL}
	M[i][j] = \min
		\begin{cases}
		M[i][j-1] + 1\\
		M[i-1][j] + 1\\
		M[i-1][j-1] + m(A[:i-1], B[:j-1]) &\text(\ref{eq:mm})\\
	\end{cases}
\end{equation}

Функция \ref{eq:mm} определена как:
\begin{equation}
	\label{eq:mm}
	m(x, y) = \begin{cases}
		0 &\text{если x = y,}\\
		1 &\text{иначе}
	\end{cases}.
\end{equation}

$M[N][L]$ содержит значение расстояния Левенштейна.

\section{Рекурсивный алгоритм нахождения расстояния Левенштейна с использованием матрицы}

Для оптимизации рекурсивного алгоритма можно использовать матрицу как промежуточный буфер, в который при выполнении рекурсии заносятся значения расстояния Левенштейна для еще не рассмотренных символов. Рассмотренные раннее символы в матрицу не добавляются.

\section{Алгоритм нахождения расстояния Дамерау-Левенштейна}

Как было сказано раннее, алгоритм Дамерау-Левенштейна отличается от алгоритма Левенштейна добавлением операции транспозиции символов при переводе одной строки в другую. Расстояние Дамерау-Левенштейна может быть найдено по формуле:

\begin{equation}
	\label{eq:DL}
	d(i, j) = \begin{cases}
		\max(i, j), &\text{если }\min(i, j) = 0,\\
		\min \lbrace \\
			\qquad d(i, j-1) + 1,\\
			\qquad d(i-1, j) + 1,\\
			\qquad d(i-1, j-1) + m(a_{i-1}, b_{j-1}), &\text{иначе}\\
			\qquad \left[ \begin{array}{cc}d(i-2, j-2) + 1, &\text{если }i,j > 1;\\
			\qquad &\text{}a_{i} = b_{j-1}; \\
			\qquad &\text{}b_{j} = a_{i-1}\\
			\qquad \infty, & \text{иначе}\end{array}\right.\\
		\rbrace
		\end{cases},
\end{equation}

Пусть $d(i,j)$ - есть расстояние Дамерау-Левенштейна между строками $A'= A[:i]$ и $B' = B[:j]$. $B' = B[:j]$ может быть получен из $A'= A[:i]$ следующими способами:

\begin{itemize}
	\item если символ $a_{i-1}$ был удален при редактировании, тогда необходимо $A[:i-1]$ превратить в $B[:j]$, на что необходимо $d(i-1,j)$ операций редактирования. В этом случае $d(i,j) = d(i-1,j) + 1$;
	\item если символ $a_{j-1}$ был добавлен при редактировании, то необходимо $A[:i]$ превратить в $B[:j-1]$, на что необходимо $d(i,j-1)$ операций редактирования. В этом случае $d(i,j) = d(i,j-1) + 1$;
	\item если последние символы префиксов совпадают, т. е. $a_{i-1} = b_{j-1}$, то в этом случае можно не менять эти последние символы. Тогда $d(i,j) = d(i-1,j-1)$;
	\item если $\medspace a_{i-1} \neq b_{j-1}$, то тогда можно потратить 1 операцию на замену символа $a_{i-1}$ на $b_{j-1}$ и также потратить  операцию $D(i-1,j-1)$ на превращение $A[:i-1]$ в $B[:j-1]$. Тогда $d(i,j) = d(i-1,j-1) + 1$;
	\item eсли последние два символа $a_{i}$ и $b_{j}$ были переставлены, то $d(i,j)$ может быть равно $d(i-2,j-2) + 1$.
\end{itemize}

Далее при вычислении  необходимо взять минимум из всех перечисленных возможностей (при этом из случаев 3 или 4 рассматривается только один, в зависимости от условия $a_{i-1} = b_{j-1}$).

Начальные значения: $D(i,0) = i$, $D(0,j) = j$.

\section{Вывод}

Были изучены способы нахождения расстояния перевода одной строки в другую при помощи рекурсивной и матричной версий алгоритма Левенштейна, алгоритма Левенштейна с использованием матрицы и рекурсивной версии алгоритма Дамерау-Левенштейна.

Программе, реализующей данные алгоритмы, на вход будет подаваться две строки. Выходными данными такой программы должно быть число - редакционное расстояние. Программа должна работать в рамках следующих ограничений: строки, поданные на вход, могут состоять из букв русского и английского языков. Кроме того, должен быть выдан корректный результат при вводе пустых строк.

Пользователь должен иметь возможность выбора алгоритма нахождения редакционного расстояния и вывода его значения на экран. Также должны быть реализованы сравнение алгоритмов по времени работы с выводом результатов на экран и получение графического представления результатов сравнения. Данные действия пользователь должен выполнять при помощи меню.