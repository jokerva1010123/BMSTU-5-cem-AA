\chapter*{Введение}
\addcontentsline{toc}{chapter}{Введение}

Задача поиска и сравнения последовательностей возникает при обработке больших объемов информации, при работе с неструктурированными данными или при поисковых запросах. Алгоритмы для решения этой задачи разделяются на три группы: точный поиск подстрок, неточный поиск и поиск наибольшей общей подпоследовательности. Алгоритмы нечеткого поиска применяются в компьютерной лингвистике (коррекция ошибок, поиск с учетом формы одного и того же слова) и в биоинформатике (сравнение ДНК). В основе методов лежит определение расстояния между строками, которое задается в конкретном алгоритме.

Данная задача впервые была поставлена советским математиком Владимиром Левенштейном для двоичных кодов. Расстояние между строками для прозвольного алфавита назвали расстоянием Левенштейна и определили как минимальное количество операций вставки, удаления и замены одного символа, необходимых для превращения одной строки в другую \cite{foxford}.

Модификацией алгоритма Левенштейна является алгоритм Дамерау-Левенштейна. В данном алгоритме к операциям, необходимым для перевода одной строки в другую, добавляется транспозиция символов (перестановка двух соседних символов).

Данные алгоритмы реализуются методом динамического программирования. Существуют их рекурсивная и нерекурсивная реализации, которые различаются временной эффективностью.

Целью данной лабораторной работы является изучение метода динамического программирования на материале алгоритмов Левенштейна и Дамерау-Левенштейна. Для достижения поставленной цели требуется выполнить следующие задачи:

\begin{itemize}
	\item изучить алгоритмы Левенштейна и Дамерау-Левенштейна нахождения расстояния между строками;
	\item привести схемы изучаемых алгоритмов;
	\item описать используемые типы и структуры данных;
	\item оценить объем памяти для хранения данных;
	\item описать структуру разрабатываемого программного обеспечения;
	\item определить средства программной реализации выбранных алгоритмов;
	\item реализовать разработанные алгоритмы;
	\item провести функциональное тестирование программного обеспечения;
	\item провести сравнительный анализ по времени реализованных алгоритмов;
	\item подготовить отчет о выполненной лабораторной работе.
\end{itemize}