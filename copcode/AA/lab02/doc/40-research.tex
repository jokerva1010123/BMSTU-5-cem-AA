\chapter{Исследовательская часть}

\section{Технические характеристики}

\begin{itemize}
	\item Операционная система: Ubuntu 19.10 64-bit.
	\item Память: 3,8 GiB.
	\item Процессор: Intel® Core™ i3-6006U CPU @ 2.00GHz
\end{itemize}

\section{Время выполнения алгоритмов}

Алгоритмы тестировались c помощью функции замера процессорного времени \code{std::chrono::high\_resolution\_clock::now()}

Результаты замеров приведены в таблицах \ref{tbl:runtime_even} и \ref{tbl:runtime_odd}.
На рисунках \ref{plt:runtime_even} и \ref{plt:runtime_odd} приведены графики зависимостей времени работы алгоритмов от размеров матриц. Здесь и далее: СА — стандартный алгоритм, АКВ — алгоритм Копперсмита — Винограда, УАКВ — улучшенный алгоритм Копперсмита — Винограда.

\begin{table}[h!]
	\begin{center}
		\begin{tabular}{|c|c|c|c|}
			\hline
			                         & \multicolumn{3}{c|}{\bfseries Время, мс}      \\ \cline{2-4}
			\bfseries Размер матрицы & \bfseries СА & \bfseries АКВ & \bfseries УАКВ
			\csvreader{inc/csv/runtime-even.csv}{}
			{\\\hline \csvcoli&\csvcolii&\csvcoliii&\csvcoliv}
			\\\hline
		\end{tabular}
	\end{center}
	\caption{Время работы алгоритмов при чётных размерах матриц}
	\label{tbl:runtime_even}
\end{table}

\begin{table}
	\begin{center}
		\begin{tabular}{|c|c|c|c|}
			\hline
			                         & \multicolumn{3}{c|}{\bfseries Время, мс}      \\ \cline{2-4}
			\bfseries Размер матрицы & \bfseries СА & \bfseries АКВ & \bfseries УАКВ
			\csvreader{inc/csv/runtime-odd.csv}{}
			{\\\hline \csvcoli&\csvcolii&\csvcoliii&\csvcoliv}
			\\\hline
		\end{tabular}
	\end{center}
	\caption{Время работы алгоритмов при нечётных размерах матриц}
	\label{tbl:runtime_odd}
\end{table}

\begin{figure}
	\centering
	\begin{tikzpicture}
		\begin{axis}[
			axis lines=left,
			xlabel=Размер,
			ylabel={Время, мс},
			legend pos=north west,
			ymajorgrids=true
		]
			\addplot table[x=size,y=product,col sep=comma]{inc/csv/runtime-even.csv};
			\addplot table[x=size,y=bad_winograd,col sep=comma]{inc/csv/runtime-even.csv};
			\addplot table[x=size,y=good_winograd,col sep=comma]{inc/csv/runtime-even.csv};
			\legend{СА, АКВ, УАКВ}
		\end{axis}
	\end{tikzpicture}
	\captionsetup{justification=centering}
	\caption{Зависимость времени работы алгоритма от размера квадратной матрицы (чётного)}
	\label{plt:runtime_even}
\end{figure}

\begin{figure}
	\centering
	\begin{tikzpicture}
		\begin{axis}[
			axis lines=left,
			xlabel=Размер,
			ylabel={Время, мс},
			legend pos=north west,
			ymajorgrids=true
		]
			\addplot table[x=size,y=product,col sep=comma]{inc/csv/runtime-odd.csv};
			\addplot table[x=size,y=bad_winograd,col sep=comma]{inc/csv/runtime-odd.csv};
			\addplot table[x=size,y=good_winograd,col sep=comma]{inc/csv/runtime-odd.csv};
			\legend{СА, АКВ, УАКВ}
		\end{axis}
	\end{tikzpicture}
	\captionsetup{justification=centering}
	\caption{Зависимость времени работы алгоритма от размера квадратной матрицы (нечётного)}
	\label{plt:runtime_odd}
\end{figure}

\section*{Вывод}

Время работы алгоритма Винограда незначительно меньше стандартного алгоритма умножения, однако оптимизированная реализации имеет заметный прирост в скорости работы, на матрицах размером 1000х1000 уже около 18\%.
