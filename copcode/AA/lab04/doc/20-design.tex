\chapter{Конструкторская часть}

\section{Разработка алгоритмов}

\subsection{Алгоритм Копперсмита — Винограда}

На рисунке \ref{img:winograd} представлена схема улучшенного алгоритма Копперсмита — Винограда.

\imgext{width=165mm}{winograd}{pdf}{Схема улучшенного алгоритма Копперсмита — Винограда}

\subsection{Параллельная реализация алгоритма Копперсмита — Винограда}

Рассмотрим способы распараллеливания алгоритма.
\begin{itemize}
	\item Прежде всего предварительные вычисления MH и MV не зависят друг от друга, значит, их можно вычислить параллельно;
	\item Если количество потоков больше 2-х, то вычисления и MH, и MV можно распараллеть, выделив каждому из \code{n\_threads/2} потоков, где \code{n\_threads} - кол-во доступных потоков, вычисление своего участка длинной \code{l/n\_threads} и \code{n/n\_threads} соответственно;
	\item Аналогичным способом распараллеливается вычисление матрицы.
\end{itemize}

Подсчёты для каждой строки (циклы i на рисунке \ref{img:winograd}) выполняются независимо, а значит их можно распараллелить, выполняя цикл не от 0 до условного \code{R}, а поделив \code{R} на \code{n\_threads} частей.
Схема распараллеливания цикла приведена на рисунке \ref{img:parallelize}

\imgext{height=200mm}{parallelize}{pdf}{Схема распараллеливания процедуры $p$}

\section*{Вывод}

На основе теоретических данных, полученных из аналитического раздела, была построена схема алгоритма Копперсмита — Винограда и приведены способы его распараллелить.
