\chapter{Аналитическая часть}

\section{Описание задачи}


Пусть даны две прямоугольные матрицы
\begin{equation}
	A_{lm} = \begin{pmatrix}
		a_{11} & a_{12} & \ldots & a_{1m}\\
		a_{21} & a_{22} & \ldots & a_{2m}\\
		\vdots & \vdots & \ddots & \vdots\\
		a_{l1} & a_{l2} & \ldots & a_{lm}
	\end{pmatrix},
	\quad
	B_{mn} = \begin{pmatrix}
		b_{11} & b_{12} & \ldots & b_{1n}\\
		b_{21} & b_{22} & \ldots & b_{2n}\\
		\vdots & \vdots & \ddots & \vdots\\
		b_{m1} & b_{m2} & \ldots & b_{mn}
	\end{pmatrix},
\end{equation}

тогда матрица $C$
\begin{equation}
	C_{ln} = \begin{pmatrix}
		c_{11} & c_{12} & \ldots & c_{1n}\\
		c_{21} & c_{22} & \ldots & c_{2n}\\
		\vdots & \vdots & \ddots & \vdots\\
		c_{l1} & c_{l2} & \ldots & c_{ln}
	\end{pmatrix},
\end{equation}

где
\begin{equation}
	\label{eq:M}
	c_{ij} =
	\sum_{r=1}^{m} a_{ir}b_{rj} \quad (i=\overline{1,l}; j=\overline{1,n})
\end{equation}

будет называться произведением матриц $A$ и $B$ \cite{Cohn}.

Рассматривая результат умножения двух матриц, очевидно, что каждый элемент в нем представляет собой скалярное произведение соответствующих строки и столбца исходных матриц.
Такое умножение допускает предварительную обработку, позволяющую часть работы выполнить заранее \cite{Coppersmith}.

Рассмотрим два вектора $V = (v_1, v_2, v_3, v_4)$ и $W = (w_1, w_2, w_3, w_4)$.
Их скалярное произведение равно: $V\cdot W = v_1w_1 + v_2w_2 + v_3w_3 + v_4w_4,$, что эквивалентно
\begin{equation}
V\cdot W = (v_1 + w_2)(v_2 + w_1) + (v_3 + w_4)(v_4 + w_3) - v_1v_2 - v_3v_4 - w_1w_2 - w_3w_4.
\end{equation}

Несмотря на то, что второе выражение требует вычисления большего количества операций, чем стандартный алгоритм: вместо четырех умножений - шесть, а вместо трех сложений - десять, выражение в правой части последнего равенства допускает предварительную обработку: его части можно вычислить заранее и запомнить для каждой строки первой матрицы и для каждого столбца второй, то для каждого элемента будет необходимо выполнить лишь первые два умножения и последующие пять сложений, а также дополнительно два сложения.
Из-за того, что операция сложения быстрее операции умножения, алгоритм должен работать быстрее стандартного \cite{Pogorelov}.

В данной лабораторной работе стоит задача распараллеливания алгоритма Винограда.
Так как каждый элемент матрицы $C$ вычисляется независимо от других и матрицы $A$ и $B$ не изменяются, то для того, чтобы вычислить произведение параллельно, достаточно просто указать, какие элементы $C$ какому потоку вычислять.


\section*{Вывод}
В лабораторной работе стоит задача реализации многопоточной версии алгоритма Копперсмита — Винограда.
Необходимо сравнить её с однопоточной версией.
