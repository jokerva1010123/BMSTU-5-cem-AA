\chapter{Конструкторская часть}

\section{Разработка алгоритмов}

\subsubsection{Алгоритм полного перебора}

В листинге \ref{lst:pseudo-full} приведен псевдокод алгоритма полного перебора для решения задачи коммивояжера.
\begin{algorithm}
	\caption{\label{lst:pseudo-full}Алгоритм полного перебора для решения задачи коммивояжера $ex\_search(G, E)$}
	\begin{algorithmic}
		\State $best\_path\_len \gets -1$
		\State $queue \gets (0, 0)$
		\While{$queue \text{ is not empty} $}
		\State $ u <- queue.pop() $
		\For{$v \text{ in } E(u[u.size()-1])$}
		\If {$v \text{ in } u$}
		\State $continue$
		\EndIf
		\State $ new\_path \gets u $
		\State $ new\_path.add(v) $
		\State $ update\_best\_path $
		\EndFor
		\EndWhile
	\end{algorithmic}
\end{algorithm}

На рис. \ref{img:ant} отображена работа одного муравья.

\imgext{height=90mm}{ant}{pdf}{Схема работы одного муравья}

Описание этапов:
\begin{itemize}
	\item инициализация муравья — установка муравья в стартовый город;
	\item цикл продолжает работу до тех пор, пока все вершины не будут посещены, внутри тела цикла высчитывается вероятность посещения следующего города по формуле (\ref{possibility}). Для определения конкретного города применяется метод рулетки, в котором случайно выбирается значение $ 0, \dots , 1 $ и на основе полученного значения с учетом вероятностей перехода в непосещённые города определяется следующий город.
\end{itemize}

Описанный выше алгоритм применяется $N$ раз, где $N$ — количество вершин.
Каждый муравей помещается в отдельную вершину на карте, после чего ищет маршрут.
Когда последний муравей завершил обход всех вершин, производится поиск оптимального из полученных маршрутов, после чего феромон обновляется и в случае, если полученный результат не удовлетворяет поставленной задаче, алгоритм запускается заново.

На рис.\ref{img:ants} представлена работа алгоритма для всей колонии.

\imgext{height=110mm}{ants}{pdf}{Схема работы колонии}

\section*{Вывод}

В результате работы над конструкторским разделом была разработана схемы алгоритма полного перебора и муравьиного алгоритма.
