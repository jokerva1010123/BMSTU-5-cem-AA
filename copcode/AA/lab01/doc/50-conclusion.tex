\chapter*{Заключение}
\addcontentsline{toc}{chapter}{Заключение}

В ходе выполнения работы были выполнены все поставленные задачи и изучены методы динамического программирования на основе алгоритмов вычисления расстояния Левенштейна.

Экспериментально были установлены различия в производительности различных алгоритмов вычисления расстояния Левенштейна. Рекурсивный алгоритм Левенштейна работает на несколько порядков медленнее матричной реализации. Если длина сравниваемых строк превышает 10, рекурсивный алгоритм становится неприемлимым для использования. Матричная реализация алгоритма Дамерау — Левенштейна работает дольше алгоритма Левенштейна, т.к. в нем добавлены дополнительные проверки.

Теоретически было рассчитано использования памяти в каждом из алгоритмов вычисления расстояния Левенштейна. Обычные матричные алгоритмы потребляют намного больше памяти, чем рекурсивный, одна модфицированные их версии, держащие в памяти лишь текущую и предыдущую строки матрицы, и работают быстрее, и память используют меньше.