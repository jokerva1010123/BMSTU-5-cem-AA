\chapter{Аналитическая часть}

Как говорилось ранее, расстояние Левенштейна между двумя строками — это минимальное количество операций вставки, удаления и замены, необходимых для превращения одной строки в другую.

Цены операций могут зависеть от вида операции (вставка, удаление, замена) и/или от участвующих в ней символов, отражая разную вероятность разных ошибок при вводе текста, и т. п. В общем случае:
\begin{itemize}
	\item $w(a,b)$ — цена замены символа $a$ на символ $b$
	\item $w(\lambda,b)$ — цена вставки символа $b$
	\item $w(a,\lambda)$ — цена удаления символа $a$
\end{itemize}
Расстояние Левенштейна является частным случаем этой задачи \cite{ifmo} при
\begin{itemize}
	\item $w(a,a)=0$
	\item $w(a,b)=1, \medspace a \neq b$
	\item $w(\lambda,b)=1$
	\item $w(a,\lambda)=1$
\end{itemize}

\section{Рекурсивный алгоритм нахождения расстояния Левенштейна}

Расстояние Левенштейна между двумя строками a и b может быть вычислено по формуле $D(|a|, |b|)$, где $|a|$ означает длину строки $a$; $a[i]$ — i-ый символ строки $a$ , функция $D(i, j)$ определена как:
\begin{equation}
	\label{eq:D}
	D(i, j) = \begin{cases}
		0 &\text{i = 0, j = 0}\\
		i &\text{j = 0, i > 0}\\
		j &\text{i = 0, j > 0}\\
		\min \lbrace \\
			\qquad D(i, j-1) + 1\\
			\qquad D(i-1, j) + 1 &\text{i > 0, j > 0}\\
			\qquad D(i-1, j-1) + m(a[i], b[j])\\
		\rbrace
	\end{cases},
\end{equation}

а функция $m(a, b)$ определена как
\begin{equation}
	m(a, b) = \begin{cases}
		0 &\text{если a = b,}\\
		1 &\text{иначе}
	\end{cases}.
\end{equation}

Рекурсивный алгоритм реализует данную формулу.
Функция $D$ составлена из следующих соображений:
\begin{enumerate}
	\item для перевода из пустой строки в пустую требуется ноль операций;
	\item для перевода из пустой строки в строку $a$ требуется $|a|$ операций, аналогично, для перевода из строки $a$ в пустую требуется $|a|$ операций;
	\item для перевода из строки $a$ в строку $b$ требуется выполнить последовательно некоторое кол-во операций (удаление, вставка, замена) в некоторой последовательности. Как можно показать сравнением, последовательность проведения любых двух операций можно поменять, и, как следствие, порядок проведения операций не имеет никакого значения. Тогда цена преобразования из строки $a$ в строку $b$ может быть выражена как (полагая, что $a', b'$  — строки $a$ и $b$ без последнего символа
	соответственно):
	\begin{itemize}
		\item сумма цены преобразования строки $a$ в $b$ и цены проведения операции удаления, которая необходима для преобразования $a'$ в $a$;
		\item сумма цены преобразования строки $a$ в $b$  и цены проведения операции вставки, которая необходима для преобразования $b'$ в $b$;
		\item сумма цены преобразования из $a'$ в $b'$ и операции замены, предполагая, что $a$ и $b$ оканчиваются разные символы;
		\item цена преобразования из $a'$ в $b'$, предполагая, что $a$ и $b$ оканчиваются на один и тот же символ.
	\end{itemize}
	Очевидно, что минимальной ценой преобразования будет минимальное
	значение этих вариантов.
\end{enumerate}

\section{Алгоритм Вагнера — Фишера (построчный)}

Прямая реализация приведенной выше формулы $D$ может быть малоэффективна при больших $i, j$, т. к. множество промежуточных значений $D(i, j)$ вычисляются заново множество раз подряд. Для оптимизации нахождения расстояния Левенштейна можно использовать матрицу в целях хранения соответствующих промежуточных значений. В таком случае алгоритм представляет собой построчное заполнение матрицы 
$A_{|a|,|b|}$ значениями $D(i, j)$.

Можно заметить, что при каждом заполнении новой строки значения
предыдущей становятся ненужными. Поэтому можно провести оптимизацию по памяти и использовать дополнительно только одномерный массив размером $|b|$. Такой вариант алгоритма называется построчным и именно он реализован в данной работе в качестве нерекурсивного.\\ \\

\section{Расстояния Дамерау — Левенштейна}

Расстояние Дамерау — Левенштейна может быть найдено по рекурсивной формуле $d_{a,b}(|a|, |b|)$, где $d_{a,b}(i, j)$ задана как
\begin{equation}
	d_{a,b}(i, j) = \begin{cases}
		\max(i, j) &\text{если }\min(i, j) = 0,\\
		\min \lbrace \\
			\qquad d_{a,b}(i, j-1) + 1\\
			\qquad d_{a,b}(i-1, j) + 1 &\text{иначе}\\
			\qquad d_{a,b}(i-1, j-1) + m(a[i], b[j])\\
			\qquad M_{a,b}(i, j)\\
		\rbrace
		\end{cases},
\end{equation}

где $M_{a,b}(i, j)$ задается как
\begin{equation}
	M_{a,b}(i, j) = \begin{cases}
		d_{a,b}(i-2, j-2) + 1 &\text{если }i,j > 1; a[i] = b[j-1]; b[j] = a[i-1],\\
		+\infty &\text{иначе}
	\end{cases}.
\end{equation}

Формула выводится по тем же соображениям, что и формула (\ref{eq:D}).
Т.к. прямое применение этой формулы неэффективно, то аналогично действиям из предыдущего пункта производится добавление матрицы для хранения промежуточных значений рекурсивной формулы и оптимизация по памяти. В таком случае необходимо хранить одномерный массив длиной $3\min(|a|,|b|)$.

Пусть $S_{1}$ и $S_{2}$ — две строки (длиной $M$ и $N$ соответственно) над некоторым алфавитом, тогда расстояние Левенштейна $d(S_{1},S_{2})$ можно подсчитать по рекуррентной формуле $d(S_{1},S_{2}) = D(M,N)$, где\\
\begin{equation}
D(i,j) = 
\left \{ \begin{aligned}
& 0, & i = 0, \medspace j = 0\\
& i, & i > 0, \medspace j = 0\\
& j, & i = 0, \medspace j > 0\\
& \min\{ &\\
& \qquad D(i, j - 1) + 1, & \\
& \qquad D(i - 1, j) + 1, & i > 0, \medspace j > 0\\
& \qquad D(i - 1, j - 1) + m(S_{1}[i], S_{2}[j]) &\\
& \} &
\end{aligned} \right. 
\end{equation}


\section*{Вывод}

Формулы Левенштейна и Дамерау — Левенштейна для рассчета расстояния между строками задаются рекурсивно, а следовательно, алгоритмы могут быть реализованы рекурсивно или итерационно.
