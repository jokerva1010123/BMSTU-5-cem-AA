\chapter*{Введение}
\addcontentsline{toc}{chapter}{Введение}

Алгоритмы сортировки имеют большое практическое применение.
Их можно встретить почти везде, где речь идет об обработке и хранении больших объемов информации.
Сортировки используются в самом широком спектре задач, включая обработку коммерческих, сейсмических, космических и прочих данных \cite{support-office}.
Часто сортировка является просто вспомогательной операцией для упорядочивания данных, упрощения последующих алгебраических действий над данными и т. п.

Сортировка применяется во всех без исключения областях программирования, например, базы данных или математические программы.
Упорядоченные объекты содержатся в телефонных книгах, ведомостях налогов, в библиотеках, в оглавлениях, в словарях.

В учебнике \cite[стр. 21]{Knut} Д. Кнута упоминается что «по оценкам производителей компьютеров в 60-х годах в среднем более четверти машинного времени тратилось на сортировку.
Во многих вычислительных системах на нее уходит больше половины машинного времени.
Исходя из этих статистических данных, можно заключить, что либо (i) сортировка имеет много важных применений, либо (ii) ею часто пользуются без нужды, либо (iii) применяются в основном неэффективные алгоритмы сортировки». 

В настоящее время, в связи с экспоненциально возросшими объемами данных, вопрос эффективной сортировки данных снова стал актуальным.

К примеру, на сайте \cite{sort-benchmark} можно найти результаты производительности алгоритмов сортировки для ряда ведущих центров данных.
При этом используются различные критерии оценки эффективности.

\section*{Задачи работы}

В рамках выполнения работы необходимо решить следующие задачи:
\begin{itemize}
	\item рассмотреть и изучить сортировки пузырьком, вставками и выбором;
	\item реализовать каждую из этих сортировок;
	\item рассчитать их трудоемкость;
	\item сравнить их временные характеристики экспериментально;
	\item на основании проделанной работы сделать выводы.
\end{itemize}
