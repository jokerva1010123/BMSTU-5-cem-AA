\chapter{Технологическая часть}

\section{Выбор ЯП}

В данной лабораторной работе использовался язык программирования - python. \cite{Python}
Я знакома с ним.
Поэтому данный язык был выбран.
В качестве среды разработки я использовала Visual Studio Code \cite{Vs}.
Visual Studio Code подходит не только для  Windows \cite{Win},
но и для Linux \cite{Lin}, это причина,
по которой я выбрала VS code,
т.к. у меня установлена ОС Ubuntu 18.04.4 \cite{Ubuntu}.
В моей архитектуре присутствует 8 ядер.

\section{Сведения о модулях программы}

Данная программа разбита на модули.

\begin{itemize}
	\item main.py - главный файл, содержащий точку входа в программу.
	\item dictionary.py - файл, содержащий класс dictionary.
	\item world\_info.py - файл, содержащий класс, который описывает информацию о игре.
\end{itemize}

На листингах 3.1-3.4 представлен основной код программы.


\begin{lstlisting}[label=some-code,caption=Поиск полным перебором]
def SequentialSearch(self, key):
for elem in self.data:
	if key == elem:
		return self.data[elem]
return -1
\end{lstlisting}

\begin{lstlisting}[label=some-code,caption=Бинарный поиск]
def BinarySearch(self, key, list_keys):
	l, r = 0, len(list_keys) - 1

	while l <= r:
		middle = (l + (r - l) // 2)  

		if list_keys[middle] == key:
			return self.data[elem]

		elif list_keys[middle] < key:
			l = middle + 1
		
		else:
			r = middle - 1

	return -1
\end{lstlisting}

\begin{lstlisting}[label=some-code,caption=Оптимизация бинарного поиска]
def BinarySearch(self, key, list_keys, l, r): 
	while l <= r:
		middle = (r + l) // 2 # Optimization.
		elem = list_keys[middle] # Optimization.

		if elem == key:
			return self.data[elem]

		elif elem < key:
			l = middle + 1

		else:
			r = middle - 1

	return -1
\end{lstlisting}

\begin{lstlisting}[label=some-code,caption=Частичный анализ]
def Search(self, key, new_dict):
	# Loop by letter
	for k in new_dict:
		if key[0] == k:
			for elem in new_dict[k]:
				if elem == key:
					return new_dict[k][elem]
			return -1
	
	return -1
\end{lstlisting}


\section{Тестирование}

В данном разделе будет приведена таблица с тестами (таблица \ref{table:ref1}).

\begin{table}[ht]
	\centering
	\caption{Таблица тестов}
	\label{table:ref1}
	\begin{tabular}{ | l | l | l |}
		\hline
		Входные данные                        & Пояснение              & Результат    \\ \hline
		A Good Librarian Like a Good Shepherd & Первый элемент         & Ответ верный \\ \hline
		Championship Manager 2008             & Средний элемент        & Ответ верный \\ \hline
		Patapon 3                             & Последний элемент      & Ответ верный \\ \hline
		Adrift                                & Произвольный элемент   & Ответ верный \\ \hline
		Adrift 333                            & Несуществующий элемент & Ответ верный \\ \hline
		\hline
	\end{tabular}
\end{table}

Все тесты пройдены.

\section{Вывод}

В данном разделе были разобраны листинги рис 3.1-3.4, показывающие работу каждого алгоритма и
приведена таблица с тестами (таблица \ref{table:ref1})..