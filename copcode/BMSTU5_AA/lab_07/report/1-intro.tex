\newpage
\chapter*{Введение}
\addcontentsline{toc}{chapter}{Введение}

С появлением словарей появилась нужда в том, чтобы уметь быстро
находить нужное значение по ключу. Со временем стали разрабатывать
алгоритмы поиска в словаре. В данной лабораторной работе мы рассмотрим
три алгоритма:

\begin{enumerate}
	\item Поиск полным перебором;
	\item Бинарный поиск;
	\item Частичный анализ.
\end{enumerate}

% В данной лабораторной работе будет рассмотрен и реализован метода конвейерных вычислений.

Целью данной работы является изучение и реализация трех алгоритмов.

В рамках выполнения работы необходимо решить следующие задачи:

\begin{enumerate}
	\item Изучения трех алгоритмов поиска в словаре;
	\item Применение изученных основ для реализации поиска значений в словаре по ключу;
	\item Получения практических навыков;
	\item Получение замеров времени;
	\item Описание и обоснование полученных результатов;
	\item Выбор и обоснование языка программирования, для решения данной задачи.
\end{enumerate}