\newpage
\chapter*{Введение}
\addcontentsline{toc}{chapter}{Введение}
Параллельные вычисления часто используются для увеличения скорости выполнения
программ. Однако приемы, применяемые для однопоточных машин, для
параллельных могут не подходить.

В данной лабораторной работе будет рассмотрено и реализованно параллельное
программирование на примере задачи трассировки лучей.

Целью данной работы является изучения параллельных вычислений на материале
трассировки лучей.

В рамках выполнения работы необходимо решить следующие задачи.

\begin{enumerate}
	\item Изучения основ параллельных вычислений.
	\item Применение изученных основ для реализации многопоточности на материале трассировки лучей.
	\item Получения практических навыков.
	\item Сравнительный анализ параллельной и однопоточной реализации алгоритма трассировки лучей.
	\item Экспериментально подтверждение различий во временной эффективность реализации однопоточной и многопоточной трассировки лучей.
	\item Описание и обоснование полученных результатов.
	\item Выбор и обоснование языка программирования, для решения данной задачи.
\end{enumerate}