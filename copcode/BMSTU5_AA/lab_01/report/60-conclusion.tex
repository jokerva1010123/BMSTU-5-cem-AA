\Conclusion % заключение к отчёту

%<shape name="Loop Limit" h="60" w="98" aspect="variable" strokewidth="inherit">
%<connections>
%<constraint x="0.5" y="0" perimeter="0" name="N" />
%<constraint x="0.5" y="1" perimeter="0" name="S" />
%<constraint x="0" y="0.5" perimeter="0" name="W" />
%<constraint x="1" y="0.5" perimeter="0" name="E" />
%<constraint x="0.1" y="0.15" perimeter="0" name="NW" />
%<constraint x="0.02" y="0.985" perimeter="0" name="SW" />
%<constraint x="0.9" y="0.15" perimeter="0" name="NE" />
%<constraint x="0.98" y="0.985" perimeter="0" name="SE" />
%</connections>
%<background>
%<path>
%<move x="19" y="0" />
%<line x="79" y="0" />
%<line x="98" y="20" />
%<line x="98" y="55" />
%
%<line x="0" y="55" />
%<line x="0" y="20" />
%<line x="19" y="0" />
%<close />
%</path>
%</background>
%<foreground>
%<fillstroke />
%</foreground>
%</shape>


% Изменить мы !!! OK.
% Алгоритм Левенштейна -> Алгоритм поиска расстояния Левенштейна. OK.
% Схема на рисунке -> рисунок OK.
% В Фомруле ДЛ. добавить квадратную схему.
% Добавить в рекуррентной схеме комментарий с названием. OK.
% Формулы для памяти рекуррентного алг. высота дерева равна длина первого слова + длина второго слова !!! 

Алгоритмы поиска расстояний Левенштейна и ДамерауЛевенштейна являются самыми популярными алгоритмами, которые помогают найти редакторское расстояние. \\
В этой лабораторной работе мы познакомились с алгоритмами поиска расстояний Левенштейна (Формула \ref{eq:ref1}) и Дамерау-Левенштейна (Формула \ref{eq:ref2}).
Построили схемы (Рисунок \ref{fg:ref1}, Рисунок \ref{fg:ref2}), соответствующие данным алгоритмам, также разобрали рекуррентные реализации (Рисунок \ref{fg:ref3}).
Написали полностью готовый и протестированный (Таблица \ref{table:ref1}) программный продукт, который считает дистанцию 4 способами.

В рамках выполнения работы решены следующие задачи.

\begin{enumerate}
	\item Изучены алгоритмы поиска расстояний Дамерау-Левенштейна и Левенштейна.
	\item Реализованы изученные алгоритмы, а также матричную и рекурсивную реализации алгоритма.
	\item Проиллюстрированы алгоритмы схемами.
%	\item Описали выбранную среду разработки и ЯП.
	\item Проведено сравнение временных характеристик, а также затраченной памяти.
\end{enumerate}

%По окончании изучения данного материала можно смело идти и реализовывать алгоритмы нахождения редакционного расстояния!