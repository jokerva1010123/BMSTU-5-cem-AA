\Introduction
% мы познакомимся -> рассмотрим. Более формально. ОК.
В данной лабораторной работе будет рассмотрено расстояние Левенштейна. Данное расстояние показывает минимальное количество редакторских операций (вставки, замены и удаления), которые необходимы для перевода одной строки в другую. Это расстояние помогает определить схожесть двух строк.

Упомянутое расстояние используется в задачах автозамены. В частности, учитываются ошибки, которые человек может допускать при наборе текста. Кроме упомянутых трех ошибок (вставка лишнего символа, пропуск символа, замена одного символа другим), его пальцы могут нажимать на нужные клавиши не в том порядке. С этой проблемой поможет справиться расстояние Дамерау-Левенштейна. Данное расстояние задействует еще одну редакторскую операцию -- транспозицию, или перестановку.

Практическое применение расстояние Левенштейна:

\begin{itemize}
	\item сравнение введенной строки со словарными словами в поисковой системе, такой как 'yandex' или 'google';
	\item помогает найти разницу двух ДНК, дать оценку мутации.
\end{itemize}

Целью данной работы является разбор и реализация алгоритма Дамерау-Левенштейна и Левенштейна.

В рамках выполнения работы необходимо решить следующие задачи.

\begin{enumerate}
	\item Изучить расстояния Дамерау-Левенштейна и Левенштейна.
	\item Реализовать алгоритмы поиска расстояний, в частности, нерекурсивный (матричный) и рекурсивные алгоритмы поиска расстояния Левенштейна и нерекурсивный алгоритм поиска расстояния Дамерау-Левенштейна.
	\item Подсчет времени поиска расстояния.
	\item Сравнить временные характеристики, а также затраченную память.
%	\item Описать выбранную среду разработки и ЯП.
\end{enumerate}