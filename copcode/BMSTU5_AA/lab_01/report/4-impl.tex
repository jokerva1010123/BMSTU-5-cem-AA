\chapter{ Технологический раздел}
\label{cha:design}

\section{Выбор ЯП}

В данной лабораторной работе использовался язык программирования - python \cite{bib1}.
Данный язык простой и понятный, также я знакома с ним.
Поэтому данный язык был выбран. 
В качестве среды разработки я использовала Visual Studio Code \cite{bib2}, т.к. считаю его достаточно удобным и легким.
Visual Studio Code подходит не только для  Windows \cite{bib3}, но и для Linux \cite{bib4}, это еще одна причина, по которой я выбрала VS code, т.к. у меня установлена ОС Ubuntu 18.04.4 \cite{bib5}.
При замерах времени использовалась функция process\_time \cite{bib7}.
% сюда инструменты замера!!! OK.

\section{Требования к программному обеспечению}

Входными данными являются две строки. Строки регистрозависимые. На выходе имеется матрица и дистанция, полученная алгоритмом Левенштейна. Результат алгоритма Дамерау-Левенштейна выводится только в том случае, когда его результат не совпадает с результатом алгоритма Левенштейна. Также требуется замерить время работы каждой реализации. 


\section{Сведения о модулях программы}

Данная программа разбита на модули:

\begin{itemize}
	\item main.py - Файл, содержащий точку входа в программу. В нем происходит общение с пользователем и вызов алгоритмов;
	\item algorithms.py - Файл содержит непосредственно сами алгоритмы;
	\item test\_time.py - Файл с замером времени работы алгоритмов.
\end{itemize}

На листингах 3.1-3.6 представлены коды разобранных ранее алгоритмов.

\begin{lstlisting}[label=some-code,caption=Главная функция main]
def main():
	output("Enter the first string", GREEN)  # Column
	strFirst = input()
	output("Enter the second string:", GREEN)  # Row
	strSecond = input()
	distanceLev = Levenshtein(strFirst, strSecond, True)
	distanceDamLev = DamerauLevenshtein(strFirst, strSecond)
	output("distance Levenshtein: " + str(distanceLev), GREEN)
	if distanceLev != distanceDamLev:
		output("distance Damerau-Levenshtein: " + str(distanceDamLev), GREEN)
\end{lstlisting}

\begin{lstlisting}[label=some-code,caption=Функция нахождения расстояния Левенштейна матрично]
def Levenshtein(strFirst, strSecond, flag=False):
	n, m = len(strFirst), len(strSecond)
	matrix = np.full((n + 1, m + 1), 0)  # math.inf)
	
	matrix[0][0] = 0
	for i in range(1, n + 1):
		matrix[i][0] = i
	for i in range(1, m + 1):
		matrix[0][i] = i
	
	for i in range(1, n + 1):
		for j in range(1, m + 1):
			matrix[i][j] = min(
			matrix[i - 1][j - 1] +
			int(not(strFirst[i-1] == strSecond[j-1])),  # R
			matrix[i - 1][j] + 1,                       # D
			matrix[i][j - 1] + 1)                       # I
		
	if flag:
		OutputMatrix(matrix, strFirst, strSecond)
	
	return matrix[-1][-1]
\end{lstlisting}


\begin{lstlisting}[label=some-code,caption=Рекурсивная функция нахождения расстояния Левенштейна ]
def LevenshteinRecursion(strFirst, strSecond):
	if (strFirst == "" or strSecond == ""):
		return abs(len(strFirst) - len(strSecond))
	
	temp = 0 if strFirst[-1] == strSecond[-1] else 1
	return min(
	LevenshteinRecursion(strFirst, strSecond[:-1]) + 1, # I
	LevenshteinRecursion(strFirst[:-1], strSecond) + 1, # D
	LevenshteinRecursion(strFirst[:-1], strSecond[:-1]) + temp # R
	)
\end{lstlisting}


\begin{lstlisting}[label=some-code,caption=Функция нахождения расстояния Дамерау-Левенштейна матрично]
def DamerauLevenshtein(strFirst, strSecond, flag=False):
	n, m = len(strFirst), len(strSecond)
	matrix = np.full((n + 1, m + 1), 0)  # math.inf)
	
	matrix[0][0] = 0
		for i in range(1, n + 1):
	matrix[i][0] = i
		for i in range(1, m + 1):
	matrix[0][i] = i
	
	for i in range(1, n + 1):
		for j in range(1, m + 1):
			matrix[i][j] = min(
			matrix[i - 1][j - 1] +
			int(not(strFirst[i-1] == strSecond[j-1])),  # R
			matrix[i - 1][j] + 1,                       # D
			matrix[i][j - 1] + 1)                       # I
			if i>1 and j>1 and strFirst[i-1] == strSecond[j-2] \
			and strFirst[i-2] == strSecond[j-1]:
			matrix[i][j] = min(matrix[i][j], matrix[i-2][j-2] + 1) # T
	
	if flag:
		OutputMatrix(matrix, strFirst, strSecond)
	return matrix[-1][-1]
\end{lstlisting}


\begin{lstlisting}[label=some-code,caption=Рекурсивная функция нахождения расстояния Дамерау-Левенштейна]
def DamerauLevenshteinRecursion(strFirst, strSecond):
	if (strFirst == "" or strSecond == ""):
		return abs(len(strFirst) - len(strSecond))
	
	temp = 0 if strFirst[-1] == strSecond[-1] else 1
	result = min(
	DamerauLevenshteinRecursion(
	strFirst, strSecond[:-1]) + 1,           # I
	DamerauLevenshteinRecursion(
	strFirst[:-1], strSecond) + 1,           # D
	DamerauLevenshteinRecursion(
	strFirst[:-1], strSecond[:-1]) + temp    # R
	)
	
	if len(strFirst) > 1 and len(strSecond) > 1 and \
		strFirst[-1] == strSecond[-2] and strFirst[-2] == strSecond[-1]:
		result = min(result, DamerauLevenshteinRecursion(
		strFirst[:-2], strSecond[:-2]) + 1)  # T
	
	return result
\end{lstlisting}

\begin{lstlisting}[label=some-code,caption=функция замера времени]
def TimeTest(strFirst, strSecond, countOperations):
	t1 = time.process_time()
	for _ in range(countOperations):
		Levenshtein(strFirst, strSecond)
	t2 = time.process_time()
	print("Levenshtein = ", t2 - t1)
	
	t1 = time.process_time()
	for _ in range(countOperations):
		LevenshteinRecursion(strFirst, strSecond)
	t2 = time.process_time()
	print("Levenshtein (Recursion)= ", t2 - t1)
	
	t1 = time.process_time()
	for _ in range(countOperations):
		DamerauLevenshtein(strFirst, strSecond)
	t2 = time.process_time()
	print("DamerauLevenshtein = ", t2 - t1)
	
	t1 = time.process_time()
	for _ in range(countOperations):
		DamerauLevenshteinRecursion(strFirst, strSecond)
	t2 = time.process_time()
	print("DamerauLevenshtein (Recursion)= ", t2 - t1)
\end{lstlisting}


\section{Тестирование}

В данном разделе будет приведена таблица \ref{table:ref1}, в которой четко отражено тестирование программы. Первый и второй столбец отвечают за введенные пользователем слова.\\

\begin{table}[ht]
	\centering
	\caption{Таблица тестов}
	\label{table:ref1}
	\begin{tabular}{ | l | l | l | l |}
		\hline
		Слово 1 & Слово 2 & Ожидаемый вывод &  Вывод программы  \\ \hline
		сито & столб & 3 & 3 \\ \hline
		exponential & polynomial & 6 & 6 \\ \hline
		Alice & Alice & 0 & 0 \\ \hline
		Alice & alice & 1 & 1 \\ \hline
		ma  & am & 2 1 &  2 1 \\ \hline
		& & 0 & 0 \\ \hline
		abc & cab & 2 & 2 \\ \hline
		\hline
	\end{tabular}
\end{table} 


\fcolorbox{yellow}{green}{Все тесты пройдены успешно.}

\section{Вывод}

В данном разделе были рассмотрены листинги кода, обоснован выбор использованного в данной работе языка программирования и среды разработки, а также была произведена проверка корректной работы программы, благодаря таблице \ref{table:ref1}.
Сравнив представленные листинги, можно сказать, что написание рекуррентных подпрограмм проще, чем матричных.
