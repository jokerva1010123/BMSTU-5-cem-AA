\newpage
\chapter*{Введение}
\addcontentsline{toc}{chapter}{Введение}

\textit{Муравьиный алгоритм} -- алгоритм оптимизации подражанием муравьиной колонии.
один из эффективных полиномиальных алгоритмов
для нахождения приближённых решений задачи коммивояжёра,
а также решения аналогичных задач поиска маршрутов на графах.

Муравьиные алгоритмы серьезно исследуются европейскими учеными с середины 1990-х годов.
Муравьиный алгоритм предоставляет хорошие результаты оптимизации для
многих сложных комбинаторных задач, таких как:

\begin{itemize}
	\item задачи коммивояжера;
	\item раскраски графа;
	\item оптимизации маршрутов грузовиков;
	\item квадратичной задачи о назначениях;
	\item оптимизации сетевых графиков;
	\item задачи календарного планирования.
\end{itemize}


Целью данной работы является изучение и реализация двух алгоритмов:

\begin{enumerate}
	\item полный перебор;
	\item муравьиный алгоритм.
\end{enumerate}

В рамках выполнения работы необходимо решить следующие задачи:

\begin{enumerate}
	\item изучить два, описанных выше, алгоритма для решения задачи коммивояжера;
	\item применить изученные основы для реализации двух алгоритмов;
	\item получить практические навыки;
	\item провести параметризацию муравьиного алгоритма;
	\item провести сравнительный анализ скорости работы реализованных алгоритмов;
	\item выбрать и обосновать выбор язык программирования, для решения поставленной задачи.
\end{enumerate}