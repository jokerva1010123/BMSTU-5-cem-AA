\chapter{Технологическая часть}

\section{Выбор ЯП}

\section{Выбор языка программирования}
В данной лабораторной работе использовался язык программирования - С  \cite{Microsoft},
так как данный язык программирования предоставляет удобные библиотеки и инструменты для
работы со структурами данных и обеспечивает хорошую производительность программного продукта.
В качестве среды разработки я использовала Visual Studio Code \cite{Vs}.
Visual Studio Code подходит не только для  Windows \cite{Win},
но и для Linux \cite{Lin}, это причина,
по которой я выбрала VS code,
т.к. у меня установлена ОС Ubuntu 18.04.4 \cite{Ubuntu}.

\section{Сведения о модулях программы}

Программа состоит из следующих модулей:

\begin{itemize}
	\item main.c - главный файл программы, в котором располагается точка входа в программу;
	\item ant\_algorithm.c - реализация муравьиного алгоритма;
	\item brute\_force.c - реализация алгоритма полного перебора;
	\item matrix.c - модуль для работы с матрицей смежности;
	\item array.c - модуль для организации списка;
	\item city.c - модуль для организации списка городов;
	\item parser.c - модуль для представления результата в графическом виде.
\end{itemize}

На листингах 3.1-3.3 представлен основной код программы.


\begin{lstlisting}[label=some-code,caption=Алгоритм полного перебора]
int get_path_cost(array cities, int matrix[LEN][LEN])
{
	int cost = 0;

	for (int i = 0; i < cities.count - 1; i++)
		cost += matrix[cities.arr[i]][cities.arr[i + 1]];

	return cost;
}

array get_shortest_path(array cities, int matrix[LEN][LEN])
{
	array result[DEPTH_OF_RECURSION];
	array res_arr;

	int min_cost;
	int curr_cost;
	int index = 0;

	int routes_count = 0;

	del_elem(&cities, 0);
	add_elem(&res_arr, 0);
	get_routes(&cities, &res_arr, result, &routes_count);

	min_cost = get_path_cost(result[index], matrix);
	for (int i = 1; i < routes_count; i++)
	{
		curr_cost = get_path_cost(result[i], matrix);
		if (curr_cost < min_cost)
		{
			min_cost = curr_cost;
			index = i;
		}
	}

	return result[index];
}

void get_routes(array *cities, array *res_arr, array result[DEPTH_OF_RECURSION], int *count)
{
	int elem;
	array tmp;

	if (!cities->count)
	{
		add_elem(res_arr, get_elem(*res_arr, 0));
		result[*count] = *res_arr;
		(*count)++;
		del_elem(res_arr, res_arr->count - 1);
	}

	for (int i = 0; i < cities->count; i++)
	{
		elem = get_elem(*cities, i);
		add_elem(res_arr, elem);

		tmp = copy_arr(*cities);
		del_elem(&tmp, i);

		get_routes(&tmp, res_arr, result, count);

		del_elem(res_arr, res_arr->count - 1);
	}
}
\end{lstlisting}

\begin{lstlisting}[label=some-code,caption=Функции для работы с муравьями]
void generate_ants_array(ant ants[ANTS_MAX_COUNT], int count)
{
	int elem;
	for (int i = 0; i < count; i++)
	{
		ants[i].way.count = 0;
		ants[i].route.count = 0;
		elem = rand() % count;
		add_elem(&(ants[i].way), elem);
		for (int j = 0; j < count; j++)
		{
			if (j == elem)
				continue;
			add_elem(&(ants[i].route), j);
		}
	}
}

void choice_next_city(ant *ants, float matrix_pheromones[LEN][LEN], int matrix[LEN][LEN], float alpha, float beta)
{
	float numerator = 0;
	float denominator = 0;

	float tao, reverse_cost;
	int cost;

	int city_curr = ants->way.arr[ants->way.count - 1];
	for (int i = 0; i < ants->route.count; i++)
	{
		cost = matrix[city_curr][ants->route.arr[i]];
		tao = matrix_pheromones[city_curr][ants->route.arr[i]];

		if (!cost)
			continue;

		reverse_cost = 1.0 / cost;

		denominator += powf(tao, alpha) + powf(reverse_cost, beta);
	}

	float p_array[LEN];
	float sum = 0;
	for (int i = 0; i < ants->route.count; i++)
	{
		cost = matrix[city_curr][ants->route.arr[i]];
		tao = matrix_pheromones[city_curr][ants->route.arr[i]];

		if (!cost)
			continue;

		reverse_cost = 1.0 / cost;
		p_array[i] = (powf(tao, alpha) + powf(reverse_cost, beta)) / denominator;
		sum += p_array[i];
	}
	float x = (float)rand() / RAND_MAX;
	int index = 0;
	while (x >= 0)
	{
		x -= p_array[index];
		index++;
	}
	add_elem(&ants->way, get_elem(ants->route, index - 1));
	del_elem(&ants->route, index - 1);
}

void ants_choose_way(ant ants[ANTS_MAX_COUNT], float matrix_pheromones[LEN][LEN], int matrix[LEN][LEN], int count, float alpha, float beta)
{
	for (int i = 0; i < count; i++)
		choice_next_city(&ants[i], matrix_pheromones, matrix, alpha, beta);
}
\end{lstlisting}

\begin{lstlisting}[label=some-code,caption=Муравьиный алгоритм]
int calculate_Q(int matrix[LEN][LEN], int count)
{
	int Q = 0;

	for (int i = 0; i < count; i++)
		for (int j = 0; j < i; j++)
			Q += matrix[i][j];

	return Q * 2;
}

void add_pheromones(int matrix[LEN][LEN], float matrix_pheromones[LEN][LEN], int count, int Q, ant ants[ANTS_MAX_COUNT])
{
	int city_first, city_second;
	int curr_cost;
	float delta_tao = 0;

	for (int i = 0; i < count; i++)
	{
		curr_cost = get_path_cost(ants[i].way, matrix);

		delta_tao += (float)Q / curr_cost;
	}

	for (int i = 0; i < count; i++)
	{
		for (int j = 0; j < ants[i].way.count - 1; j++)
		{
			city_first = ants[i].way.arr[j];
			city_second = ants[i].way.arr[j + 1];
			matrix_pheromones[city_first][city_second] = matrix_pheromones[city_first][city_second] + delta_tao;
			matrix_pheromones[city_second][city_first] = matrix_pheromones[city_second][city_first] + delta_tao;
		}
	}
}

void correct_pheromones(float matrix_pheromones[LEN][LEN], int count)
{
	for (int i = 0; i < count; i++)
		for (int j = 0; j < i; j++)
			if (matrix_pheromones[i][j] <= 0.1)
				matrix_pheromones[i][j] = matrix_pheromones[j][i] = 0.1;
}

void evaporation(float matrix_pheromones[LEN][LEN], int count, float p)
{
	float tmp = 1 - p;
	for (int i = 0; i < count; i++)
		for (int j = 0; j < i; j++)
		{
			matrix_pheromones[i][j] = tmp * matrix_pheromones[i][j];
			matrix_pheromones[j][i] = tmp * matrix_pheromones[j][i];
		}
}
array ant_algorithm(int matrix[LEN][LEN], int count, array cities, int tmax, float p, float alpha, float beta)
{
	int Q = calculate_Q(matrix, count);

	array best_way = copy_arr(cities);
	add_elem(&best_way, get_elem(best_way, 0));

	int best_cost = get_path_cost(best_way, matrix);
	int curr_cost = 0;

	float matrix_pheromones[LEN][LEN];
	fill_matrix(matrix_pheromones, count, PHEROMONE_MIN);

	ant ants[ANTS_MAX_COUNT];

	for (int t = 0; t < tmax; t++)
	{
		generate_ants_array(ants, count);

		for (int i = 0; i < count - 1; i++)
		{
			ants_choose_way(ants, matrix_pheromones, matrix, count, alpha, beta);
		}
		for (int i = 0; i < count; i++)
			add_elem(&ants[i].way, get_elem(ants[i].way, 0));

		for (int i = 0; i < count; i++)
		{
			curr_cost = get_path_cost(ants[i].way, matrix);
			if (curr_cost < best_cost)
			{
				best_cost = curr_cost;
				best_way = copy_arr(ants[i].way);
			}
		}

		evaporation(matrix_pheromones, count, p);
		add_pheromones(matrix, matrix_pheromones, count, Q, ants);
		correct_pheromones(matrix_pheromones, count);
	}

	return best_way;
}
\end{lstlisting}


\section{Вывод}

В данном разделе были разобраны листинги рис 3.1-3.2, показывающие работу алгоритмов для решения задачи коммивояжера.