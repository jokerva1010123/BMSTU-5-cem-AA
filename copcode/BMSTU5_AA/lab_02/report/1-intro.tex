\Introduction

В данной лабораторной работе будут рассмотрены алгоритмы умножения матриц.

Матрицы A и B могут быть перемножены, если число столбцов матрицы A равно числу строк B. 

Умножение матриц активно используется в компьютерной графике.
В частности для того, чтобы передвинуть персонажа с координатами x, y, z
на некоторое смещение dx, dy, dz. В этом случае нужно умножить 
координаты персонажа на матрицу перемещения. 
Аналогичная ситуация, если нужно повернуть персонажа. 
В этом случае матрица перемещения заменяется на матрицу вращения
и производится та же операция умножения матриц.  

Целью данной работы является изучение, программная реализация, а также 
сравнение алгоритмов умножения матриц.

В рамках выполнения работы необходимо решить следующие задачи.

\begin{enumerate}
	\item Изучить и реализовать на выбранном ЯП стандартный алгоритм умножения матриц.
	\item Изучить и реализовать алгоритм Винограда умножения матриц.
	\item Оптимизировать алгоритм Винограда умножения матриц.
	\item Сравнить временные характеристики вышеизложенных алгоритмов.
	\item Оценить алгоритмы.
\end{enumerate}
