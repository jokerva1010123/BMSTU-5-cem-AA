\chapter{Экспериментальная часть}

В данном разделе будет произведено сравнение вышеизложенных алгоритмов.

\section{Временные характеристики}

Для сравнения возьмем квадратные матрицы размерностью [10, 20, 30,\dots,100]. 
Так как подсчет умножения матриц считается короткой задачей, воспользуемся усреднением массового эксперимента. 
Для этого сложим результат работы алгоритма n раз (n >= 10), после чего поделим на n. 
Тем самым получим достаточно точные характеристики времени. 
Сравнение произведем при n = 50.
Результат можно увидеть на рис. \ref{fg:ref3}. 

\begin{figure}[ht!]
	\centering{
		\includegraphics[width=0.8\textwidth]{img/time1.png}
		\caption{Временные характеристики на четных размерах матриц}
		\label{fg:ref3}}
\end{figure}

На рис. \ref{fg:ref4} показана работа алгоритмов с матрицами, размерностью [11, 21, 31,\dots,91].

\begin{figure}[ht!]
	\centering{
		\includegraphics[width=0.8\textwidth]{img/time2.png}
		\caption{Временные характеристики на нечетных размерах матриц}
		\label{fg:ref4}}
\end{figure}

\section{Сравнительный анализ алгоритмов}

Введем модель вычислений трудоемкости алгоритма.
Пусть трудоемкость 1 у следующих базовых операций: +, -, *, /, \%, =, ==, !=, <, <=, >, >=, [].
Трудоемкость цикла: fцикла = fиниц + fсравн + Nитер ∗ (fтела +
fинкрем + fсравн ).Трудоемкость условного перехода 1.

Стандартная реализация алгоритма не эффективна по времени, так как
обладает трудоемкостью 5qmn + 4n + 4mn + 5.
Оценка трудоемкости данного алгоритма составляет 5qmn. 
По памяти в стандартном алгоритме умножения матриц требуется m*n памяти под результат.

Теперь рассмотрим алгоритм Винограда умножения матриц. 
Реализация алгоритма Винограда обла­дает трудоемкостью формула \ref{eq:ref8}.
Оценка трудоемкости данного алгоритма составляет 3qmn.
В алгоритме Винограда умножения матриц требуется дополнительно m+n памяти под результат.

\begin{equation}
	3qmn + 7mn + 2m + 5q + 6n + 11 +
	\left[ 
	\begin{gathered} 
		0 $ л.с.$ \\ 
		5mn+4n+2 $ х.с.$ \\ 
	\end{gathered}
	\right.
	\label{eq:ref8}
\end{equation}

\section{Вывод}

В данном разделе было произведено сравнение количества затраченного вре­мени вышеизложенных алгоритмов.
Самым быстрым оказался модифицированный алгоритм Винограда.
При этом в алгоритме Винограда умножения матриц требуется дополнительно m+n памяти под результат.