
\chapter{ Аналитический раздел}
\label{cha:analysis}

\section{Описание алгоритмов}

% Все изучаемые алгоритмы будут сортировать список в порядке возрастания ключевого значения.

\textbf{Сортировка вставками.}

Основная идея сортировки вставками состоит в том, что при добавлении нового элемента в уже отсортированный список его стоит сразу
вставлять в нужное место вместо того, чтобы вставлять его в произвольное место, а затем заново сортировать весь список.
В алгоритме сортировки вставками первый элемент любого списка считается отсортированным списком длиной один.
Двухэлементный отсортированный список создается добавлением второго элемента исходного списка в нужное место одноэлементного списка, содержащего первый элемент. 
Данный процесс вставки продолжается до тех пор, 
пока все элементы исходного списка не окажутся в расширяющейся отсортированной части списка.

\textbf{Пузырьковая сортировка.}

Алгоритм пузырьковой сортировки совершает несколько проходов по списку. 
При каждом проходе происходит сравнение соседних элементов. 
Если порядок соседних элементов неправильный, то они меняются местами.
Каждый проход начинается с начала списка. 
% После нулевого прохода по массиву в начале массива оказывается самый маленький элемент.
% Если при каком-то проходе не произошло ни одной перестановки элементов, то все они стоят в нужном порядке, и исполнение алгоритма
% можно прекратить. 


\textbf{Быстрая сортировка}

В алгоритме быстрой сортировки (Quicksort) используется рекурсивный подход.
Выбрав опорный элемент в списке данный алгоритм сортировки делит список на две части, относительно выбранного элемента.
Далее в первую часть попадают все элементы, меньшие выбранного, а во вторую — большие элементы. 
Если в данных частях более двух элементов, рекурсивно запускается для него та же процедура. 
В конце получится полностью отсортированная последовательность.

\section{Вывод}

Пузырьковая сортировка сравнивает элементы попарно, переставляя между собой элементы тех пар, порядок в которых нарушен.
Сортировка вставками, сортирует список, вставляя очередной элемент в нужное место уже отсортированного списка.
Быстрая сортировка определяет опорный элемент и далее переставляет элементы, относительно выбранного элемента.

Были рассмотрены основополагающие материалами, которые в дальнейшем потребуются при реализации алгоритмов сортировки.  



