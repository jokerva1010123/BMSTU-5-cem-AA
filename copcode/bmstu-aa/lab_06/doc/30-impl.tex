\chapter{Технологическая часть}

В данном разделе приведены средства программной реализации и листинг кода.

\section{Требования к ПО}

К программе предъявляется ряд требований:
\begin{itemize}
    \item на вход подаётся матрица смежности графа;
    \item на выходе программа выдаёт минимальную длину и путь, на котором данная длина достигнута.
\end{itemize}

\section{Средства реализации}

В качестве языка программирования для реализации данной лабораторной работы был выбран современный компилируемый ЯП Rust \cite{rustlang}. Данный выбор обусловлен популярностью языка и скоростью его выполнения, а также тем, что данный язык предоставляет широкие возможности для написания тестов \cite{rusttest}.

\section{Листинги кода}

В листинге \ref{lst:brute} представлена реализация алгоритма полного перебора, в листинге \ref{lst:antsolver} представлена реализация муравьиного алгоритма. В листингах \ref{lst:config} и \ref{lst:ant} представлены конфигурационная структура и структура муравья с методами соотвественно.

\begin{lstinputlisting}[
        caption={Реализация полного перебора.},
        label={lst:brute},
        style={rust}
    ]{../src/lib/brute_solver.rs}
\end{lstinputlisting}

\begin{lstinputlisting}[
        caption={Реализация муравьиного алгоритма.},
        label={lst:antsolver},
        style={rust}
    ]{../src/lib/ant_solver/mod.rs}
\end{lstinputlisting}

\begin{lstinputlisting}[
        caption={Конфигурационная структура.},
        label={lst:config},
        style={rust}
    ]{../src/lib/ant_solver/config.rs}
\end{lstinputlisting}

\begin{lstinputlisting}[
        caption={Структура муравья и её методы.},
        label={lst:ant},
        style={rust}
    ]{../src/lib/ant_solver/ant.rs}
\end{lstinputlisting}

\section{Тестирование функций.}

В таблице~\ref{tab:tests} приведены тесты для функции, реализующей алгоритм для решения задачи коммивояжера. Тесты пройдены успешно.

\begin{table}[h!]
    \begin{center}
        \caption{\label{tab:tests} Тестирование функций}
        \begin{tabular}{|c@{\hspace{7mm}}|c@{\hspace{7mm}}|c@{\hspace{7mm}}|c@{\hspace{7mm}}|}
            \hline
            Матрица смежности & Ожидаемый наименьший путь \\ \hline
            $\begin{pmatrix}
                0 &  9 &  12 &  21\\
                9 &  0 &  9 &  21\\
                12 &  9 &  0 &  21\\
                21 &  21 &  21 &  0
            \end{pmatrix}$ &
            60, [0, 1, 2, 3, 0]\TBstrut \\ \hline
            $\begin{pmatrix}
                0 &  21 &  12 &  9\\
                21 &  0 &  12 &  15\\
                12 &  12 &  0 &  15\\
                9 &  15 &  15 &  0
            \end{pmatrix}$ &
            48, [0, 2, 1, 3, 0]\TBstrut \\ \hline
            $\begin{pmatrix}
                0 &  24 &  27 &  21\\
                24 &  0 &  12 &  12\\
                27 &  12 &  0 &  6\\
                21 &  12 &  7 &  0
            \end{pmatrix}$ &
            63, [0, 1, 2, 3, 0]\TBstrut \\ \hline
        \end{tabular}
    \end{center}
\end{table}

\section*{Вывод}
Спроектированные алгоритмы были реализованы и протестированы.
