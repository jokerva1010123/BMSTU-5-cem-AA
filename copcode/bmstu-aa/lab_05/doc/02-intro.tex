\chapter*{Введение}
\addcontentsline{toc}{chapter}{Введение}

Конвейер — способ организации вычислений, используемый в современных процессорах и контроллерах с целью повышения их производительности (увеличения числа инструкций, выполняемых в единицу времени — эксплуатация параллелизма на уровне инструкций), технология, используемая при разработке компьютеров и других цифровых электронных устройств.

Сам термин <<конвейер>> пришёл из промышленности, где используется подобный принцип работы — материал автоматически подтягивается по ленте конвейера к рабочему, который осуществляет с ним необходимые действия, следующий за ним рабочий выполняет свои функции над получившейся заготовкой, следующий делает ещё что-то. Таким образом, к концу конвейера цепочка рабочих полностью выполняет все поставленные задачи, сохраняя высокий темп производства. Например, если на самую медленную операцию затрачивается одна минута, то каждая деталь будет сходить с конвейера через одну минуту. В процессорах роль рабочих исполняют функциональные модули, входящие в состав процессора.

Цель данной работы: получить навык организации асинхронного взаимодействия потоков на примере конвейерной обработки данных.

В рамках выполнения работы необходимо решить следующие задачи:
\begin{itemize}
	\item рассмотреть и изучить конвейерную обработку данных;
	\item реализовать конвейер с количеством лент не меньше трех в многопоточной среде;
	\item на основании проделанной работы сделать выводы.
\end{itemize}
