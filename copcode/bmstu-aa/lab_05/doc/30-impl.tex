\chapter{Технологическая часть}

В данном разделе приведены средства программной реализации и листинг кода.

\section{Требования к ПО}

К программе предъявляется ряд требований:
\begin{itemize}
    \item на вход ковейера подаётся массив задач, которые на нём нужно обработать;
	\item на выходе - лог-запись, в которой записаны в упорядоченном по времени порядке события начала и конца обработки определённого задания на ленте.
\end{itemize}

\section{Средства реализации}

В качестве языка программирования для реализации данной лабораторной работы был выбран современный компилируемый ЯП Rust \cite{rustlang}. Данный выбор обусловлен популярностью языка и скоростью его выполнения, а также тем, что данный язык предоставляет широкие возможности для написания тестов \cite{rusttest}.

\section{Листинг кода}

В листинге \ref{lst:conveyor} приведена реализация конвейера. В листингах \ref{lst:task} и \ref{lst:add} приведены реализации задачи и допольнительных используемых структур.

\begin{lstinputlisting}[
	caption={Реализация конвейера},
	label={lst:conveyor},
	style={rust}
]{../src/lib/conveyor.rs}
\end{lstinputlisting}

\begin{lstinputlisting}[
	caption={Структура задачи},
	label={lst:task},
	style={rust},
    linerange={1-152}
]{../src/lib/task.rs}
\end{lstinputlisting}

\begin{lstinputlisting}[
	caption={Дополнительные структуры},
	label={lst:add},
	style={rust}
]{../src/lib/additional_structs.rs}
\end{lstinputlisting}

\section{Тестирование функций.}

В таблице \ref{tab:tests} представлены данные для тестирования. Все тесты пройдены успешно.

\begin{table}[h!]
	\begin{center}
		\begin{tabular}{|c@{\hspace{7mm}}|c@{\hspace{7mm}}|c@{\hspace{7mm}}|c@{\hspace{7mm}}|}
            \hline
            Строка & Подстрока & Ожидаемый результат & Результат \\
            \hline
            ``ааааааа'' & ``аааа'' & $[0, 1, 2, 3]$ & $[0, 1, 2, 3]$ \\
            ``aaaabaaaa'' & ``aaa'' & $[0, 1, 5, 6]$ & $[0, 1, 5, 6]$ \\
            ``aaaaaaaaaa'' & ``b'' & $[]$ & $[]$ \\
            \hline
		\end{tabular}
	\end{center}
	\caption{\label{tab:tests} Тестирование функций.}
\end{table}

\section*{Вывод}

Была разработана реализацияя конвейерных вычислений.
