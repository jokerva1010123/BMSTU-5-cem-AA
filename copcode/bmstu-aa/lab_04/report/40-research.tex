\chapter{Исследовательская часть}

\section{Технические характеристики}

\begin{itemize}
	\item Операционная система: Manjaro \cite{manjaro} Linux \cite{linux} x86\_64.
	\item Память: 8 ГБ.
	\item Процессор: Intel® Core™ i7-8550U\cite{intel}.
\end{itemize}

Тестирование проводилось на ноутбуке, включенном в сеть электропитания. Во время тестирования ноутбук был нагружен только встроенными приложениями окружения, окружением, а также непосредственно системой тестирования.

\section{Время выполнения алгоритмов}

Результаты замеров приведены в таблицах \ref{tbl:cmp} и \ref{tbl:threads}.
На рисунке \ref{plt:cmp} приведено сравнение простого алгоритма и параллельных алгоритмах при исполнении на 8 потоках, а на рисунке \ref{plt:threads} приведен график зависимости времени работы 2 параллельных алгоритмов от кол-ва используемых потоков при перемножении матриц размера 500.
\clearpage

\begin{table}[h!]
	\begin{center}
		\begin{tabular}{|c|c|c|c|}
			\hline
                 & \multicolumn{3}{c|}{\bfseries Сравнение времени работы алгоритмов, с}                           \\ \cline{2-4}
			\bfseries Размер & \bfseries Простой & \bfseries Параллельный 1 & \bfseries Параллельный 2
			\csvreader{assets/csv/cmp.csv}{}
			{\\\hline \csvcoli&\csvcolii&\csvcoliii&\csvcoliv}
			\\\hline
		\end{tabular}
	\end{center}
	\caption{Время работы алгоритмов при различных размерах}
	\label{tbl:cmp}
\end{table}

\begin{figure}
	\centering
	\begin{tikzpicture}[scale=1.5]
		\begin{axis}[
			axis lines=left,
			xlabel=Размер,
			ylabel={Время, с},
			legend pos=north west,
			ymajorgrids=true
		]
			\addplot table[x=size,y=simple,col sep=comma]{assets/csv/cmp.csv};
			\addplot table[x=size,y=parall1,col sep=comma]{assets/csv/cmp.csv};
			\addplot table[x=size,y=parall2,col sep=comma]{assets/csv/cmp.csv};

			\legend{Однопоточный, Параллельный 1, Параллельный 2}
		\end{axis}
	\end{tikzpicture}
	\captionsetup{justification=centering}
	\caption{Зависимость времени работы алгоритмов от размеров квадратной матрицы}
	\label{plt:cmp}
\end{figure}

\clearpage

\begin{table}
	\begin{center}
		\begin{tabular}{|c|c|c|}
			\hline
                 & \multicolumn{2}{c|}{\bfseries Время при кол-ве потоков, с}                                           \\ \cline{2-3}
			\bfseries Потоки & \bfseries Параллельный 1 & \bfseries Параллельный 2
			\csvreader{assets/csv/threads.csv}{}
			{\\\hline \csvcoli&\csvcolii&\csvcoliii}
			\\\hline
		\end{tabular}
	\end{center}
	\caption{Сравнение времени работы 2 параллельных алгоритмов при различном кол-ве потоков}
	\label{tbl:threads}
\end{table}

\begin{figure}
	\centering
	\begin{tikzpicture}[scale=1.5]
		\begin{axis}[
			axis lines=left,
            xlabel={Кол-во потоков},
			ylabel={Время, с},
			legend pos=north west,
			ymajorgrids=true
		]
			\addplot table[x=threads,y=parall1,col sep=comma]{assets/csv/threads.csv};
			\addplot table[x=threads,y=parall2,col sep=comma]{assets/csv/threads.csv};
			\legend{Параллельный 1, Параллельный 2}
		\end{axis}
	\end{tikzpicture}
	\captionsetup{justification=centering}
	\caption{Зависимость времени работы параллельных алгоритмов от кол-ва потоков на квадратных матрицах размера 500}
	\label{plt:threads}
\end{figure}

\clearpage

Примечание: на одном потоке параллельные алгоритмы выполняются 7.379062295 секунд.

\section*{Вывод}

Наилучшее время параллельные алгоритмы показали на 8 потоках (можно заметить, что минимум времени на графике \ref{plt:threads} находится в точке 8), соответствующих количеству логических ядер ноутбука, на котором проводилось тестирование. На матрицах размеров 500 на 500 параллельные алгоритмы улучшают время однопоточной реализации примерно на $72\%$.
Количество потоков, большее 8, в итоге немногим замедляет время необходимостью переключения между потоками, а так же большим количеством инициализаций.
