\chapter{Конструкторская часть}

\section{Разработка алгоритмов}

\subsection{Алгоритм Винограда}

На рисунке \ref{img:vino_simple} представлена схема однопоточного алгоритма Винограда с выделением этапов, а на рисунке \ref{img:subroutines} представлены подробные схемы 1 и 2 этапов (предварительный расчёт для строк первой матрицы и столбцов второй).

\imgw{170mm}{vino_simple}{Схема однопоточного алгоритма Винограда}
\clearpage
\img{220mm}{subroutines}{Схема предварительного расчета для строк первой матрицы и столбцов второй}
\clearpage

\subsection{Параллельная реализация алгоритма Винограда}


Пусть размеры перемножаемых матриц непосредственно равны $M \times K$ и $K \times N$.

Рассмотрим необходимые для распараллеливания замечания:
\begin{itemize}
    \item каждый из выделенных этапов может быть выполнен независимо от других;
    \item в следствие независимости этапов, каждый их них может быть выполнен в любой момент, в том числе и параллельно с другими;
    \item каждый из выделенных этапов содержит цикл в некотором промежутке, который может быть разбит на некоторое множество меньших промежутков, в сумме составляющих исходный;
    \item трудоёмкости первого и второго этапов - величины одного порядка и относятся $M / N$;
    \item трудоёмкость третьего этапа в $N$ раз больше трудоёмкости первого этапа и в $M$ раз больше трудоёмкости второго этапа, что, не позволяет распараллелить третий этап с первым и (или) вторым;
    \item четвертый этап требует обращения к матрице на каждой итерации цикла, что при распараллеливании приведёт к большому числу блокирований разделяемой памяти, и, в купе с затратами на порождение потоков, будет неэффективно.
\end{itemize}
\clearpage

На рисунке \ref{img:parallel} представлена схема алгоритма функции, запускающая в требуемом количестве потоков функцию-аргумент, передавая ей равные по размеру промежутки из разбиения исходного. С помощью этой функции распараллеливаются этапы, описанные в рисунке \ref{img:vino_simple}.
\imgw{170mm}{parallel}{Схема распараллеливания произвольной функции, способной выполняться в некоторых независимых промежутках}

Исходя из всего, описанного выше, можно сделать вывод, что третий этап следует параллелить независимо от других, в то время как первый и второй этапы можно сделать как параллельно (где каждому из этапов достается половина от общего числа потоков), так и последовательно (где каждому из этапов достается общее число потоков). Так же очевидно, что не следует параллелить четвертый этап.

На рисунке \ref{img:parallel1} представлена схема с параллельным выполнением первого и второго этапов, а на рисунке \ref{img:parallel2} представлена схема с последовательным выполнением первого и второго этапов.

\imgw{170mm}{parallel1}{Схема с параллельным выполнением первого и второго этапов}
\img{220mm}{parallel2}{Схема с последовательным выполнением первого и второго этапов}
\clearpage

\section*{Вывод}

На основе теоретических данных, полученных из аналитического раздела, была построена схема алгоритма Винограда, а так же после разделения алгоритма на этапы были предложены 2 схемы параллельного выполнения данных этапов.
