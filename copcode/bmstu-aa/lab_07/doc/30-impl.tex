\chapter{Технологическая часть}

В данном разделе приведены средства программной реализации и листинг кода.

\section{Требования к ПО}

К программе предъявляется ряд требований:
\begin{itemize}
    \item на вход подается ключ;
    \item на выход программа выдает значение, хранящееся в словаре по ключу, если таковое присутствует, ``пустое'' значение в противном случае.
\end{itemize}

\section{Средства реализации}

В качестве языка программирования для реализации данной лабораторной работы был выбран современный компилируемый ЯП Rust \cite{rustlang}. Данный выбор обусловлен популярностью языка и скоростью его выполнения, а также тем, что данный язык предоставляет широкие возможности для написания тестов \cite{rusttest}.

\section{Листинг кода}

В листинге \ref{lst:maps} приведена реализация словарей.

\begin{lstinputlisting}[
	caption={Реализация словарей.},
	label={lst:maps},
	style={rust}
]{../src/lib/maps.rs}
\end{lstinputlisting}

\section{Тестирование функций.}

В таблице \ref{tab:tests} представлены данные для тестирования. Все тесты пройдены успешно.

\begin{table}[h!]
	\begin{center}
		\begin{tabular}{|c | c | c | c |}
            \hline
            Ключ & Словарь & Ожидание & Результат \\
            \hline
            1 & \texttt{\{1: "Иван М."\,, 2: "Олег К." \}} & "Иван М." & "Иван М." \\
            3 & \texttt{\{1: "Иван М."\,, 2: "Олег К." \}} & \text{NOT\_FOUND} & \text{NOT\_FOUND} \\
            1 & \texttt{\{\}} & \text{NOT\_FOUND} & \text{NOT\_FOUND} \\
            \hline
		\end{tabular}
	\end{center}
	\caption{\label{tab:tests} Тестирование функций.}
\end{table}

\section*{Вывод}

Была разработана и протестирована реализация словарей.
