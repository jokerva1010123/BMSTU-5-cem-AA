\chapter{Исследовательская часть}

\section{Технические характеристики}

\begin{itemize}
	\item Операционная система: Manjaro \cite{manjaro} Linux \cite{linux} x86\_64.
	\item Память: 8 GiB.
	\item Процессор: Intel® Core™ i7-8550U\cite{intel}.
\end{itemize}

Тестирование проводилось на ноутбуке, включенном в сеть электропитания. Во время тестирования ноутбук был нагружен только встроенными приложениями окружения, окружением, а также непосредственно системой тестирования.

\section{Время выполнения алгоритмов}

Результаты замеров приведены в таблицах \ref{tbl:sorted}, \ref{tbl:reversed} и \ref{tbl:random}.
На рисунках \ref{plt:sorted}, \ref{plt:reversed} и \ref{plt:random} приведены графики зависимостей времени работы алгоритмов сортировки от размеров массивов на отсортированных, обратно отсортированных и случайных данных.
\clearpage

\begin{table}[h!]
	\begin{center}
		\begin{tabular}{|c|c|c|c|}
			\hline
			                 & \multicolumn{3}{c|}{\bfseries Время сортировки, мс}           \\ \cline{2-4}
			\bfseries Размер & \bfseries Пузырьком & \bfseries Вставками & \bfseries Выбором
			\csvreader{assets/csv/sorted.csv}{}
			{\\\hline \csvcoli&\csvcolii&\csvcoliii&\csvcoliv}
			\\\hline
		\end{tabular}
	\end{center}
	\caption{Время работы алгоритмов сортировки на отсортированных данных}
	\label{tbl:sorted}
\end{table}

\begin{figure}[h!]
	\centering
	\begin{tikzpicture}
		\begin{axis}[
			axis lines=left,
			xlabel=Размер,
			ylabel={Время, мс},
			legend pos=north west,
			ymajorgrids=true
		]
			\addplot table[x=size,y=bubble,col sep=comma]{assets/csv/sorted.csv};
			\addplot table[x=size,y=insertion,col sep=comma]{assets/csv/sorted.csv};
			\addplot table[x=size,y=selection,col sep=comma]{assets/csv/sorted.csv};
			\legend{Пузырьком, Вставками, Выбором}
		\end{axis}
	\end{tikzpicture}
	\captionsetup{justification=centering}
	\caption{Зависимость времени работы алгоритма сортировки от размера отсортированного массива}
	\label{plt:sorted}
\end{figure}
\clearpage

\begin{table}[h]
	\begin{center}
		\begin{tabular}{|c|c|c|c|}
			\hline
			                 & \multicolumn{3}{c|}{\bfseries Время сортировки, мс}           \\ \cline{2-4}
			\bfseries Размер & \bfseries Пузырьком & \bfseries Вставками & \bfseries Выбором
			\csvreader{assets/csv/reversed.csv}{}
			{\\\hline \csvcoli&\csvcolii&\csvcoliii&\csvcoliv}
			\\\hline
		\end{tabular}
	\end{center}
	\caption{Время работы алгоритмов сортировки на обратно отсортированных данных}
	\label{tbl:reversed}
\end{table}

\begin{figure}[h]
	\centering
	\begin{tikzpicture}
		\begin{axis}[
			axis lines=left,
			xlabel=Размер,
			ylabel={Время, мс},
			legend pos=north west,
			ymajorgrids=true
		]
			\addplot table[x=size,y=bubble,col sep=comma]{assets/csv/reversed.csv};
			\addplot table[x=size,y=insertion,col sep=comma]{assets/csv/reversed.csv};
			\addplot table[x=size,y=selection,col sep=comma]{assets/csv/reversed.csv};
			\legend{Пузырьком, Вставками, Выбором}
		\end{axis}
	\end{tikzpicture}
	\captionsetup{justification=centering}
	\caption{Зависимость времени работы алгоритма сортировки от размера массива, отсортированного в обратном порядке}
	\label{plt:reversed}
\end{figure}
\clearpage

\begin{table}[h]
	\begin{center}
		\begin{tabular}{|c|c|c|c|}
			\hline
			                 & \multicolumn{3}{c|}{\bfseries Время сортировки, мс}           \\ \cline{2-4}
			\bfseries Размер & \bfseries Пузырьком & \bfseries Вставками & \bfseries Выбором
			\csvreader{assets/csv/random.csv}{}
			{\\\hline \csvcoli&\csvcolii&\csvcoliii&\csvcoliv}
			\\\hline
		\end{tabular}
	\end{center}
	\caption{Время работы алгоритмов сортировки на случайных данных}
	\label{tbl:random}
\end{table}

\begin{figure}[h]
	\centering
	\begin{tikzpicture}
		\begin{axis}[
			axis lines=left,
			xlabel=Размер,
			ylabel={Время, мс},
			legend pos=north west,
			ymajorgrids=true
		]
			\addplot table[x=size,y=bubble,col sep=comma]{assets/csv/random.csv};
			\addplot table[x=size,y=insertion,col sep=comma]{assets/csv/random.csv};
			\addplot table[x=size,y=selection,col sep=comma]{assets/csv/random.csv};
			\legend{Пузырьком, Вставками, Выбором}
		\end{axis}
	\end{tikzpicture}
	\captionsetup{justification=centering}
	\caption{Зависимость времени работы алгоритма сортировки от размера случайного массива}
	\label{plt:random}
\end{figure}
\clearpage

\section*{Вывод}

Алгоритм сортировки вставками работает лучше остальных двух на случайных числах и уже отсортированных, практический интерес, конечно, представляет лишь первый случай, на котором сортировка вставками почти стабильно быстрее других рассматриваемых алгоритмов. Например, на массиве в 5000 элементов сортировка вставками выигрывает по времени на случайных данных пузырьковую сортировку на $66\%$, выбором — на $12\%$.
