\chapter{Технологическая часть}
В данном разделе будут приведены требования к программному обеспечению, средства реализации и листинги кода.

\section{Выбор языка программирования}

В данной лабораторной работе использовался язык программирования - С\# \cite{csharplang}.

Данный язык позволяет создавать нативные потоки.

В качестве среды разработки выбор сделан в сторону Visual Studio 2019 \cite{wind}. 

Замеры времени производились с помощью структуры DateTime \cite{csharplang1} и свойства DateTime.Now.Ticks \cite{csharplang2}.

\section{Сведения о модулях программы}
Программа состоит из одного модуля: Program.cs - главный файл программы, в котором содержится точка входа в программу, генерация массивов и очередей, а также реализация конвейера, и всех лент данного конвейера.

\section{Реализация алгоритмов}

За счёт средств языка данный модуль позволяет раздавать задачи из главного потока рабочим потокам. 

Таким образом, в одном модуле описана работа соответствующих потоков, моделирующих ленты конвейера, и все необходимые блокировки: блокировка мьютекса очереди для потокобезопасного добавления и извлечения заявок, а также специфичная для языка т.н. блокировка потока -- средство выдачи задания потоку.


В листинге \ref{lst:con} представлена реализация конвейера.

\captionsetup{singlelinecheck = false, justification=raggedright}
\begin{lstlisting}[label=lst:con,caption=Реализация конвейера]
public static void Conveyor(object obj)
{ 
	ThreadArgs args = (ThreadArgs)obj;
	int avg = 0, count = 0, proc = 0;
	
	intPtr array;
	
	if (args.firstQueue.Count != 0)
	{ 
		lock (args.firstQueue)
		{
			array = args.firstQueue.Dequeue();
		}
		
		lock (FStage)
		{
			Int64 t1, t2;
			t1 = DateTime.Now.Ticks;
			Console.WriteLine("Lenta 1 \t{0} \t1 \t{1}",
			Thread.CurrentThread.Name,
			t1);
			
			avg = AvgArrayInt(array);
			
			t2 = DateTime.Now.Ticks;
			Console.WriteLine("Lenta 1 \t{0} \t0 \t{1} \t{2}",
			Thread.CurrentThread.Name,
			t1, t2 - t1);
		}
		
		lock (args.secondQueue)
		{
			Int64 t = DateTime.Now.Ticks;
			time2in.Add(t);
			args.secondQueue.Enqueue(array);
		}
	}
	
	if (args.secondQueue.Count != 0)
	{
		lock (SStage)
		{
			Int64 t1, t2;
			t1 = DateTime.Now.Ticks;
			Console.WriteLine("Lenta 2 \t{0} \t1 \t{1}",
			Thread.CurrentThread.Name,
			t1);
			
			lock (args.secondQueue)
			{
				Int64 t = DateTime.Now.Ticks;
				time2out.Add(t);
				array = args.secondQueue.Dequeue();
			}
			
			count = CountBigAvgInt(array, avg);
			
			t2 = DateTime.Now.Ticks;
			Console.WriteLine("Lenta 2 \t{0} \t0 \t{1} \t{2}",
			Thread.CurrentThread.Name,
			t1, t2 - t1);
		}
		
		lock (args.thirdQueue)
		{
			Int64 t = DateTime.Now.Ticks;
			time3in.Add(t);
			args.thirdQueue.Enqueue(array);
		}
	}
	
	if (args.thirdQueue.Count != 0)
	{
		lock (TStage)
		{
			Int64 t1, t2;
			t1 = DateTime.Now.Ticks;
			Console.WriteLine("Lenta 3 \t{0} \t1 \t{1}",
			Thread.CurrentThread.Name,
			t1);
			
			lock (args.thirdQueue)
			{
				Int64 t = DateTime.Now.Ticks;
				time3out.Add(t);
				array = args.thirdQueue.Dequeue();
			}
			
			proc = CountIsProc(count);
			
			t2 = DateTime.Now.Ticks;
			Console.WriteLine("Lenta 3 \t{0} \t0 \t{1} \t{2}",
			Thread.CurrentThread.Name,
			t1, t2 - t1);
			
			res.Add(proc);
		}
	}
}
\end{lstlisting}

В листинге \ref{lst:potok} представлен метод создания и запуска потоков, метод MainTread основного класса программы.

\begin{lstlisting}[label=lst:potok,caption=Метод создания и запуска потоков]
public static void MainTread(Queue<intPtr> queue)
{	
	ThreadArgs args = new ThreadArgs(queue);
	
	Thread FThread = new Thread(new ParameterizedThreadStart(Conveyor));
	FThread.Name = "Potok 1";
	
	Thread SThread = new Thread(new ParameterizedThreadStart(Conveyor));
	SThread.Name = "Potok 2";
	
	Thread TThread = new Thread(new ParameterizedThreadStart(Conveyor));
	TThread.Name = "Potok 3";
	
	FThread.Start(args);
	SThread.Start(args);
	TThread.Start(args);

    if (FThread.IsAlive)
	{
		FThread.Join();
	}
	if (SThread.IsAlive)
	{
		SThread.Join();
	}
	if (TThread.IsAlive)
	{
		TThread.Join();
	}
}
\end{lstlisting}

В листинге \ref{lst:c1} представлен код задачи, выполняемой на первой ленте конвейера.

\begin{lstlisting}[label=lst:c1,caption=Подсчет среднего арифметического значения в массиве]
public static int AvgArrayInt(intPtr array)
{
	int avg_int = 0, count = 0;
	for (int i = 0; i < CountOfOperations; i++)
	{
		avg_int = 0;
		count = 0;
		foreach (var el in array)
		{
			avg_int += el;
			count++;
		}
	}
	avg_int /= count;
	return avg_int;
}
\end{lstlisting}

В листинге \ref{lst:c2} представлен код задачи, выполняемой на второй ленте конвейера.

\begin{lstlisting}[label=lst:c2,caption=Подсчет количества элементов в массиве больших среднего арифметического значения]
public static int CountBigAvgInt(intPtr array, int num)
{
	int count = 0;
	for (int i = 0; i < CountOfOperations; i++)
	{
		count = 0;
		foreach (var el in array)
			if (el > num)
				count++;
	}
	return count;
}
\end{lstlisting}


В листинге \ref{lst:c3} представлен код задачи, выполняемой на третьей ленте конвейера.

\begin{lstlisting}[label=lst:c3,caption=Реализация алгоритма определния числа на простоту]
public static int CountIsProc(int num)
{
	int res = 1;
	for (int k = 0; k < CountOfOperations; k++)
	{
		if (num > 1)
			for (int i = 2; i < num; i++)
				if (num % i == 0)
				{
					res = 0;
					break;
				}
		else
			res = 0;
	}
	return res;
}
\end{lstlisting}

\captionsetup{singlelinecheck = false, justification=centering}

\section{Тестирование}

В таблице \ref{tbl:functional_test} приведены тесты для функций, реализующих алгоритмы сортировки. Тесты пройдены успешно.

Сокращение для таблицы: КЗ и ДМ - количество задач и длина массива, М - массив, ОР - ожидаемый результат, ДР - действительный результат.

\begin{table}[h]
	\begin{center}
	\caption{\label{tbl:functional_test} Функциональные тесты}
		\begin{tabular}{|c|c|c|c|}
			\hline
			КЗ и ДМ & М & ОР & ДР \\ 
			\hline
			5, 4  & $[1,4,8,2]$        & 1 & 1\\
			1, 3  & $[9,12,-1]$        & 1 & 1\\
			2, 7  & $[1,3,2,1,0,1,2]$  & 1 & 1\\
			2, 3  & $[4,2,1]$          & 0 & 0\\
			1, 4  & $[0,3,-1,3]$       & 1 & 1\\
			\hline
		\end{tabular}
	\end{center}
\end{table}

В таблице \ref{tbl:ft2} приведены тесты для функции, реализующей алгоритм на ленте 1.

\begin{table}[h]
	\begin{center}
		\caption{\label{tbl:ft2} Функциональные тесты для ленты 1}
		\begin{tabular}{|c|c|c|}
			\hline
			Массив & Ожидаемый результат & Действительный результат \\ 
			\hline
			 $[1,4,8,2]$        & 3 & 3\\
			 $[9,12,-1]$        & 6 & 6\\
			 $[1,3,2,1,0,1,2]$  & 1 & 1\\
			 $[4,2,1]$          & 2 & 2\\
			 $[0,3,-1,3]$       & 1 & 1\\
			\hline
		\end{tabular}
	\end{center}
\end{table}

В таблице \ref{tbl:ft3} приведены тесты для функции, реализующей алгоритм на ленте 2.

\begin{table}[!h]
	\begin{center}
		\caption{\label{tbl:ft3} Функциональные тесты для ленты 2}
		\begin{tabular}{|c|c|c|}
			\hline
			Массив и число & Ожидаемый результат & Действительный результат \\ 
			\hline
			$[1,4,8,2]$ , 3        & 2 & 2\\
			$[9,12,-1]$ , 6        & 2 & 2\\
			$[1,3,2,1,0,1,2]$ , 1  & 3 & 3\\
			$[4,2,1]$ , 2          & 1 & 1\\
			$[0,3,-1,3]$ , 1       & 3 & 3\\
			\hline
		\end{tabular}
	\end{center}
\end{table}

В таблице \ref{tbl:ft4} приведены тесты для функции, реализующей алгоритм на ленте 3.

\begin{table}[h]
	\begin{center}
		\caption{\label{tbl:ft4} Функциональные тесты для ленты 3}
		\begin{tabular}{|c|c|c|}
			\hline
			Число & Ожидаемый результат & Действительный результат \\ 
			\hline
			2  & 1 & 1\\
			2  & 1 & 1\\
			3  & 1 & 1\\
			1  & 0 & 0\\
			3  & 1 & 1\\
			\hline
		\end{tabular}
	\end{center}
\end{table}


\section{Вывод}
В данном разделе были разобраны листинги  показывающие
работу конвейера, а также каждой ленты.
