\chapter{Аналитическая часть}
В данном разделе представлены теоретические сведения о рассматриваемых алгоритмах.

\section{Задача коммивояжера}
Коммивояжёр (фр. commis voyageur) — бродячий торговец. Задача коммивояжёра — одна из самых важных задач транспортной логистики, отрасли, занимающейся планированием транспортных перевозок \cite{kommivoyazh}. В описываемой задаче рассматривается несколько городов и матрица попарных расстояний между ними (матрица смежности, если рассматривать граф). 

Требуется найти такой порядок посещения городов, чтобы суммарное пройденное расстояние было минимальным, каждый город посещался ровно один раз и коммивояжер вернулся в тот город, с которого начал свой маршрут. Другими словами, во взвешенном полном графе требуется найти гамильтонов цикл минимального веса \cite{gamilt}.


\section{Алгоритм полного перебора для решения задачи коммивояжера}
Алгоритм полного перебора для решения задачи коммивояжера предполагает рассмотрение всех возможных путей в графе и выбор наименьшего из них. Смысл перебора состоит в том, что мы перебираем все варианты объезда городов и выбираем оптимальный. Однако, при таком подходе количество возможных маршрутов очень быстро возрастает с ростом $n$ (сложность алгоритма равна $n!$).

Такой подход гарантирует точное решение задачи, однако, уже при небольшом числе городов решение за приемлемое количество времени невозможно.



\section{Муравьиный алгоритм для решения задачи коммивояжера}
Муравьиные алгоритмы представляют собой перспективный метод решения задач коммивояжера, в основе которого лежит моделирование поведения колонии муравьев \cite{Ulianov}.

Каждый муравей определяет для себя маршрут, который необходимо пройти на основе феромона, который он ощущает, во время прохождения, каждый муравей оставляет феромон на своем пути, чтобы остальные муравьи могли по нему ориентироваться. В результате при прохождении каждым муравьем различного маршрута наибольшее число феромона остается на оптимальном пути. 

Самоорганизация колонии является результатом взаимодействия следующих компонентов:
\begin{itemize}
	\item случайность — муравьи имеют случайную природу движения (как будто бросили монетку);
	\item многократность -- колония допускает число муравьев, достигающее от нескольких десятков до миллионов особей;
	\item положительная обратная связь -- во время движения муравей откладывает феромон, позволяющий другим особям определить для себя оптимальный маршрут;
	\item отрицательная обратная связь -- по истечении определенного времени феромон испаряется.
\end{itemize}

Пусть каждый из муравьев обладает следующими характеристиками:
\begin{itemize}
	\item память - запоминает маршрут, который прошел;
	\item зрение - определяет длину ребра;
	\item обоняние - чувствует феромон.
\end{itemize}

Введем некую целевую функцию \eqref{d}
\begin{equation}
	\label{d}
	\eta_{ij} = 1 / D_{ij},
\end{equation}
где $D_{ij}$ — расстояние из текущего пункта $i$ до заданного пункта $j$.


Посчитаем вероятности перехода в заданную точку по формуле \eqref{possibility}:
\begin{equation}
	\label{possibility}
	P_{kij} = \begin{cases}
		\frac{\tau_{ij}^a\eta_{ij}^b}{\sum_{q=1}^m \tau^a_{iq}\eta^b_{iq}}, \textrm{вершина не была посещена ранее муравьем k,} \\
		0, \textrm{иначе}
	\end{cases}
\end{equation}
где 
\begin{itemize}
	\item $a$ -- параметр влияния длины пути;
	\item $b$ -- параметр влияния феромона;
	\item $\tau_{ij}$ -- расстояния от города $i$ до $j$;
	\item $\eta_{ij}$ -- количество феромонов на ребре $ij$.
\end{itemize}


Когда все муравьи завершили движение происходит обновление феромона по формуле \eqref{pheromone1}:
\begin{equation}
	\label{pheromone1}
		\tau_{ij}(t+1) = (1-p)\tau_{ij}(t) + \Delta \tau_{ij}.
\end{equation}
При этом
\begin{equation}
\label{pheromone11}
 \Delta \tau_{ij} = \sum_{k=1}^N \tau^k_{ij},
\end{equation}
где
\begin{equation}
	\label{pheromone2}
		 \Delta\tau^k_{ij} = \begin{cases}
		Q/L_{k}, \textrm{ребро посещено k-ым муравьем,} \\
		0, \textrm{иначе}
	\end{cases}
\end{equation}
где
\begin{itemize}
	\item $L_{k}$ — длина пути k-ого муравья;
	\item $N$ — количество муравьев;
	\item $Q$ — настраивает концентрацию нанесения/испарения феромона.
\end{itemize}

\subsubsection{Описание поведения муравьев при выборе пути.}

\begin{enumerate}
	\item Муравьи имеют собственную «память».
Поскольку каждый город может быть посещён только один раз, то у каждого муравья есть список уже посещенных городов - список запретов.
Обозначим через $J_{ik}$ список городов, которые необходимо посетить муравью $k$, находящемуся в городе $i$.

\item Муравьи обладают «зрением» - желание посетить город $j$, если муравей находится в городе $i$.
Будем считать, что видимость обратно пропорциональна расстоянию между городами.

\item Муравьи обладают «обонянием» - они могут улавливать след феромона, подтверждающий желание посетить город $j$ из города $i$ на основании опыта других муравьёв.
Количество феромона на ребре $(i,j)$ в момент времени $t$ обозначим через  $\tau_{i,j} (t)$.

\item Пройдя ребро $(i,j)$ , муравей откладывает на нём некоторое количество феромона, которое должно быть связано с оптимальностью сделанного выбора.
Пусть $T _{k} (t)$ есть маршрут, пройденный муравьем $k$ к моменту времени $t$ , $L _{k} (t)$ - длина этого маршрута, а $Q$ - параметр,
имеющий значение порядка длины оптимального пути. Тогда откладываемое количество феромона может быть задано формулой \eqref{pheromone2}.

\end{enumerate}

\section{Вывод}
В данном разделе были рассмотрены основополагающие материалы, которые в дальнейшем потребуются для реализации алгоритмов. Были рассмотрены задача коммивояжера, муравьиный алгоритм и алгоритм полного перебора. 

