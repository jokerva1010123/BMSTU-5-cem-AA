\chapter*{Введение}
\addcontentsline{toc}{chapter}{Введение}
Муравьиный алгоритм – алгоритм оптимизации подражанием муравьиной колонии. Один из эффективных полиномиальных алгоритмов для нахождения приближённых решений задачи коммивояжёра, а также решения аналогичных задач поиска маршрутов на графах.

Муравьиные алгоритмы серьезно исследуются европейскими учеными с середины 1990-х годов. Муравьиный алгоритм предоставляет хорошие результаты оптимизации для многих сложных комбинаторных задач, таких как:
\begin{itemize}
	\item задачи коммивояжера;
	\item задачи оптимизации маршрутов грузовиков;
	\item задачи календарного планирования;
	\item задачи оптимизации сетевых графиков.
\end{itemize}

Целью данной лабораторной работы является реализация муравьиного алгоритма и приобретение навыков параметризации методов на примере реализованного алгоритма, примененного к задаче коммивояжера.

Для достижения данной цели необходимо решить следующие задачи.

\begin{enumerate}
	\item Изучить алгоритм полного перебора для решения задачи коммивояжера.
	\item Реализовать алгоритм полного перебора для решения задачи коммивояжера.
	\item Изучить муравьиный алгоритм для решения задачи коммивояжера.
	\item Реализовать муравьиный алгоритм для решения задачи коммивояжера.
	\item Провести параметризацию муравьиного алгоритма на трех классах
	данных.
	\item Провести сравнительный анализ скорости работы реализованных
	алгоритмов.
	\item Описание и обоснование полученных результатов в отчете о выполненной лабораторной работе, выполненного как расчётно-пояснительная записка к работе.
\end{enumerate}
