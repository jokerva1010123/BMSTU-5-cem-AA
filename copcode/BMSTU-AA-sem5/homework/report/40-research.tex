\chapter{Исследовательская часть}
В данном разделе будет произведено сравнение вышеизложенных алгоритмов.

\section{Технические характеристики}

Технические характеристики устройства, на котором выполнялось тестирование, следующие.

\begin{itemize}
	\item Операционная система: Windows 10 \cite{oswind} x86\_64.
	\item Память: 8 GiB.
	\item Процессор: 11th Gen Intel® Core™ i5-1135G7 @ 2.40GHz \cite{intel}.
	\item 4 физических ядра и 8 логических ядра.
\end{itemize}

Тестирование проводилось на ноутбуке, включенном в сеть электропитания. Во время тестирования ноутбук был нагружен только встроенными приложениями окружения, а также непосредственно системой тестирования.

\section{Время выполнения алгоритмов}

Результаты замеров приведены в таблицах \ref{tbl:allpotok} и \ref{tbl:only4}.
На рисунках \ref{img:f1}, \ref{img:f2} и \ref{img:f3} приведены графики зависимостей времени работы реализации алгоритма ранговой сортировки от размеров массивов для различного количества потоков. 





\section*{Вывод}

В данном разделе было произведено сравнение алгоритма ранговой сортировки при простой реализации и многопоточной. Результат показал, что выгоднее всего использовать все ядра процессора.