\chapter{Технологическая часть}
В данном разделе будут приведены требования к программному обеспечению, средства реализации и листинги кода.

\section{Требования к ПО}

К программе предъявляется ряд требований:
\begin{itemize}
	\item на вход подаётся ключ, значение которого необходимо найти;
	\item на выходе -- три результата поиска в словаре (для каждого алгоритма выводится отдельный результат).
\end{itemize}

\section{Средства реализации}

В качестве языка программирования для реализации данной лабораторной работы был выбран ЯП Python \cite{pythonlang}. 

Данный язык достаточно удобен и гибок в использовании. 

Время работы алгоритмов было замерено с помощью функции
 process\_time() из библиотеки time \cite{pythonlangtime}

В качестве среды разработки выбор сделан в сторону Visual Studio Code \cite{wind}. Данная среда подходит как для Windows, так и для Linux.

\section{Сведения о модулях программы}
Данная программа разбита на модули:
\begin{itemize}
	\item main.py -- файл, содержащий меню программы и основную функцию;
	\item info\_for\_prog.py -- файл, содержащий класс, который описывает информацию о данных;
	\item dictionary.py -- файл, содержащий класс ''Словарь''.
\end{itemize}

\section{Реализация алгоритмов}
В листинге \ref{lst:bfs} представлена реализация алгоритма поиска в словаре полным перебором.

\begin{center}
	\captionsetup{justification=raggedright,singlelinecheck=off}
	
\begin{lstlisting}[label=lst:bfs,caption=Реализация алгоритма поиска полным перебором]
def BruteForceSearch(self, key):
	k = 0
	kk = list(self.data.keys())
	for elem in self.data:
		k += 1
		if key == elem:
			// Writing to the log file 
			self.f.write(f"{kk.index(key)},{key},{k}\n") 
			return self.data[elem]
	return -1
\end{lstlisting}
\end{center}

В листинге \ref{lst:bs} представлена реализация алгоритма бинарного поиска в словаре.

\begin{center}
	\captionsetup{justification=raggedright,singlelinecheck=off}
\begin{lstlisting}[label=lst:bs,caption=Реализация алгоритма бинарного поиска]
def BinarySearch(self, key, list_keys):
	l, r = 0, len(list_keys) - 1
	k = 0
	kk = list(self.data.keys())
	while l <= r:
		middle = (r + l) 
		elem = list_keys[middle]
		k += 1
		if elem == key:
			// Writing to the log file 
			self.f1.write(f"{kk.index(key)},{key},{k}\n")
			return self.data[elem]
		elif elem < key:
			l = middle + 1
		else:
			r = middle - 1
	return -1
\end{lstlisting}
\end{center}

В листинге \ref{lst:ss} представлена реализация алгоритма поиска в словаре с помощью разбиения на сегменты.

\begin{center}
	\captionsetup{justification=raggedright,singlelinecheck=off}
\begin{lstlisting}[label=lst:ss,caption=Реализация алгоритма поиска в словаре с помощью разбиения на сегменты]
def SegmentalSearch(self, key, new_dict):
	c = 0
	kk = list(self.data.keys())
	for k in new_dict:
		c += 1
		if key[0] == k:
			c += 1
			for elem in new_dict[k]:
				if elem == key:
					// Writing to the log file 
					self.f2.write(f"{kk.index(key)},{key},{c}\n")
					return new_dict[k][elem]
			return -1
	return -1
\end{lstlisting}
\end{center}


\section{Функциональное тестирование}

В данном разделе будет приведена таблица с тестами (таблица \ref{table:ref1}).
\begin{center}
	\captionsetup{justification=raggedleft,singlelinecheck=off}
\begin{table}[ht]
	\centering
	\caption{Таблица тестов}
	\label{table:ref1}
	\begin{tabular}{ |c|c|c|}
		\hline
		Входные данные    & Пояснение   	  & Результат    \\ \hline
		\hline
		Moscow			  & Средний элемент   & Ответ верный \\ \hline
		Tokyo 			  & Первый элемент    & Ответ верный \\ \hline
		Belgium 		  & Последний элемент & Ответ верный \\ \hline
		123 & Несуществующий элемент & Ответ верный (-1) \\ \hline
		Mosccow & Несуществующий элемент & Ответ верный (-1) \\ \hline
	\end{tabular}
\end{table}
\end{center}
Все тесты пройдены успешно для всех реализаций алгоритмов.


\section{Вывод}
В данном разделе были представлены листинги рассматриваемых алгоритмов поиска в словаре, приведена информация о средствах реализации, сведения о модулях программы и было проведено функциональное тестирование.