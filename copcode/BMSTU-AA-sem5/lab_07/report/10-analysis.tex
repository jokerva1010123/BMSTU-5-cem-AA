\chapter{Аналитическая часть}
В данном разделе описаны определение словаря как структуры данных, а также алгоритмы поиска по словарю.

\section{Словарь как структура данных}

Словарь (или ''\textit{ассоциативный массив}'') \cite{dict} - абстрактный тип данных (интерфейс к хранилищу данных), позволяющий хранить пары вида «(ключ, значение)» и поддерживающий операции добавления пары, а также поиска и удаления пары по ключу:
\begin{itemize}
	\item \texttt{INSERT(k, v)};
	\item \texttt{FIND(k)};
	\item \texttt{REMOVE(k)}.
\end{itemize}

В паре \texttt{(k, v)}: \texttt{v} называется значением, ассоциированным с ключом \texttt{k}. Где \texttt{k} — это ключ, a \texttt{v} — значение. Семантика и названия вышеупомянутых операций в разных реализациях ассоциативного массива могут отличаться.

Операция \texttt{ПОИСК(k)} возвращает значение, ассоциированное с заданным ключом, или некоторый специальный объект \texttt{НЕ\_НАЙДЕНО}, означающий, что значения, ассоциированного с заданным ключом, нет. Две другие операции ничего не возвращают (за исключением, возможно, информации о том, успешно ли была выполнена данная операция).

Ассоциативный массив с точки зрения интерфейса удобно рассматривать как обычный массив, в котором в качестве индексов можно использовать не только целые числа, но и значения других типов — например, строки.

В данной лабораторной работе в качестве ключа будет использоваться строка: название города, а в качестве значения -- строка: страна, в которой расположен город.

\section{Алгоритм полного перебора}
Алгоритмом полного перебора \cite{AI} называют метод решения задачи, при котором по очереди рассматриваются все возможные варианты. В случае реализации алгоритма в рамках данной работы будут последовательно перебираться ключи словаря до тех пор, пока не будет найден нужный. 

Трудоёмкость алгоритма зависит от того, присутствует ли искомый ключ в словаре, и, если присутствует -- насколько он далеко от начала массива ключей.

Пусть на старте алгоритм затрагивает $k_{0}$ операций, а при сравнении $k_{1}$ операций. 

Пусть алгоритм нашёл элемент на первом сравнении (лучший случай), тогда будет затрачено $k_0 + k_1$ операций, на втором - $k_0 + 2 \cdot k_1$, на последнем (худший случай) - $k_0 + N \cdot k_1$. Если ключа нет в массиве ключей, то мы сможем понять это, только перебрав все ключи, таким образом трудоёмкость такого случая равно трудоёмкости случая с ключом на последней позиции. Трудоёмкость в среднем может быть рассчитана как математическое ожидание по формуле (\ref{for:brute}), где $\Omega$ -- множество всех возможных случаев.

\begin{equation}
\label{for:brute}
\begin{aligned}
\sum\limits_{i \in \Omega} p_i \cdot f_i = & (k_0 + k_1) \cdot \frac{1}{N + 1} + (k_0 + 2 \cdot k_1) \cdot \frac{1}{N+1} +\\& + (k_0 + 3 \cdot k_1) \cdot \frac{1}{N + 1} + (k_0 + Nk_1)\frac{1}{N + 1} + (k_0 + N \cdot k_1) \cdot \frac{1}{N + 1} =\\& = k_0\frac{N+1}{N+1}+k_1+\frac{1 + 2 + \cdots + N + N}{N + 1} = \\& = k_0 + k_1 \cdot \left(\frac{N}{N + 1} + \frac{N}{2}\right) = k_0 + k_1 \cdot \left(1 + \frac{N}{2} - \frac{1}{N + 1}\right)
\end{aligned}
\end{equation}

\section{Алгоритм поиска в упорядоченном словаре двоичным поиском}
Бинарный поиск базируется на том, что словарь изначально отсортирован, что позволяет сравнивать ключ с средним элементом, и, если, он меньше, то продолжать искать в левой части, таким же методом, иначе в правой.

Таким образом, при двоичном поиске \cite{binary} обход можно представить деревом, поэтому трудоёмкость в худшем случае составит $k_{0} + \log_2 N$ (в худшем случае нужно спуститься по двоичному дереву от корня до листа).

Лучшим случаем будет случай, если искомый ключ окажется средним элементом, тогда трудоемкость будет равна $k_{0} + \log_2 1$.

Скорость роста функции $\log_2 N$ меньше, чем скорость роста линейной функции, полученной для полного перебора.

\section{Поиск с сегментированием словаря}
Алгоритм на вход получает словарь. Далее словарь разбивается на сегменты так, что все элементы с некоторым общим признаком попадают в один сегмент (для букв это может быть первая буква, для чисел -- остаток от деления).

В данной работе словать содержит ключи в виде строк, поэтому будет использоваться подход с формированием сегментов, объединяющих все ключи, начинающиеся на один и тот же символ (для слов -- первую юукву слова).

Сегменты упорядочиваются по значению частотной характеристики так, чтобы к элементам с наибольшей частотной характеристикой (частотой обращения) был самый быстрый доступ.
Например, такой характеристикой может быть размер сегмента.

Вероятность обращения к определенному сегменту равна сумме вероятностей обращений к его ключам, то есть $P_i = \sum_{j}p_j = N \cdot p$, где $P_i$ -- вероятность обращения к $i$-ому сегменту, $p_j$ -- вероятность обращения к $j$-ому элементу, который принадлежит $i$-ому сегменту. 

Если обращения ко всем ключам равновероятны, то можно заменить сумму на произведение, где $N$ - количество элементов в $i$-ом сегменте, а $p$ - вероятность обращения к произвольному ключу.

Далее ключи в каждом сегменте сортируются для того, чтобы внутри каждого сегмента можно было использовать бинарный поиск, который обеспечит эффективный поиск со сложностью $O(\log_2 m)$ (где $m$ - количество ключей в сегменте) внутри сегмента.

Таким образом, сначала выбирается нужный сегмент, а затем в нем проводится бинарный поиск нужного элемента.

Трудоёмкость в среднем при множестве всех возможных случаев $\Omega$ может быть рассчитана по формуле (\ref{for:anal}). 
\begin{equation}
\label{for:anal}
\sum_{i \in \Omega}{\left(f_{\text{выбор сегмента i-ого элемента}} + f_{\text{бинарный поиск i-ого элемента}}\right)} \cdot p_i
\end{equation}

Худший случай -- это случай, когда выбор сегмента -- $N \cdot p_i$, при этом $N$ равно числу сегментов. А поиск элемента (ключа) в данном сегменте -- $\log_2 N$, где $N$ -- число элементов в сегменте.

Лучший случай -- если искомый сегмент оказывается первым, и в этом сегменте исходный ключ -- серединный элемент, так как бинарный поиск сначала укажет в середину. 

\section{Вывод}
В данном разделе был рассмотрен абстрактный тип данных словарь и возможные реализации алгоритмов поиска в нём.

Программа будет получать на вход ключ, по кторому необходимо проивести поиск в словаре. При неверном вводе ключа (пустая строка) будет выведено сообщение об ошибке.

Реализуемое ПО дает возможность получить лог программы для трех алгоритмов (в лог файле содержится количество сравнения при поиске каждого ключа). Также имеется возможность построения графиков зависимотси времени поиска каждого ключа.
