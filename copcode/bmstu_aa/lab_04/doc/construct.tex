
\chapter{Конструкторская часть}
В данном разделе будут рассмотрены схемы алгоритмов нахождения определителя с распараллеливанием и без него.

\section{Описание используемых типов данных}

При реализации алгоритмов будут использованы следующие типы данных:
\begin{itemize}[label=---]
	\item количество строк — целое число типа \textit{int};
	\item количество столбцов — целое число типа \textit{int};
	\item матрица — двумерный массив значений типа \textit{int}.
\end{itemize}

\section{Требования к программному обеспечению}
Выдвинут ряд требований к программе:
\begin{itemize}[label=---]
	\item на вход подается матрица;
    \item выполняется проверка на предмет того, является ли матрица квадратной;
    \item если матрица квадратная, то на выходе необходимо получить определитель матрицы и время (в с), потраченное на его вычисление, в противном случае сообщить о том, что определитель не может быть вычислен.
\end{itemize}

\section{Разработка алгоритмов}
На рисунках \ref{img:det}--\ref{img:det_parallel} представлены схемы алгоритмов вычисления определителя матрицы без распараллеливания и с ним.
\imgScale{0.5}{det}{Последовательный алгоритм поиска определителя матрицы}
\imgScale{0.5}{det_parallel}{Алгоритм поиска определителя матрицы с распараллеливанием}
\clearpage

\section*{Вывод}

В данном разделе были построены схемы алгоритма нахождения определителя матриц через миноры с распараллеливанием и без.
