\chapter{Технологическая часть}

В данном разделе будут выбраны язык программирования, среда разработки, а также представлены листинги кода реализации алгоритмов поиска определителя.

\section{Средства реализации}
Для реализации алгоритмов был выбран язык программирования C++, в котором существует библиотека time.h~\cite{time.h} для замера процессорного времени. С помощью полученных результатов времени будут построены графики для наглядного отображения временной эффективности алгоритмов. Будет использована среда разработки CLion.

\section{Сведения о модулях программы}
Программа состоит из одного модуля --- $main.c$ — файл, содержащий код алгоритмов, создания и заполнения матрицы, вывода замеренного процессорного времени и весь второстепенный код.

В листингах \ref{lst:del}, \ref{lst:det} и \ref{lst:detparallel} представлены реализации функции удаления матрицы \textit{i}\textit{-ой} строки и \textit{j}\textit{-ого} столбца и алгоритма поиска определителя с получением замеров процессорного времени, затраченного на их выполнение.

\begin{center}
    \captionsetup{justification=raggedright,singlelinecheck=off}
    \begin{lstlisting}[label=lst:del,caption=Функция удаления \textit{i}\textit{-ой} строки и \textit{j}\textit{-ого} столбца]
void getMatrixWithoutRowAndCol(int **matrix, int size, int row, int col, int **newMatrix)
{
    int offsetRow = 0;
    int offsetCol = 0;
    for(int i = 0; i < size-1; i++)
	{
        if(i == row)
		{
            offsetRow = 1;
        }
        offsetCol = 0;
        for(int j = 0; j < size-1; j++)
		{
            if(j == col)
			{
                offsetCol = 1;
            }
            newMatrix[i][j] = matrix[i + offsetRow][j + offsetCol];
        }
    }
}
\end{lstlisting}
\end{center}

\begin{center}
    \captionsetup{justification=raggedright,singlelinecheck=off}
    \begin{lstlisting}[label=lst:det,caption=Алгоритм нахождения определителя матрицы через миноры]
int matrixDet(int **matrix, int size)
{
    time_t startt = 0, endt = 0;
    int det = 0;
    int degree = 1;
    startt = clock();
    
    if(size == 1)
	{
        return matrix[0][0];
    }
    else if(size == 2)
	{
        return matrix[0][0]*matrix[1][1] - matrix[0][1]*matrix[1][0];
    }
    else
	{
        int **newMatrix = new int*[size-1];
        for(int i = 0; i < size-1; i++)
		{
            newMatrix[i] = new int[size-1];
        }
        for(int j = 0; j < size; j++)
		{
            getMatrixWithoutRowAndCol(matrix, size, 0, j, newMatrix);
            det = det + (degree * matrix[0][j] * matrixDet(newMatrix, size-1));
            degree = -degree;
        }
        for(int i = 0; i < size-1; i++)
		{
            delete [] newMatrix[i];
        }
        delete [] newMatrix;
    }
    
    endt = clock();
    return endt - startt;
}
\end{lstlisting}
\end{center}

\begin{center}
    \captionsetup{justification=raggedright,singlelinecheck=off}
    \begin{lstlisting}[label=lst:detparallel,caption= Функция вспомогательного потока]
void for_parallel(int size, int **matrix, int *det, int *degree)
{
    int **newMatrix = new int*[size-1];
    for(int i = 0; i < size-1; i++)
	{
	    newMatrix[i] = new int[size-1];
    }
    for(int j = 0; j < size; j++)
	{
        getMatrixWithoutRowAndCol(matrix, size, 0, j, newMatrix);
        *det = *det + (*degree * matrix[0][j] * matrixDet(newMatrix, size-1));
        *degree = -(*degree);
    }
    for(int i = 0; i < size-1; i++)
	{
        delete [] newMatrix[i];
    }
    delete [] newMatrix;
}
        
int matrixDetparallel(int **matrix, int size)
{
    clock_t startt, endt;
    int det = 0;
    int degree = 1;
    
    startt = clock();
    if(size == 1)
	{
        return matrix[0][0];
    }
    else if(size == 2)
	{
        return matrix[0][0]*matrix[1][1] - matrix[0][1]*matrix[1][0];
    }
    else
	{
	    std::thread mas_potoks[size];
	    for (int k = 0; k < size; k++)
	    {
	        mas_potoks[k] = std::thread(for_parallel, size, matrix, &det, &degree);
	    }
	    
	    for (int k = 0; k < size; k++)
	        mas_potoks[k].join();
    }
    
    endt = clock();
    return endt - startt;
}
\end{lstlisting}
\end{center}

\section{Классы эквивалентности при тестировании}

Для тестирования выделены следующие классы эквивалентности.
\begin{enumerate}
	\item Пустая матрица.
	\item Матрица не квадратная.
	\item Квадратная матрица.
\end{enumerate}

\section{Функциональные тесты}

В таблице \ref{tbl:functional_test} приведены функциональные тесты для реализаций алгоритма поиска определителя матрицы.  

\begin{table}[h]
	\begin{center}
		\begin{threeparttable}
		\captionsetup{justification=raggedright,singlelinecheck=off}
		\caption{\label{tbl:functional_test} Функциональные тесты}
		\begin{tabular}{|c|c|c|c|}
			\hline
			Входная матрица & Ожидаемый результат & Результат \\ 
			\hline
			$"пустая матрица"$ & матрица пустая & матрица пустая\\
			$[1, 2] [3, 4]$ & $-2$  & $-2$\\
			$[1, 2] [3, 4] [5, 6]$ & матрица не квадратная & матрица не квадратная\\
			\hline
		\end{tabular}
    \end{threeparttable}
	\end{center}
\end{table}

Все тесты пройдены успешно реализациями последовательного алгоритма поиска определителя матрицы и алгоритма с распараллеливанием.

\section*{Вывод}

В технологической части лабораторной работы были выбраны язык программирования и среда разработки, приведены листинги алгоритма, проведено функциональное тестирование, а также приведены технические характеристики устройства, на котором проводятся замеры времени работы алгоритма.