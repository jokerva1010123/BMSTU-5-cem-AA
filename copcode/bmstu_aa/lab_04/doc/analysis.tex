\chapter{Аналитическая часть}
В этом и последующих разделах будет рассмотрен алгоритм нахождения определителя матриц через миноры.

\textbf{Матрица} \cite{matrix} — это набор чисел, записываемый в виде прямоугольной таблицы. Строки и столбцы матрицы можно считать векторами. Матрицы, у которых число строк равно числу столбцов, называют квадратными.

Обычно для обозначения элемента матрицы \textit{A}, стоящего в \textit{i-той} строке и в \textit{j-ом} столбце используется следующая запись: \textit{A}[i][j].

Самыми распространенными операциями над матрицами являются сложение, вычитание, транспонирование, умножение и нахождение определителя \cite{matrix}.

Способ вычисления определителя матрицы через миноры реализует формулу, описанную далее. 

\textbf{Минором} \textit{Mij} элемента \textit{aij} матрицы \textit{A} \textit{n}\textit{-го} порядка называется определитель $(n-1)$\textit{-го} порядка, полученного из исходного определителя вычеркиванием \textit{i}\textit{-ой} строки и \textit{j}\textit{-го} столбца:
\begin{equation}
	M_{ij} = \begin{pmatrix}
		a_{11} & \ldots & a_{1,j-1} & a_{1,j+1} & \ldots & a_{1,n}\\
		\ldots & \ldots & \ldots & \ldots & \ldots & \ldots\\
		a_{i-1,1} & \ldots & a_{i-1,j-1} & a_{i-1,j+1} & \ldots & a_{i-1,n}\\
		a_{i+1,1} & \ldots & a_{i+1,j-1} & a_{i+1,j+1} & \ldots & a_{i+1,n}\\
		\ldots & \ldots & \ldots & \ldots & \ldots & \ldots\\
		a_{n1} & \ldots & a_{n,j-1} & a_{n,j+1} & \ldots & a_{n,n}
	\end{pmatrix}.
\end{equation}

\textbf{Замечание:} определитель можно считать только для квадратных матриц, то есть тех матриц, у которых количество строк равно количеству столбцов.

\textbf{Алгебраическим дополнением \textit{Aij}} элемента \textit{aij} матрицы \textit{A} \textit{n}\textit{-го} порядка называется число, равное произведению минора \textit{Mij} на $(−1)^{i+j}$: 
\begin{equation}
 Aij=(−1)^{i+j}\cdot M_{ij}.
\end{equation}

Определители $n$\textit{-го} порядка вычисляются с помощью метода понижения порядка --- по формуле \textit{detA}=$\sum\limits_{j=1}^n$\textit{aij}\textit{Aij} (\textit{i} фиксировано) --- разложение по $i$\textit{-ой} строке.

Метод приведения к треугольному виду заключается в преобразовании определителя, когда все элементы, лежащие по одну сторону главной диагонали рассматриваемой матрицы, становятся равными нулю. В этом случае определитель равен произведению элементов главной диагонали \cite{dig}.

\section*{Вывод}
В аналитической части был описан последовательный алгоритм поиска определителя матриц через миноры. Далее необходимо произвести оценку его эффективности и проверить её экспериментально.
