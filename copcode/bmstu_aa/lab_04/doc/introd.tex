\chapter*{Введение}
\addcontentsline{toc}{chapter}{Введение}
Зачастую в сфере информационных технологий используют параллельную обработку данных, которая позволяет уменьшить время работы программы. Одним из примеров такой обработки является конвейерная обработка.
Суть та же, что и при работе реальных конвейрных лент - материал (данное) поступает на обработку, после окончания обработки материал передается на место следующего обработчина, при этом предыдыдущий обработчик не ждет полного цикла обработки материала, а получает новый материал и работает с ним.


\textbf{Целью данной работы} является изучение принципов конвейрной обработки данных. 
Для достижения поставленной цели необходимо выполнить следующие задачи:
\begin{itemize}[label=---]
	\item изучить основы конвейрной обработки данных;
    \item описать алгоритмы обработки матрицы, которые будут использоваться в текущей лабораторной работе;
    \item привести схемы конвейрной и линейной обработок;
    \item описать используемые типы и структуры данных;
    \item описать структуру разрабатываемого программного обеспечения;
    \item реализовать разработанный алгоритм;
    \item провести функциональное тестирование разработанного алгоритма;
    \item провести сравнительный анализ по времени для реализованного алгоритма;
    \item подготовить отчет по лабораторной работе.
\end{itemize}

\textbf{Многопоточность} \cite{mnog} --- способность центрального процессора (CPU) или одного ядра в многоядерном процессоре выполнять несколько процессов или потоков, соответствующим образом поддерживаемых операционной системой. Этот подход отличается от многопроцессорности, так как многопоточность процессов и потоков совместно использует ресурсы одного или нескольких ядер: вычислительных блоков, кэш-памяти центрального процессора (ЦП) или буфера перевода с преобразованием (TLB).

В тех случаях, когда многопроцессорные системы включают в себя несколько полных блоков обработки, многопоточность направлена на максимизацию использования ресурсов одного ядра, используя параллелизм на уровне потоков, а также на уровне инструкций. Поскольку эти два метода являются взаимодополняющими, их иногда объединяют в системах с несколькими многопоточными ЦП и в ЦП с несколькими многопоточными ядрами.

Ниже описаны достоинства и недостатки многопоточности.

\textbf{Преимущества}:
\begin{itemize}[label=---]
	\item использование общего адресного пространства программы для набора потоков;
    \item меньшие затраты на создание потока в сравнении с процессами;
    \item повышение производительности процесса за счёт распараллеливания процессорных вычислений;
    \item если теряется кэш, выделенный одному потоку, другие потоки могут продолжать использовать неиспользованные вычислительные ресурсы.
\end{itemize}

\textbf{Недостатки}:
\begin{itemize}[label=---]
	\item один поток может использовать адресное пространство другого потока при совместном использовании аппаратных ресурсов;
    \item с программной точки зрения, аппаратная поддержка многопоточности более трудоемка для программного обеспечения;
    \item проблема планирования потоков.
\end{itemize}

\textbf{Целью данной работы} является получить навык организации параллельных вычислений на примере алгоритма нахождения определителя матрицы через миноры.

Для достижения поставленной цели необходимо выполнить следующие задачи:
\begin{itemize}[label=---]
	\item описать основы многопоточного программирования;
    \item описать алгоритм нахождения определителя матрицы через миноры;
    \item разработать параллельный алгоритм нахождения определителя матрицы через миноры;
    \item реализовать оба алгоритма нахождения определителя матрицы через миноры;
    \item провести сравнительный анализ по времени на одинаковых матрицах и различном количестве потоков;
    \item провести сравнительный анализ затрат реализаций последовательного и параллельного алгоритмов по времени при разных размерах матриц с использованием многопоточности и без;
	\item описать и обосновать полученные результаты.
\end{itemize}