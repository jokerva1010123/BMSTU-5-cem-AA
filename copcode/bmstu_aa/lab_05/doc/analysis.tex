\chapter{Аналитическая часть}
В этом разделе будет представлено описание сути конвейрной обработки данных и используемых алгоритмов.

\section{Описание конвейерной обработки данных}

Конвейер~\cite{bib1} — способ организации вычислений, используемый в современных процессорах и контроллерах с целью повышения их производительности (увеличения числа инструкций, выполняемых в единицу времени — эксплуатация параллелизма на уровне инструкций), технология, используемая при разработке компьютеров и других цифровых электронных устройств.

Конвейерную обработку можно использовать для совмещения этапов выполнения разных команд. Производительность при этом возрастает благодаря тому, что одновременно на различных ступенях конвейера выполняются несколько команд. Такая обработка данных в общем случае основана на разделении подлежащей исполнению функции на более мелкие части, называемые лентами, и выделении для каждой из них отдельного блока аппаратуры. Так, обработку любой машинной команды можно разделить на несколько этапов (лент), организовав передачу данных от одного этапа к следующему.

\section{Описание алгоритмов}

В данной лабораторной работе на основе конвейрной обработки данных будет обрабатываться матрица. В качестве алгоритмов на каждую из трех лент были выбраны следующие действия.

\begin{itemize}[label=---]
    \item Найти наименьший элемент в матрице $min\_elem$.
    \item Записать в каждую ячейку матрицы остаток от деления текущего элемента на $min\_elem$.
    \item Найти сумму элементов полученной матрицы.
\end{itemize}

\section*{Вывод}

В этом разделе было рассмотрено понятие конвейрной обработки данных, а также выбраны алгоритмы для обработки матрицы на каждой из трех лент конвейера.

 На вход программе будет поступать кол-во матриц и её размер (кол-во строк и столбцов). При попытке задать некорректные данные, будет выдано сообщение об ошибке. Реализуемая программа будет давать возможность выбрать метод обработки данных (конвейрный или линейный) и вывести для него результат вычисления, а также возможность произвести сравнение алгоритмов по затраченному времени.