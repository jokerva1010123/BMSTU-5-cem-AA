\chapter{Исследовательская часть}

\section{Технические характеристики устройства}

Ниже представлены характеристики компьютера, на котором проводилось тестирование программы:
\begin{itemize}[label=---]
    \item операционная система Windows 10 Домашняя 21H2;
    \item оперативная память 16 Гб;
    \item процессор Intel(R) Core(TM) i7-10870H CPU @ 2.20 ГГц.
\end{itemize}

Во время тестирования ноутбук был подключен к сети электропитания. Из программного обеспечения были запущены только среда разработки \textit{Clion} и браузер \textit{Chrome}.

Процессор был загружен на 19\%, оперативная память -- на 50\%.

\clearpage

% \section{Демонстрация работы программы}

% \img{220mm}{example_parallel}{Пример работы программы (конвейерная реализация)}

% \clearpage

% \img{220mm}{example_linear}{Пример работы программы (линейная реализация)}

% \clearpage

\section{Время выполнения алгоритмов}

Результаты замеров времени работы алгоритмов обработки матриц для конвейерной и ленейной реализаций представлены на рисунках \ref{img:graph_diff_quantities} -- \ref{img:graph_diff_sizes}. Замеры времени проводились в секундах и усреднялись для каждого набора одинаковых экспериментов.

% \imgScale{1}{time}{Замеры времени работы алгоритмов для конвейерной и ленейной реализаций}

\imgScale{0.5}{graph_diff_quantities}{Зависимость времени работы алгоритмов от кол-ва матриц (размеры матриц 100х100)}

\imgScale{0.5}{graph_diff_sizes}{Зависимость времени работы алгоритмов от размера матриц (кол-во матриц = 100)}

\clearpage

\section{Вывод}

В этом разделе были указаны технические характеристики машины, на которой происходило сравнение времени работы алгоритмов обработки матриц для конвейерной и ленейной реализаций.

В результате замеров времени было установлено, что конвейерная реализация обработки лучше линейной
при большом кол-ве матриц (в 2.5 раза при 400 матрицах, в 2.6 раза при 800 и в 2.7 при 1600). Так же конвейерная обработка показала себя лучше при увеличении размеров обрабатываемых матриц (в 2.8 раза при размере матриц 160х160, в 2.9 раза при размере 320х320 и в 2.9 раза при матрицах 640х640). Значит при большом кол-ве обрабатываемых матриц, а так же при матрицах большого размера стоит использовать конвейерную реализацию обработки, а не линейную.


