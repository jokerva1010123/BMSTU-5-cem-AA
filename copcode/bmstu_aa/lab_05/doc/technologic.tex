\chapter{Технологическая часть}

В данном разделе будут приведены требования к программному обеспечению, средства реализации, листинги кода, а также функциональные тесты.

\section{Требования к программному обеспечению}

В качестве входных данных задается количество строк и столбцов матрицы $matr$, которое должно быть больше 0, а все элементы матрицы имеют тип $int$. Количество матриц больше 0.
Выходные данные --- табличка с номерами матриц, номерами этапов (лент) её обработки, временем начала обработки текущей матрицы на текущей ленте, временем окончания обработки текущей матрицы на текущей ленте.

\section{Выбор языка программирования}

В данной работе для реализации был выбран язык программирования \textit{C++}~\cite{bib2}, так как он предоставляет весь необходимый функционал для выполнения работы. Для замера времени работы использовалась функция \textit{std::chrono::system\_clock::now()}~\cite{bib3}.

\clearpage

\section{Реализация алгоритмов}

В листингах \ref{lst:parallel_processing} - \ref{lst:stage_3} представлены функции для конвейерного и ленейного алгоритмов обработки матриц.

\begin{center}
\captionsetup{justification=raggedright,singlelinecheck=off}
\begin{lstlisting}[label=lst:parallel_processing,caption=Функция алгоритма конвейерной обработки матрицы]
void parallel_processing(int n_matr, int m_matr, int cnt_matr, bool matr_is_print, bool compare_time)
{
	std::queue<matrix_t> q1;
	std::queue<matrix_t> q2;
	std::queue<matrix_t> q3;

	std::mutex m;

	for (int i = 0; i < cnt_matr; i++)
	{
		matrix_t matr = generate_matrix(n_matr, m_matr);
		q1.push(matr);

		if (matr_is_print && i == cnt_matr - 1)
		{
			m.lock();
			printf("Первоначальная матрица:\n");
			print_matrix(matr);
			m.unlock();
		}
	}
	
	bool q1_is_empty = false;
	bool q2_is_empty = false;

	std::vector<matrix_state_t> matrix_state_arr;
	init_matrix_state_arr(matrix_state_arr, cnt_matr);

	std::chrono::time_point<std::chrono::system_clock> time_begin = 
	std::chrono::system_clock::now();

	std::vector<res_time_t> time_result_arr;
	init_time_result_arr(time_result_arr, time_begin, cnt_matr, THREADS_COUNT);

	std::thread threads[THREADS_COUNT];

	threads[0] = std::thread(parallel_stage_1, std::ref(q1), std::ref(q2), std::ref(time_result_arr), std::ref(matrix_state_arr), std::ref(q1_is_empty));
	threads[1] = std::thread(parallel_stage_2, std::ref(q2), std::ref(q3), std::ref(time_result_arr), std::ref(matrix_state_arr), std::ref(q1_is_empty), std::ref(q2_is_empty));
	threads[2] = std::thread(parallel_stage_3, std::ref(q3), std::ref(time_result_arr), std::ref(matrix_state_arr), std::ref(q2_is_empty), cnt_matr, matr_is_print);

	for (int i = 0; i < THREADS_COUNT; i++)
	{
		threads[i].join();
	}

	if (compare_time)
	{
		printf("     %4d      %s|%s     %4d      %s|%s   %.6f  \n",
			n_matr, PURPLE, BASE_COLOR, 
			cnt_matr, PURPLE, BASE_COLOR,
			time_result_arr[cnt_matr - 1].end);
	}
	else
	{
		print_res_time(time_result_arr, cnt_matr * THREADS_COUNT);
	}
}
\end{lstlisting}
\end{center}

\clearpage

\begin{center}
\captionsetup{justification=raggedright,singlelinecheck=off}
\begin{lstlisting}[label=lst:linear_processing,caption=Функция алгоритма линейной обработки матрицы]
void linear_processing(int n_matr, int m_matr, int cnt_matr, bool matr_is_print, bool compare_time)
{
	std::queue<matrix_t> q1;
	std::queue<matrix_t> q2;
	std::queue<matrix_t> q3;

	std::mutex m;

	for (int i = 0; i < cnt_matr; i++)
	{
		matrix_t matr = generate_matrix(n_matr, m_matr);
		q1.push(matr);

		if (matr_is_print && i == cnt_matr - 1)
		{
			m.lock();
			printf("Первоначальная матрица:\n");
			print_matrix(matr);
			m.unlock();
		}
	}

	std::vector<matrix_state_t> matrix_state_arr;
	init_matrix_state_arr(matrix_state_arr, cnt_matr);

	std::chrono::time_point<std::chrono::system_clock> time_start, time_end, 
	time_begin = std::chrono::system_clock::now();

	std::vector<res_time_t> time_result_arr;
	init_time_result_arr(time_result_arr, time_begin, cnt_matr, THREADS_COUNT);

	for (int i = 0; i < cnt_matr; i++)
	{
		time_start = std::chrono::system_clock::now();
		stage_1(std::ref(q1), std::ref(q2));
		time_end = std::chrono::system_clock::now();
		
		save_result(time_result_arr, time_start, time_end, time_begin, i + 1, 1);

		time_start = std::chrono::system_clock::now();
		stage_2(std::ref(q2), std::ref(q3));
		time_end = std::chrono::system_clock::now();

		save_result(time_result_arr, time_start, time_end, time_begin, i + 1, 2);

		time_start = std::chrono::system_clock::now();
		stage_3(std::ref(q3), i + 1, cnt_matr, matr_is_print);
		time_end = std::chrono::system_clock::now();

		save_result(time_result_arr, time_start, time_end, time_begin, i + 1, 3);
	}

	if (compare_time)
	{
		printf("     %4d      %s|%s     %4d      %s|%s   %.6f  \n",
			n_matr, PURPLE, BASE_COLOR, 
			cnt_matr, PURPLE, BASE_COLOR,
			time_result_arr[cnt_matr - 1].end);
	}
	else
	{
		print_res_time(time_result_arr, cnt_matr * THREADS_COUNT);
	}
}
\end{lstlisting}
\end{center}


\begin{center}
\captionsetup{justification=raggedright,singlelinecheck=off}
\begin{lstlisting}[label=lst:parallel_stage_1,caption=Функция 1-ой ленты конвейерной обработки матрицы]
void parallel_stage_1(std::queue<matrix_t> &q1, std::queue<matrix_t> &q2,
                      std::vector<res_time_t> &time_result_arr,
                      std::vector<matrix_state_t> &matrix_state_arr, 
                      bool &q1_is_empty)
{
    std::chrono::time_point<std::chrono::system_clock> time_start, time_end;
    int task_num = 1;

    while(!q1.empty())
    {   
        time_start = std::chrono::system_clock::now();
        stage_1(std::ref(q1), std::ref(q2));
        time_end = std::chrono::system_clock::now();

        save_result(time_result_arr, time_start, time_end, time_result_arr[0].time_begin, task_num, 1);

        matrix_state_arr[task_num - 1].stage_1 = true;
        task_num++;
    }

    q1_is_empty = true;
}
\end{lstlisting}
\end{center}

\clearpage

\begin{center}
\captionsetup{justification=raggedright,singlelinecheck=off}
\begin{lstlisting}[label=lst:parallel_stage_2,caption=Функция 2-ой ленты конвейерной обработки матрицы]
void parallel_stage_2(std::queue<matrix_t> &q2, std::queue<matrix_t> &q3,
                      std::vector<res_time_t> &time_result_arr,
                      std::vector<matrix_state_t> &matrix_state_arr, 
                      bool &q1_is_empty, bool &q2_is_empty)
{
    std::chrono::time_point<std::chrono::system_clock> time_start, time_end;
    int task_num = 1;

    while(true)
    {      
        if (!q2.empty())
        {   
            if (matrix_state_arr[task_num - 1].stage_1 == true)
            {
                time_start = std::chrono::system_clock::now();
                stage_2(std::ref(q2), std::ref(q3));
                time_end = std::chrono::system_clock::now();

                save_result(time_result_arr, time_start, time_end, time_result_arr[0].time_begin, task_num, 2);

                matrix_state_arr[task_num - 1].stage_2 = true;
                task_num++;
            }
        }
        else if(q1_is_empty)
        {
            break;
        }
    }

    q2_is_empty = true;
}
\end{lstlisting}
\end{center}

\clearpage

\begin{center}
\captionsetup{justification=raggedright,singlelinecheck=off}
\begin{lstlisting}[label=lst:parallel_stage_3,caption=Функция 3-ей ленты конвейерной обработки матрицы]
void parallel_stage_3(std::queue<matrix_t> &q3, 
                      std::vector<res_time_t> &time_result_arr,
                      std::vector<matrix_state_t> &matrix_state_arr, 
                      bool &q2_is_empty, int cnt_matr, bool matr_is_print)
{
    std::chrono::time_point<std::chrono::system_clock> time_start, time_end;
    int task_num = 1;

    while(true)
    {      
        if (!q3.empty())
        {   
            if (matrix_state_arr[task_num - 1].stage_2 == true)
            {
                time_start = std::chrono::system_clock::now();
                stage_3(std::ref(q3), task_num, cnt_matr, matr_is_print);
                time_end = std::chrono::system_clock::now();

                save_result(time_result_arr, time_start, time_end, time_result_arr[0].time_begin, task_num, 3);

                matrix_state_arr[task_num - 1].stage_3 = true;
                task_num++;
            }
        }
        else if(q2_is_empty)
        {
            break;
        } 
    }
}
\end{lstlisting}
\end{center}

\clearpage

\begin{center}
\captionsetup{justification=raggedright,singlelinecheck=off}
\begin{lstlisting}[label=lst:stage_1,caption=Функция реализации 1-ого этапа обработки матрицы]
void stage_1(std::queue<matrix_t> &q1, std::queue<matrix_t> &q2)
{
	std::mutex m;

	m.lock();
	matrix_t matr = q1.front();
	m.unlock();

	matr.min_elem = get_min_elem(matr);

	m.lock();
	q2.push(matr);
	m.unlock();

	m.lock();
	q1.pop();
	m.unlock();
}
\end{lstlisting}
\end{center}

\begin{center}
\captionsetup{justification=raggedright,singlelinecheck=off}
\begin{lstlisting}[label=lst:stage_2,caption=Функция реализации 2-ого этапа обработки матрицы]
void stage_2(std::queue<matrix_t> &q2, std::queue<matrix_t> &q3)
{
	std::mutex m;

	m.lock();
	matrix_t matr = q2.front();
	m.unlock();

	mod_by_min_elem(matr);
	
	m.lock();
	q3.push(matr);
	m.unlock();
	
	m.lock();
	q2.pop();
	m.unlock();
}
\end{lstlisting}
\end{center}

\clearpage

\begin{center}
\captionsetup{justification=raggedright,singlelinecheck=off}
\begin{lstlisting}[label=lst:stage_3,caption=Функция реализации 3-его этапа обработки матрицы]
void stage_3(std::queue<matrix_t> &q3, int task_num, int cnt_matr, bool matr_is_print)
{
	std::mutex m;

	m.lock();
	matrix_t matr = q3.front();
	m.unlock();

	matr.sum_elem = get_sum_elements(matr);
	
	if (matr_is_print && task_num == cnt_matr)
	{
		printf("\nmin_elem = %d\n\nМатрица после 2 этапа:\n", matr.min_elem);        
		print_matrix(matr);
		printf("\nsum_elem = %d\n\n", matr.sum_elem);
	}

	m.lock();
	q3.pop();
	m.unlock();
}
\end{lstlisting}
\end{center}

\clearpage

\section{Функциональные тесты}

В таблице \ref{tbl:functional_test} приведены функциональные тесты для конвейерного и ленейного алгоритмов обработки матриц. Все тесты пройдены успешно.

\begin{table}[h]
	\begin{center}
	\begin{threeparttable}
		\captionsetup{justification=raggedright,singlelinecheck=off}
		\caption{\label{tbl:functional_test} Функциональные тесты}
		\begin{tabular}{|c|c|c|c|c|}
			\hline
			Строк & Столбцов & Метод обр. & Алгоритм & Ожидаемый результат 
			\\ \hline
			0 & 10 & 10 & Конвейерный & Сообщение об ошибке 
			\\ \hline
			k & 10 & 10 & Конвейерный & Сообщение об ошибке 
			\\ \hline
			10 & 0 & 10 & Конвейерный & Сообщение об ошибке 
			\\ \hline
			10 & k & 10 & Конвейерный & Сообщение об ошибке 
			\\ \hline
			10 & 10 & -5 & Конвейерный & Сообщение об ошибке 
			\\ \hline
			10 & 10 & k & Конвейерный & Сообщение об ошибке 
			\\ \hline
			100 & 100 & 20 & Конвейерный & Вывод результ. таблички
			\\ \hline
			100 & 100 & 20 & Линейный & Вывод результ. таблички
			\\ \hline
			50 & 100 & 100 & Линейный & Вывод результ. таблички
			\\ \hline
		\end{tabular}
	\end{threeparttable}
	\end{center}
\end{table}

\section*{Вывод}

В данном разделе были разработаны алгоритмы для конвейерного и ленейного алгоритмов обработки матриц, проведено тестирование, описаны средства реализации и требования к программе.
