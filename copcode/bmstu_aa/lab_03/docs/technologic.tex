\chapter{Технологическая часть}

В данном разделе будут рассмотрены средства реализации лабораторной работы, представлены листинги алгоритмов сортировки, а также тестовые данные, которые использовались для проверки корректности работы алгоритмов.


\section{Выбор языка программирования и среды разработки}
Для реализации алгоритмов был выбран язык программирования \textit{Python}, так как он предоставляет необходимую функциональность, позволяющую решить поставленные задачи на реализацию алгоритмов и выполнение замеров процессорного времени работы их реализаций, а в качестве среды разработки -- \textit{PyCharm}.
Для замеров процессорного времени использовалась функция \textit{process\_time()} \cite{time} из библиотеки \textit{time}, а для построения графиков -- библиотека \textit{matplotlib} \cite{mpl}.

\section{Реализация алгоритмов}
Ниже представлены реализации алгоритмов сортировки перемешиванием, выбором и пирамидальной сортировки (листинги \ref{lst:heapify}--\ref{lst:shakersort}).

\begin{center}
    \captionsetup{justification=raggedright,singlelinecheck=off}
    \begin{lstlisting}[label=lst:heapify,caption= Пирамидальная сортировка (перестройка сортирующего дерева)]
def heapify(arr, n, i):
    largest = i
    left = 2 * i + 1
    right = 2 * i + 2

    if left >= n and right >= n:
        return

    if left < n and arr[i] < arr[left]:
        largest = left

    if right < n and arr[largest] < arr[right]:
        largest = right

    if largest != i:
        arr[i], arr[largest] = arr[largest], arr[i]
        heapify(arr, n, largest)
\end{lstlisting}
\end{center}


\begin{center}
    \captionsetup{justification=raggedright,singlelinecheck=off}
    \begin{lstlisting}[label=lst:heapsort,caption=Пирамидальная сортировка]
def heapsort(arr):
    n = len(arr)

    for i in range(n, -1, -1):
        heapify(arr, n, i)

    for i in range(n - 1, 0, -1):
        arr[i], arr[0] = arr[0], arr[i]
        heapify(arr, i, 0)

    return arr
\end{lstlisting}
\end{center}


\begin{center}
    \captionsetup{justification=raggedright,singlelinecheck=off}
    \begin{lstlisting}[label=lst:selectionsort,caption=Сортировка выбором]
def selectionsort(arr):
    for i in range(len(arr)):
        minind = i
        for j in range(i + 1, len(arr)):
            if arr[j] < arr[minind]:
                minind = j
        arr[i], arr[minind] = arr[minind], arr[i]
    return arr
\end{lstlisting}
\end{center}


\begin{center}
    \captionsetup{justification=raggedright,singlelinecheck=off}
    \begin{lstlisting}[label=lst:shakersort,caption=Сортировка перемешиванием (шейкерная)]
def shakersort(arr):
    left = 0
    right = len(arr) - 1
    while left < right:
        r_new = left
        for i in range(left, right):
            if arr[i] > arr[i + 1]:
                arr[i], arr[i + 1] = arr[i + 1], arr[i]
                r_new = i
        right = r_new
        l_new = right
        for i in range(right - 1, left - 1, -1):
            if arr[i] > arr[i + 1]:
                arr[i], arr[i + 1] = arr[i + 1], arr[i]
                l_new = i
        left = l_new
    return arr
\end{lstlisting}
\end{center}


\section{Тестирование}
Классы эквивалентности:

\begin{itemize}[label=---]
    \item пустой массив;
    \item массив из одного элемента;
    \item массив из одинаковых элементов;
    \item отсортированный по возрастанию массив;
    \item отсортированный по убыванию массив;
    \item неотсортированный массив.
\end{itemize}

В таблице \ref{tbl:tests} представлены модульные тесты. Все тесты пройдены успешно.

\begin{table}[h]
	\begin{center}
        \begin{threeparttable}
        \captionsetup{justification=raggedright,singlelinecheck=off}
		\caption{\label{tbl:tests} Модульные тесты}
		\begin{tabular}{|c|c|c|c|c|}
			\hline
			&Входные данные& \multicolumn{3}{c|}{Ожидаемый результат} \\
			\hline
			№&Массив&Пирамидальная&Выбором&Перемешиванием \\
			\hline
            1&[ ]&[ ]&[ ]&[ ] \\
            \hline
            2&[5]&[5]&[5]&[5] \\
            \hline
            3&[8, 8, 8, 8, 8]&[8, 8, 8, 8, 8]&[8, 8, 8, 8, 8]&[8, 8, 8, 8, 8] \\
            \hline
            4&[1, 5, 8, 10]&[1, 5, 8, 10]&[1, 5, 8, 10]&[1, 5, 8, 10] \\
			\hline
			5&[10, 7, 3, -1]&[-1, 3, 7, 10]&[-1, 3, 7, 10]&[-1, 3, 7, 10] \\
			\hline
            6&[8, 0, 12, -4]&[-4, 0, 8, 12]&[-4, 0, 8, 12]&[-4, 0, 8, 12] \\
			\hline
		\end{tabular}
        \end{threeparttable}
	\end{center}
\end{table}

\section*{Вывод}

В этом разделе были рассмотрены средства реализации лабораторной работы, представлены листинги алгоритмов сортировок перемешиванием, выбором и пирамидальной сортировки, а также модульные тесты.
