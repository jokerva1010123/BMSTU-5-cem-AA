\chapter{Конструкторская часть}
В данном разделе будут представлены схемы алгоритмов сортировок, рассмотренных в аналитической части, выбранные типы данных и вывод оценки трудоемкости алгоритмов.

\section{Алгоритмы сортировки}
Схемы алгоритмов сортировки представлены на рисунках \ref{img:heapify}--\ref{img:shakersort}.

\imgScale{0.65}{heapify}{Пирамидальная сортировка (перестройка сортирующего дерева)}
\imgScale{0.65}{heapsort}{Пирамидальная сортировка}
\imgScale{0.7}{selectionsort}{Сортировка выбором}
\imgScale{0.5}{shakersort}{Сортировка перемешиванием (шейкерная)}
\clearpage


\section{Описание типов данных}
Выбранные типы данных:

\begin{itemize}[label=---]
    \item массив типа \textit{int};
    \item длина массива --- тип \textit{int}. 
\end{itemize}


\section{Оценка трудоемкости алгоритмов}
Для вывода оценки трудоемкости алгоритмов следует описать модель оценки трудоемкости.

\begin{enumerate}
    \item Трудоемкость базовых операций:
    %\begin{itemize}[label=---]
    \begin{enumerate}[label=1.\arabic*.]
       \item /, \%, * --- 2;
        \item =, +=, --=, ++, $--$, ==, !=, <, <=, >, >=, +, -, $<<$, $>>$, [ ] --- 1.
    \end{enumerate}
    \item Трудоемкость цикла с N итерациями:
    
        $f_{for}$ = $f_{\text{инициализации}}$ + $f_{\text{сравнения}}$ + N($f_{\text{тела}}$ + $f_{\text{инкремент}}$ + $f_{\text{сравнения}}$).
        
    \item Трудоемкость условного оператора:
    
        $f_{if}$ = $f_{\text{условия}}$ +
		\begin{cases}
			f_A, & \text{если условие выполняется,}\\
			f_B, & \text{иначе.}
		\end{cases}
\end{enumerate}

Вывод оценки трудоемкости алгоритмов заданных сортировок приведен ниже.

%\begin{itemize}
    %\item
    \textbf{Пирамидальная сортировка}
    
    В таблице \ref{tbl:heap_weight} представлена построчная оценка трудоемкости пирамидальной сортировки.
    \begin{table}[h]
    \begin{center}
        \begin{threeparttable}
        \captionsetup{}
        \caption{\label{tbl:heap_weight}Построчная оценка трудоемкости пирамидальной сортировки}
        \begin{tabular}{|l|c|}
            \hline
            Строка кода & Вес \\
            \hline
            largest = i & 1 \\ 
            \hline
            left = 2 * i + 1 & 4 \\ 
            \hline
            right = 2 * i + 2 & 4 \\ 
            \hline
            if left >= n and right >= n: & 2 \\ 
            \hline
            ~~~~return & 0 \\ 
            \hline
            if left < n and arr[i] < arr[left]: & 4 \\ 
            \hline
            ~~~~largest = left & 1 \\ 
            \hline
            if right < n and arr[largest] < arr[right]: & 4 \\ 
            \hline
            ~~~~largest = right & 1 \\ 
            \hline
            if largest != i: & 1 \\ 
            \hline
            ~~~~arr[i], arr[largest] = arr[largest], arr[i] & 5 \\ 
            \hline
            ~~~~heapify(arr, n, largest) & 0 \\ 
            \hline
		\end{tabular}
        \end{threeparttable}
    \end{center}
    \end{table}

    \FloatBarrier
    
    Для процесса перестройки бинарного дерева  из $N$ элементов в худшем случае потребуется $log_2 N$ операций. После $N$ итераций все элементы будут отсортированы, а значит сложность сортировки определяестя формулой $N \cdot log_2 N$.
    
    Итого трудоемкость -- O($N \cdot log_2 N$).
    
    %\item 
    \textbf{Сортировка выбором}
    
    В таблице \ref{tbl:selection_weight} представлена построчная оценка трудоемкости сортировки выбором.
    \begin{table}[h]
    \begin{center}
        \begin{threeparttable}
        \captionsetup{}
        \caption{\label{tbl:selection_weight}Построчная оценка трудоемкости сортировки выбором}
        \begin{tabular}{|l|c|}
            \hline
            Строка кода & Вес \\
            \hline
            for i in range(len(arr)): & 3 \\ 
            \hline
            ~~~~minind = i & 1 \\ 
            \hline
            ~~~~for j in range(i + 1, len(arr)): & 4 \\ 
            \hline
            ~~~~~~~~if arr[j] < arr[minind]: & 3 \\ 
            \hline
            ~~~~~~~~~~~~minind = j & 1 \\ 
            \hline
            ~~~~arr[i], arr[minind] = arr[minind], arr[i] & 5 \\ 
            \hline
            return arr & 0 \\ 
            \hline
		\end{tabular}
        \end{threeparttable}
    \end{center}
    \end{table}

    \FloatBarrier
    
    В лучшем случае (когда массив уже отсортирован в порядке возрастания):

    f = 2 + N(3 + N(3 + 5 + 2)) = O($N^2$).
    
    В худшем случае (массив отсортирован в обратном порядке):
    
    f = 2 + N(3 + N(3 + 6 + 2)) = O($N^2$).
     
    Итого трудоемкость -- O($N^2$).

    %\item
    \textbf{Сортировка перемешиванием}
    
    В таблице \ref{tbl:shaker_weight} представлена построчная оценка трудоемкости сортировки перемешиванием.
    \begin{table}[h]
    \begin{center}
        \begin{threeparttable}
        \captionsetup{}
        \caption{\label{tbl:shaker_weight}Построчная оценка трудоемкости сортировки перемешиванием}
        \begin{tabular}{|l|c|}
            \hline
            Строка кода & Вес \\
            \hline
            left = 0 & 1 \\ 
            \hline
            right = len(arr) - 1 & 2 \\ 
            \hline
            while left < right: & 1 \\ 
            \hline
            ~~~~rnew = left & 2 \\ 
            \hline
            ~~~~for i in range(left, right): & 3 \\ 
            \hline
            ~~~~~~~~if arr[i] > arr[i + 1]: & 4 \\ 
            \hline
            ~~~~~~~~~~~~arr[i], arr[i + 1] = arr[i + 1], arr[i] & 7 \\ 
            \hline
            ~~~~~~~~~~~~rnew = i & 1 \\ 
            \hline
            ~~~~right = rnew & 1 \\ 
            \hline
            ~~~~lnew = right & 1 \\ 
            \hline
            ~~~~for i in range(right - 1, left - 1, -1): & 5 \\ 
            \hline
            ~~~~~~~~if arr[i] > arr[i + 1]: & 4 \\ 
            \hline
            ~~~~~~~~~~~~arr[i], arr[i + 1] = arr[i + 1], arr[i] & 7 \\ 
            \hline
            ~~~~~~~~~~~~lnew = i & 1 \\ 
            \hline
            ~~~~left = lnew & 1 \\ 
            \hline
            return arr & 0 \\ 
            \hline
		\end{tabular}
        \end{threeparttable}
    \end{center}
    \end{table}

    \FloatBarrier
    
    В лучшем случае (массив отсортирован):

    f = 4 + N(4 + 5 + 5) = O(N).
    
    В худшем случае (массив отсортирован в обратном порядке):
    
    f = 2 + N(4 + 2 + N(2 + 12) + 2 + N(2 + 12)) = O($N^2$).
     
    Итого трудоемкость -- O($N^2$).
    
%\end{itemize}



\section{Исходные файлы программы}
Программа состоит из следующих модулей:
\begin{itemize}[label=---]
	\item $main.py$ -- файл с главной функцией, вызывающей меню программы;
    \item $sorts.py$ -- файл, содержащий код алгоритмов сортировок;
    \item $proc\_time.py$ -- файл с функциями подсчета процессорного времени алгоритмов.
\end{itemize}

\section*{Вывод}
В этом разделе были представлены схемы алгоритмов пирамидальной сортировки и сортировок выбором и перемешиванием, вывод оценки их трудоемкости и выбранные типы данных.

Проведя анализ трудоемкости алгоритмов, можно сказать, что в большинстве случаев пирамидальная сортировка является более предпочтительной, однако сортировка перемешиванием менее затратна при работе с отсортированным массивом.
