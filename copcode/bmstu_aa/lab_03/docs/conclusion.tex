\chapter*{Заключение}
\addcontentsline{toc}{chapter}{Заключение}

Цель, которая была поставлена в начале лабораторной работы, была достигнута: изучены и исследованы трудоемкости алгоритмов сортировки.

Решены все поставленные задачи:

\begin{itemize}[label=---]
	\item описаны и реализованы алгоритмы сортировок выбором, перемешиванием и пирамидальной сортировки;
	\item выведена оценка трудоемкости алгоритмов;
    \item выполнены замеры процессорного времени работы реализаций алгоритмов;
	\item проведен сравнительный анализ заданных алгоритмов сортировки по затраченному времени работы реализаций.
\end{itemize}

В результате исследования было определено, что при увеличении длины входного массива время работы реализаций алгоритмов сортировки увеличивается по разным законам.

Трудоемкости (в худшем случае):
\begin{itemize}[label=---]
    \item пирамидальная -- O($N\cdot log_2N$);
    \item выбором -- O($N^2$);
    \item перемешиванием -- O($N^2$).
\end{itemize}

Теоретическая оценка трудоемкости алгоритмов подтвердилась при замерах в ходе экспериментов.

Стоит отметить, что сортировка перемешиванием является самой быстродействующей среди рассмотренных трех при входном отсортированном массиве, однако в остальных случаях она проигрывает двум другим. 

Исходя из проведенных экспериментов, можно сделать вывод: для упорядочивания элементов в массивах общего вида лучше использовать пирамидальную сортировку.