\chapter{Исследовательская часть}

В данном разделе будут представлены примеры работы программы, проведены замеры процессорного времени и предоставлена информация о технических характеристиках устройства.

\section{Технические характеристики устройства}

Ниже представлены характеристики компьютера, на котором проводилось тестирование программы:

\begin{itemize}[label=---]
    \item операционная система Windows 10 Домашняя 21H2;
    \item оперативная память 16 Гб;
    \item процессор Intel(R) Core(TM) i7-10870H CPU @ 2.20 ГГц.
\end{itemize}

Во время тестирования ноутбук был подключен к сети электропитания. Из программного обеспечения были запущены только среда разработки браузер \textit{Chrome} и \textit{PyCharm}.

Загруженность компонентов:

\begin{itemize}[label=---]
    \item процессор -- 15~\%;
    \item оперативная память -- 45~\%.
\end{itemize}

\section{Примеры работы программы}

На рисунках \ref{img:example1}--\ref{img:example2} представлен результат работы программы. В каждом примере пользователем введен массив целых чисел и получены результаты его сортировки.

\imgScale{0.75}{example1}{Пример работы программы №1}
\imgScale{0.75}{example2}{Пример работы программы №2}
\clearpage


\section{Время выполнения реализаций алгоритмов}

Для замера процессорного времени использована функция \textit{process\_time()} библиотеки \textit{time}. Возвращаемый результат --- время в миллисекундах, число типа \textit{float}.

Чтобы получить достаточно точное значение, производилось усреднение времени для серии запусков расчета. В замерах использовались массивы длиной от 200 до 2800 символов, а сортировка каждого массива запускалась по 1000 раз.

В таблицах \ref{tbl:time_res_1}--\ref{tbl:time_res_3} представлено процессорное время работы алгоритмов сортировки.

\begin{table}[h]
    \begin{center}
        \begin{threeparttable}
        \captionsetup{justification=raggedright,singlelinecheck=off}
        \caption{\label{tbl:time_res_1}Результаты замеров времени (неотсортированный массив)}
        \begin{tabular}{|c|c|c|c|}
            \hline
            Длина массива & Пирамидальная & Выбором & Перемешиванием \\
            \hline
            200 & 0.000391 & 0.000625 & 0.001234 \\ 
            \hline
            400 & 0.000859 & 0.002859 & 0.005063 \\ 
            \hline
            600 & 0.001422 & 0.006578 & 0.012047 \\ 
            \hline
            800 & 0.001969 & 0.011797 & 0.022141 \\ 
            \hline
            1000 & 0.002594 & 0.018625 & 0.036766 \\ 
            \hline
            1200 & 0.003203 & 0.026750 & 0.051250 \\ 
            \hline
            1400 & 0.003859 & 0.039563 & 0.078734 \\ 
            \hline
            1600 & 0.004797 & 0.048188 & 0.096422 \\ 
            \hline
            1800 & 0.005188 & 0.060828 & 0.120266 \\ 
            \hline
            2000 & 0.005797 & 0.074297 & 0.151266 \\ 
            \hline
            2200 & 0.006562 & 0.090563 & 0.181062 \\ 
            \hline
            2400 & 0.007281 & 0.108969 & 0.212422 \\ 
            \hline
            2600 & 0.007859 & 0.129266 & 0.281609 \\ 
            \hline
            2800 & 0.009672 & 0.164656 & 0.354766 \\ 
            \hline
		\end{tabular}
    \end{threeparttable}
\end{center}
\end{table}


\begin{table}[h]
    \begin{center}
        \begin{threeparttable}
        \captionsetup{justification=raggedright,singlelinecheck=off}
        \caption{\label{tbl:time_res_2}Результаты замеров времени (отсортированный по возрастанию массив)}
        \begin{tabular}{|c|c|c|c|}
            \hline
            Длина массива & Пирамидальная & Выбором & Перемешиванием \\
            \hline
            200 & 0.000766 & 0.001000 & 0.000016 \\ 
            \hline
            400 & 0.001375 & 0.003625 & 0.000016 \\ 
            \hline
            600 & 0.002031 & 0.008438 & 0.000031 \\ 
            \hline
            800 & 0.002812 & 0.014984 & 0.000047 \\ 
            \hline
            1000 & 0.003625 & 0.023609 & 0.000063 \\ 
            \hline
            1200 & 0.004484 & 0.033719 & 0.000078 \\ 
            \hline
            1400 & 0.005406 & 0.046594 & 0.000094 \\ 
            \hline
            1600 & 0.006281 & 0.056734 & 0.000094 \\ 
            \hline
            1800 & 0.005672 & 0.059719 & 0.000078 \\ 
            \hline
            2000 & 0.006250 & 0.074250 & 0.000078 \\ 
            \hline
            2200 & 0.007078 & 0.092594 & 0.000125 \\ 
            \hline
            2400 & 0.008703 & 0.118984 & 0.000141 \\ 
            \hline
            2600 & 0.009547 & 0.145531 & 0.000156 \\ 
            \hline
            2800 & 0.011969 & 0.186812 & 0.000188 \\ 
            \hline
		\end{tabular}
    \end{threeparttable}
\end{center}
\end{table}


\begin{table}[h]
    \begin{center}
        \begin{threeparttable}
        \captionsetup{justification=raggedright,singlelinecheck=off}
        \caption{\label{tbl:time_res_3}Результаты замеров времени (отсортированный по убыванию массив)}
        \begin{tabular}{|c|c|c|c|}
            \hline
            Длина массива & Пирамидальная & Выбором & Перемешиванием \\
            \hline
            200 & 0.000484 & 0.000906 & 0.002719 \\ 
            \hline
            400 & 0.001016 & 0.003891 & 0.011469 \\ 
            \hline
            600 & 0.001687 & 0.009000 & 0.027750 \\ 
            \hline
            800 & 0.002406 & 0.016094 & 0.051094 \\ 
            \hline
            1000 & 0.003141 & 0.025844 & 0.082578 \\ 
            \hline
            1200 & 0.003906 & 0.038750 & 0.109063 \\ 
            \hline
            1400 & 0.004047 & 0.042969 & 0.138953 \\ 
            \hline
            1600 & 0.004719 & 0.056172 & 0.189469 \\ 
            \hline
            1800 & 0.005500 & 0.071625 & 0.231750 \\ 
            \hline
            2000 & 0.006141 & 0.088234 & 0.298125 \\ 
            \hline
            2200 & 0.006906 & 0.120609 & 0.402500 \\ 
            \hline
            2400 & 0.008797 & 0.140891 & 0.434625 \\ 
            \hline
            2600 & 0.008672 & 0.148734 & 0.483844 \\ 
            \hline
            2800 & 0.009141 & 0.176016 & 0.612531 \\ 
            \hline
		\end{tabular}
    \end{threeparttable}
\end{center}
\end{table}

\clearpage



На рисунках \ref{img:graph1}-\ref{img:graph3} также приведены результаты замеров процессорного времени.

\imgScale{1}{graph1}{Сравнение процессорного времени работы алгоритмов сортировки на неотсортированном массиве}
\FloatBarrier
\imgScale{1}{graph2}{Сравнение процессорного времени работы алгоритмов сортировки на отсортированном по возрастанию массиве}
\FloatBarrier
\imgScale{1}{graph3}{Сравнение процессорного времени работы алгоритмов сортировки на отсортированном по убыванию массиве}
\FloatBarrier


\section{Вывод}

В результате замеров процессорного времени выделены следущие аспекты:
\begin{itemize}[label=---]
    \item на неотсортированном массиве наименее затратной по времени сортировкой является пирамидальная, а самой медленной оказалась сортировка перемешиванием;
    \item в случае, если массив уже отсортирован, сортировка выбором не предпочтительна к использованию, а сортировка перемешиванием, наоборот, тратит меньше всего времени.
\end{itemize}

