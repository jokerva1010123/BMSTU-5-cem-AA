\chapter{Аналитическая часть}

\section{Цель и задачи}

 \textbf{Цель} --- изучение и исследование трудоемкости алгоритмов сортировки.\newline

Для достижения поставленной цели следует решить следующие задачи:
\begin{itemize}[label=---]
        \item описать алгоритмы сортировки (пирамидальную, выбором и перемешиванием);
        \item реализовать описанные алгоритмы;
        \item вывести оценку трудоемкости алгоритмов;
        \item выполнить замеры процессорного времени работы реализаций алгоритмов;
        \item провести сравнительный анализ заданных алгоритмов сортировки по затраченному времени работы реализаций.
\end{itemize} 

\section{Пирамидальная сортировка}
\textbf{Пирамидальная сортировка} \cite{heapsort} --- сортировка, основанная на структуре, называемой двоичной кучей или бинарным сортирующим деревом. Для этого дерева обязательно должны быть выполнены следующие условия:
\begin{itemize}[label=---]
    \item каждый лист дерева имеет глубину \textit{h}, либо $h - 1$, где \textit{h} --- максимальная глубина дерева;
    \item значение в любой вершине не меньше (не больше) значения ее потомков. 
\end{itemize}

Для представления такого дерева обычно используется массив \textit{arr}, в котором корень дерева хранится в первом элементе массива $arr[0]$, а потомки элемента $arr[i]$ --- в $arr[2\cdot i + 1]$ и $arr[2\cdot i + 2]$.

\clearpage

Сам алгоритм сортировки включает в себя два этапа:
\begin{enumerate}[label=\arabic*)]
    \item построение бинарного сортирующего дерева, удовлетворяющего перечисленным выше условиям;
    \item замена корня на последний элемент кучи, уменьшение ее размера на 1 и построение сортирующего дерева для нового корня.
\end{enumerate}

Второй этап алгоритма повторяется до тех пор, пока размер двоичной кучи не станет равным 1.

\section{Сортировка выбором}
\textbf{Сортировка выбором} \cite{selectionsort} --- метод сортировки, идея которого состоит в том, чтобы на очередном шаге найти минимум (максимум) и поставить его в начало (конец) неотсортированной части массива.

Тогда можно выделить следующие шаги алгоритма:
\begin{enumerate}[label=\arabic*)]
    \item двигаясь по текущему подмассиву, произвести поиск индекса минимального (максимального) элемента;
    \item поменять местами найденный минимум (максимум) и элемент, стоящий в начале (конце)  неотсортированной части массива.
\end{enumerate}

Описанные шаги выполняются, пока длина анализируемого подмассива не станет равна 1.

\section{Сортировка перемешиванием}
\textbf{Сортировка перемешиванием} (или шейкерная сортировка) \cite{shakersort} --- разновидность сортировки пузырьком, особенностью которой является двунаправленное действие, то есть при движении от начала массива к концу "тонет"\ максимальный элемент, а при проходе от конца к началу "всплывает"\ минимальный.

\textit{Замечание:}
алгоритм сортировки пузырьком заключается в выполнении $N - 1$ проходов, в каждом из которых производится последовательное сравнение значений соседних элементов и обмен чисел местами, если предыдущее оказывается больше последующего. В очередном $i$-м проходе выполняется сравнение каждого из $N - i$ первых элементов с правым соседом.


\section*{Вывод}
В данном разделе были теоретически разобраны алгоритмы пирамидальной сортировки, сортировки выбором и перемешиванием.

