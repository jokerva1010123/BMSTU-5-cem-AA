\chapter{Технологическая часть}

В данном разделе будут рассмотрены средства реализации задания, представлены листинги алгоритмов поиска расстояния Левенштейна и Дамерау-Левенштейна, а также тестовые данные, которые использовались для проверки корректности работы алгоритмов.


\section{Выбор языка программирования и среды разработки}
Для реализации алгоритмов был выбран язык программирования \textit{Python}, а в качестве среды разработки - \textit{PyCharm}.
Для замеров процессорного времени использовалась функция \textit{process\_time()} \cite{time} из библиотеки \textit{time}, а для построения графиков - библиотека \textit{matplotlib} \cite{mpl}.


\section{Реализация алгоритмов}
Ниже представлены реализации алгоритмов поиска расстояний Левенштейна и Дамерау-Левенштейна (листинги \ref{lst:lev}--\ref{lst:d_l_rec_cache}).

\begin{center}
    \captionsetup{justification=raggedright,singlelinecheck=off}
    \begin{lstlisting}[label=lst:lev,caption=Итерационный алгоритм поиска расстояния Левенштейна]
def levenstein(str_1, str_2):
    n, m = len(str_1), len(str_2)
    matrix = [[0] * (n + 1) for _ in range(m + 1)]

    for i in range(m + 1):
        for j in range(n + 1):
            if j == 0 and i > 0:
                matrix[i][j] = i
            if i == 0 and j > 0:
                matrix[i][j] = j
            if i > 0 and j > 0:
                if str_1[j - 1] == str_2[i - 1]:
                    res = 0
                else:
                    res = 1
                matrix[i][j] = min(
                    matrix[i][j - 1] + 1,
                    matrix[i - 1][j] + 1,
                    matrix[i - 1][j - 1] + res)

    return matrix[m][n]
\end{lstlisting}
\end{center}


\begin{center}
    \captionsetup{justification=raggedright,singlelinecheck=off}
    \begin{lstlisting}[label=lst:d_l_matr,caption=Итерационный алгоритм поиска расстояния Дамерау-Левенштейна]
def damerau_levenstein(str_1, str_2):
    n, m = len(str_1), len(str_2)
    matrix = [[0] * (n + 1) for _ in range(m + 1)]

    for i in range(m + 1):
        for j in range(n + 1):
            if i == 0 and j == 0:
                matrix[i][j] = 0
            if j == 0 and i > 0:
                matrix[i][j] = i
            if i == 0 and j > 0:
                matrix[i][j] = j
            if i > 0 and j > 0:
                if str_1[j - 1] == str_2[i - 1]:
                    res = 0
                else:
                    res = 1
                matrix[i][j] = min(
                    matrix[i][j - 1] + 1,
                    matrix[i - 1][j] + 1,
                    matrix[i - 1][j - 1] + res)

                if str_1[j - 1] == str_2[i - 2] and str_1[j - 2] == str_2[i - 1]:
                    matrix[i][j] = min(
                        matrix[i][j],
                        matrix[i - 2][j - 2] + 1)

    return matrix[m][n]
\end{lstlisting}
\end{center}


\begin{center}
    \captionsetup{justification=raggedright,singlelinecheck=off}
    \begin{lstlisting}[label=lst:d_l_rec,caption=Рекурсивный алгоритм поиска расстояния Дамерау-Левенштейна]
def damerau_levenstein_recursive(str_1, str_2):
    n, m = len(str_1), len(str_2)

    if m == 0:
        return n
    if n == 0:
        return m

    if str_1[-1] == str_2[-1]:
        mistake = 0
    else:
        mistake = 1

    res = min(damerau_levenstein_recursive(str_1, str_2[:-1]) + 1,
              damerau_levenstein_recursive(str_1[:-1], str_2) + 1,
              damerau_levenstein_recursive(str_1[:-1], str_2[:-1]) + mistake)
    if n > 1 and m > 1 and str_1[-1] == str_2[-2] and str_1[-2] == str_2[-1]:
        res = min(res, damerau_levenstein_recursive(str_1[:-2], str_2[:-2]) + 1)
    return res
\end{lstlisting}
\end{center}


\begin{center}
    \captionsetup{justification=raggedright,singlelinecheck=off}
    \begin{lstlisting}[label=lst:d_l_rec_cache,caption=Рекурсивный алгоритм поиска расстояния Дамерау-Левен- штейна с кешем]
def damerau_levenstein_recursive_cache(str_1, str_2, matrix):
    n, m = len(str_1), len(str_2)

    if m == 0:
        matrix[m][n] = n
    elif n == 0:
        matrix[m][n] = m
    else:
        if matrix[m][n - 1] == -1:
            damerau_levenstein_recursive_cache(str_1[:-1], str_2, matrix)
        if matrix[m - 1][n] == -1:
            damerau_levenstein_recursive_cache(str_1, str_2[:-1], matrix)
        if matrix[m - 1][n - 1] == -1:
            damerau_levenstein_recursive_cache(str_1[:-1], str_2[:-1], matrix)

        matrix[m][n] = min(matrix[m][n - 1] + 1,
                           matrix[m - 1][n] + 1,
                           matrix[m - 1][n - 1] + int(str_1[-1] != str_2[-1]))

        if n > 1 and m > 1 and str_1[-1] == str_2[-2] and str_1[-2] == str_2[-1]:
            matrix[m][n] = min(matrix[m][n],
                               matrix[m - 2][n - 2] + 1)

        return


def damerau_levenstein_recursive_cache_matrix(str_1, str_2):
    n, m = len(str_1), len(str_2)

    matrix = [[-1] * (n + 1) for _ in range(m + 1)]
    damerau_levenstein_recursive_cache(str_1, str_2, matrix)

    if info:
        print_matrix(str_1, str_2, matrix)
    return matrix[m][n]
\end{lstlisting}
\end{center}


\section{Тестирование}
Классы эквивалентности:

\begin{itemize}
    \item обе строки пустые;
    \item одна строка пустая, другая - нет;
    \item одинаковые строки;
    \item строки, дающие один и тот же результат и для алгоритма Левенштейна, и для алгоритмов Дамерау-Левенштейна;
    \item строки, дающие разные результаты для алгоритма Левенштейна и для алгоритмов Дамерау-Левенштейна.
\end{itemize}

В таблице \ref{tbl:tests} представлены модульные тесты. Все тесты пройдены успешно.

\begin{table}[h]
	\begin{center}
        \begin{threeparttable}
        \captionsetup{justification=raggedright,singlelinecheck=off}
		\caption{\label{tbl:tests} Модульные тесты}
		\begin{tabular}{|c|c|c|c|c|}
			\hline
			&\multicolumn{2}{c|}{Входные данные}& \multicolumn{2}{c|}{Ожидаемый результат} \\
			\hline
			№&str\_1&str\_2&Левенштейн&Дамерау-Левенштейн \\
			\hline
            1&"Пустая строка"&"Пустая строка"&0&0 \\
            \hline
            2&"Пустая строка"&hugo&4&4 \\
            \hline
            3&applejack&"Пустая строка"&9&9 \\
            \hline
            4&house&house&0&0 \\
			\hline
			5&car&cars&1&1 \\
			\hline
            6&decode&decoding&3&3 \\
			\hline
			7&class&clsas&2&1 \\
			\hline
            8&kingtoys&ikdfotpq&7&6 \\
			\hline
            9&unique&nueqeu&5&3 \\
			\hline
            10&qwertyuiop&wqreytiupo&6&5 \\
			\hline
			11&sweetbloody&croatlbiawe&10&9 \\
			\hline
		\end{tabular}
        \end{threeparttable}
	\end{center}
\end{table}

\section*{Вывод}

В этом разделе были рассмотрены средства реализации задания, представлены листинги алгоритмов поиска расстояния Левенштейна и Дамерау-Левенштейна, а также тесты.




% \begin{table}[h]
% 	\begin{center}
%         \begin{threeparttable}
%         \captionsetup{justification=raggedright,singlelinecheck=off}
% 		\caption{\label{tbl:tests} Тесты}
% 		\begin{tabular}{|c|c|c|c|c|c|}
% 			\hline
% 			& & & & \multicolumn{2}{c|}{Ожидаемый результат} \\
% 			\hline
% 			№&Описание теста&str\_1&str\_2&Левенштейн&Дамерау-Левенштейн \\
% 			\hline
%             1&Обе строки пустые&"Пустая строка"&"Пустая строка"&0&0 \\
%             \hline
%             2&Первая строка пустая&"Пустая строка"&hugo&4&4 \\
%             \hline
%             3&Вторая строка пустая&applejack&"Пустая строка"&9&9 \\
%             \hline
%             4&Одинаковые строки&house&house&0&0 \\
% 			\hline
% 			5&Строки отличаются 1 символом&car&cars&1&1 \\
% 			\hline
%             6&Строки отличающиеся 3 символами&decode&decoding&3&3 \\
% 			\hline
% 			7&Разные результаты для Л. и Д.-Л.&class&clsas&2&1 \\
% 			\hline
%             8&Разные результаты для Л. и Д.-Л.&kingtoys&ikdfotpq&7&6 \\
% 			\hline
%             9&Разные результаты для Л. и Д.-Л.&unique&nueqeu&5&3 \\
% 			\hline
%             10&Разные результаты для Л. и Д.-Л.&qwertyuiop&wqreytiupo&6&5 \\
% 			\hline
% 			11&Разные результаты для Л. и Д.-Л.&sweetbloody&croatlbiawe&10&9 \\
% 			\hline
% 		\end{tabular}
%         \end{threeparttable}
% 	\end{center}
% \end{table}