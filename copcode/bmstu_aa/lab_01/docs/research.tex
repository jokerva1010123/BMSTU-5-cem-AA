\chapter{Исследовательская часть}

В данном разделе будут представлены примеры работы программы, проведены замеры процессорного времени и предоставлена информация о технических характеристиках устройства.

\section{Примеры работы программы}

На рисунках \ref{img:example1}-\ref{img:example2} представлен результат работы программы. В каждом примере пользователем введены две строки и получены результаты вычислений редакционных расстояний.

\imgScale{0.7}{example1}{Пример работы программы (1)}
\imgScale{0.7}{example2}{Пример работы программы (2)}
\clearpage


\section{Время выполнения алгоритмов}

Для замера процессорного времени использовалась функция \textit{process\_time()} библиотеки \textit{time}. Возвращаемый результат - время в миллисекундах, число типа \textit{float}.
Чтобы получить достаточно точное значение, производилось усреднение времени.
Количество запусков замера процессорного времени:
\begin{itemize}
    \item для итерационного алгоритма поиска Левенштейна - 10000;
    \item для итерационного алгоритма поиска Дамерау-Левенштейна - 10000;
    \item для рекурсивного алгоритма поиска расстояния Дамерау-Левенштейна - 50;
    \item для рекурсивного алгоритма (с кешем) поиска расстояния Дамерау-Левенштейна - 10000.
\end{itemize}

В замерах использовались строки длиной от 0 до 10 символов.

В таблице \ref{tbl:time_res} представлено процессорное время работы реализаций алгоритмов поиска редакционного расстояния.

\begin{table}[h]
    \begin{center}
        \begin{threeparttable}
        \captionsetup{justification=raggedright,singlelinecheck=off}
        \caption{\label{tbl:time_res}Результаты замеров времени}
        \begin{tabular}{|c|c|c|c|c|}
            \hline
            Длина &Лев.&Д.-Л.(итер.)&Д.-Л.(рек.)&Д.-Л.(рек. с кешем)\\
            \hline
            0 & 0.000002 & 0.000002 & 0.000000 & 0.000000 \\ 
            \hline
            1 & 0.000003 & 0.000002 & 0.000000 & 0.000003 \\ 
            \hline
            2 & 0.000002 & 0.000003 & 0.000000 & 0.000005 \\ 
            \hline
            3 & 0.000005 & 0.000008 & 0.000000 & 0.000008 \\ 
            \hline
            4 & 0.000009 & 0.000009 & 0.000313 & 0.000014 \\ 
            \hline
            5 & 0.000013 & 0.000016 & 0.000625 & 0.000022 \\ 
            \hline
            6 & 0.000017 & 0.000019 & 0.003125 & 0.000028 \\ 
            \hline
            7 & 0.000022 & 0.000025 & 0.017500 & 0.000037 \\ 
            \hline
            8 & 0.000028 & 0.000033 & 0.095000 & 0.000050 \\ 
            \hline
            9 & 0.000034 & 0.000041 & 0.523438 & 0.000061 \\ 
            \hline
            10 & 0.000041 & 0.000048 & 2.912500 & 0.000078 \\ 
            \hline
		\end{tabular}
    \end{threeparttable}
\end{center}
\end{table}

На рисунках \ref{img:graph1}-\ref{img:graph3} также приведены результаты замеров процессорного времени. 

При данном масштабе на рисунке \ref{img:graph2} графики времени работы итерационной и рекурсивной реализации с кешем алгоритма Дамерау-Левенштейна совпадают.

\imgScale{1}{graph1}{Сравнение итерационных алгоритмов поиска расстояний Левенштейна и Дамерау-Левенштейна}
\FloatBarrier
\imgScale{1}{graph2}{Сравнение 3 алгоритмов поиска расстояний Дамерау-Левенштейна}
\FloatBarrier
\imgScale{1}{graph3}{Сравнение итерационного алгоритма поиска расстояний Дамерау-Левенштейна и рекурсивного с кешем}
\FloatBarrier


\section{Технические характеристики устройства}

Ниже представлены характеристики компьютера, на котором проводилось тестирование программы:

\begin{itemize}
    \item операционная система Windows 10 Домашняя 21H2;
    \item оперативная память 16 Гб;
    \item процессор Intel(R) Core(TM) i7-10870H CPU @ 2.20 ГГц.
\end{itemize}

Во время тестирования ноутбук был подключен к сети электропитания. Из программного обеспечения были запущены только среда разработки \textit{PyCharm} и браузер \textit{Chrome}.

Загруженность компонентов:

\begin{itemize}
    \item процессор - 20\%;
    \item оперативная память - 49\%.
\end{itemize}


\section{Вывод}

В результате замеров процессорного времени выделены следущие аспекты:
\begin{itemize}
    \item итерационные алгоритмы Левенштейна и Дамерау-Левенштейна работают за сходное время;
    \item рекурсивный алгоритм Дамерау-Левенштейна уже при длине строк равной 4 символа проигрывает в 26 раз по времени итерационной и рекурсивной с кешем реализациям;
    \item итерационный алгоритм поиска расстояний Дамерау-Левенштейна в среднем на 42\% - 54\% быстрее рекурсивного с кешем для длин строк от 0 до 10 символов;
\end{itemize}

