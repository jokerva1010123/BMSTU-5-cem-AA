\chapter{Конструкторская часть}
В данном разделе будут представлены схемы алгоритмов поиска расстояний Левенштейна и Дамерау-Левенштейна, выбранные типы данных и теоретическая оценка затрат по памяти.

\section{Алгоритмы поиска редакционных расстояний}
Схемы алгоритмов поиска редакционных расстояний представлены на рисунках \ref{img:lev}-\ref{img:d_l_rec_cache_helper}.
\imgScale{0.45}{lev}{Схема итерационного алгоритма нахождения расстояния Левенштейна}
\imgScale{0.43}{d_l_matr}{Схема итерационного алгоритма нахождения расстояния Дамерау-Левенштейна}
\imgScale{0.55}{d_l_rec}{Схема рекурсивного алгоритма нахождения расстояния Дамерау-Левенштейна}
\imgScale{0.7}{d_l_rec_cache}{Схема рекурсивного алгоритма нахождения расстояния Дамерау-Левенштейна с кешем (создание матрицы)}
\imgScale{0.5}{d_l_rec_cache_helper}{Схема рекурсивного алгоритма нахождения расстояния Дамерау-Левенштейна с кешем}
\clearpage


\section{Описание типов данных}
Выбранные типы данных:

\begin{itemize}
    \item две строки - тип \textit{str};
    \item длина строки - тип \textit{int};
    \item матрица (для итерационного представления алгоритма Левенштейна и Дамерау-Левенштейна и для рекурсивной реализации с кешем) - двумерный массив типа \textit{int}. 
\end{itemize}


\section{Оценка затрат алгоритмов по памяти}

Затраты по памяти для алгоритма поиска расстояния Левенштейна (итерационный):
\begin{itemize}
    \item матрица - (n + 1) * (m + 1) * sizeof(int);
    \item строки str\_1, str\_2 - (n + m + 2) * sizeof(char);
    \item длины строк n, m - 2 * sizeof(int);
    \item дополнительные переменные (i, j, res) - 3 * sizeof(int);
    \item адрес возврата.
\end{itemize}

Итого:

(n + 1) * (m + 1) * sizeof(int) + (n + m + 2) * sizeof(char) + 5 * sizeof(int)\newline
    
Затраты по памяти для алгоритма поиска расстояния\newlineДамерау-Левенштейна (итерационный):
\begin{itemize}
    \item матрица - (n + 1) * (m + 1) * sizeof(int);
    \item строки str\_1, str\_2 - (n + m + 2) * sizeof(char);
    \item длины строк n, m - 2 * sizeof(int);
    \item дополнительные переменные (i, j, res) - 3 * sizeof(int);
    \item адрес возврата.
\end{itemize}

Итого:

(n + 1) * (m + 1) * sizeof(int) + (n + m + 2) * sizeof(char) + 5 * sizeof(int)\newline

Затраты по памяти для алгоритма поиска расстояния\newlineДамерау-Левенштейна (рекурсивный), для одного вызова:
\begin{itemize}
    \item строки str\_1, str\_2 - (n + m + 2) * sizeof(char);
    \item длины строк n, m - 2 * sizeof(int);
    \item дополнительные переменные (mistake) - sizeof(int);
    \item адрес возврата.
\end{itemize}

Итого (K - количество вызовов рекурсии):

((n + m + 2) * sizeof(char) + 3 * sizeof(int)) * K\newline

Затраты по памяти для алгоритма поиска расстояния\newlineДамерау-Левенштейна (рекурсивный с кешем), для одного вызова:
\begin{itemize}
    \item строки str\_1, str\_2 - (n + m + 2) * sizeof(char);
    \item длины строк n, m - 2 * sizeof(int);
    \item адрес возврата;
\end{itemize}
и матрица - (n + 1) * (m + 1) * sizeof(int).\newline

Итого (K - количество вызовов рекурсии):

((n + m + 2) * sizeof(char) + 2 * sizeof(int)) * K + (n + 1) * (m + 1) * sizeof(int)\newline

\section{Исходные файлы программы}
Программа состоит из следующих модулей:
\begin{itemize}
	\item $main.py$ -- файл с главной функцией, вызывающей меню программы;
    \item $algorithms.py$ -- файл, содержащий код алгоритмов поиска редакционных расстояний и функцию вывода полученной матрицы;
    \item $proc\_time.py$ -- файл с функциями подсчета процессорного времени алгоритмов;
    \item $tests.py$ -- файл с тестирующей функцией.
    \newline
\end{itemize}

\section*{Вывод}
В этом разделе были представлены схемы алгоритмов поиска расстояний Левенштейна и Дамерау-Левенштейна, выбранные типы данных и теоретическая оценка затрат по памяти.

Проведя анализ оценки алгоритмов по памяти, можно сказать, что рекурсивные алгоритмы менее затратны, так как для них максимальный размер памяти растет прямо пропорционально сумме длин строк, а в итерационных -- пропорционально их произведению.