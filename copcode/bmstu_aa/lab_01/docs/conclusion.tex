\chapter*{Заключение}
\addcontentsline{toc}{chapter}{Заключение}

Цель, которая была поставлена в начале лабораторной работы, была достигнута: изучен метод динамического программирования на основании определения редакционных расстояний по алгоритмам Левенштейна и Дамерау-Левенштейна.

Решены все поставленные задачи:

\begin{itemize}
	\item изучены, разработаны и реализованы алгоритмы поиска расстояния Левенштейна и Дамерау-Левенштейна;
	\item выполнена оценка затрат алгоритмов по памяти;
    \item выполнены замеры алгоритмов процессорного времени работы реализаций алгоритмов;
	\item проведен сравнительный анализ нерекурсивных алгоритмов поиска расстояний Левенштейна и Дамерау-Левенштейна;
	\item проведен сравнительный анализ трех алгоритмов поиска расстояний Дамерау-Левенштейна.
\end{itemize}

В результате проведенных экспериментов было определено, что при увеличении длины строк время работы изученных алгоритмов увеличивается в геометрической прогрессии. Самой медленной реализацией оказалась рекурсивная реализация алгоритма поиска расстояния Дамерау-Левенштейна. Из оставшихся трех вариантов наиболее быстрым является итерационный алгоритм Левенштейна. Но стоит отметить, что матричная реализация алгоритма Дамерау-Левенштейна отрабатывает примерно за такое же время. 

Несмотря на то, что итерационные алгоритмы обладают высоким быстродействием, при больших длинах строк они занимают довольно много памяти под матрицу.