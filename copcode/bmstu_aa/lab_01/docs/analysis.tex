\chapter{Аналитическая часть}

\section{Цель и задачи}

 \textbf{Цель} -- изучение метода динамического программирования на основании определения редакционных расстояний по алгоритмам Левенштейна и Дамерау-Левенштейна.\newline

Для достижения поставленной цели следует решить следующие задачи:
\begin{itemize}
        \item изучить расстояния Левенштейна и Дамерау-Левенштейна;
        \item разработать алгоритмы поиска расстояний Левенштейна и Дамерау-Левенштейна;
        \item реализовать разработанные алгоритмы;
        \item выполнить оценку затрат алгоритмов по памяти;
        \item выполнить замеры процессорного времени работы реализаций алгоритмов;
        \item провести сравнительный анализ нерекурсивных алгоритмов для поиска расстояний Левенштейна и Дамерау-Левенштейна;
        \item провести сравнительный анализ трех алгоритмов поиска расстояний Дамерау-Левенштейна. \newline
\end{itemize} 

\section{Итерационный алгоритм нахождения расстояния Левенштейна}
\textbf{Расстояние Левенштейна} - минимальное количество редакционных мероприятий, необходимых для преобразования одной строки в другую.\newline

Пусть $S_{1}$ и $S_{2}$ -- две строки, длиной \textit{N} и \textit{M} соответственно, а редакционные операции:
\begin{itemize}
        \item вставка символа в произвольной позиции (I - Insert);
        \item удаление символа в произвольной позиции (D - Delete);
        \item замена символа на другой (R - Replace). \newline
        % \item совпадение двух символов (M - Match). \newline
\end{itemize} 

Принято, что для этих операций "штраф"\ равен 1. \newline

Для поиска расстояния Левенштейна используют рекуррентную формулу, то есть такую, которая использует предыдущие члены ряда или последовательности для вычисления последующих:


\begin{equation}
	\label{eq:Lev}
	D(i, j) = \begin{cases}
		
		0, &\text{i = 0, j = 0}\\
		i, &\text{j = 0, i > 0}\\
		j, &\text{i = 0, j > 0}\\
		\min \lbrace \\
		\qquad D(i, j-1) + 1,\\
		\qquad D(i-1, j) + 1, &\text{i > 0, j > 0}\\
		\qquad D(i-1, j-1) + \begin{cases}
                        		0, &\text{если $S_{i}$ = $S_{j}$,}\\
                        		1, &\text{иначе}
                        	\end{cases}\\
		\rbrace,
	\end{cases}
\end{equation}

Итерационный алгоритм поиска расстояния Левенштейна будет выполнять расчёт по формуле (\ref{eq:Lev}).

\section{Итерационный алгоритм нахождения\newline расстояния Дамерау-Левенштейна}
Поскольку одна из самых частых ошибок при наборе текстов на естественном языке -- это перестановка двух соседних символов местами, Дамерау предложил включить в число редакционных операций операцию перестановки двух соседних символов со "штрафом"\ 1.

Тогда для данного алгоритма определены следующие редакционные операции:
\begin{itemize}
        \item вставка символа в произвольной позиции (I - Insert);
        \item удаление символа в произвольной позиции (D - Delete);
        \item замена символа на другой (R - Replace);
        \item транспозиция двух символов (M - Match).
\end{itemize} 


Таким образом, чтобы получить формулу нахождения расстояния Дамерау-Левенштейна, необходимо в формулу для поиска расстояния Левенштейна добавить еще одно определение минимума, но уже из четырех вариантов, если возможна перестановка двух соседних символов:

\begin{equation}
	\label{eq:D}
	D(i, j) = \begin{cases}
		
		0, &\text{i = 0, j = 0,}\\
		i, &\text{j = 0, i > 0,}\\
		j, &\text{i = 0, j > 0,}\\
		
		\min \lbrace \\
		\qquad D(i, j-1) + 1,&\text{i > 0, j > 0,}\\
		\qquad D(i-1, j) + 1,&\text{$S_{i}$ = $S_{j-1}$,}\\
		\qquad D(i-2, j-2) + 1,&\text{$S_{i-1}$ = $S_{j}$,}\\
		\qquad D(i-1, j-1) + \begin{cases}
                        		0, &\text{если $S_{i}$ = $S_{j}$,}\\
                        		1, &\text{иначе}
                        	\end{cases}\\
		\rbrace,\\
		
		\min \lbrace \\
		\qquad D(i, j-1) + 1,\\
		\qquad D(i-1, j) + 1,&\text{иначе}\\
		\qquad D(i-1, j-1) + \begin{cases}
                        		0, &\text{если $S_{i}$ = $S_{j}$,}\\
                        		1, &\text{иначе}
                        	\end{cases}\\
		\rbrace
	\end{cases}
\end{equation}

Итерационный алгоритм поиска расстояния Дамерау-Левенштейна будет выполнять расчёт по формуле (\ref{eq:D}).


\section{Рекурсивный алгоритм нахождения\newlineрасстояния Дамерау-Левенштейна}
Рекурсивный алгоритм поиска расстояния Дамерау-Левенштейна решает ту же задачу, что и его итерационная реализация. Отличие лишь в том, что в данном случае не используется матрица, так как все значения необходимые для расчета последующих, вычисляются рекурсивно. Но в связи с этим такая реализация не отличается быстродействием.


\section{Рекурсивный алгоритм нахождения\newlineрасстояния Дамерау-Левенштейна с использованием кеша}

Данный алгоритм является оптимизацией рекурсивной реализации. Суть оптимизации заключается в использовнии кеша, который представляет собой матрицу.

При выполнении рекурсии выполняется заполнение матрицы значениями редакционных расстояний для подстрок заданной длины. Если данные, для которых выполняется расчет на очередном шаге рекурсии, еще не обрабатывались (значение ячейки матрицы равно -1), то результат будет занесен в матрицу. Если же это расстояние уже было найдено, то вычисление не выполняется, а алгоритм переходит к следующему шагу рекурсии.

\section*{Вывод}
В данном разделе были теоретически разобраны алгоритмы Левенштейна и Дамерау-Левенштейна. Эти алгоритмы позволяют найти редакционное расстояние для двух строк.

