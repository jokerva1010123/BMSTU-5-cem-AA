\chapter{Аналитическая часть}

\section{Цель и задачи}

 \textbf{Цель} -- изучение и исследование алгоритмов перемножения матриц.\newline

Для достижения поставленной цели следует решить следующие задачи:
\begin{itemize}
        \item разработать алгоритмы перемножения матриц (классический, Винограда и Винограда с оптимизациями);
        \item реализовать разработанные алгоритмы;
        \item вывести оценку трудоемкости алгоритмов;
        \item выполнить замеры процессорного времени работы реализаций алгоритмов;
        \item провести сравнительный анализ заданных алгоритмов сортировки по затрачнному времени работы реализаций. \newline
\end{itemize} 

\section{Классический алгоритм умножения матриц}
Произведением матриц $A$ и $B$ называется матрица $C$ такая, что число строк и столбцов матрицы $C$ равно количеству строк матрицы $A$ и столбцов матрицы $B$ соответственно \cite{classic}.

Необходимым условием умножения двух матриц является равнство количества столбцов первой матрицы количеству строк второй матрицы.

Пусть даны матрицы $A$ [$a\times b$] и $B$ [$c\times d$]:
\begin{equation}
A = \left(
\begin{array}{cccc}
a_{11} & a_{12} & \ldots & a_{1n}\\
a_{21} & a_{22} & \ldots & a_{2n}\\
\vdots & \vdots & \ddots & \vdots\\
a_{n1} & a_{n2} & \ldots & a_{nn}
\end{array}
\right),
\end{equation}

\begin{equation}
B = \left(
\begin{array}{cccc}
b_{11} & b_{12} & \ldots & b_{1n}\\
b_{21} & b_{22} & \ldots & b_{2n}\\
\vdots & \vdots & \ddots & \vdots\\
b_{n1} & b_{n2} & \ldots & b_{nn}
\end{array}
\right),
\end{equation}

тогда произведением матриц $A$ и $B$ будет считаться матрица $C$ [$a$ x $d$]:

\begin{equation}
C = \left(
\begin{array}{cccc}
c_{11} & c_{12} & \ldots & c_{1n}\\
c_{21} & c_{22} & \ldots & c_{2n}\\
\vdots & \vdots & \ddots & \vdots\\
c_{n1} & c_{n2} & \ldots & c_{nn}
\end{array}
\right),
\end{equation}

где

\begin{equation}
\label{eq:classic}
    C_{ij} = \sum\limits_{k=1}^m a_{ik}b_{kj},\ i = 1, 2, ..., a;\ j = 1, 2, ..., d. 
\end{equation}

Классический алгоритм умножения двух матриц работает по формуле \ref{eq:classic}.

\section{Алгоритм Винограда}
В матрице, которая является результатом произведения двух других матриц, каждый элемент представляет собой скалярное произведение соответствующих строки и столбца исходных матриц \cite{vinograd}. 
Рассмотрим два вектора:

\begin{equation}
P = \left(
\begin{array}{cccc}
p_{1}, & p_{2}, & p_{3}, & p_{4}\\
\end{array}
\right),
\end{equation}
\begin{equation}
Q = \left(
\begin{array}{cccc}
q_{1}, & q_{2}, & q_{3}, & q_{4}\\
\end{array}
\right).
\end{equation}

Их скалярное произведение будет равно:
\begin{equation}
\label{eq:optim}
P \cdot Q = p_{1}q_{1} + p_{2}q_{2} + p_{3}q_{3} + p_{4}q_{4};
\end{equation}
\begin{equation}
P \cdot Q = (p_{1} + q_{2})(p_{2} + q_{1}) + (p_{3} + q_{4})(p_{4} + q_{3}) - p_{1}p_{2} - p_{3}p_{4} - q_{1}q_{2} - q_{3}q_{4}.
\end{equation}

После внимательного изучения формулы \ref{eq:optim} можно сказать, что ее слагаемые могут быть вычислены заранее для каждой строки первой матрицы и для каждого столбца второй. В итоге некоторые операции умножения заменяются на операции сложения, которые являются менее времязатратными.

Именно этот прием и используется в алгоритме Винограда.

\section{Алгоритм Винограда с оптимизациями}
Для того, чтобы дополнительно уменьшить время работы алгоритма Винограда необходимо применить следующие методы оптимизации:
\begin{itemize}
    \item заменить операцию $x = x + k$ на $x += k$;
    \item заменить умножение на 2 на побитовый сдвиг;
    \item предвычислять слагаемые для алгоритма, например те, которые считаются в цикле каждый раз заново, но не зависят от исполняемых итераций.
\end{itemize}


\section*{Вывод}
В данном разделе был теоретически разобран классический алгоритм умножения матриц и алгоритм Винограда.