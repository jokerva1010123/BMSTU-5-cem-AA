\chapter{Исследовательская часть}

В данном разделе будут представлены примеры работы программы, проведены замеры процессорного времени и предоставлена информация о технических характеристиках устройства.

\section{Примеры работы программы}

На рисунках \ref{img:example1}-\ref{img:example2} представлен результат работы программы. В каждом примере пользователем введены две матрицы и получены результаты их умножения.

\imgScale{0.6}{example1}{Пример работы программы №1}
\imgScale{0.75}{example2}{Пример работы программы №2}
\clearpage


\section{Время выполнения реализаций алгоритмов}

Для замера процессорного времени использовалась функция \textit{process\_time()} библиотеки \textit{time}. Возвращаемый результат -- время в миллисекундах, число типа \textit{float}.

Чтобы получить достаточно точное значение, производилось усреднение времени. В замерах использовались матрицы размером от [$50\times 50$] до \newline [$450\times 450$], а сам процесс перемножения для каждых двух матриц запускался по 200 раз.

В таблице \ref{tbl:time_res} представлено процессорное время работы алгоритмов умножения матриц.

\begin{table}[h]
    \begin{center}
        \begin{threeparttable}
        \captionsetup{justification=raggedright,singlelinecheck=off}
        \caption{\label{tbl:time_res}Результаты замеров времен}
        \begin{tabular}{|c|c|c|c|}
            \hline
            Размер матрицы & Классический & Винограда & Винограда с опт. \\
            \hline
            50 & 0.026328 & 0.023984 & 0.021094 \\ 
            \hline
            100 & 0.165547 & 0.174063 & 0.164062 \\ 
            \hline
            150 & 0.565937 & 0.456016 & 0.458047 \\ 
            \hline
            200 & 1.135781 & 1.183828 & 1.235234 \\ 
            \hline
            250 & 2.311016 & 2.591016 & 2.154219 \\ 
            \hline
            300 & 4.498516 & 4.429297 & 3.884453 \\ 
            \hline
            350 & 6.387656 & 7.152188 & 6.172891 \\ 
            \hline
            400 & 9.993828 & 10.394141 & 9.592266 \\ 
            \hline
            450 & 13.757500 & 15.271719 & 13.475078 \\ 
            \hline
		\end{tabular}
    \end{threeparttable}
\end{center}
\end{table}

\FloatBarrier

На рисунке \ref{img:graph1} также приведены результаты замеров процессорного времени.

\imgScale{1}{graph1}{Сравнение процессорного времени работы алгоритмов умножения матриц}
\FloatBarrier


\section{Технические характеристики устройства}

Ниже представлены характеристики компьютера, на котором проводилось тестирование программы:
\begin{itemize}
    \item операционная система Windows 10 Домашняя 21H2;
    \item оперативная память 16 Гб;
    \item процессор Intel(R) Core(TM) i7-10870H CPU @ 2.20 ГГц.
\end{itemize}

Во время тестирования ноутбук был подключен к сети электропитания. Из программного обеспечения были запущены только среда разработки \textit{PyCharm} и браузер \textit{Chrome}.

Процессор был загружен на 19\%, оперативная память -- на 50\%.


\section{Вывод}

В результате замеров процессорного времени выделены следущие аспекты: работа алгоритма Винограда без оптимизаций занимает больше времени, чем два остальных алгоритма; классический алгоритм умножения матриц и алгоритм Винограда с оптимизациями производят вычисления приблизительно за одно время.

