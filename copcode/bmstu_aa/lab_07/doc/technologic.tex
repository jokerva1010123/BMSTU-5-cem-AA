\chapter{Технологическая часть}

В данном разделе будут рассмотрены средства реализации лабораторной работы, представлены тестовые данные, которые использовались для проверки корректности работы алгоритмов.

\section{Выбор языка программирования и среды \\разработки}
Для реализации алгоритмов был выбран язык \textit{Python}, так как он предоставляет необходимую функциональность, позволяющую решить поставленные задачи на реализацию алгоритмов, а в качестве среды разработки --- \textit{PyCharm}.
Для построения графиков использовалась библиотека \textit{matplotlib}~\cite{mpl}.


\section{Тестирование}
Классы эквивалентности:

\begin{itemize}[label=---]
    \item использование одиночных термов;
    \item использование составных термов с <<очень>>;
    \item использование составных термов с <<не>> и <<очень>>.
\end{itemize}

В таблице \ref{tbl:tests} представлены тесты. Все тесты пройдены успешно.

\begin{table}[h]
	\begin{center}
        \begin{threeparttable}
        \captionsetup{justification=raggedright,singlelinecheck=off}
		\caption{\label{tbl:tests} Функциональные тесты}
		\begin{tabular}{|c|c|c|}
			\hline
			Запрос& Ожидаемы результат & Фактический результат \\
			\hline
            медленные автомобили & Niva 4x4 & Niva 4x4 \\
			\hline
                                     & Toyota Mark II & Toyota Mark II \\
                                     & Subaru Impreza & Subaru Impreza \\
            очень быстрые            & Ford Mustang & Ford Mustang \\
              автомобили             & BMW X6 & BMW X6 \\
                                     & BMW X5 & BMW X5 \\
                                     & BMW M4 & BMW M4 \\
            \hline
            не очень быстрые & Niva 4x4 & Niva 4x4 \\
              автомобили       & Lada Granta & Lada Granta \\
            \hline
                                     & Lada Granta & Lada Granta \\
                                     & Mazda 6 & Mazda 6 \\
            не очень медленные       & Mazda 3 & Mazda 3 \\
              автомобили             & Nissan Skyline & Nissan Skyline \\
                                     & Nissan Silvia & Nissan Silvia \\
                                     & Toyota Land Cruiser & Toyota Land Cruiser \\
            \hline
		\end{tabular}
        \end{threeparttable}
	\end{center}
\end{table}
\FloatBarrier