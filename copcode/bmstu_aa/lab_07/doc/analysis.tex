\chapter{Аналитическая часть}

\section{Цель и задачи}
\textbf{Цель} --- получение навыка поиска по словарю при ограничении на значение признака, заданном при помощи лингвистической переменной.

Для достижения поставленной цели необходимо выполнить следующие задачи:

\begin{itemize}[label=---]
	\item формализовать объект и его признак;
	\item провести анкетирование респондентов;
	\item построить функцию принадлежности термам числовых значений признака, описываемого лингвистической переменной, на основе статистической обработки мнений респондентов, выступающих в роли экспертов;
	\item описать алгоритм поиска в словаре объектов;
	\item описать структуру данных словаря;
	\item реализовать описанный алгоритм поиска в словаре.
\end{itemize}

\section{Структура данных <<словарь>>}

Словарь \cite{dict} --- абстрактный тип данных, позволяющий хранить пары вида (ключ, значение) и поддерживающий операции добавления пары, а также поиска и удаления пары по ключу:
\begin{enumerate}[label=\arabic*)]
	\item \textit{insert(k, v)};
	\item \textit{find(k)};
	\item \textit{remove(k)}.
\end{enumerate}

В паре \textit{(k, v)}: \textit{v} называется значением, ассоциированным с ключом \textit{k}. Где \textit{k} — это ключ, a \textit{v} — значение. Семантика и названия вышеупомянутых операций в разных реализациях ассоциативного массива могут отличаться.

Операция поиска \textit{find(k)} возвращает значение, ассоциированное с заданным ключом, или некоторый специальный объект, означающий, что значения, ассоциированного с заданным ключом, нет. Две другие операции ничего не возвращают.

Словарь с точки зрения интерфейса удобно рассматривать как обычный массив, в котором в качестве индексов можно использовать не только целые числа, но и значения других типов --- например, строки (именно по этой причине словарь также иногда называют <<ассоциативным массивом>>).

\section{Алгоритм полного перебора}
Алгоритмом полного перебора называют метод решения задачи, при котором по очереди рассматриваются все возможные варианты. В случае реализации алгоритма в рамках данной работы будут последовательно перебираться ключи словаря до тех пор, пока не будет найден нужный.

Трудоёмкость алгоритма зависит от того, присутствует ли искомый ключ в словаре, и, если присутствует -- насколько он далеко от начала массива ключей.
Пусть на старте алгоритм затрагивает $k_{0}$ операций, а при сравнении $k_{1}$ операций.

Пусть алгоритм нашёл элемент на первом сравнении (лучший случай), тогда будет затрачено $k_0 + k_1$ операций, на втором --- $k_0 + 2 \cdot k_1$, на последнем (худший случай) --- $k_0 + N \cdot k_1$. Если ключа нет в массиве ключей, то мы сможем понять это, только перебрав все ключи, таким образом трудоёмкость такого случая равно трудоёмкости случая с ключом на последней позиции. Трудоёмкость в среднем может быть рассчитана как математическое ожидание по формуле (\ref{for:brute}), где $\Omega$ --- множество всех возможных случаев.

\begin{equation}
	\label{for:brute}
	\begin{aligned}
		\sum\limits_{i \in \Omega} p_i \cdot f_i = k_0 + k_1 \cdot \left(1 + \frac{N}{2} - \frac{1}{N + 1}\right)
	\end{aligned}
\end{equation}

\section{Формализация объектов и его признаков}

В данной лабораторной работе словарь используется для описания <<максимальной скорости автомобиля>> со следующими параметрами: ключ --- терм (словесное описание признака), значение --- массив числовых значений признака.

Доступные термы:
\begin{itemize}[label=---]
    \item очень медленные;
    \item медленные;
    \item средние;
    \item быстрые;
    \item очень быстрые;
    \item не очень быстрые;
    \item не очень иедленные.
\end{itemize}

Доступные числовые значения максимальной скорости: от 0 до 500 км/ч.

\section{Анкетирование респондентов}

На рисунке \ref{img:vote} приведены результаты анкетирования респонденотов.

\imgScale{0.65}{vote}{Результаты анкетирования}
\FloatBarrier

\section{Функция пренадлежности термам числовых \\значений признака}

Построим графики функций принадлежности числовых значений переменной термам, описывающим группы значений лингвистической переменной.

Для этого для каждого значения максимальной скорости автомобиля для каждого терма из перечисленных найдём количество респондентов, согласно которым значение удовлетворяет сопоставляемому терму.
Данное значение поделим на количество респондентов --- это и будет значением функции $\mu$ для терма в точке.

На рисунке \ref{img:func_res} показан результат подсчета значений функции принадлежности.

\imgScale{0.65}{func_res}{Функция пренадлежности термам числовых значений признака}
\FloatBarrier

На рисунке \ref{img:graph} представлены графики функций принадлежности термам числовых значений признака.
\imgScale{0.9}{graph}{Графики функций принадлежности термам числовых значений признака}
