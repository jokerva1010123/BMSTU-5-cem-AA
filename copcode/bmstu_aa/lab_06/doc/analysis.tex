\chapter{Аналитическая часть}

В данном разделе будет представлено описание задачи коммивояжера и используемых для её решения алгоритмов (полного перебора и муравьиного алгоритма).

\section{Цель и задачи}
\textbf{Цель} --- изучение муравьиного алгоритма на примере задачи коммивояжера.

Для достижения поставленной цели необходимо выполнить следующие задачи:

\begin{itemize}[label=---]
	\item исследовать задачу коммивояжера;
	\item изучить алгоритм полного перебора и муравьиный алгоритм для решения задачи коммивояжера;
	\item провести параметризацию муравьиного алгоритма на двух классах данных;
	\item привести схемы используемых алгоритмов;
	\item реализовать описанные алгоритмы;
	\item провести сравнительный анализ времени работы алгоритмов;
	\item провести тестирование.
\end{itemize}

\section{Задача коммивояжера}

Коммивояжер --- бродячий торговец. Задача коммивояжера~\cite{bib1} --- одна из самых важных задач транспортной логистики, отрасли, занимающейся планированием транспортных перевозок. Коммивояжеру, чтобы распродать товары, следует объехать $n$ пунктов и в конце концов вернуться в исходный пункт. Требуется определить наиболее выгодный маршрут объезда. В качестве меры выгодности маршрута может служить суммарное время в пути, суммарная стоимость дороги, или, в простейшем случае, длина маршрута.

В описываемой задаче рассматривается несколько городов и матрица смежности.


\section{Алгоритм полного перебора}


Алгоритм полного перебора~\cite{bib2} для решения задачи коммивояжера предполагает рассмотрение всех возможных путей в графе и выбор наименьшего из них. Смысл перебора состоит в том, что мы перебираем все варианты объезда городов и выбираем оптимальный. Однако, при таком подходе количество возможных маршрутов очень быстро возрастает с ростом $n$ (сложность алгоритма равна $n!$).

Алгоритм полного перебора гарантирует точное решение задачи, однако, уже при небольшом числе городов будут большие затраты по времени выполнения.

\section{Муравьиный алгоритм}


Муравьиный алгоритм~\cite{bib3} -- метод решения задач коммивояжера, в основе которого лежит моделирование поведения колонии муравьев.

Каждый муравей определяет для себя маршрут, который необходимо пройти на основе феромона, который он ощущает во время прохождения, каждый муравей оставляет феромон на своем пути, чтобы остальные муравьи могли по нему ориентироваться. В результате при прохождении каждым муравьем различного маршрута наибольшее число феромона остается на оптимальном пути.


Пусть муравей имеет следующие характеристики:
\begin{itemize}[label=---]
	\item зрение --- способен определить длину ребра;
	\item память --- запоминает пройденный маршрут;
	\item обоняние --- чувствует феромон.
\end{itemize}


Также введем целевую функцию \ref{eq:d_func}.

\begin{equation}
	\label{eq:d_func}
	\eta_{ij} = 1 / D_{ij},
\end{equation}
где $D_{ij}$ — расстояние из текущего пункта $i$ до заданного пункта $j$.


А также понадобится формула вычисления вероятности перехода в заданную точку \ref{eq:posib}.

\begin{equation}
	\label{eq:posib}
	P_{kij} = \begin{cases}
		\frac{\tau_{ij}^a\eta_{ij}^b}{\sum_{q=1}^m \tau^a_{iq}\eta^b_{iq}}, \textrm{вершина не была посещена ранее муравьем k,} \\
		0, \textrm{иначе}
	\end{cases}
\end{equation}
где $a$ -- параметр влияния длины пути, $b$ -- параметр влияния феромона, $\tau_{ij}$ -- расстояния от города $i$ до $j$, $\eta_{ij}$ -- количество феромонов на ребре $ij$.

После завершения движения всех муравьев, формула обновляется феромон по формуле \ref{eq:update_phero_1}:
\begin{equation}
	\label{eq:update_phero_1}
		\tau_{ij}(t+1) = (1-p)\tau_{ij}(t) + \Delta \tau_{ij}.
\end{equation}
При этом
\begin{equation}
\label{update_phero_2}
 \Delta \tau_{ij} = \sum_{k=1}^N \tau^k_{ij},
\end{equation}
где
\begin{equation}
	\label{eq:update_phero_3}
		 \Delta\tau^k_{ij} = \begin{cases}
		Q/L_{k}, \textrm{ребро посещено k-ым муравьем,} \\
		0, \textrm{иначе}
	\end{cases}
\end{equation}


Описание поведения муравьев при выборе пути.

\begin{enumerate}
	\item  Муравьи имеют собственную <<память>>. Поскольку каждый город может быть посещён только один раз, то у каждого муравья есть списокуже посещенных городов - список запретов. Обозначим через $J_{ik}$ список городов, которые необходимо посетить муравью $k$, находящемуся в городе $i$.
	\item Муравьи обладают «зрением» - желание посетить город $j$, если муравей находится в городе $i$. Будем считать, что видимость обратно пропорциональна расстоянию между городами.
	\item Муравьи обладают «обонянием» - они могут улавливать след феромона, подтверждающий желание посетить город $j$ из города $i$ на основании опыта других муравьёв. Количество феромона на ребре $(i, j)$ в момент времени $t$ обозначим через $\tau_{i,j}(t)$.
	\item Пройдя ребро $(i, j)$, муравей откладывает на нём некоторое количество феромона, которое должно быть связано с оптимальностью сделанного выбора. Пусть $T_{k}(t)$ есть маршрут, пройденный муравьем $k$ к моменту времени $t$, $L_{k}(t)$ - длина этого маршрута, а $Q$ - параметр, имеющий значение порядка длины оптимального пути. Тогда откладываемое количество феромона может быть задано формулой \ref{eq:update_phero_3}.
\end{enumerate}
