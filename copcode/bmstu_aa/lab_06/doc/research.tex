\chapter{Исследовательская часть}

В данном разделе будут представлены примеры работы программы, проведены замеры процессорного времени и предоставлена информация о технических характеристиках устройства.

\section{Технические характеристики устройства}

Ниже представлены характеристики компьютера, на котором проводилось тестирование программы:
\begin{itemize}[label=---]
    \item операционная система Windows 10 Домашняя 21H2;
    \item оперативная память 16 Гб;
    \item процессор Intel(R) Core(TM) i7-10870H CPU @ 2.20 ГГц.
\end{itemize}

Во время тестирования ноутбук был подключен к сети электропитания. Из программного обеспечения были запущены только среда разработки \textit{PyCharm} и браузер \textit{Chrome}.

Процессор был загружен на 19\%, оперативная память -- на 50\%.

\section{Примеры работы программы}

На рисунке \ref{img:example}--\ref{img:example1} представлен результат работы программы.

\imgScale{0.75}{example}{Пример работы программы №1}
\imgScale{0.75}{example1}{Пример работы программы №2}
\clearpage


\section{Время выполнения реализаций алгоритмов}

Результаты замеров времени работы алгоритмов решения задачи коммивояжера представлены на рисунках \ref{img:graph}. Замеры времени проводились в секундах и усреднялись для каждого набора одинаковых экспериментов.

\imgScale{0.5}{graph}{Зависимость времени работы алгоритмов от размера матриц}

\section{Автоматическая параметризация}

Автоматическая параметризация была проведена на двух классах данных. Для проведение эксперимета были взяты матрицы размером 10x10. Муравьиный алгоритм был запущен для всех значений $\alpha, \rho \in (0, 1)$ с шагом 0.1.

В качестве эталонного значения был взят результат работы алгоритма полного перебора. 

Далее будут представлены матрицы смежности (матрица \ref{eq:kd1} для первого класса данных и \ref{eq:kd2} для второго), на которых происходила параметризация и таблицы с результами её выполнения.

\subsection{Класс данных 1}

В качестве первого класса данных была взята матрица смежности, в которой все значения незначительно отличаются друг от друга, находятся в диапазоне [1, 3]. Таблица с результатами параметризации представлена в приложении А.

\begin{equation}
    \label{eq:kd1}
	M_{1} = \begin{pmatrix}
		0 & 2 & 1 & 1 & 3 & 3 & 2 & 2 & 3 & 3 \\
		2 & 0 & 3 & 3 & 3 & 1 & 3 & 2 & 2 & 3 \\
		1 & 3 & 0 & 2 & 3 & 3 & 1 & 1 & 2 & 3 \\
		1 & 3 & 2 & 0 & 2 & 3 & 2 & 3 & 1 & 3 \\
		3 & 3 & 3 & 2 & 0 & 3 & 2 & 3 & 1 & 1 \\
		3 & 1 & 3 & 3 & 3 & 0 & 1 & 1 & 2 & 1 \\
		2 & 3 & 1 & 2 & 2 & 1 & 0 & 2 & 1 & 2 \\
		2 & 2 & 1 & 3 & 3 & 1 & 2 & 0 & 2 & 3 \\
		3 & 2 & 2 & 1 & 1 & 2 & 1 & 2 & 0 & 2 \\
		3 & 3 & 3 & 3 & 1 & 1 & 2 & 3 & 2 & 0 
	\end{pmatrix}
\end{equation}

\subsection{Класс данных 2}

В качестве второго класса данных была взята матрица смежности, в которой все значения отличюатся на большое значение друг от друга, находятся в диапазоне [1, 1000]. Таблица с результатами параметризации представлена в приложении А.

\begin{equation}
    \label{eq:kd2}
	M_{2} = \begin{pmatrix}
		0 & 157 & 611 & 117 & 341 & 452 & 579 & 773 & 370 & 343 \\
		157 & 0 & 170 & 696 & 238 & 669 & 302 & 633 & 111 & 427 \\
		611 & 170 & 0 & 95 & 829 & 14 & 661 & 118 & 871 & 754 \\
		117 & 696 & 95 & 0 & 37 & 535 & 308 & 996 & 419 & 456 \\
		341 & 238 & 829 & 37 & 0 & 854 & 454 & 908 & 806 & 455 \\
		452 & 669 & 14 & 535 & 854 & 0 & 646 & 262 & 400 & 799 \\
		579 & 302 & 661 & 308 & 454 & 646 & 0 & 139 & 982 & 423 \\
		773 & 633 & 118 & 996 & 908 & 262 & 139 & 0 & 887 & 59 \\
		370 & 111 & 871 & 419 & 806 & 400 & 982 & 887 & 0 & 578 \\
		343 & 427 & 754 & 456 & 455 & 799 & 423 & 59 & 578 & 0 
	\end{pmatrix}
\end{equation}


\section{Вывод}

В этом разделе были указаны технические характеристики машины, на которой происходило сравнение времени работы алгоритмов (муравьиного алгоритма и алгоритма полного перебора) решения задачи коммивояжера, также была рассмотрена автоматическая параметризация.

В результате замеров времени было установлено, что муравьиный алгоритм работает хуже алгоритма полного перебора на матрицах, размер которых меньше 9 (при размере 8 он работает хуже в 3.7 раза). Но при больших размерах муравьиный алгоритм существенно превосходит алгоритм полного перебора (на матрицах 9x9 он лучше в 2.1 раз, а на матрицах 10x10 уже в 15.4 раза). Во время проведения замеров времени кол-во дней для муравьиного алгоритма было взято 300.

На основе проведённой параметризации по двум классам данных можно сделать следующие выводы.

Для первого класса данных \ref{eq:kd1} лучше всего подходят следующие параметры:
\begin{itemize}[label=---]
    \item $\alpha = 0.1, \beta = 0.9, \rho = 0.2, 0.5$;
    \item $\alpha = 0.2, \beta = 0.8, \rho = 0.3, 0.4, 06$;
    \item $\alpha = 0.3, \beta = 0.7, \rho = 0.2$;
\end{itemize}  

Для второго класса данных \ref{eq:kd2} лучше всего подходят следующие коэффиценты:
\begin{itemize}[label=---]
    \item $\alpha = 0.1, \beta = 0.9, \rho = 0.1, 0.6, 0.8$;
    \item $\alpha = 0.2, \beta = 0.8, \rho = 0.5$;
    \item $\alpha = 0.3, \beta = 0.7, \rho = 0.4$;
\end{itemize}  

Таким образом, можно сделать вывод о том, что для лучшей работы муравьиного алгоритма на используемых классах данных (\ref{eq:kd1} и \ref{eq:kd2}) нужно использовать полученные коэффиценты.

