\chapter*{Заключение}
\addcontentsline{toc}{chapter}{Заключение}

Было экспериментально подтверждено различие во временной эффективности муравьиного алгоритма и алгоритма полного перебора решения задачи коммивояжера. В результате исследований можно сделать вывод о том, что при матрицах большого размера (больше 9) стоит использовать муравьиный алгоритм решения задачи коммивояжера, а не алгоритм полного перебора (на матрице размером 10x10 он работает в 15.4 раза быстрее). Также было установлено по результатам параметризации на экспериментальных класса данных, что при коэффиценте $\alpha$ = 0.1, 0.2, 0.3 муравьиный алгоритм работает наилучшим образом.
\vspace{5mm}

В ходе выполнения данной лабораторной работы были решены следующие задачи:
\begin{itemize}[label=---]
	\item исследована задача коммивояжера;
	\item изучен алгоритм полного перебора и муравьиный алгоритм для решения задачи коммивояжера;
	\item проведена параметризация муравьиного алгоритма;
	\item приведены схемы используемых алгоритмов;
	\item реализованы описанные алгоритмы;
	\item проведен сравнительный анализ времени работы алгоритмов;
	\item проведено тестирование.
\end{itemize}

Поставленная цель была достигнута.