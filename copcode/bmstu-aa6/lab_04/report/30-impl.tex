\chapter{Технологическая часть}

В данном разделе приведены средства программной реализации и листинг кода.

\section{Требования к ПО}

К программе предъявляется ряд требований:
\begin{itemize}
	\item на вход подаются размеры 2 матриц, а также их элементы;
	\item на выходе — матрица, которая является результатом умножения входных матриц.
\end{itemize}

\section{Средства реализации}

В качестве языка программирования для реализации данной лабораторной работы был выбран современный компилируемый ЯП Rust \cite{rustlang}. Данный выбор обусловлен популярностью языка и скоростью его выполнения, а также тем, что данный язык предоставляет широкие возможности для написания тестов \cite{rusttest}.

\section{Листинг кода}

В листинге \ref{lst:simple} приведена реализация простого алгоритма Винограда. В листингах \ref{lst:parallel1} и \ref{lst:parallel2} приведены реализации параллельных алгоритмов Винограда. В листингах \ref{lst:utils} и \ref{lst:parallel_utils} приведены вспомогательные однопоточные и многопоточные функции соответственно.

\begin{lstinputlisting}[
	caption={Последовательный алгоритм Винограда},
	label={lst:simple},
	style={rust},
	linerange={24-83}
]{../src/lib/algorithms.rs}
\end{lstinputlisting}

\begin{lstinputlisting}[
	caption={Параллельный алгоритм Винограда 1},
	label={lst:parallel1},
	style={rust},
	linerange={3-14}
]{../src/lib/algorithms/parallel1.rs}
\end{lstinputlisting}

\begin{lstinputlisting}[
	caption={Параллельный алгоритм Винограда 2},
	label={lst:parallel2},
	style={rust},
	linerange={3-15}
]{../src/lib/algorithms/parallel2.rs}
\end{lstinputlisting}

\begin{lstinputlisting}[
	caption={Вспомогательные однопоточные функции},
	label={lst:utils},
	style={rust},
	linerange={6-36}
]{../src/lib/algorithms/utils.rs}
\end{lstinputlisting}

\begin{lstinputlisting}[
	caption={Вспомогательные многопоточные функции},
	label={lst:parallel_utils},
	style={rust},
]{../src/lib/algorithms/parallel_utils.rs}
\end{lstinputlisting}

\section{Тестирование функций}

В таблице~\ref{tabular:test_rec} приведены тесты для функций, реализующих однопоточный и многопоточный алгоритмы Винограда. Тесты пройдены успешно.

\begin{table}[h!]
	\begin{center}
		\begin{tabular}{c@{\hspace{7mm}}c@{\hspace{7mm}}c@{\hspace{7mm}}c@{\hspace{7mm}}c@{\hspace{7mm}}c@{\hspace{7mm}}}
			\hline
			Матрица 1 & Матрица 2 &Ожидаемый результат \\ \hline
			\vspace{4mm}
			$\begin{pmatrix}
			1 & 2 & 3\\
			1 & 2 & 3\\
			1 & 2 & 3
			\end{pmatrix}$ &
			$\begin{pmatrix}
			1 & 2 & 3\\
			1 & 2 & 3\\
			1 & 2 & 3
			\end{pmatrix}$ &
			$\begin{pmatrix}
			6 & 12 & 18\\
			6 & 12 & 18\\
			6 & 12 & 18
			\end{pmatrix}$ \\
			\vspace{2mm}
			\vspace{2mm}
			$\begin{pmatrix}
			1 & 2\\
			1 & 2
			\end{pmatrix}$ &
			$\begin{pmatrix}
			1 & 2\\
			1 & 2
			\end{pmatrix}$ &
			$\begin{pmatrix}
			3 & 6\\
			3 & 6
			\end{pmatrix}$ \\
			\vspace{2mm}
			\vspace{2mm}
			$\begin{pmatrix}
			2
			\end{pmatrix}$ &
			$\begin{pmatrix}
			2
			\end{pmatrix}$ &
			$\begin{pmatrix}
			4
			\end{pmatrix}$ \\
			\vspace{2mm}
			\vspace{2mm}
			$\begin{pmatrix}
			1 & -2 & 3\\
			1 & 2 & 3\\
			1 & 2 & 3
			\end{pmatrix}$ &
			$\begin{pmatrix}
			-1 & 2 & 3\\
			1 & 2 & 3\\
			1 & 2 & 3
			\end{pmatrix}$ &
			$\begin{pmatrix}
			0 & 4 & 6\\
			4 & 12 & 18\\
			4 & 12 & 18
			\end{pmatrix}$\\
			\vspace{2mm}
			\vspace{2mm}
			$\begin{pmatrix}
			1 & 2
			\end{pmatrix}$ &
			$\begin{pmatrix}
			1 & 2
			\end{pmatrix}$ &
			Не могут быть перемножены\\
		\end{tabular}
	\end{center}
	\caption{\label{tabular:test_rec} Тестирование функций}
\end{table}

\section*{Вывод}

Правильный выбор инструментов разработки позволил эффективно реализовать алгоритмы, настроить модульное тестирование и выполнить исследовательский раздел лабораторной работы.
