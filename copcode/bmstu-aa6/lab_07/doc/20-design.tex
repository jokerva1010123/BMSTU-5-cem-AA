\chapter{Конструкторская часть}

\section{Структура словаря}

Словарь состоит из пар вида \texttt{<id - ФИ>}, где \texttt{id} - id игрока NHL \cite{nhl} на оффициальном сайте лиги \cite{nhl}, \texttt{ФИ} - его фамилия и имя.

\section{Схемы алгоритмов}

На рисунке \ref{img:brute} представлена схема алгоритма поиска полным перебором, на рисунке \ref{img:binary_search} представлена схема поиска с использованием двоичного поиска, на рисунках \ref{img:segments1} и \ref{img:segments2} представлена схема поиска по сегментам, которые в результате частотного анализа (анализа длины) упорядочены в порядке убывания длины и отсортированы для возможности использования двоичного поиска внутри сегмента.

\img{110mm}{brute}{Схема алгоритма поиска полным перебором.}
\clearpage
\img{120mm}{binary_search}{Схема алгоритма поиска с использованием двоичного поиска.}
\clearpage
\img{120mm}{segments1}{Схема алгоритма поиска с использованием разделения на сегменты и частотного анализа.}
\clearpage
\img{120mm}{segments2}{Схема алгоритма поиска с использованием разделения на сегменты и частотного анализа. Продолжение.}
\clearpage

\section*{Вывод}

В данном разделе были рассмотрены структура словаря, на котором будут проводиться эксперименты, а также схемы алгоритмов поисков.
