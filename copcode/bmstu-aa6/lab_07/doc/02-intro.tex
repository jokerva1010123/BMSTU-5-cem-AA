\chapter*{Введение}
\addcontentsline{toc}{chapter}{Введение}

Словарь, как тип данных, применяется везде, где есть связь ``ключ - значение'' или ``объект - данные'': поиск истории болезни пациента по номеру его амбулаторной карты, поиск налогов по ИНН и другое. Поиск - основная задача при использовании словаря. Данная задача решается различными способами, которые дают различную скорость решения.

Цель данной работы: получить навык работы со словарём, как структурой данных, реализовать алгоритмы поиска по словарю с применением оптимизаций.

В рамках выполнения работы необходимо решить следующие задачи:
\begin{itemize}
	\item реализовать алгоритм поиска по словарю, использующий полный перебор;
	\item реализовать алгоритм поиска по словарю, использующий двоичный поиск;
	\item применить частотный анализ для эффективного поиска по словарю;
    \item сравнить полученные результаты;
    \item сделать выводы по проделанной работе.
\end{itemize}
