\chapter*{Введение}
\addcontentsline{toc}{chapter}{Введение}

Одной из важнейших процедур обработки структурированной информации является сортировка \cite{Knut}. Сортировкой называют процесс перегруппировки заданной последовательности (кортежа) объектов в некотором определенном порядке. Определенный порядок (например, упорядочение в алфавитном порядке, по возрастанию или убыванию количественных характеристик, по классам, типам и.т.п.) в последовательности объектов необходимо для удобства работы с этим объектом . В частности, одной из целей сортировки является облегчение последующего поиска элементов в отсортированном множестве. 

Алгоритмы сортировки используются практически в любой программной системе. Целью алгоритмов сортировки является упорядочение последовательности элементов данных. Поиск элемента в последовательности отсортированных данных занимает время, пропорциональное логарифму количеству элементов в последовательности, а поиск элемента в последовательности не отсортированных данных занимает время, пропорциональное количеству элементов в последовательности, то есть намного больше. Существует множество различных методов сортировки данных. Однако любой алгоритм сортировки можно разбить на три основные части:
\begin{itemize}
    \item сравнение, определяющее упорядоченность пары элементов;
    \item перестановка, меняющая местами пару элементов;
    \item собственно сортирующий алгоритм, который осуществляет сравнение и перестановку элементов данных до тех пор, пока все эти элементы не будут упорядочены.
\end{itemize}

Важнейшей характеристикой любого алгоритма сортировки является скорость его работы, которая определяется функциональной зависимостью среднего времени сортировки последовательностей элементов данных, заданной длины, от этой длины. Время сортировки будет пропорционально количеству сравнений и перестановки элементов данных в процессе их сортировки.

Как уже было сказано, в любой сфере, использующей какое-либо программное обеспечение, с большой долей вероятности используются сортировки. К примеру, на сайте \cite{sort-benchmark} можно найти результаты производительности алгоритмов сортировки для ряда ведущих центров данных. При этом используются различные критерии оценки эффективности.

Задачи лабораторной работы:
\begin{itemize}
	\item изучить и реализовать 3 алгоритма сортировки: пузырёк, вставками, выбором;
	\item провести сравнительный анализ трудоёмкости алгоритмов на основе теоретических расчетов и выбранной модели вычислений;
	\item провести сравнительный анализ алгоритмов на основе экспериментальных данных;
    \item подготовить отчета по лабораторной работе.
\end{itemize}
