\chapter{Технологическая часть}

В данном разделе приведены средства реализации и листинг кода.

\section{Требования к ПО}

К программе предъявляется ряд требований:
\begin{itemize}
	\item на вход подаются размеры 2 матриц, а также их элементы;
	\item на выходе — матрица, которая является результатом умножения входных матриц.
\end{itemize}

\section{Средства реализации}

В качестве языка программирования для реализации данной лабораторной работы был выбран современный компилируемый ЯП Rust \cite{rustlang}. Данный выбор обусловлен моим желанием расширить свои знания в области применения данного языка, а также тем, что данный язык предоставляет широкие возможности для написания тестов \cite{rusttest}.

\section{Листинг кода}

В листингах \ref{lst:simple}--\ref{lst:fast_subroutines} приведены реализации алгоритмов умножения матриц, в листингах \ref{lst:tests}--\ref{lst:benches} представлены примеры написания тестов и бенчмарков.

\begin{lstinputlisting}[
	caption={Стандартный алгоритм умножения матриц},
	label={lst:simple},
	style={rustlang},
	linerange={19-31}
]{../src/lib/algorithms.rs}
\end{lstinputlisting}

\begin{lstinputlisting}[
	caption={Алгоритм Копперсмита — Винограда},
	label={lst:vino},
	style={rustlang},
	linerange={89-113}
]{../src/lib/algorithms.rs}
\end{lstinputlisting}

\begin{lstinputlisting}[
	caption={Функции алгоритма Копперсмита — Винограда},
	label={lst:subroutines},
	style={rustlang},
	linerange={33-55}
]{../src/lib/algorithms.rs}
\end{lstinputlisting}

\begin{lstinputlisting}[
	caption={Оптимизированный алгоритм Копперсмита — Винограда},
	label={lst:vino_improved},
	style={rustlang},
	linerange={115-145}
]{../src/lib/algorithms.rs}
\end{lstinputlisting}

\begin{lstinputlisting}[
	caption={Функции оптимизированного алгоритма Копперсмита — Винограда},
	label={lst:fast_subroutines},
	style={rustlang},
	linerange={57-79}
]{../src/lib/algorithms.rs}
\end{lstinputlisting}

\begin{lstinputlisting}[
	caption={Пример написания теста для алгоритмов},
	label={lst:tests},
	style={rustlang},
	linerange={38-42}
]{../src/lib/algorithms/tests.rs}
\end{lstinputlisting}

\begin{lstinputlisting}[
	caption={Функция run\_check},
	label={lst:run_check},
	style={rustlang},
	linerange={27-36}
]{../src/lib/algorithms/tests.rs}
\end{lstinputlisting}

\begin{lstinputlisting}[
	caption={Пример написания бенчмарка для алгоритма},
	label={lst:benches},
	style={rustlang},
	linerange={57-61}
]{../src/lib/algorithms/tests.rs}
\end{lstinputlisting}


\section{Тестирование функций}

В таблице~\ref{tabular:test_rec} приведены тесты для функций, реализующих стандартный алгоритм умножения матриц, алгоритм Винограда и оптимизированный алгоритм Винограда. Тесты пройдены успешно.

\begin{table}[h!]
	\begin{center}
		\begin{tabular}{c@{\hspace{7mm}}c@{\hspace{7mm}}c@{\hspace{7mm}}c@{\hspace{7mm}}c@{\hspace{7mm}}c@{\hspace{7mm}}}
			\hline
			Матрица 1 & Матрица 2 &Ожидаемый результат \\ \hline
			\vspace{4mm}
			$\begin{pmatrix}
			1 & 2 & 3\\
			1 & 2 & 3\\
			1 & 2 & 3
			\end{pmatrix}$ &
			$\begin{pmatrix}
			1 & 2 & 3\\
			1 & 2 & 3\\
			1 & 2 & 3
			\end{pmatrix}$ &
			$\begin{pmatrix}
			6 & 12 & 18\\
			6 & 12 & 18\\
			6 & 12 & 18
			\end{pmatrix}$ \\
			\vspace{2mm}
			\vspace{2mm}
			$\begin{pmatrix}
            1 & 2 & 2\\
            1 & 2 & 2
			\end{pmatrix}$ &
			$\begin{pmatrix}
			1 & 2\\
			1 & 2\\
            1 & 2
			\end{pmatrix}$ &
			$\begin{pmatrix}
			5 & 10\\
			5 & 10
			\end{pmatrix}$ \\
			\vspace{2mm}
			\vspace{2mm}
			$\begin{pmatrix}
			2
			\end{pmatrix}$ &
			$\begin{pmatrix}
			2
			\end{pmatrix}$ &
			$\begin{pmatrix}
			4
			\end{pmatrix}$ \\
			\vspace{2mm}
			\vspace{2mm}
			$\begin{pmatrix}
			1 & -2 & 3\\
			1 & 2 & 3\\
			1 & 2 & 3
			\end{pmatrix}$ &
			$\begin{pmatrix}
			-1 & 2 & 3\\
			1 & 2 & 3\\
			1 & 2 & 3
			\end{pmatrix}$ &
			$\begin{pmatrix}
			0 & 4 & 6\\
			4 & 12 & 18\\
			4 & 12 & 18
			\end{pmatrix}$\\
			\vspace{2mm}
			\vspace{2mm}
			$\begin{pmatrix}
			1 & 2
			\end{pmatrix}$ &
			$\begin{pmatrix}
			1 & 2
			\end{pmatrix}$ &
			Не могут быть перемножены\\
		\end{tabular}
	\end{center}
	\caption{\label{tabular:test_rec} Тестирование функций}
\end{table}

\section*{Вывод}

Правильный выбор инструментов разработки позволил эффективно реализовать алгоритмы, настроить модульное тестирование и выполнить исследовательский раздел лабораторной работы.
