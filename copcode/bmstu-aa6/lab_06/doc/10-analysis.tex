\chapter{Аналитическая часть}
В данном разделе описаны задача коммивояжёра, а также алгоритм полного перебора и муравьиный алгоритм для решения данной задачи.

\section{Задача коммивояжера}
Коммивояжёр (фр. \textit{commis voyageur}) — бродячий торговец. Задача коммивояжёра — важная задача транспортной логистики, отрасли, занимающейся планированием транспортных перевозок \cite{Commie}. В описываемой задаче рассматривается несколько городов  и матрица попарных расстояний между ними. Требуется найти такой порядок  посещения  городов,  чтобы  суммарное  пройденное  расстояние было минимальным, каждый город посещался ровно один раз и  коммивояжер  вернулся  в  тот  город,  с  которого  начал  свой  маршрут.  Другими  словами,  во  взвешенном  полном  графе  требуется найти гамильтонов цикл минимального веса.

\section{Алгоритм полного перебора для решения задачи коммивояжера}
Алгоритм полного перебора для решения задачи коммивояжера предполагает рассмотрение всех возможных путей в графе и выбор наименьшего из них.

Такой подход гарантирует точное решение задачи, однако, так как задача относится к числу $NP$-сложных \cite{np_tasks}, то уже при небольшом числе городов решение за приемлемое время невозможно.

\section{Муравьиный алгоритм для решения задачи коммивояжера}
Муравьиные алгоритмы представляют собой новый перспективный метод решения задач оптимизации, в основе которого лежит моделирование поведения колонии муравьев \cite{Ulianov}. Колония представляет собой систему с очень простыми правилами автономного поведения особей.

Каждый муравей определяет для себя маршрут, который необходимо пройти на основе феромона, который он ощущает, во время прохождения, каждый муравей оставляет феромон на своем пути, чтобы остальные муравьи могли по нему ориентироваться. В результате при прохождении каждым муравьем различного маршрута наибольшее число феромона остается на оптимальном пути.

Самоорганизация колонии является результатом взаимодействия следующих компонентов:
\begin{itemize}
    \item случайность — муравьи имеют случайную природу движения;
    \item многократность — колония допускает число муравьев, достигающее от нескольких десятков до миллионов особей;
    \item положительная обратная связь — во время движения муравей откладывает феромон, позволяющий другим особям определить для себя оптимальный маршрут;
    \item отрицательная обратная связь — по истечении определенного времени феромон испаряется;
    \item целевая функция.
\end{itemize}

Пусть муравей обладает следующими характеристиками:
\begin{itemize}
    \item память — запоминает маршрут, который прошел;
    \item зрение — определяет длину ребра;
    \item обоняние — чувствует феромон.
\end{itemize}

Введем целевую функцию $\eta_{ij} = 1 / D_{ij}$, где $D_{ij}$ — расстояние из пункта $i$ до пункта $j$.

Посчитаем вероятности перехода в заданную точку по формуле \eqref{possibility}:
\begin{equation}
    \label{possibility}
    P_{k,ij} = \begin{cases}
        \frac{\tau_{ij}^\alpha\eta_{ij}^\beta}{\sum_{q \in J_{i,k}}^m \tau^\alpha_{iq}\eta^\beta_{iq}}, \textrm{вершина не была посещена ранее муравьем k,} \\
        0, \textrm{иначе}
    \end{cases}
\end{equation}
где $\alpha, \beta$ -- настраиваемые параметры, $J_{i,k}$ - список городов, которые надо посетить $k$-ому муравью, находящемуся в $i$-ом городе, $\tau$ - концентрация феромона, а при $\alpha = 0$ алгоритм вырождается в жадный.

Когда все муравьи завершили движение происходит обновление феромона по формуле \eqref{pheromone1}:
\begin{equation}
    \label{pheromone1}
    \tau_{ij}(t+1) = (1-p)\tau_{ij}(t) + \Delta \tau_{ij}, \Delta \tau_{ij} = \sum_{k=1}^N \tau^k_{ij}
\end{equation}
где
\begin{equation}
    \label{pheromone2}
    \Delta \tau^k_{ij} = \begin{cases}
        Q/L_{k}, \textrm{ребро посещено k-ым муравьем,} \\
        0, \textrm{иначе}
    \end{cases}
\end{equation}
$L_{k}$ — длина пути k-ого муравья, $Q$ — некоторая константа порядка длины путей, $N$ — количество муравьев.

\section*{Вывод}
Были рассмотрены задача Коммивояжера, муравьиный алгоритм и алгоритм полного перебора для решения данной задачи.
