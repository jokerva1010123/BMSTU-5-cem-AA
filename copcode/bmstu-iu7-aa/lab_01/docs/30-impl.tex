\chapter{Технологическая часть}

В данном разделе приведены требования к программному обеспечению, средства реализации и листинги кода.

\section{Выбор языка программирования}

В качестве языка программирования для реализации данной лабораторной работы был выбран язык Golang \cite{golang}. Данный выбор обусловлен тем, что я имею некоторый опыт разработки на нем, а так же наличием у языка встроенных высокоточных средств тестирования и анализа разработанного ПО.

\section{Требования к программному обеспечению}

К программе предъявляется ряд требований:
\begin{itemize}
	\item На вход подаётся две регистрозависимых строки.
	\item На выходе — результат выполнения каждого из вышеуказанных алгоритмов.
\end{itemize}

\section{Сведения о модулях программы}

Данная программа разбита на следующие модули:
\begin{itemize}
\item \texttt{main.go} -- Файл, содержащий точку входа в программу. В нем происходит общение с пользователем и вызов алгоритмов.
\item \texttt{io.go} -- Файл содержит функции для вввода и вывода информации.
\item \texttt{iterative\_cash.go} -- Файл содержит реализацию поиска расстояния Левенштейна с кэшированием.
\item \texttt{iterative\_dl\_cash.go} -- Файл содержит реализацию поиска расстояния Дамерау-Левенштейна с кэшированием.
\item \texttt{recursive\_dl.go} -- Файл содержит рекурсивную реализацию поиска расстояния Дамерау-Левенштейна.
\item \texttt{recursive.go} -- Файл содержит рекурсивную реализацию поиска расстояния Левенштейна.
\item \texttt{types.go} -- Файл содержит пользовательские типы данных, используемые в алгоритмах.
\item \texttt{utils.go} -- Файл содержит различные функции для вычислений.
\end{itemize}

В листингах \ref{lst:recursive}--\ref{lst:iterative_dl_cash} представлены исходные коды разобранных ранее алгоритмов.

\listingfile{main.go}{main}{Go}{Основной файл программы main}{linerange={9-25}}

\listingfile{types.go}{types}{Go}{Определение пользовательских типов данных}{linerange={3-10}}

\clearpage

\listingfile{utils.go}{utils}{Go}{Различные функции для вычислений}{linerange={3-19}}

\listingfile{recursive.go}{recursive}{Go}{Рекурсивная функция нахождения расстояния Левенштейна}{linerange={3-19}}

\clearpage

\listingfile{iterative_cash.go}{iterative_cash}{Go}{Функция нахождения расстояния Левенштейна с кешем}{linerange={3-25}}

\listingfile{recursive_dl.go}{recursive_dl}{Go}{Рекурсивная функция нахождения расстояния Дамерау--Левенштейна}{linerange={7-30}}

\clearpage

\listingfile{iterative_dl_cash.go}{iterative_dl_cash}{Go}{Функция нахождения расстояния Дамерау--Левенштейна с кэшем}{linerange={3-42}}

\section{Вывод}

Были реализованы алгоритмы: вычисления расстояния Левенштейна рекурсивно, с заполнением кэша, а также вычисления расстояния Дамерау--Левенштейна рекурсивно и вычисления расстояния Дамерау--Левенштейна с заполнением кэша.
