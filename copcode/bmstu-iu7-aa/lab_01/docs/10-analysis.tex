\chapter{Аналитическая часть}

\section{Некоторые теоретические сведения}

При преобразовании одного слова в другое можно использовать следующие операции.
\begin{enumerate}
\item D (от англ. delete) - удаление;
\item I (от англ. insert) - вставка;
\item R (от англ replace) - замена;
\end{enumerate}

Будем считать стоимость каждой вышеизложенной операции равной 1.

Введем понятие совпадения - M ( от англ. match ). Его стоимость будет
равна 0.

\section{Рекурсивный алгоритм нахождения расстояния Левенштейна}

Расстояние Левенштейна между двумя строками a и b может быть вычислено по формуле \ref{eq:D}, где $|a|$ означает длину строки $a$; $a[i]$ — i-ый символ строки $a$ , функция $D(i, j)$ определена как:

\begin{equation}
	\label{eq:D}
	D(i, j) = \begin{cases}
		0 &\text{i = 0, j = 0}\\
		i &\text{j = 0, i > 0}\\
		j &\text{i = 0, j > 0}\\
		\min \lbrace \\
			\qquad D(i, j-1) + 1\\
			\qquad D(i-1, j) + 1 &\text{i > 0, j > 0}\\
			\qquad D(i-1, j-1) + m(a[i], b[j]) &\text(\ref{eq:m})\\
		\rbrace
	\end{cases},
\end{equation}

а функция \ref{eq:m} определена как:
\begin{equation}
	\label{eq:m}
	m(a, b) = \begin{cases}
		0 &\text{если a = b,}\\
		1 &\text{иначе}
	\end{cases}.
\end{equation}\\

\clearpage
Рекурсивный алгоритм реализует формулу 1.1. Функция \ref{eq:D} составлена из следующих соображений.

\begin{enumerate}
\item Для перевода из пустой строки в пустую требуется ноль операций.
\item Для перевода из пустой строки в строку $a$ требуется $|a|$ операций.
\item Для перевода из строки $a$ в пустую требуется $|a|$ операций.
\item Для перевода из строки $a$ в строку $b$ требуется выполнить последовательно некоторое количество операций (удаление, вставка, замена) в некоторой последовательности. Последовательность проведения любых двух операций можно поменять, порядок проведения операций не имеет никакого значения. Полагая, что $a'$, $b'$ -- строки $a$ и $b$ без последнего символа соответственно, цена преобразования из строки $a$ в строку $b$ может быть выражена как:
	\begin{itemize}
	\item cумма цены преобразования строки $a$ в $b$ и цены проведения операции удаления, которая необходима для преобразования $a'$ в $a$;
	\item cумма цены преобразования строки $a$ в $b$ и цены проведения операции вставки, которая необходима для преобразования $b'$ в $b$;
	\item cумма цены преобразования из $a'$ в $b'$ и операции замены, предполагая, что $a'$ и $b'$ оканчиваются разные символы;
	\item цена преобразования из $a'$ в $v'$ , предполагая, что $a$ и $b$ оканчиваются на один и тот же символ.
	\end{itemize}
\end{enumerate}
Минимальной ценой преобразования будет минимальное значение приведенных вариантов.

\section{Итеративный алгоритм нахождения расстояния Левенштейна с кэшированием}

Прямая реализация формулы \ref{eq:D} может быть малоэффективна по времени исполнения при больших $i$, $j$, т. к. множество промежуточных значений $D(i, j)$ вычисляются заново множество раз подряд. Для оптимизации нахождения расстояния Левенштейна можно использовать матрицу в целях хранения соответствующих промежуточных значений. В таком случае алгоритм представляет собой построчное заполнение матрицы
$A_{|a|,|b|}$ значениями $D(i, j)$.

Однака, матричный алгоритм является малоэффективным по памяти при больших $i$, $j$, т. к. множество промежуточных значений $D(i, j)$ хранится в памяти после их использования. Для оптимизации нахлождения расстояния Левенштейна можно использовать кэш, т.е. пару строк, содержащюю значения $D(i, j)$, вычисленные в предыдущей итерации алгоритма и значения $D(i, j)$, вычисляемые в текущей итерации.

\section{Расстояние Дамерау-Левенштейна}

В алгоритме поиска расстояния Дамерау-Левенштейна, помимо 3 операций (удаление, вставка, замена), присутсвующих в алгоритме поиска расстояния Левенштейна, задействуется еще одна редакторская операция - транспозиция T (от англ. transposition). Расстояние Дамерау-Левенштейна рассчитывается по рекуррентной формуле \ref{eq:d}.

\begin{equation}
	\label{eq:d}
	d_{a,b}(i, j) = \begin{cases}
		\max(i, j), &\text{если }\min(i, j) = 0,\\
		\min \lbrace \\
			\qquad d_{a,b}(i, j-1) + 1,\\
			\qquad d_{a,b}(i-1, j) + 1, &\text{иначе}\\
			\qquad d_{a,b}(i-1, j-1) + m(a[i], b[j]),\\
			\qquad \left[ \begin{array}{cc}d_{a,b}(i-2, j-2) + 1, &\text{если }i,j > 1;\\
			\qquad &\text{}a[i] = b[j-1]; \\
			\qquad &\text{}b[j] = a[i-1]\\
			\qquad \infty, & \text{иначе}\end{array}\right.\\
		\rbrace
		\end{cases},
\end{equation}

\section{Вывод}

В данном разделе были рассмотрены алгоритмы нахождения расстояния Левенштейна и Дамерау-Левенштейна, который является модификаций первого, учитывающего возможность перестановки соседних символов. Формулы Левенштейна и Дамерау -- Левенштейна для рассчета расстояния между строками задаются рекурсивно, а следовательно, алгоритмы могут быть реализованы рекурсивно или итерационно.
