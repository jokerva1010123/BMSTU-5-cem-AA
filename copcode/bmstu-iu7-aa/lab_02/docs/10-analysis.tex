\chapter{Аналитическая часть}

\section{Некоторые теоретические сведения}

Для начала нужно ввести собственно понятие матрицы.

\textit{Матрица} -- объект, записываемый в виде прямоугольной таблицы элементов,
которая представляет собой совокупность строк и столбцов, на пересечении которых находятся её элементы (формула \ref{eq:ref1}).
\begin{equation}
	A = \left(
	\begin{array}{cccc}
			a_{11} & a_{12} & \ldots & a_{1m} \\
			a_{21} & a_{22} & \ldots & a_{2m} \\
			\vdots & \vdots & \ddots & \vdots \\
			a_{n1} & a_{n2} & \ldots & a_{nm}
		\end{array}
	\right)
	\label{eq:ref1}
\end{equation}

\textit{Произведение матриц} $AB$ состоит из всех возможных комбинаций скалярных произведений 
вектор-строк матрицы $A$ и вектор-столбцов матрицы $B$ (рис. \ref{img:AMulB}).

\imgw{AMulB}{ht!}{0.3\textwidth}{Произведение матриц}

Операция умножения двух матриц выполнима только в том случае, если число столбцов в первой матрице равно числу строк во второй.

\section{Стандартный алгоритм умножения матриц}
Пусть даны матрицы $A$ (формула \ref{eq:ref1}) размерностью $n \times m$ и $B$ (формула \ref{eq:ref2}) $m \times q$.
\begin{equation}
	B = \left(
	\begin{array}{cccc}
			b_{11} & b_{12} & \ldots & b_{1q} \\
			b_{21} & b_{22} & \ldots & b_{2q} \\
			\vdots & \vdots & \ddots & \vdots \\
			b_{m1} & b_{m2} & \ldots & b_{mq}
		\end{array}
	\right)
	\label{eq:ref2}
\end{equation}

Матрица $C = AB$ будет размерностью $n \times q$.
Тогда каждый элемент матрицы $C$ выражается формулой (\ref{eq:ref3}).
\begin{equation}
	\begin{array}{cc}
		c_{ij} = \sum\limits_{k=1}^m a_{ik}b_{kj} & (i=1,2,\dots n; j=1,2,\dots q)
	\end{array}
	\label{eq:ref3}
\end{equation}

\section{Умножение матриц по Винограду}

Каждый элемент в матрице $C$, которая является результатом умножения двух матриц, представляет собой скалярное произведение соответствующих строки и столбца исходных матриц. 

В алгоритме умножение матриц по Винограду предложено сделать предварительную обработку, позволяющую часть работы выполнить заранее.

Рассмотрим два вектора $V$ (формула \ref{eq:ref4}) и $W$ (формула \ref{eq:ref5}).
\begin{equation}
	V = (v_1, v_2, v_3, v_4)
	\label{eq:ref4}
\end{equation}
\begin{equation}
	W = (w_1, w_2, w_3, w_4)
	\label{eq:ref5}
\end{equation}

Их скалярное произведение вычисляется по формуле  \ref{eq:ref6}.
\begin{equation}
	V * W = v_1w_1 + v_2w_2 + v_3w_3 + v_4w_4
	\label{eq:ref6}
\end{equation}

Равенство \ref{eq:ref6} можно записать в виде \ref{eq:ref7}.
\begin{equation}
	\begin{array}{l}
		V * W = (v_1 + w_2)(v_2 + w_1) + (v_3 + w_4)(v_4 + w_3) - \\
		\quad \quad \quad v_1v_2 - v_3v_4 - w_1w_2 - w_3w_4
	\end{array}
	\label{eq:ref7}
\end{equation}

Несмотря на то, что второе выражение \ref{eq:ref7} требует вычисления большего количества операций, чем стандартный алгоритм: вместо четырех умножений - шесть, а вместо трех сложений - десять, выражение в правой части последнего равенства допускает предварительную обработку: его части можно вычислить заранее и запомнить для каждой строки первой матрицы и для каждого столбца второй, то для каждого элемента будет необходимо выполнить лишь первые два умножения и последующие пять сложений, а также дополнительно два сложения. Из-за того, что операция сложения быстрее операции умножения, алгоритм должен работать быстрее стандартного


\section{Вывод}
Были рассмотрены основополагающие материалы, которые в дальнейшем потребуются при реализации алгоритмов умножения матриц.  
