\chapter*{Введение}
\addcontentsline{toc}{chapter}{Введение}

В данной лабораторной работе будут рассмотрены конвейерные на примере вычисления хеш-сумм файлов в файловой системе.

Конвейерный принцип обработки \cite{conveyer} - способ организации вычислений, используемый в современных процессорах и контроллерах с целью повыщения их производительности, путем увеличения числа инструкций, выполняемых в единицу времени (эксплуатация параллелизма на уровне инструкций.

Основная идея ковейерной обработки заключается в разделении подлежащей выполнению функции на более мелкие части и выделении для каждой из них отдельного блока аппаратуры. Прирост производительности достигается именно за счет параллельного выполнения частей более сложной функции.

Целью данной работы является реализация и изучение конвейерной обработки.

Для достижения данной цели необходимо решить следующие задачи:

\begin{itemize}
    \item исследовать основы конвейерных вычислений; 
	\item исследовать основные методы организации конвейерных вычислений;
	\item сравнить существующие методы организации конвейерных вычислений;
	\item привести схемы рассматриваемых алгоритмов, а именно:
	\begin{itemize}
	    \item схему конвейера, содержащего 3 ленты;
	    \item схему конвейера, содержащего 3 ленты и реализующего \texttt{fan--in}--\texttt{fan--out} подходы;
	\end{itemize}
	\item описать использующиеся структуры данных;
	\item описать структуру разрабатываемого програмного обеспечения;
	\item определить средства программной реализации;
	\item определить требования к программному обеспечению;
	\item привести сведения о модулях программы;
	\item провести тестирование реализованного программного обеспечения;
	\item провести экспериментальные замеры временных характеристик реализованного конвейера.
\end{itemize}
