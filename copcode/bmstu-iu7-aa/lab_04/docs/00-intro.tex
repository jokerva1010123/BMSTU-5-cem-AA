\chapter*{Введение}
\addcontentsline{toc}{chapter}{Введение}

В данной лабораторной работе будут рассмотрены параллельные вычисления на примере умножения матриц.

Многопоточность -- способность центрального процессора (ЦПУ) или одного ядра в многоядерном процессоре одновременно выполнять несколько процессов или потоков, соответствующим образом поддерживаемых операционной системой.

Этот подход отличается от многопроцессорности, так как многопоточность процессов и потоков совместно использует ресурсы одного или нескольких ядер: вычислительных блоков, кэш-памяти ЦПУ или буфера перевода с преобразованием.

В тех случаях, когда многопроцессорные системы включают в себя несколько полных блоков обработки, многопоточность направлена на максимизацию использования ресурсов одного ядра, используя параллелизм на уровне потоков, а также на уровне инструкций.

Поскольку эти два метода являются взаимодополняющими, их иногда объединяют в системах с несколькими многопоточными ЦП и в ЦП с несколькими многопоточными ядрами.

Многопоточная парадигма стала более популярной с конца 1990-х годов, поскольку усилия по дальнейшему использованию параллелизма на уровне инструкций застопорились.

Целью данной работы является реализация и изучение следующих алгоритмов:
\begin{itemize}
	\item обычное умножение матриц по строкам;
	\item параллельное умножение матриц по строкам;
	\item параллельное умножение матриц по столбцам.
\end{itemize}

Для достижения данной цели необходимо решить следующие задачи:

\begin{itemize}
	\item изучить основные методы параллельных вычислений;
	\item реализовать каждый из указанных алгоритмов умножения матриц;
	\item сравнить временные характеристики реализованных алгоритмов экспериментально.
\end{itemize}
