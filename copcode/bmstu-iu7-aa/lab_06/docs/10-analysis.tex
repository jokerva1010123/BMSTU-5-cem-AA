\chapter{Аналитическая часть}

\section{Постановка задачи}

Задача коммивояжера - одна из самых известных задач комбинаторной оптимизации, заключающаяся в поиске самого выгодного маршрута, проходящего через указанные города хотя бы по одному разу с последующим возвратом в исходный город. 

В условиях задачи указываются критерий выгодности маршрута (кратчайший, самый дешёвый, совокупный критерий и тому подобное) и соответствующие матрицы расстояний, стоимости и тому подобного. 

Как правило, указывается, что маршрут должен проходить через каждый город только один раз -- в таком случае выбор осуществляется среди гамильтоновых циклов.

Существует несколько частных случаев общей постановки задачи, в частности:

\begin{itemize}
\item геометрическая задача коммивояжёра (также называемая планарной или евклидовой, когда матрица расстояний отражает расстояния между точками на плоскости);
\item метрическая задача коммивояжёра (когда на матрице стоимостей выполняется неравенство треугольника), симметричная;
\item асимметричная задачи коммивояжёра.
\end{itemize} 

Также существует обобщение задачи, так называемая обобщённая задача коммивояжёра

\section{Описание алгоритмов}

\textbf{Алгоритм полного перебора}

Чтобы решить задачу алгоритмом полного перебора, нужно вввести соответствующую модель. 

Пусть:
\begin{itemize}
\item существует $n$ городов;
\item все города пронумерованы целыми числами от 1 до $n$;
\item базовый (начальный) город имеет номер $n$.
\end{itemize}

Тогда каждый тур по городам однозначно соответствует перестановке целых чисел \newline 1, 2, \dots, $n$ - 1.

Задачу коммивояжера можно решить образуя все перестановки первых $n-1$ натуральных чисел. Для каждой перестановки строится соответствующий тур и вычисляется его стоимость. Обрабатывая таким образом все перестановки, запоминается тур, который к текущему моменту имеет наименьшую стоимость. Если находится тур с более низкой стоимостью, то дальнейшие сравнения производятся с ним.

Сложность алгоритма полного перебора составляет $O(n!)$ \cite{analyse}.

\textbf{Муравьиный алгоритм}

В реальном мире муравьи (первоначально) ходят в случайном порядке и по нахождении продовольствия возвращаются в свою колонию, прокладывая феромонами тропы. Если другие муравьи находят такие тропы, они, вероятнее всего, пойдут по ним. Вместо того, чтобы отслеживать цепочку, они укрепляют её при возвращении, если в конечном итоге находят источник питания. 

Со временем феромонная тропа начинает испаряться, тем самым уменьшая свою привлекательную силу. Чем больше времени требуется для прохождения пути до цели и обратно, тем сильнее испарится феромонная тропа. 

На коротком пути, для сравнения, прохождение будет более быстрым, и, как следствие, плотность феромонов остаётся высокой. 

Испарение феромонов также имеет функцию избежания стремления к локально-оптимальному решению. Если бы феромоны не испарялись, то путь, выбранный первым, был бы самым привлекательным. В этом случае, исследования пространственных решений были бы ограниченными. 

Таким образом, когда один муравей находит (например, короткий) путь от колонии до источника пищи, другие муравьи, скорее всего пойдут по этому пути, и положительные отзывы в конечном итоге приводят всех муравьёв к одному, кратчайшему, пути.

Пример приведен на Рисунке \ref{img:Aco_branches}.

\imgw{Aco_branches}{ht!}{\textwidth}{Пример работы муравьиного алгоритма}

Применение муравьиного алгоритма к решению задачи коммивояжера имеет некоторые особенности:
\begin{itemize}
\item муравьи имеют собственную «память». Поскольку каждый город может быть посещен только один раз, у каждого муравья есть список уже посещенных городов. Обозначим через $J_{i,k}$ список городов, которые необходимо посетить муравью $k$, находящемуся в городе $i$;

\item муравьи обладают «зрением» -- видимость есть эвристическое желание посетить город $j$, если муравей находится в городе $i$. Будем считать, что видимость обратно пропорциональна расстоянию между городами $i$ и $j$ -- $D_{ij}$

\begin{equation}
	\eta_{ij} = \frac{1}{D_{ij}}
\end{equation}	

\item муравьи обладают «обонянием» – они могут улавливать след феромона, подтверждающий желание посетить город $j$ из города $i$, на основании опыта других муравьев. Количество феромона на ребре $(i, j)$ в момент времени $t$ обозначим через $\tau_{ij} (t)$.
\end{itemize}

Приняв во внимание вышеописанные особенности конкретной реализации алгоритма, можно сформулировать правило \ref{for:probability}, которое определяет вероятность перехода $k$-ого муравья из города $i$ в город $j$:

\begin{equation}
\label{for:probability}
	\begin{cases}
		P_{i,j,k} = { \frac{[\tau_{ij}]^\alpha * [\eta_{ij}]^\beta}{\sum_{l \in J_{j,k}}[\tau_{il}]^\alpha * [\eta_{il}]^\beta }}, & j \in J_{i,k}\\
		P_{i,j,k} = 0, & j \not\in J_{i,k}
	\end{cases}
\end{equation}

где $\alpha, \beta$ -- параметры, задающие веса следа феромона, при $\alpha = 0$ алгоритм вырождается до наивного (жадного) алгоритма. 

Выбор города является вероятностным, правило \ref{for:probability} определяет ширину зоны города $j$; в общую зону всех городов $J_{i,k}$ бросается случайное число, которое и определяет выбор муравья.

Правило \ref{for:probability} не изменяется в ходе алгоритма, но у двух разных муравьев значение вероятности перехода будут отличаться, т. к. они имеют разный список разрешенных городов.

Пройдя ребро $(i, j)$, муравей откладывает на нем некоторое количество феромона, которое должно быть связано с оптимальностью сделанного выбора. Пусть $T_k(t)$ есть маршрут, пройденный муравьем $k$ к моменту времени $t$, а $L_k(T)$ -- длина этого маршрута. Пусть также $Q$ -- параметр, имеющий значение порядка длины оптимального пути. Тогда откладываемое количество феромона может быть задано в виде:

\begin{equation}
\label{for:pheromon}
	\begin{cases}
		P_{i,j,k}(t) = \frac{Q}{L_k(t)}, & (i, j) \in T_k(t)\\
		P_{i,j,k} = 0, & (i, j) \not\in T_k(t)
	\end{cases}
\end{equation}

Правила внешней среды определяют, в первую очередь, испарение феромона. Пусть $\rho \in [0,1]$ есть коэффициент испарения, тогда правило испарения имеет вид:
\begin{equation}
\label{for:pheromon_rho}
	\tau_{ij}(t + 1) = (1 - \rho)*\tau_{ij}(t) + \Delta\tau_{ij}(t), \Delta\tau_{ij}(t) = \sum_{k=1}^m \Delta\tau_{ij,k}(t + 1)
\end{equation}

где $m$ -- количество муравьев в колонии.

В начале алгоритма количество феромона на ребрах принимается равным небольшому положительному числу. Общее количество муравьев остается постоянным и равным количеству городов, каждый муравей начинает маршрут из своего города.

Сложность алгоритма: $O(t_{max} * max(m, n^2))$, где $t_{max}$ -- время жизни колонии, $m$ -- количество муравьев в колонии, $n$ – размер графа \cite{analyse_2}.

\section{Вывод}

Были рассмотрены основополагающие материалы, которые в дальнейшем потребуются при реализации алгоритмов сортировки. 