\chapter*{Заключение}
\addcontentsline{toc}{chapter}{Заключение}

В ходе выполнения лабораторной работы были решены следующие задачи:

\begin{itemize}
    \item изучены алгоритмы нахождения расстояний Левенштейна и Дамерау-Левенштейна;
	\item реализованы алгоритмы поиска расстояния Левенштейна и расстояния Дамерау-Левенштейна без рекурсии;
	\item реализованы рекурсивные алгоритмы поиска расстояния Дамерау-Левенштейна с и без матрицы-кеша;
	\item проведен сравнительный анализ линейной и рекурсивной реализаций алгоритмов определения расстояния между строками по затрачиваемым ресурсам (времени и памяти);
	\item подготовлен отчет о лабораторной работе.
\end{itemize}

Экспериментально было подтверждено различие во временной эффективности рекурсивной и нерекурсивной реализаций выбранного алгоритма определения расстояния между строками при помощи разработанного программного обеспечения на материале замеров процессорного времени выполнения реализаций на различных длин строк.
Было получено, что рекурсивная реализация алгоритмов без кеширования в 3-4 раза проигрывает по памяти нерекурсивной реализации. Однако ее можно улучшить, добавив кеширование, и получить небольшой (1.5 раза) выигрыш по памяти по сравнению с нерекурсивными алгоритмами. Анализ временных затрат показал, что для длинных строк (10 символов и больше) рекурсивная реализация алгоритмов работает в 1.5 раза дольше, чем нерекурсивная.
