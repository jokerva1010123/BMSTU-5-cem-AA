\chapter{Технологическая часть}

В данном разделе будут приведены требования к программному обеспечению, средства реализации и листинги кода.

\section{Требования к ПО}

Программа принимает две строки (регистрозависимые). В качестве результата возвращается число, равное редакторскому расстоянию. Необходимо реализовать возможность подсчета процессорного времени и пиковой использованной памяти для каждого из алго­ритмов.

\section{Средства реализации}

В качестве языка программирования для реализации данной лабораторной работы был выбран ЯП Python \cite{pythonlang}. 

Данный язык имеет все небходимые инструменты для решения поставленной задачи.

Время работы алгоритмов было замерено с помощью функции time() из библиотеки time \cite{pythonlangtime}.

\section{Сведения о модулях программы}
Программа состоит из четырех модулей:
\begin{enumerate}
	\item algorithms.py - хранит реализацию алгоритмов;
	\item unit\_tests.py - хранит реализацию тестирующей системы и тесты;
	\item time\_memory.py - хранит реализацию системы замера памяти и времени;
	\item tools.py - хранит реализацию вспомогательных функций.
\end{enumerate}


\section{Листинг кода}

 В листингах \ref{lst:non_rec_l}, \ref{lst:non_rec_dl}, \ref{lst:rec_dl}, \ref{lst:rec_dl_cache} приведены реализации алгоритмов нахождения расстояния Левенштейна и Дамерау--Левенштейна.

\begin{lstlisting}[label=lst:non_rec_l,caption=Функция нахождения расстояния Левенштейна нерекурсивным методом.]
def non_recursive_levenshtein(source: str, target: str) -> int:
    source_len = len(source)
    target_len = len(target)

    matrix_dist = [[0 for i in range(target_len + 1)] for j in range(source_len + 1)]

    for i in range(source_len + 1):
        for j in range(target_len + 1):
            if i == 0 and j == 0:
                matrix_dist[i][j] = 0
            elif i == 0 and j > 0:
                matrix_dist[i][j] = j
            elif i > 0 and j == 0:
                matrix_dist[i][j] = i
            else:
                matrix_dist[i][j] = min(matrix_dist[i][j - 1] + 1,
                                        matrix_dist[i - 1][j] + 1,
                                        matrix_dist[i - 1][j - 1] + (source[i - 1] != target[j - 1]))
    return matrix_dist[source_len][target_len]
	
\end{lstlisting}

\begin{lstlisting}[label=lst:non_rec_dl,caption=Функция нахождения расстояния Дамерау--Левенштейна нерекурсивным методом.]
def non_recursive_damerau_levenshtein(source: str, target: str) -> int:
    source_len = len(source)
    target_len = len(target)

    matrix_dist = [[0 for i in range(target_len + 1)] for j in range(source_len + 1)]

    for i in range(source_len + 1):
        for j in range(target_len + 1):
            if i == 0 and j == 0:
                matrix_dist[i][j] = 0
            elif i == 0 and j > 0:
                matrix_dist[i][j] = j
            elif i > 0 and j == 0:
                matrix_dist[i][j] = i
            else:
                matrix_dist[i][j] = min(matrix_dist[i][j - 1] + 1,
                                        matrix_dist[i - 1][j] + 1,
                                        matrix_dist[i - 1][j - 1] + (source[i - 1] != target[j - 1]))

                if i > 1 and j > 1 and source[i - 1] == target[j - 2] and source[i - 2] == target[j - 1]:
                    exchange_dist = matrix_dist[i - 2][j - 2] + 1
                    matrix_dist[i][j] = min(matrix_dist[i][j], exchange_dist)

    return matrix_dist[source_len][target_len]
	
\end{lstlisting}

\begin{lstlisting}[label=lst:rec_dl,caption=Функция нахождения расстояния Дамерау--Левенштейна с использованием рекурсии.]
def _rdl_wrap(source: str, target: str, i: int, j: int) -> int:
    answer = -1

    if i == 0 and j == 0:
        answer = 0
    elif i == 0:
        answer = j
    elif j == 0:
        answer = i
    else:

        left = _rdl_wrap(source, target, i - 1, j) + 1
        up = _rdl_wrap(source, target, i, j - 1) + 1
        left_up = _rdl_wrap(source, target, i - 1, j - 1) + (source[i - 1] != target[j - 1])

        answer = min(left, up, left_up)

        if i > 1 and j > 1 and source[i - 1] == target[j - 2] and source[i - 2] == target[j - 1]:
            exchange_dist = _rdl_wrap(source, target, i - 2, j - 2) + 1
            answer = min(answer, exchange_dist)

    return answer


def recursive_damerau_levenshtein(source: str, target: str) -> int:
    source_len = len(source)
    target_len = len(target)

    return _rdl_wrap(source, target, source_len, target_len)
	
\end{lstlisting}

\begin{lstlisting}[label=lst:rec_dl_cache,caption=Функция нахождения расстояния Дамерау--Левенштейна рекурсивным методом с использованием кеша.]
def _rdl_cache_wrap(source: str, target: str, i: int, j: int, cache: list[list[int]]) -> int:
    if cache[j][i] != -1:
        return cache[j][i]

    if i == 0 and j == 0:
        cache[j][i] = 0
        return cache[j][i]
    elif i == 0 and j > 0:
        cache[j][i] = j
        return cache[j][i]
    elif j == 0 and i > 0:
        cache[j][i] = i
        return cache[j][i]
    else:
        left = _rdl_cache_wrap(source, target, i - 1, j, cache) + 1
        up = _rdl_cache_wrap(source, target, i, j - 1, cache) + 1
        left_up = _rdl_cache_wrap(source, target, i - 1, j - 1, cache) + (source[i - 1] != target[j - 1])

        cache[j][i] = min(left, up, left_up)

        if i > 1 and j > 1 and source[i - 1] == target[j - 2] and source[i - 2] == target[j - 1]:
            exchange = _rdl_cache_wrap(source, target, i - 2, j - 2, cache) + 1
            cache[j][i] = min(cache[j][i], exchange)

    return cache[j][i]


def recursive_damerau_levenshtein_with_cache(source: str, target: str) -> int:
    source_len = len(source)
    target_len = len(target)

    cache = [[-1 for i in range(source_len + 1)] for j in range(target_len + 1)]
    return _rdl_cache_wrap(source, target, source_len, target_len, cache)
\end{lstlisting}

\section{Функциональные тесты}
В таблице \ref{tabular:functional_test} приведены функциональные тесты для алгоритмов вычисления расстояния Левенштейна (в таблице столбец подписан "Левенштейн") и Дамерау-Левенштейна (в таблице - "Дамерау-Л."). Все тесты пройдены успешно.


\begin{table}[h]
    \begin{center}
    \begin{threeparttable}
    \captionsetup{justification=raggedright,singlelinecheck=off}	\caption{\label{tabular:functional_test} Функциональные тесты}
		\begin{tabular}{|l|l|l|l|l|}
			\hline
			& & & \multicolumn{2}{c|}{Ожидаемый результат} \\
			\hline
			№&Строка 1&Строка 2&Левенштейн&Дамерау-Л. \\
			\hline
			1&"пустая строка"&"пустая строка"&0&0 \\
			\hline
			2&"пустая строка"&a&1&1 \\
			\hline
			3&b&"пустая строка"&1&1 \\
			\hline
			4&asdfv&"пустая строка"&5&5 \\
			\hline
			5&"пустая строка"&sdfvs&5&5 \\
			\hline
			8&lll&lll&0&0 \\
			\hline
			9&qwem&qwem&0&0 \\
			\hline
			10&aa&cg&2&2 \\
			\hline
			11&kot&sobaka&5&5 \\
			\hline
			12&stroka&sobaka&3&3 \\
			\hline
			13&kot&kod&1&1 \\
			\hline
			14&cat&caaat&2&2 \\
			\hline
			15&cat&catty&2&2 \\
			\hline
			16&cat&tac&2&2 \\
			\hline
			17&recur&norecur&2&2 \\
			\hline
			18&1234&2143&3&2 \\
			\hline
			19&mriak&mriakmriak&5&5 \\
			\hline
			20&aaaaa&aa&3&3 \\
			\hline
		\end{tabular}
	\end{threeparttable}
	\end{center}
\end{table}


\newpage

\section*{Вывод}

Были разработаны и протестированы алгоритмы: нахождения расстояния Левенштейна нерекурсивно, нахождения расстояния Дамерау -- Левенштейна нерекурсивно, рекурсивно, а также рекурсивно с кешированием.
