\chapter{Конструкторская часть}
 В данном разделе будут представлены схемы рассматриваемых алгоритмов и требования к вводу.

\section{Требования к вводу}
На вход подается массив точек и связей между ними. 

\section{Требования к программе}

Программа не должна аварийно завершаться при некорректном вводе. Вывод программы - матрица, в каждом узле которой находится значение цвета, соответствующее текущему состоянию сцены.

\section{Разработка алгоритмов}

На рисунках \ref{img:s3}, \ref{img:s1} и \ref{img:s5} представлены схемы алгоритмов потока конвейера, потока диспетчера, и последовательного алгоритма удаления невидимых граней использующий Z-буфер.

\section{Работа с разделяемой памятью}

Алгоритм работает по принципу постолбцового развертывания изображения. Такая модификация гарантирует монопольный доступ к памяти при параллельном выполнении алгоритма, а также уменьшает количество требуемой для работы алгоритма памяти.

\newpage
\img{220mm}{s3}{Схема алгоритма потока конвейера}

\newpage
\img{200mm}{s1}{Схема алгоритма потока диспетчера}

\newpage
\img{240mm}{s5}{Схема алгоритма линейной программы}

\newpage
\section*{Вывод}
Была разработана схема последовательного алгоритма удаления невидимых граней. На ее основе была построена схема потока конвейера и схема потока диспетчера для многопоточной реализации алгоритма удаления невидимых граней.