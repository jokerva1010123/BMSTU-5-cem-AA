\chapter*{Заключение}
\addcontentsline{toc}{chapter}{Заключение}

В результате исследования было получено, что эвристический метод на базе муравьиного алгоритма следует использовать при большом количество городов - от 8 и более, так как для 2 городов мураьвиный алгоритм работает медленнее полного перебора в 32 раза, а для 10 городов - быстрее в 75 раз.

Лучшие значения муравьиный алгоритм показывает при меньших значениях коэффициента видимости и при большом числе дней.

Цель, поставленная перед началом работы, была достигнута. В ходе лабораторной работы были решены следующие задачи:

\begin{enumerate}
	\item Изучены основы алгоритма полного перебора.
	\item Применены изученные основы алгоритма полного перебора для реализации.
	\item Изучены основы муравьиного алгоритма.
	\item Применены изученные основы муравьиного алгоритма для реализации.
	\item Проведена параметризация муравьиного алгоритма.
	\item Проведен сравнительный анализ времени работы реализованных
	алгоритмов.
	\item Подготовлен отчет о лабораторной работе.
\end{enumerate}

Поставленная цель достигнута.
