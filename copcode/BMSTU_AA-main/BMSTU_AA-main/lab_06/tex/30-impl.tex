\chapter{Технологическая часть}

В данном разделе были приведены средства реализации и листинги кода.

\section{Средства реализации}

В качестве языка программирования для реализации данной лабораторной работы был выбран язык программирования C++ \cite{Cpp}.
Данный язык имеет все небходимые инструменты для решения поставленной задачи.

\section{Сведения о модулях программы}
Программа состоит из четырех модулей:
\begin{enumerate}
        \item main.cpp - главный файл программы, в котором располагается точка входа;
	\item algorithms.cpp - хранит реализацию алгоритмов;
	\item unit\_tests.cpp - хранит реализацию тестирующей системы и тесты;
	\item tools.cpp - хранит реализацию вспомогательных функций.
\end{enumerate}

\section{Реализация алгоритмов}

В листинге \ref{lst:con} представлен алгоритм полного перебора.

\captionsetup{singlelinecheck = false, justification=raggedright}
\begin{lstlisting}[label=lst:con,caption=Реализация алгоритма полного перебора]
sarray get_shortest_path(sarray *cities, int matrix[N][N])
{
    sarray result[MAXRECURSION];
    sarray res_;

    int min_index = 0;
    int min_cost;
    int cur_cost;
    
    int len_routes = 0;
	
    del_element(cities, 0);
    add_element(&res_, 0);

    get_routes(cities, &res_, result, &len_routes);
    min_cost = get_cost(result[min_index], matrix);

    for (int i = 1; i < len_routes; i++)
    {
        cur_cost = get_cost(result[i], matrix);
        if (cur_cost < min_cost)
        {
            min_cost = cur_cost;
            min_index = i;
        }
    }
    return result[min_index];
}
\end{lstlisting}

В листинге \ref{lst:potok} представлен алгоритм нахождения всех перестановок в графе.

\begin{lstlisting}[label=lst:potok,caption=Реализация алгоритма нахождения всех перестановок в графе]
void get_routes(sarray *cities, sarray *res_,
sarray result[MAXRECURSION], int *len)
{
	int element;
	sarray buf;
	
	if (!cities->len)
	{
		add_element(res_, get_element(*res_, 0));
		result[*len] = *res_;
		(*len)++;
		del_element(res_, res_->len - 1);
	}

	for (int i = 0; i < cities->len - 1; i++)
	{
		element = get_element(*cities, i);
		add_element(res_, element);
		buf = copy_array(*cities);
		del_element(&buf, i);
		get_routes(&buf, res_, result, len);
		del_element(res_, res_->len - 1);
	}
}
\end{lstlisting}

В листинге \ref{lst:c1} представлена реализация муравьиного алгоритма.

\begin{lstlisting}[label=lst:c1,caption=Реализация муравьиного алгоритма]
sarray ant_alg(int matrix[N][N], int len, sarray cities, int tmax,
float p, float alpha, float beta)
{
	int q = calc_q(matrix, len);
	sarray best_way = copy_array(cities);
	add_element(&best_way, get_element(best_way, 0));

	int best_cost = get_cost(best_way, matrix);
	int cur_cost = 0;
	float matrix_p[N][N];

	fill_matrix(matrix_p, len, MINPHENOM);
	sant ants[MAXCOUNTOFANT];
	
	for (int t = 0; t < tmax; t++)
	{
		generate_ants_array(ants, len);
		for (int i = 0; i < len - 1; i++)
		{
			ants_choose_way(ants, matrix_p, matrix, len, alpha, beta);
		}
		
		for (int i = 0; i < len; i++)
		{
			add_element(&ants[i].way, get_element(ants[i].way, 0));
		}

		for (int i = 0; i < len; i++)
		{
			cur_cost = get_cost(ants[i].way, matrix);

			if (cur_cost < best_cost)
			{
				best_cost = cur_cost;

				best_way = copy_array(ants[i].way);
			}
		}

		delete_phenom(matrix_p, len, p);
		add_phenom(matrix, matrix_p, len, q, ants);
		phenom_correct(matrix_p, len);
	}
	return best_way;
}
\end{lstlisting}

В листинге \ref{lst:c2} представлена реализация алгоритма выбора оптимального пути для муравья.

\begin{lstlisting}[label=lst:c2,caption=Реализация алгоритма выбора оптимального пути для муравья]
void next_city(sant *ants, float matrix_p[N][N],
int matrix[N][N], float alpha, float beta)
{
	float number = 0;
	float denominator = 0;
	float tao, rev_cost;
	int cost;
	int cur_city = ants->way.array[ants->way.len - 1];

	for (int i = 0; i < ants->quere.len; i++)
	{
		cost = matrix[cur_city][ants->quere.array[i]];
		tao = matrix_p[cur_city][ants->quere.array[i]];

		if (!cost)
			continue;
		
		rev_cost = 1.0 / cost;
		denominator += powf(tao, alpha) + powf(rev_cost, beta);
	}
	
	float p_array[N] = {0};
	float sum = 0;

	for (int i = 0; i < ants->quere.len; i++)
	{
		cost = matrix[cur_city][ants->quere.array[i]];
		tao = matrix_p[cur_city][ants->quere.array[i]];

		rev_cost = 1.0 / cost;
		
		p_array[i] = (powf(tao, alpha) + powf(rev_cost, beta)) / denominator;
	}
	
	float x = (float)rand() / RAND_MAX;
	int index = 0;
	
	while (x >= 0)
	{
		x -= p_array[index];
		index++;
	}
	
	add_element(&ants->way, get_element(ants->quere, index - 1));
	del_element(&ants->quere, index - 1);
}
\end{lstlisting}

В листинге \ref{lst:c3} представлена реализация вспомогательных функций.

\begin{lstlisting}[label=lst:c3,caption=Вспомогательные функции]
void add_phenom(int matrix[N][N], float matrix_p[N][N], int len,
int q, sant ants[MAXCOUNTOFANT])
{
	int fcity, scity;
	int cur_cost;
	float dtao = 0;

	for (int i = 0; i < len; i++)
	{
		cur_cost = get_cost(ants[i].way, matrix);
		dtao += (float)q / cur_cost;
	}
	
	for (int i = 0; i < len; i++)
	{
		for (int j = 0; j < ants[i].way.len - 1; j++)
		{
			fcity = ants[i].way.array[j];
			scity = ants[i].way.array[j + 1];
			
			matrix_p[fcity][scity] = matrix_p[fcity][scity] + dtao;
			matrix_p[scity][fcity] = matrix_p[scity][fcity] + dtao;
		}
	}
}

void delete_phenom(float matrix_p[N][N], int len, float p)
{
	float qq = 1 - p;

	for (int i = 0; i < len; i++)
	{
		for (int j = 0; j < i; j++)
		{
			matrix_p[i][j] = matrix_p[i][j] * qq;
			matrix_p[j][i] = matrix_p[j][i] * qq;
		}
	}
}

void phenom_correct(float matrix_p[N][N], int len)
{
	for (int i = 0; i < len; i++)
	{
		for (int j = 0; j < i; j++)
		{
			if (matrix_p[i][j] <= 0.1)
			{
		    	matrix_p[i][j] = matrix_p[j][i] = 0.1;
			}
		}
	}
}
\end{lstlisting}

\captionsetup{singlelinecheck = false, justification=centering}

\section{Тестирование}
В таблице~\ref{tabular:test_rec} приведены тесты для функции, реализующей алгоритм для решения задачи коммивояжера (алгоритм полного перебора и муравьиный алгоритм). Тесты пройдены успешно.

При тестировании муравьиного алгоритма тесты прошли успешно, так как матрицы смежности маленькие, и поэтому муравьиный алгоритм работает без видимых ошибок, то есть выдает такие же результаты как и алгоритм полного перебора.

Ошибки в результатах муравьиного алгоритма чаще всего возникают при больших размерах матриц (начиная от 9х9). Погрешности бывают чаще всего как разность между наименьшим путем и предпоследним (тот который мог бы быть наименьшим).

\begin{table}[h!]
	\begin{center}
		\captionsetup{justification=raggedright,singlelinecheck=off}
		\caption{\label{tabular:test_rec} Функциональные тесты}
		\begin{tabular}{|c@{\hspace{1mm}}|c@{\hspace{1mm}}|c@{\hspace{1mm}}|c@{\hspace{1mm}}|c@{\hspace{1mm}}|c@{\hspace{1mm}}|}
			\hline
			Матрица смежности & Ожидаемый результат & Действительный результат \\ \hline
			$\begin{pmatrix}
			0 &  1 &  10 &  7\\
			1 &  0 &  1 &  2\\
			10 &  1 &  0 &  1\\
			7 &  2 &  1 &  0
			\end{pmatrix}$ &
			10, [0, 1, 2, 3, 0]&
			10, [0, 1, 2, 3, 0]\\ \hline 			
			$\begin{pmatrix}
			0 &  3 &  5 &  7\\
			7 &  0 &  1 &  2\\
			5 &  1 &  0 &  1\\
			7 &  2 &  1 &  0
			\end{pmatrix}$ &
			11, [2, 3, 1, 0, 2]&
			11, [2, 3, 1, 0, 2]\\ \hline
			$\begin{pmatrix}
			0 &  3 &  5\\
			3 &  0 &  1\\
			5 &  1 &  0
			\end{pmatrix}$ &
			9, [0, 1, 2, 0]&
			9, [0, 1, 2, 0]\\ \hline
		\end{tabular}
	\end{center}
\end{table}
\newpage

\section*{Вывод}
В данном разделе были представлены реализации алгоритма полного перебора и муравьиного алгоритма. Алгоритмы были протестированы. 
