\chapter{Аналитическая часть}
В этом разделе были представлены описания алгоритма стандартного умножения матриц, алгоритма Винограда и оптимизированного алгоритма Винограда.

\section{Матрица}

\textbf{Матрица} \cite{matrix} - математический объект, который представляет собой двумерный массив, в котором элементы располагаются по строкам и столбцам.

Две матрицы одинакового размера можно поэлементно сложить или вычесть друг из друга. Если число столбцов в первой матрице совпадает с числом строк во второй, то эти две матрицы можно перемножить. У произведения будет столько же строк, сколько в первой матрице, и столько же столбцов, сколько во второй. 

Умножение матриц некоммутативно: оба произведения АВ и ВА двух квадратных матриц одинакового размера можно вычислить, однако результаты, вообще говоря, будут отличаться друг от друга.

\section{Стандартный алгоритм}

Пусть даны две прямоугольные матрицы (\ref{aaa}).
\begin{equation}
        \label{aaa}
	A_{lm} = \begin{pmatrix}
		a_{11} & a_{12} & \ldots & a_{1m}\\
		a_{21} & a_{22} & \ldots & a_{2m}\\
		\vdots & \vdots & \ddots & \vdots\\
		a_{l1} & a_{l2} & \ldots & a_{lm}
	\end{pmatrix},
	\quad
	B_{mn} = \begin{pmatrix}
		b_{11} & b_{12} & \ldots & b_{1n}\\
		b_{21} & b_{22} & \ldots & b_{2n}\\
		\vdots & \vdots & \ddots & \vdots\\
		b_{m1} & b_{m2} & \ldots & b_{mn}
	\end{pmatrix},
\end{equation}

тогда матрица $C$ (\ref{bbb}).
\begin{equation}
        \label{bbb}
	C_{ln} = \begin{pmatrix}
		c_{11} & c_{12} & \ldots & c_{1n}\\
		c_{21} & c_{22} & \ldots & c_{2n}\\
		\vdots & \vdots & \ddots & \vdots\\
		c_{l1} & c_{l2} & \ldots & c_{ln}
	\end{pmatrix},
\end{equation}

где
\begin{equation}
	\label{eq:M}
	c_{ij} =
	\sum_{r=1}^{m} a_{ir}b_{rj} \quad (i=\overline{1,l}; j=\overline{1,n})
\end{equation}

будет называться произведением матриц $A$ и $B$.

Стандартный алгоритм реализует данную формулу (\ref{eq:M}).

\section{Алгоритм Винограда}

\textbf{Алгоритм Винограда} \cite{vinogr} — алгоритм умножения квадратных матриц. Начальная версия, предложенная в 1987 году Д. Копперсмитом и Ш. Виноградом, имела асимптотическую сложность примерно $O(n^{2,3755})$, где $n$ - размер стороны матрицы, но после доработки алгоритм стал обладать лучшей асимптотикой среди всех алгоритмов умножения матриц \cite{Cohn}.

\bigskip

Рассмотрим два вектора $V = (v_1, v_2, v_3, v_4)$ и $W = (w_1, w_2, w_3, w_4)$.
Их скалярное произведение равно: $V \cdot W = v_1w_1 + v_2w_2 + v_3w_3 + v_4w_4$, что эквивалентно:
\begin{equation}
	\label{for:new}
	V \cdot W = (v_1 + w_2)(v_2 + w_1) + (v_3 + w_4)(v_4 + w_3) - v_1v_2 - v_3v_4 - w_1w_2 - w_3w_4.
\end{equation}

Несмотря на то, что второе выражение требует вычисления большего количества операций, чем стандартный алгоритм: вместо четырех умножений - шесть, а вместо трех сложений - десять, выражение в правой части последнего равенства допускает предварительную обработку: его части можно вычислить заранее и запомнить для каждой строки первой матрицы и для каждого столбца второй, что позволит для каждого элемента выполнять лишь два умножения и пять сложений, складывая затем только лишь с 2 предварительно посчитанными суммами соседних элементов текущих строк и столбцов.
Из-за того, что операция сложения быстрее операции умножения в ЭВМ, на практике алгоритм должен работать быстрее стандартного.

В случае нечетного значений размера изначальной матрицы ($n$), следует произвести еще одну операцию - добавление произведения последних элементов соответствующих строк и столбцов.

\section{Оптимизированный алгоритм Винограда}

Оптимизированный алгоритм Винограда представляет собой обычный алгоритм Винограда, за исключением следующих оптимизаций:
\begin{itemize}
	\item вычисление происходит заранее;
	\item используется битовый сдвиг, вместо деления на 2;
	\item последний цикл для нечётных элементов включён в основной цикл,
	используя дополнительные операции в случае нечётности N.
\end{itemize}


\section*{Вывод}
Были рассмотрены алгоритм классического умножения матриц, алгоритм Винограда и оптимизированный алгоритм Винограда. Было выяснено, что основные отличия алгоритма Винограда от классического алгоритма — наличие предварительной обработки и количество операций умножения.


