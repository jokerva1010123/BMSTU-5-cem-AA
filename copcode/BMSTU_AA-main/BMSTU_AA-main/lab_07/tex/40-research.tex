\chapter{Исследовательская часть}

В данном разделе приведена постановка эксперимента.

\section{Исследование}

В данной работе, в качестве респондентов, принимали участия следующие студенты.
\begin{itemize}
	\item[---] Малышев И.Н.
	\item[---] Николаев С.С.
	\item[---] Фролова В.А.
	\item[---] Шабанова А.В.
	\item[---] Щербина М.А. 
\end{itemize}

Объектом данного исследования является продолжительность жизни человека. Исследуемый признак --- количество лет, прожитых человеком. 

Возможные значения признака (описывают количество):
\begin{itemize}
	\item[---] мало;
	\item[---] средне;
	\item[---] много;
	\item[---] очень много;
	\item[---] не очень много; 
	\item[---] не очень мало.
\end{itemize}

\newpage
По результатам опроса была сформирована и занесена в таблицу обобщённая статистика. Упомянутая таблица представлена на рисунке \ref{img:f1}.

\img{48mm}{f1}{Обобщённые результаты опроса.}

В данной таблице приведены количество голосов, отданных респондентами в пользу истинности разных утверждений. В узлах таблицы расположено количество голосов в пользу высказывания, формируемого по принципу  «количество лет это тезис» (Например, за истинность высказывания «50 лет это много» проголосовал один человек).

На рисунке \ref{img:f2} приведена таблица, содержащая нормализованные значения из таблицы \ref{img:f1}.

\img{50mm}{f2}{Нормализованные значения}

Построенные функции принадлежности термам числовых значений признака, описываемого лингвистической переменной, на основе статистической обработки мнений респондентов, выступающих в роли экспертов, приведены на рисунке \ref{img:f3}
 
\newpage
\img{120mm}{f3}{Функции принадлежности}

\section*{Вывод}

Таким образом, на основании экспертной оценки, были точечно заданы функции принадлежности числового значения указанного признака некоторому термину. Благодаря данным функциям можно с некоторой точностью определить описанную принадлежность.

