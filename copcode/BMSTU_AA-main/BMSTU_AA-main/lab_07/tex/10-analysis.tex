\chapter{Аналитическая часть}
В данном разделе были рассмотрены словарь как структура данных и алгоритм полного перебора по словарю.

\section{Словарь}

Обычные списки (массивы) представляют собой набор пронумерованных элементов. Номер элемента в списке однозначно идентифицирует сам элемент. Но идентифицировать данные по числовым номерам не всегда удобно. Например, маршруты поездов в России идентифицируются численно-буквенным кодом (число и одна цифра), также численно-буквенным кодом идентифицируются авиарейсы. Для хранения информации о рейсах поездов или самолетов в качестве идентификатора удобнее было бы использовать не число, а текстовую строку.

Структура данных, позволяющая идентифицировать ее элементы не по числовому индексу, а по произвольному, называется словарем или ассоциативным массивом \cite{dict}. Данная структура хранит пары вида \guillemotleftключ-значение\guillemotright. При поиске возвращается значение, которое ассоциируется с данным ключом, или \guillemotleftПара не найдена\guillemotright, если по данному ключу нет значений.

\section{Полный перебор}

Полный перебор \cite{search-full} --- метод решения, при котором поочередно перебираются все ключи словаря, пока не будет найден нужный.

Трудоёмкость алгоритма зависит от того, присутствует ли искомый ключ в словаре, и, если присутствует -- насколько он далеко от начала массива ключей.

\newpage

Так, если в начале алгоритм затрагивает $b$ операций, а при сравнении $k$ операций, то:
\begin{itemize}
	\item[---] элемент найден на первом сравнении за $b + k$ операций (лучший случай);
	\item[---] элемент найден на $i$-ом сравнении за $b + i \cdot k$ операций;
	\item[---] элемент найден на последнем сравнении за $b +  N \cdot k$ операций, где $N$ -- размер словаря (худший случай);
\end{itemize}

При этом средняя трудоемкость равна:

\begin{equation}
	f = b + k \cdot \left(1 + \frac{N}{2} - \frac{1}{N + 1}\right)
\end{equation}

\section*{Вывод}

Была изучена структура данных --- словарь. Был рассмотрен подход к поиску в словаре --- алгоритм полного перебора.

