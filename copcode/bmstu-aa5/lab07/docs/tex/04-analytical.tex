\chapter{Аналитическая часть}

В данном разделе представлено теоретическое описание словаря и алгоритмов
поиска.

\section{Словарь}

Словарь построен на основе пар (ключ, значение). Для данного типа данных
определено три операции:
\begin{itemize}
    \item вставка;
    \item удаление;
    \item поиск.
\end{itemize}

При поиске по заданному ключу возвращается значение или "сообщение"{}, по
которому можно понять, что в словаре нет пары c данным ключем.

В данной работе используется словарь настольных игр, где ключом является
имя игры, а значением информация о ней: год выпуска, количество отзывов и
рейтинг.

Далее будут рассмотрены алгоритмы поиска в словаре.

\section{Алгоритм полного перебора}

Алгоритм полного перебора \cite{McConnel} подразумевает поочередный просмотр
всех возможных вариантов. В случае словаря по очереди просматривают ключи
словаря до тех пор, пока не будет найден нужный. Трудоёмкость алгоритма зависит
от того, присутствует ли искомый ключ в словаре, и, если присутствует ---
насколько он далеко от начала массива ключей.

Пусть на старте алгоритм поиска затрачивает $k_0$ операций, а при каждом
сравнении $k_1$ операций. Тогда при поиске первого элемента (лучший случай)
будет затрачено $k_0 + k_1$ операций, $i$-ого --- $k_0 +
i \cdot k_1$, последнего (худший случай) --- $k_0 + N \cdot k_1$. Ситуация
отсутствия ключа обнаруживается только послед перебора всех значений, что 
соответствует трудоёмкости поиска ключа на последней позиции.
Средняя трудоёмкость равна математическому ожиданию и может быть
рассчитана по формуле \ref{eq:brute}:

\begin{equation}\label{eq:brute}
    f_{\text{ср}}= k_0 + k_1 \cdot (1 + \frac{N}{2} - \frac{1}{N + 1})
\end{equation}

\section{Алгоритм бинарного поиска}

Бинарный поиск \cite{McConnel} осуществляется в отсортированном списке ключей.
Искомый ключ сравнивается со средним элементом, если ключ меньше, то поиск
продолжается в левой части, если больше, то --- в правой части, если равен, то
искомый ключ найден.

Таким образом при бинарном поиске \cite{McConnel} обход можно представить
деревом, поэтому трудоёмкость в худшем случае составит $\log_2 N$ (в худшем
случае нужно спуститься по двоичному дереву от корня до листа). Скорость роста
функции $\log_2 N$ меньше, чем скорость линейной функции, полученной для
полного перебора.

\section{Алгоритм частотного анализа}

Алгоритм частотного анализа разбивает словарь на сегменты по какому-либо
признаку. В данной работе рассматривается случай определения ключа в сегмент
по первому символу (определены сегменты, соответствующие буквам латинского
алфавита, цифрам, а также сегмент включающий ключи, которые не попали в
другие сегменты).

Сегменты упорядочиваются по значению частотной характеристики так, чтобы к
элементам с наибольшей частотной характеристикой был самый быстрый доступ.
В данной работе такой характеристикой служит размер сегмента.

Обращение к сегменту происходит с вероятностью равной сумме вероятностей
обращений к его ключам, рассчитывающейся по формуле \ref{eq:prob}:

\begin{equation}\label{eq:prob}
    P_i = \sum_{j}p_j = N \cdot p,
\end{equation}

где $P_i$ - вероятность обращения к $i$-ому сегменту, $p_j$ - вероятность
обращения к $j$-ому элементу, который принадлежит $i$-ому сегменту. Если
обращения ко всем ключам равновероятны, то можно заменить сумму на
произведение, где $N$ - количество элементов в $i$-ом сегменте, а $p$ -
вероятность обращения к произвольному ключу.

В каждой сегменте ключи упорядочиваются по значению. Это необходимо для
реализации бинарного поиска, который обеспечит эффективный поиск со сложностью
$O(\log_2 m)$ (где $m$ - количество ключей в сегменте) внутри сегмента.
На этом предварительная обработка словаря заканчивается.

При поиске ключа сначала выбирается нужный сегмент, а затем в нем проводится
бинарный поиск нужного элемента. Средняя трудоёмкость при множестве всех
возможных случаев $\Omega$ может быть рассчитана по формуле (\ref{for:anal}).
\begin{equation}\label{for:anal}
    \sum_{i \in \Omega}{\left(f_{\text{выбор сегмента i-ого элемента}} +
    f_{\text{бинарный поиск i-ого элемента}}\right)} \cdot p_i
\end{equation}

\section{Вывод}

В данном разделе был описан словарь, так же алгоритмы поиска значений по ключу:
перебор, бинарный поиск и частотный анализ. Из представленных описаний можно
предъявить ряд требований к разрабатываемому программному обеспечению:
\begin{itemize}[left=\parindent]
    \item на вход должен подавться словарь, также ключ, поиск значения которого
        осуществляется;
    \item при отсутствии ключа в словаре должно выдаваться соответствующее
          сообщение;
    \item на выходе должно выдаваться значение, соответствующее данному ключу.
\end{itemize}
