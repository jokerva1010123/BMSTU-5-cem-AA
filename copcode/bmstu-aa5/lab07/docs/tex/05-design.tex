\chapter{Конструкторская часть}

В данном разделе разрабатываются алгоритмы поиска в словаре: перебором,
бинарный и частотный анализ, также описывается структура программы и способы её
тестироваия.

\section{Разработка алгоритмов}

На рисунке \ref{scheme:brute} представлена схема алгоритма перебора ключей.

На рисунке \ref{scheme:bin} представлена схема алгоритма бинарного поиска.

На рисунке \ref{scheme:seg} представлена схема алгоритма частотного анализа.

\noindent
\scheme{110mm}{brute}{Схема алгоритма поиска полным перебором}{brute}
\noindent
\scheme{150mm}{bin}{Схема алгоритма бинарного поиска}{bin}
\noindent
\scheme{110mm}{seg}{Схема алгоритма частотного анализа}{seg}

\clearpage
\section{Структура разрабатываемого ПО}

Для реализации взаимодействия с пользователем будет использован метод
структурного программирования. Обработка каждого пункта меню будет представлена
отдельной функцией, при необходимости будут выделены подпрограммы для каждой из
них. Будут реализованы функции для ввода-вывода и функция, вызывающая все
подпрограммы для связности и полноценности программы. Также будет реализован
класс данных словарь, который будет содержать методы, соответсвующие
разработанным алгоритмам.

\section{Классы эквивалентности при тестировании}

Для тестирования программного обеспечения во множестве тестов будут выделены
следующие классы эквивалентности:
\begin{itemize}[left=\parindent]
    \item ключа нет в словаре;
    \item ключ в словаре;
    \item ключ число, соответсвующее названию игры.
\end{itemize}

\section{Вывод}

В данном разделе были разработаны алгоритмы поиска в словаре: перебором,
бинарный и частотный анализ. Для дальнейшей проверки правильности работы
программы были выделены классы эквивалентности тестов.
