\chapter{Исследовательская часть}

\section{Технические характеристики}

Технические характеристики устройства, на котором выполнялось тестирование:

\begin{itemize}
	\item Операционная система: Manjaro \cite{manjaro} Linux x86\_64.
	\item Память: 8 GiB.
    \item Процессор: Intel® Core™ i5-8265U, 4 физических ядра, 8 логических
        ядра\cite{intel}.
\end{itemize}

Тестирование проводилось на ноутбуке, включенном в сеть электропитания. Во
время тестирования ноутбук был нагружен только встроенными приложениями
окружения, окружением, а также непосредственно системой тестирования.

\section{Примеры работы программы}

На рисунке \ref{img:find} представлены результаты работы программы при
существующем ключе, на рисунке \ref{img:notfind} -- при отстутствующем ключе. 

\img{90mm}{find}{Пример работы программы при существующем ключе}{find}

\img{60mm}{notfind}{Пример работы программы при отстутствующем ключе}{notfind}

\section{Результаты тестирования}

Программа была протестирована на входных данных, приведенных в таблице
\ref{tab:tests}. Полученные результаты работы программы совпали с ожидаемыми
результатами.

\section[Постановка эксперимента по замеру времени]
        {Постановка ~~эксперимента ~~по ~~замеру времени}

Для оценки времени работы алгоритмов поиска: перебором, бинарного и частотного
анализа; -- был проведен эксперимент, в котором определялось влияние положения
включа в словаре на время работы алгоритма и количество сравнений. Тестирование
проводилось на всех ключах, имеющихся в словаре. Так как от запуска к запуску
процессорное время, затрачиваемое на выполнение алгоритмов, менялось в
определенном промежутке, необходиом было усреднить вычисляемые значения. Для
этого каждый алгоритм на каждом ключе запускался по 10 раз, и для полученных 10
значений определялось среднее арифметическое, которое заносилось в таблицу, на
основе которой строился график. Количество сравнений находилось аналогично без
усреднений.

Результаты эксперимента были представлены в виде графиков, приведенных
в следующем подразделе.

\section{Результаты эксперимента}

На рисунке \ref{img:time1} приведен график зависимости времени работы
алгоритмов от положения ключа в словаре. На рисунке \ref{img:time2} для
наглядости приведен график только бинарного поиска и частотного анализа.

На рисунках \ref{img:bf}-\ref{img:seg} приведены диаграммы количества сравнений
для каждого ключа. На каждом рисунке приведены диаграммы отсортированные по
индексам ключей и по значению количества сравнений.

\img{72mm}{time1}{График зависимости времени работы от индекса ключа}{time1}

\img{75mm}{time2}{График зависимости времени работы от индекса ключа (бинарный
поиск и частотный анализ)}{time2}

\img{67mm}{bf}{Диаграмма количества сравнений по индексам ключей для поиска
перебором}{bf}

\img{67mm}{bin}{Диаграмма количества сравнений по индексам ключей для бинарного
поиска}{bin}

\img{67mm}{seg}{Диаграмма количества сравнений по индексам ключей для
частотного анализа}{seg}

\clearpage
\section{Вывод}

По результатам эксперимента можно сделать следующие выводы:
\begin{itemize}[]
    \item время работы поиска перебором линейно возрастает в зависимости от
        индекса ключа (чем дальше ключ от начала словаря, тем больше время);
    \item алгоритмы бинарного поиска и частотного анализа при больших объемах
        данных справляются со своей задачей за константное время;
    \item при количестве ключей до 50 алгоритм линейного поиска работает
        быстрее двух других алгоритмов в среднем в 2 раза;
    \item при больших значениях линейный поиск работает медленнее двух других,
        соотношение в скорости зависит от положения ключа в словаре;
   \item реализация частотного анализа с бинарным поиском по сегментам работает
       в 1.7 раза быстрее бинарного поиска;
   \item количество сравнений при линейном поиске увеличивается пропорционально
        индексу ключа, максимально ключ будет найден за $N$ сравнений, где $N$
        -- количество ключей в словаре, минимально ключ будет найден за 1
        сравнение, в среднем потребует $\frac{N + 1}{2}$ сравнений;
   \item количество сравнений при бинарным поиском зависит от значения ключа, а
        именно от положения ключа в отсортированном массиве ключей, максимально
        ключ будет найден за 29 сравнений, минимально -- за 7, в среднем -- за
        27 сравнений;
   \item количество сравнений при поиском с помощью частотного анализа
       количество сравнений также зависит от значения ключа, однако в среднем
        количество сравнений уменьшается за счет распределения по сегментам;
        максимально ключ будет найден за 40 сравнений, минимально -- за 7, в
        среднем -- за 25 сравнений.
\end{itemize}

Таким образом, при обработке больших данных необходимо использовать или
бинарный поиск, или метод частотного анализа. Выбор между двумя данными
алгоритмами делается за счет возможности затрат времени на предобработку
данных. Бинарный поиск требует сортировки данных в словаре по ключу, в то время
как метод частнотного анализа требует распределения ключей по сегментам и
внутри каждого сегмента дополнительно соритровку. Линейный поиск следует
использовать при объемах данных до 50 ключей, так как в таком случае он
работает быстрее двух других алгоритмов и не затрачивает время на
предварительную обработку данных.
