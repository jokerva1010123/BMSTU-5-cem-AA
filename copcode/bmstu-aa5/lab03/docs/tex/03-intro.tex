\chapter*{Введение}
\addcontentsline{toc}{chapter}{Введение}

Сортировка -- процесс перестановки объектов данного множества в определенном
порядке. Сортировка служит для последующего облегчения поиска элементов в
отсортированном множестве. При этом соритровка является примером огромного
разообразия алгоритмов, выполняющих одну и ту же задачу. Несмотря на большое
количество алгоритмов сортировки, любой из них можно разбить на три основные
части:
\begin{itemize}[left=\parindent]
    \item сравнение, определяющее порядок элементов в паре;
    \item перестановка, меняющая элементы в паре местами;
    \item сам сортирующий алгоритм, который выполняет два предыдущих шага до
          полного упорядочивания.
\end{itemize}

Ввиду большого количества алгоритмов сортировки одни из них имеют преимущества
над другими, что приводит к необходимости их сравнительного анализа
\cite{intro}.

\textbf{Целью данной работы} является получение навыков анализа алгоритмов
сортировок.

Для достижения поставленной цели необходимо выполнить следующие
\textbf{задачи}:
\begin{itemize}[left=\parindent]
    \item изучить три алгоритма сортировки: вставками, перемешиванием, выбором;
    \item разработать каждый из алгоритмов;
    \item дать теоретическую оценку трудоемкости алгоритмов сортировки;
    \item реализовать каждый алгоритм;
    \item провести тестирование реализованных алгоритмов;
    \item провести сравнительный анализ алгоритмов по процессорному времени
          работы реализации;
\end{itemize}
