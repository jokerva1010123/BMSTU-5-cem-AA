\chapter{Технологическая часть}

В данном разделе описаны требования к программному обеспечению, средства
реализации, приведены листинги кода и данные, на которых будет проводиться
тестирование.

\section{Требования к программному обеспечению}

Программа должна предоставлять следующие возможности:
\begin{itemize}[left=\parindent]
    \item ввода количества обрабатываемых заявок;
    \item вывода лога работы программы для введенного количества заявок,
        включающих для каждой заявки время начала и конца обработки на каждой
        из лент;
    \item получения времени обработки всех заявок.
\end{itemize}

\section{Средства реализации}

Для реализации данной лабораторной работы выбран компилируемый многопоточный
язык програмирования \texttt{Go}\cite{golang}, так как он предоставляет
необходимый функционал для работы с потоками и простой реализациии их
взаимодействия. Интерпретируемый язык программирования высокого уровня
Python\cite{python} был выбран для визуализации данных эксперимента, так как он
предоставляет большое число настроек параметров графика с использованием
простого синтаксиса. 

В качестве среды разработки выбран текстовый редактор Vim\cite{vim} c
установленными плагинами автодополнения и поиска ошибок в процессе написания,
так как он реализует быстрое перемещение по тексту программы и простое
взаимодействие с командной строкой.

Замеры времени проводились при помощи функции \texttt{Now()}
из библиотеки \texttt{time}\cite{time}.

\newpage
\section{Листинги кода}

В данном подразделе представлены листинги кода алгоритмов:
\begin{itemize}[]
    \item реализация генерации сообщений (листинг \ref{lst:gen});
    \item реализация перестановки букв в каждом слове в обратном
          порядке (листинг \ref{lst:rev});
    \item реализация шифра Веженера (листинг \ref{lst:Vegenere});
    \item последовательная реализация конвейера (листинг \ref{lst:LinConv});
    \item параллельная реализация конвейера (листинг \ref{lst:ParConv}).
\end{itemize}

\mylisting{Функция генерации сообщений}{gen}{33-36}{conveyor/encription.go}

\mylisting{Функции перестановки букв в каждом слове в обратном
порядке}{rev}{38-54}{conveyor/encription.go}

\begin{mdlisting}
    \captionsetup{justification=raggedright,singlelinecheck=off}
    \lstinputlisting[label=lst:Vegenere, caption=Функция шифра Веженера,
    linerange=56-76]{../../src/conveyor/encription.go}
\end{mdlisting}

\mylisting{Последовательная обработка
заявок}{LinConv}{7-29}{conveyor/conveyor.go}
\begin{mdlisting}
    \captionsetup{justification=raggedright,singlelinecheck=off}
    \lstinputlisting[label=lst:ParConv, caption=Конвейерная обработка заявок,
    linerange=31-94]{../../src/conveyor/conveyor.go}
\end{mdlisting}

\section{Описание тестирования}

В таблице \ref{tab:tests} приведены функциональные тесты программы.

\begin{table}[h!]
	\begin{center}
    \begin{threeparttable}
        \captionsetup{justification=raggedright,singlelinecheck=off}
        \caption{\label{tab:tests}Функциональные тесты}
        \begin{tabular}{|c|c|}
			\hline
            \textbf{Количество заявок} & \textbf{Ожидаемый результат} \\ [2mm]
            \hline
            -1
            &
            Сообщение об ошибке
            \\
            \hline
            0
            &
            Сообщение об ошибке
            \\
            \hline
            j
            &
            Сообщение об ошибке
            \\
            \hline
            10
            &
            Лог работы каждого из алгоритмов
            \\
            \hline
            100
            &
            Лог работы каждого из алгоритмов
            \\
            \hline
            1000
            &
            Лог работы каждого из алгоритмов
            \\
            \hline
		\end{tabular}
    \end{threeparttable} 
	\end{center}
\end{table}

\section{Вывод}

В данном разделе были реализованы последовательный и конвейрный алгоритмы
поэтапной генерации зашифрованных сообщений. Также были написаны
тесты для каждого класса эквивалентности, описанного в конструкторском разделе.
