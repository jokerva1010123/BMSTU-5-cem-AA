\chapter*{Введение}
\addcontentsline{toc}{chapter}{Введение}

Сегодня программирование используется во многих научных и социальных областях.
Компьютерам требуется производить все более трудоемкие вычисления на больших
объемах данных. При этом предъявляются требования к скорости вычислений.

При обработке больших объемов данных часто возникает ситуация, когда каждых
набор данных необходимо обработать последовательно несколькими алгоритмами. Так
как каждый набор может быть обработан независимого от другого имеется
возможность расспараллелить вычисления. В таком случае релизуют конвейерную
обработку данных, когда каждый алгоритм выполняется на отдельной "ленте"{},
запущенной в отдельном потоке, а результат каждого этапа используется в
качестве входных данных следующего.

\textbf{Целью данной работы} является получение навыков
конвейерной обработки данных.

Для достижения поставленной цели необходимо выполнить следующие
\textbf{задачи}:
\begin{itemize}[left=\parindent]
    \item описать конвейерную обработку данных;
    \item описать алгоритмы, реализуемые на каждом этапе конвейера;
    \item описать функциональные требования;
    \item разработать описанные алгоритмы;
    \item реализовать алгоритмы всех этапов;
    \item реализовать конвейерную обработку данных;
    \item реализовать линейную обработку данных;
    \item провести тестирование реализованных алгоритмов;
    \item провести сравнительный анализ алгоритмов по времени работы
          реализаций;
    \item сделать выводы по полученным результатам.
\end{itemize}
