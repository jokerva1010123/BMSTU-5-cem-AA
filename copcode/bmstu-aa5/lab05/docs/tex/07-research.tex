\chapter{Исследовательская часть}

\section{Технические характеристики}

Технические характеристики устройства, на котором выполнялось тестирование:

\begin{itemize}
	\item Операционная система: Manjaro \cite{manjaro} Linux x86\_64.
	\item Память: 8 GiB.
    \item Процессор: Intel® Core™ i5-8265U, 4 физических ядра, 8 логических
        ядра\cite{intel}.
\end{itemize}

Тестирование проводилось на ноутбуке, включенном в сеть электропитания. Во
время тестирования ноутбук был нагружен только встроенными приложениями
окружения, окружением, а также непосредственно системой тестирования.

\section{Примеры работы программы}

На рисунке \ref{img:logL} представлены результаты работы последовательной
обработки, на рисунке \ref{img:logP} -- конвейерной.

\img{110mm}{logL}{Лог последовательной обработки}{logL}

\img{110mm}{logP}{Лог конвейерной обработки}{logP}

\section{Результаты тестирования}

Программа была протестирована на входных данных, приведенных в таблице
\ref{tab:tests}. Полученные результаты работы программы совпали с ожидаемыми
результатами.

\section[Постановка эксперимента по замеру времени]
        {Постановка ~~эксперимента ~~по ~~замеру времени}

Для оценки времени работы последовательной и конвейерной реализации алгоритмов
шифрования был проведен эксперимент, в котором определялось влияние количества
заявок и количества слов в сообщении на время работы каждого из алгоритмов.
Тестирование проводилось на количестве заявок от 5 до 10 с шагом 5, от 25 до
100 с шагом 25 и от 100 до 1000 с шагом 250, а количество слов принимало
значения $5, 10, 25, 50, 75, 100$. Время работы на каждом из значений было
получено с помощью \textit{бенчмарков}\cite{gotest}, являющимися встроенными
средствами языка \texttt{Go}. В них количество повторов измерений времени
изменяется динамически до тех пор, пока не будет получен стабильный результат.

Результаты эксперимента были представлены в виде таблиц и графиков, приведенных
в следующем подразделе.

\section{Результаты эксперимента}

В таблице \ref{tab:byReq} представлены результаты измерения времени работы
последовательной и конвейерной реализаций в зависимости от числа заявок.
На рисунке \ref{img:byReq} представлен соответствующий график.

В таблице \ref{tab:byWords} представлены результаты измерения времени работы
последовательной и конвейерной реализаций в зависимости от числа слов в
сообщениях при фиксированном числе заявок, равном 50.  На рисунке
\ref{img:byWords} представлен соответствующий график.

\begin{table}[h]
    \begin{center}
    \begin{threeparttable}
        \captionsetup{format=hang,justification=raggedright,
                      singlelinecheck=off}
        \caption{\label{tab:byReq}Время работы от числа заявок}
        \begin{tabular}{|r|r|r|}
            \hline
            \bfseries Число заявок & \bfseries Последовательная, нс
            & \bfseries Конвейерная, нс
            \csvreader{../data/csv/byReq.csv}{}
            {\\\hline \csvcoli&\csvcolii&\csvcoliii}
            \\\hline
        \end{tabular}
    \end{threeparttable}
    \end{center}
\end{table} 

\img{80mm}{byReq}{График зависимости времени работы от числа заявок}{byReq}

\begin{table}[h]
    \begin{center}
    \begin{threeparttable}
        \captionsetup{format=hang,justification=raggedright,
                      singlelinecheck=off}
        \caption{\label{tab:byWords}Время работы от количества слов}
        \begin{tabular}{|r|r|r|}
            \hline
            \bfseries Точность & \bfseries Последовательная, нс
            & \bfseries Параллельная, нс
            \csvreader{../data/csv/byWords.csv}{}
            {\\\hline \csvcoli&\csvcolii&\csvcoliii}
            \\\hline
        \end{tabular}
    \end{threeparttable}
    \end{center}
\end{table} 

\img{80mm}{byWords}{График зависимости времени работы реализаций от
количества слов в сообщениях}{byWords}

\clearpage
\section{Вывод}

По результатам эксперимента можно сделать следующие выводы:
\begin{itemize}[left=\parindent]
    \item при количестве заявок до 10 последовательная и конвейерная
         реализация генерации зашифрованных сообщений отрабатывают за
         одинаковое время, а при количестве заявок до 5 последовательная
         реализация работает $1.3$ раза быстрее, что объясняется затратами
         на передачу данных между лентами с помощью очередей/каналов
         в конвейерной реализации;
    \item при большем количестве заявок от 25 конвейерная реализация работает в
        $1.7$ раза быстрее последовательной;
    \item при фиксированном количестве заявок при различных количествах слов в
        сообщениях конвейерная реализация работает в $1.7$ раза быстрее
        последовательной.
\end{itemize}

Таким образом, для обработки количества заявок большего 10 для достижения
оптимальной скорости вычислений необходимо использовать конвейерную реализацию.
Если работа просходит с количеством заявок до 10 достаточно последовательной
реализации, то есть нет необходимости реализовывать более сложный конвейерный
алгоритм.
