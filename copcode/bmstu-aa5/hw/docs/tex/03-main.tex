\chapter{Выполнение задания}

В этом разделе описаны средства реализации, а также приведен программный код и
графовые представления.

\section{Средства реализации}

Для реализации данной лабораторной работы выбран интерпретируемый язык
программирования высокого уровня Python\cite{python}, так как он позволяет
реализовывать сложные задачи за кратчайшие сроки за счет простоты синтаксиса и
наличия большого количества подключаемых библиотек. 

В качестве среды разработки выбран текстовый редактор Vim\cite{vim} c
установленными плагинами автодополнения и поиска ошибок в процессе написания,
так как он реализует быстрое перемещение по тексту программы и простое
взаимодействие с командной строкой.

\clearpage
\section{Программный код}

На листинге \ref{lst:dijkstra} приведен программный код алгоритма Дейкстры.

\mylisting{Алгоритм Дейкстры поиска минимальных расстояний}
          {dijkstra}{4-30}{dijkstra.py}

\clearpage
\section{Графовые представления}

На рисунках \ref{scheme:opGraph}-\ref{scheme:infoHist3} представлены
графовые представления программы алгоритма Дейкстры.

\noindent
\scheme{50mm}{opGraph}{Операционный граф}{opGraph}
\noindent
\scheme{70mm}{infoGraph}{Информационный граф}{infoGraph}
\noindent
\scheme{185mm}{opHist}{Операционная история}{opHist}
\noindent
\scheme{130mm}{infoHistP1}{Информационная история (часть 1)}{infoHist1}
\noindent
\scheme{115mm}{infoHistP2}{Информационная история (часть 2)}{infoHist2}
\noindent
\scheme{115mm}{infoHistP3}{Информационная история (часть 3)}{infoHist3}
