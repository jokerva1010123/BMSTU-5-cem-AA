\chapter{Исследовательская часть}

\section{Технические характеристики}

Технические характеристики устройства, на котором выполнялось тестирование:

\begin{itemize}
	\item Операционная система: Manjaro \cite{manjaro} Linux x86\_64.
	\item Память: 8 GiB.
    \item Процессор: Intel® Core™ i5-8265U, 4 физических ядра, 8 логических
        ядра\cite{intel}.
\end{itemize}

Тестирование проводилось на ноутбуке, включенном в сеть электропитания. Во
время тестирования ноутбук был нагружен только встроенными приложениями
окружения, окружением, а также непосредственно системой тестирования.

\section{Примеры работы программы}

На рисунке \ref{img:exp} представлены результаты работы программы.

\img{100mm}{exp}{Пример работы программы}{exp}

\section{Результаты тестирования}

Программа была протестирована на входных данных, приведенных в таблице
\ref{tab:tests}. Полученные результаты работы программы совпали с ожидаемыми
результатами.

\section[Постановка эксперимента по замеру времени]
        {Постановка ~~эксперимента ~~по ~~замеру времени}

Для оценки времени работы алгоритма полного перебора и муравьиного алгоритма
был проведен эксперимент, в котором определялось влияние количества вершин в
графе на время работы каждого из алгоритмов. Тестирование проводилось на
количестве вершин от 2 до 10 с шагом 1. Так как от запуска к запуску
процессорное время, затрачиваемое на выполнение алгоритмов, менялось в
определенном промежутке, необходимо было усреднить вычисляемые значения. Для
этого каждый алгоритм на каждом значении запускался по 10 раз, и для полученных
10 значений определялось среднее арифметическе, которое заносилось в таблицу.

Также был проведен эксперимент для определения оптимальных параметров
муравьиного алгоритма. Для этого алгоритм запускался на всех сочетаниях
значений параметров, где коэффициенты $\alpha$ и $p$ изменились в промежутке от
0 до 1 с шагом 0.1, а количество итераций принимало занчения 1, 5, 10, 50, 100,
500, 100; и полученный результат сравнивался с точным результатом, полученным
с помощью алгоритма полного перебора. Параметризация проводилась на двух
различных классах данных с малым (класс данных 1, формула \ref{eq:low}) и
большим (класс данных 2, формула \ref{eq:high}) разбросом значений стоимостей
перехода из одного города в другой. Размер матриц, на которых проводилась
параметризация был выбран равный 9, как наибольший размер, для которого
алгоритм полного перебора решает задачу за малое время.

\begin{equation}
    \label{eq:low}
	K_{1} = \begin{pmatrix}
        0  &  4  & 6 &  4 &  9  &  1  &  8  & 5 &  8\\
        2  &  0  & 7 &  6 &  9  &  9  &  6  & 5 &  2\\
        1  &  3  & 0 &  2 &  2  &  8  &  10 & 3 &  2\\
        9  &  7  & 8 &  0 &  10 &  6  &  1  & 1 &  4\\
        3  &  10 & 4 &  3 &  0  &  7  &  10 & 8 &  1\\
        8  &  8  & 7 &  4 &  1  &  0  &  2  & 1 &  3\\
        9  &  10 & 6 &  2 &  3  &  2  &  0  & 4 &  9\\
        10 &  7  & 4 &  7 &  3  &  7  &  3  & 0 &  2\\
        6  &  4  & 3 &  7 &  4  &  10 &  3  & 5 &  0\\
            \end{pmatrix}
\end{equation}

\begin{equation}
    \label{eq:high}
	K_{2} = \begin{pmatrix}
        0    & 1932 & 1753 & 666  & 2036 & 282  & 1055 & 1564 & 1427\\
        1608 & 0    & 314  & 897  & 396  & 509  & 141  & 1000 & 673 \\
        1657 & 2136 & 0    & 2331 & 1431 & 7    & 2203 & 713  & 1804\\
        2291 & 544  & 400  & 0    & 2171 & 1590 & 189  & 1074 & 1886\\
        1335 & 1590 & 1184 & 640  & 0    & 1266 & 296  & 738  & 698 \\
        116  & 93   & 1577 & 788  & 2439 & 0    & 1231 & 2139 & 1665\\
        315  & 1152 & 1870 & 2491 & 842  & 986  & 0    & 1235 & 13  \\
        986  & 2155 & 2397 & 1765 & 871  & 694  & 1811 & 0    & 1167\\
        784  & 15   & 1487 & 81   & 740  & 372  & 1279 & 2453 & 0   \\
	        \end{pmatrix}
\end{equation}


Результаты эксперимента были представлены в виде таблиц и графиков, приведенных
в следующем подразделе.

\section{Результаты эксперимента}

В таблице \ref{tab:times} представлены результаты измерения времени работы
алгоритмов решения задачи коммивояжера от количества вершин в графе.
На рисунке \ref{img:graph} представлен соответствующий график.

\begin{table}[h]
    \begin{center}
    \begin{threeparttable}
        \captionsetup{format=hang,justification=raggedright,
                      singlelinecheck=off}
        \caption{\label{tab:times}Время работы от количества вершин}
        \begin{tabular}{|r|r|r|}
            \hline
            \bfseries Количество врешин & \bfseries Перебор, нс
            & \bfseries Муравьиный, нс
            \csvreader{../data/csv/times.csv}{}
            {\\\hline \csvcoli&\csvcolii&\csvcoliii}
            \\\hline
        \end{tabular}
    \end{threeparttable}
    \end{center}
\end{table} 

\img{80mm}{graph}{График зависимости времени работы от числа заявок}{graph}

В таблицах \ref{tab:low}, \ref{tab:high} представлена часть результатов
параметризации муравьиного алгоритма на классе данных 1 с малым разбросом
значений и классе данных 2 с большим разбросом значений соответственно. Полный
таблицы результатов представлены в приложениях \hyperlink{apA}{А} и
\hyperlink{apB}{Б} соответственно.

\noindent
\captionsetup{format=hang,justification=raggedright,
              singlelinecheck=off,width=8.4cm}
\begin{longtable}[c]{|r|r|r|r|}
    \caption[(продолжение)]{\label{tab:low}Параметризация для класса
                            данных 2}
    \\\hline
    $\alpha$ & $p$ & Число итераций & Ошибка \\
    \hline
    \endfirsthead
    \captionsetup{labelsep=none}
    \caption[]{ (продолжение)}\\
    \hline
    $\alpha$ & $p$ & Число итераций & Ошибка \\
    \endhead
    \csvreader[
        late after line=\\\hline,
        late after last line=,
        before reading={\catcode`\#=12},
        after reading={\catcode`\#=6}
    ]
    {../data/csv/lowPart.csv}{1=\colo,2=\coltw,3=\colt,4=\colf}
    {\colo & \coltw & \colt & \colf}
    \\\hline
\end{longtable}

\noindent
\captionsetup{format=hang,justification=raggedright,
              singlelinecheck=off,width=8.4cm}
\begin{longtable}[c]{|r|r|r|r|}
    \caption[(продолжение)]{\label{tab:high}Параметризация для класса
                            данных 2}
    \\\hline
    $\alpha$ & $p$ & Число итераций & Ошибка \\
    \hline
    \endfirsthead
    \captionsetup{labelsep=none}
    \caption[]{ (продолжение)}\\
    \hline
    $\alpha$ & $p$ & Число итераций & Ошибка \\
    \endhead
    \csvreader[
        late after line=\\\hline,
        late after last line=,
        before reading={\catcode`\#=12},
        after reading={\catcode`\#=6}
    ]
    {../data/csv/highPart.csv}{1=\colo,2=\coltw,3=\colt,4=\colf}
    {\colo & \coltw & \colt & \colf}
    \\\hline
\end{longtable}

\section{Вывод}

По результатам эксперимента можно сделать следующие выводы:
\begin{itemize}[left=\parindent]
    \item при количестве вершин в графе до 8 алгоритм полного перебора работает
        быстрее муравьного алгоритма, однако если на 3 вершинах он преобладает
        над муравьиным в 700 раз, то уже на 5 он преобладает в 150 раз, а на 7
        -- в 8, что говорит о быстром росте сложности алгоритма полного
        перебора; 
    \item при количестве вершин в графе более 8 время выполнения алгоритма
        полного перебора резко возрастает, в то время как у муравьиного
        алгоритма при переходе на следующее количество вершин время возрастает
        не более чем в 1.5 раза;
    \item муравьиный алгоритм дает лучшие результаты как при малом, так при
        большой разбросе на значения коэффициента $\alpha$ до 0.5, при этом
        строгой зависимости точности алгоритма от коэффициентов $\alpha$ и $p$
        не наблюдается;
    \item при малых разбросах значений стоимостей лучшие результаты работы
        алгоритма налюдались на парах значений: (0.0, 0.1), (0.0, 0.9), (0.1,
        0.4), (0.1, 0.7), (0.1, 0.8), (0.2, 0.5), (0.3,0.0) и (0.5, 0.5); --
        при этом рост количества дней в полной мере устранял ошибки только при
        значениях (0.5, 0.5), при других значениях находилось такое количество
        дней, при котором находился субоптимальный маршрут с большей
        стоимостью;
    \item при больших разбросах значений стоимостей лучшие результаты работы
        алгоритма налюдались на парах значений: (0.2, 0.0), (0.3, 0.0), (0.4,
        0.1), (0.4, 0.5), (0.5, 0.5), (0.5, 0.6), (0.5, 0.8) и (0.6, 0.6); --
        при этом рост количества дней в полной мере устранял ошибки только при
        значениях (0.5, 0.5) и близким к ним (0.5, 0.6), при других значениях
        находилось такое количество дней, при котором находился субоптимальный
        маршрут с большей стоимостью;
\end{itemize}

Таким образом, скорость роста сложности алгоритма перебора в разы выше скорости
роста сложности муравьиного алгоритма, но при малых значениях количества вершин
в графе следует использоавать именно метод полного перебора, так как он в любом
случае дает точный результат и при малых значениях имеет преимущество в
скорости перед муравьиным алгоритмам. В то же время использовать алгоритм
полного пребора при значениях количества вершин в графе больше 8 не
представляется возможным из-за неопределенно долгого ожидания завершения работы
алгоритма, на таких значениях следует использовать муравьиный алгоритм.

При использовании муравьиного алгоритма в случае, если разброс данных не
известен, следует выбирать оптимальную по результатам эксперимента пару
значений ($\alpha$, $p$) равную (0.5, 0.5) при этом количество итераций выбирая
от 50 и выше. При известном разбросе параметров, можно найти пару параметров,
при которой алгоритм будет давать точный результат за меньшее число итераций.
Поиск такого значения требует дополнительного исследования, которое в данной
лабораторной работе не проводилось, был сделан вывод только о том, что
значения параметра $\alpha$ следует выбирать в промежутке $[0.0,0.5]$.
