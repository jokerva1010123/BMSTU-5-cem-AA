\chapter{Технологическая часть}

В данном разделе описаны требования к программному обеспечению, средства
реализации, приведены листинги кода и данные, на которых будет проводиться
тестирование.

\section{Требования к программному обеспечению}

Программа должна предоставлять следующие возможности:
\begin{itemize}[left=\parindent]
    \item ввод имени файла с матрицей смежности графа;
    \item ввод коэффициентов для муравьиного алгоритма;
    \item вывод искомого пути и минимальной стоимости;
    \item подсчет времени работы алгоритмов на различных значениях количества
        вершин;
    \item параметризация муравьиного алоритма с оценкой точности при заданных
        параметрах.
\end{itemize}

\section{Средства реализации}

Для реализации данной лабораторной работы выбран интерпретируемый язык
программирования высокого уровня Python\cite{python}, так как он позволяет
реализовывать сложные задачи за кратчайшие сроки за счет простоты синтаксиса и
наличия большого количества подключаемых библиотек.

В качестве среды разработки выбран текстовый редактор Vim\cite{vim} c
установленными плагинами автодополнения и поиска ошибок в процессе написания,
так как он реализует быстрое перемещение по тексту программы и простое
взаимодействие с командной строкой.

Замеры времени проводились при помощи функции process\_time\_ns из библиотеки
time\cite{time}.

\newpage
\section{Листинги кода}

В данном подразделе представлены листинги кода алгоритмов:
\begin{itemize}[]
    \item реализация алгоритма полного перебора (листинги
        \ref{lst:bfsm}-\ref{lst:bfgtc});
    \item реализация муравьиного алгоритма (листинги
        \ref{lst:asm}-\ref{lst:aup}).
\end{itemize}

\mylisting{Релизация алгоритма полного
пербора}{bfsm}{12-28}{brute_force.py}

\mylisting{Функция поиска стоимости пути}{bfgtc}{3-9}{brute_force.py}

\mybreaklisting{Релизация муравьиного алгоритма}{asm}{96-119}{ants.py}

\mylisting{Функция поиска параметра Q}{afco}{7-11}{ants.py}

\mylisting{Функция инициализации видимости и феромонов}{agec}{14-25}{ants.py}

\mybreaklisting{Функция получения тура и его стоимости для одного
муравья}{agan}{64-76}{ants.py}

\mylisting{Функция выбора муравьем следующего города}{aсс}{50-61}{ants.py}

\mylisting{Получение вероятностей перехода муравья в каждый
город}{ap}{28-47}{ants.py}

\mybreaklisting{Функция обновления феромона}{aup}{79-93}{ants.py}

\section{Описание тестирования}

В таблице \ref{tab:tests} приведены функциональные тесты программы.

\begin{table}[h!]
	\begin{center}
    \begin{threeparttable}
        \captionsetup{justification=raggedright,singlelinecheck=off}
        \caption{\label{tab:tests}Функциональные тесты}
        \begin{tabular}{|p{5cm}|p{6cm}|}
			\hline
            \textbf{Входные данные} & \textbf{Ожидаемый результат} \\ [2mm]
            \hline
            \textit{пустая строка}
            &
            Сообщение об ошибке
            \\
            \hline
            ./data/g5.txt\newline
            j
            &
            Сообщение об ошибке
            \\
            \hline
            ./data/g5.txt\newline
            0.5\newline
            j
            &
            Сообщение об ошибке
            \\
            \hline
            ./data/g5.txt\newline
            0.5\newline
            0.5\newline
            j
            &
            Сообщение об ошибке
            \\
            \hline
            ./data/g5.txt\newline
            0.5\newline
            0.5\newline
            100
            &
            0 -> 2 -> 1 -> 4 -> 3 -> 0\newline
            19.0
            \\
            \hline
		\end{tabular}
    \end{threeparttable} 
	\end{center}
\end{table}

\section{Вывод}

В данном разделе были реализованы алгоритмы решения задачи коммивояжера: полный
перебор и муравьиный алгоритм. Также были написаны тесты для каждого класса
эквивалентности, описанного в конструкторском разделе.
