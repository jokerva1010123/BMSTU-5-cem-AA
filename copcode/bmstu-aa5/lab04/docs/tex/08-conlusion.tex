\chapter*{Заключение}
\addcontentsline{toc}{chapter}{Заключение}

В ходе исследования был проведен сравнительный аналализ последовательной
и параллельной реализации алгорима численного интегрирования методом
средних прямоугольников. В результате исследования было выяснено, что
при распараллеливании можно добиться улучшения скорости вычислений 
в 3 раза, но при условии использования оптимального числа потоков, равного
числу логических ядер процессора, и вычисления интегралов в точностью от 5
знаков после запятой.

В ходе выполения лабораторной работы:

\begin{itemize}[left=\parindent]
    \item были описаны и разработаны последовательный и параллельный алгоритм
          численного интегрирования методом средних прямоугольников;
    \item был реализован каждый из описанных алгоритмов;
    \item по экспериментальным данным были сделаны выводы об эффективности по
          времени каждого из реализованных алгоритмов;
    \item были получены зависимости времени работы паралелльного алгоритма
          от числа потоков и времени работы каждого из алгритмов от
          точности вычислений.
\end{itemize}

Таким образом, все поставленные задачи были выполнены, а цель достигнута.
