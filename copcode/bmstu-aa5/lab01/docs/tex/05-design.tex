\chapter{Конструкторская часть}

В данном разделе разрабатываются алгоритмы, а также производится сравнительный
анализ по используемой памяти рекурсивной и нерекурсивной их реализации на
примере алгоритма поиска расстояния Левенштейна.

\section{Разработка алгоритмов}

В данном подразделе приводятся схемы разработанных алгоритмов поиска расстояния
Левенштейна и Дамерау-Левенштейна.
оло
\subsection[Схемы алгоритмов поиска расстояния Левенштейна]
           {Схемы ~~~~алгоритмов ~~~~поиска ~~~~расстояния Левенштейна}

Схема рекурсивного алгоритма поиска расстояния Левенштейна представлена
на рисунке \ref{scheme:recLiv}.

В двух следующих алгоритмах используется матрица для сохранения результатов
вычислений, так как в обоих матрица инициализируется одним и тем же способом
на рисунке \ref{scheme:matrInit} представен алгоритм инициализации матрицы.

Матричный алгоритм и рекурсивный алгоритм с кэшем для поиска расстояния
Левенштейна приведены на рисунках \ref{scheme:matrLev} и
\ref{scheme:cacheLev}-\ref{scheme:recCacheLev} соответственно.

\subsection[Схема алгоритма поиска расстояния Дамерау-Левенштейна]
           {Схема ~~~~алгоритма ~~~~поиска ~~~~расстояния Дамерау-Левенштейна}

Схема рекурсивного алгоритма поиска расстояния Дамерау-Левенштейна с кэшем
представлена на рисунке \ref{scheme:recCacheDamLiv}.

\noindent
\scheme{170mm}{recursiveLevenstein}{Схема рекурсивного алгоритма
            поиска расстояния Левенштейна}{recLiv}
\noindent
\scheme{135mm}{matrixInit}{Схема алгоритма инициализации матрицы}{matrInit}
\noindent
\scheme{183mm}{matrixLevenstein}{Схема матричного алгоритма поиска расстояния
Левенштейна}{matrLev}
\noindent
\scheme{55mm}{cacheLevenstein}{Схема рекурсивного алгоритма поиска расстояния
Левенштейна с кэшем}{cacheLev}
\noindent \scheme{172mm}{recursiveCacheLevenstein}{Схема рекурсивной
подпрограммы алгоритма поиска расстояния Левенштейна с кэшем}{recCacheLev}
\noindent
\scheme{210mm}{recursiveCacheDamerauLevenstein}{Схема рекурсивного алгоритма
               поиска расстояния Дамерау-Левенштейна с кэшем}{recCacheDamLiv}

\clearpage

\section[\nohyphens{Сравнительный ~анализ ~рекурсивной ~и нерекурсивной реализации
         алгоритмов по используемой памяти}]
         {Сравнительный ~анализ ~рекурсивной ~и нерекурсивной реализации
         алгоритмов по используемой памяти}

Схемы алгоритмов поиска расстояния Левенштейна и Дамерау"=Левенштейна
практически индентичны за исключением нескольких блоков, включающихся при
поиска расстояния Дамерау-Левенштейна из-за дополнительной операции
транспозиции. Это говорит о том, что алгоритмы релизованные одним и тем же
способом не будут отличаться друг от друга с точки зрения используемой памяти.
Поэтому проведем анализ рекурсивной и нерекурсивной реализаций алгоритмов на
примере поиска расстояния Левенштейна.

Проанализируем рекурсивную реализацию. При каждом вызове будет происходить
выделения памяти под:
\begin{itemize}
    \item 2 строки (string);
    \item длины строк (int);
    \item возвращаемое значение (int);
    \item адрес возврата (int);
    \item локальную перменную (int).
\end{itemize}

Таким образом, на каждом вызове используется память, выраженная формулой
\ref{eq:call}:

\begin{equation}\label{eq:call}
    M_{call} = 5 \cdot Size(int) + 2 \cdot Size(string)
\end{equation}

Количество вызовов в пике рекурсии соответствует сумме длин двух строк, то есть
глубина рекурсии вычисляется по формуле \ref{eq:depth}:

\begin{equation}\label{eq:depth}
    depth = |S_1| + |S_2|
\end{equation}

То есть память, используемая рекурсивной реализацией, выражается формулой
\ref{eq:rec}:
\begin{equation}\label{eq:rec}
    M_{rec} = M_{call} \cdot depth
\end{equation}

Теперь найдем память используемую нерекурсивной реализацией. Функция вызывается
единожды, происходит выделение памяти под:
\begin{itemize}
    \item 2 строки (string);
    \item длины строк (int);
    \item возвращаемое значение (int);
    \item адрес возврата (int);
    \item 5 локальных перменных (int);
    \item матрицу размерами, равными длинам строк, увеличенным на 1.
\end{itemize}

То есть память, используемая нерекурсивной реализацией, выражается формулой
\ref{eq:notrec}:
\begin{equation}\label{eq:notrec}
    M_{notrec} = (9 + (|S_1| + 1)(|S_2| + 1)) \cdot Size(int)
                 + 2 \cdot Size(string)
\end{equation}

Таким образом, память, используемая рекурсивной реализацией, растет как сумма
длин строк, в то же время память, используемая нерекурсивной реализацией,
растет как произведение длин строк, то есть по расходу памяти итеративная
реализация проигрывает рекурсивной. При рекурсивной реализации с кэшем при
каждом вызове функции, как минимум, должна выделяться память под ссылку на
матрицу, под которую также необходимо выделить память. То есть при рекурсивной
реализации с кэшем выделяемая память будет также расти, как произведение длин
строк.

\section{Структура разрабатываемого ПО}

Для реализации разрабатываемого программного обеспечения будет использоваться
метод структурного программирования. Каждый из алгоритмов будет представлен
отдельной функцией, при наличии части инициализации матрицы, она также будет
вынесена в отдельную функцию. Также будут реализованы функции для ввода-вывода
и функция, вызывающая все подпрограммы для связности и полноценности
программы.

\section{Классы эквивалентности при тестировании}

Для тестирования программного обеспечения во множестве тестов будут выделены
следующие классы эквивалентности (каждый класс делится на два для проверки
работы алгоритмов на строках в русской и английской раскладках):
\begin{itemize}[left=\parindent]
    \item две пустые строки;
    \item одна пустая, одна нет;
    \item одинаковое количество разных символов в строках;
    \item разное количество разных символов в строках;
    \item наличие транспозиции символов в строках;
    \item удаление, добавление символов в конец;
    \item удаление, добавление символов внутри строк;
    \item замена символов;
    \item разный регистр символов.
\end{itemize}

\section{Вывод}

В данном разделе были разработаны и представлены в виде схем алгоритмы поиска
расстояний Левенштейна и Дамерау-Левенштейна в рекурсивной и итеративной форме.
Также был проведен анализ используемой памяти каждой формой, в результате
которого был сделан вывод о том, что итеративные алгоритмы проигрывают
рекурсивным по используемой памяти, а использование кэша приравнивает
рекурсивную реализации к итеративной по используемой памяти.
