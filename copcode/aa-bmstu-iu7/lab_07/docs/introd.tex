\chapter*{Введение}
\addcontentsline{toc}{chapter}{Введение}

В процессе развития компьютерных систем количество данных стало достигать огромных размеров, поэтому множество операций стали выполняться очень долго, поскольку чаще всего это был обычный перебор. Это вызвало необходимость создать новые алгоритмы, которые решают поставленную задачу на порядок быстрее стандартного решения “в лоб”. В том числе это касается и словарей, в которых одной из основных операций является операция поиска.


\textbf{Целью данной работы} является изучение алгоритмов поиска в словаре -- полным перебором, бинарным поиском и сегментами.
Для достижения поставленной цели необходимо выполнить следующие задачи:
\begin{itemize}
	\item изучить понятие словаря;
    \item описать алгоритмы решения задачи поиска в словаре -- полный перебор, бинарный поиск и сегментами;
    \item привести схемы алгоритмов;
    \item описать используемые типы и структуры данных;
    \item описать структуру разрабатываемого программного обеспечения;
    \item реализовать разработанные алгоритмы;
    \item провести функциональное тестирование разработанного алгоритма;
    \item провести сравнительный анализ по времени для реализованного алгоритма;
    \item подготовить отчет по лабораторной работе.
\end{itemize}
