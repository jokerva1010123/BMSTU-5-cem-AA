\chapter{Технологическая часть}

В данном разделе будут рассмотрены средства реализации, а также представлены листинги алгоритма поиска полным перебором, бинарного алгоритма поиска и алгоритма поиска с использованием сегментации.

\section{Средства реализации}
В данной работе для реализации был выбран язык программирования $Python \cite{python-lang}$. В текущей лабораторной работе требуется замерить процессорное время для выполняемой программы, а также построить графики. Все эти инструменты присутствуют в выбранном языке программирования.

Время работы было замерено с помощью функции \textit{process\_time(...)} из библиотеки $time \cite{python-lang-time}$.


\section{Листинги кода}

В листинге \ref{lst:full_search} представлен алгоритм поиска полного перебора в словаре, а в листинге \ref{lst:bin_search} -- алгоритм бинарного поиска. Также в листинге \ref{lst:make_segm} представлен алгоритм разбиения словаря на сегменты по общему признаку -- 
ключи, начинающиеся на одну букву, а в листинге \ref{lst:segm_search} -- поиск в словаре сегментами.

\clearpage

\begin{center}
    \captionsetup{justification=raggedright,singlelinecheck=off}
    \begin{lstlisting}[label=lst:full_search,caption=Алгоритм поиска в словаре полным перебором]
def full_search(data_dict, key, output = True):
	count = 0 # count of comparsions

	keys = data_dict.keys()

	for elem in keys:
		count += 1

		if (elem == key):
			if (output):
				record = [key, data_dict[key]]
				print("\n\nSearch result:\n")
						
				print_record(record)
				
			return count

	return -1
\end{lstlisting}
\end{center}

\clearpage

\begin{center}
    \captionsetup{justification=raggedright,singlelinecheck=off}
    \begin{lstlisting}[label=lst:bin_search,caption=Алгоритм бинарного поиска]
def binary_search(sorted_dict, key, output = True):
	count = 0 # count of comparsions

	keys = list(sorted_dict.keys())

	left = 0 
	right = len(keys)

	while (left <= right):
		count += 1
		middle = (left + right) // 2
		elem = keys[middle]

		if (elem == key):
			if (output):
				record = [key, sorted_dict[key]]
				print("\n\nSearch result:\n")     
				print_record(record)
				
			return count

		if (elem < key):
			left = middle + 1
		else:
			right = middle - 1

	return -1
\end{lstlisting}
\end{center}


\clearpage


\begin{center}
    \captionsetup{justification=raggedright,singlelinecheck=off}
    \begin{lstlisting}[label=lst:make_segm,caption=Алгоритм разбиения словаря на сегменты, объединенные общим признаком (начинаются на одну букву)]
def make_segments(self):
	temp_dict = {i: 0 for i in "ABCDEFGHIJKLMNOPQRSTUVWXYZ"}

	for key in self.data_dict:
		temp_dict[key[FIRST_LETTER]] += 1

	temp_dict = self.sort_value(temp_dict)

	segmented_dict = {i: dict() for i in temp_dict}

	for key in self.data_dict:
		segmented_dict[key[0]].update({key: self.data_dict[key]})

	return segmented_dict
\end{lstlisting}
\end{center}


\clearpage


\begin{center}
    \captionsetup{justification=raggedright,singlelinecheck=off}
    \begin{lstlisting}[label=lst:segm_search,caption=Алгоритм поиска в словаре с использованием сегментации]
def segment_search(segmented_dict, key, output = True):
	count = 0

	keys = list(segmented_dict.keys())

	for key_letter in keys:
		count += 1

		if (key[FIRST_LETTER] == key_letter):
			count_search = 0

			for elem in segmented_dict[key_letter]:
				count_search += 1

				if (elem == key):
					if (output):
						record = [key, segmented_dict[key_letter][key]]
						print("\n\nSearch result:\n")     
						print_record(record)

					return count_search + count

			return -1

	return -1
\end{lstlisting}
\end{center}


\section{Сведения о модулях программы}
Программа состоит из двух модулей:
\begin{itemize}
	\item \textit{main.py} - файл, содержащий меню программы, а также весь служебный код;
    \item \textit{dictionary.py} - файл, содержащий реализацию класса \textit{Dictionary}, в котором содержатся методы для работы со словарем, а также алгоритмы поиска -- полным перебором, бинарный, сегментами.
\end{itemize}


\section{Функциональные тесты}

В таблице \ref{tbl:functional_test} приведены тесты для функций программы. Тесты \textit{для всех функций} пройдены успешно.

\begin{center}
    \captionsetup{justification=raggedright,singlelinecheck=off}
    \begin{longtable}[c]{|c|c|c|c|c|}
    \caption{Функциональные тесты\label{tbl:functional_test}} \\ \hline
		\textbf{Ключ} & \textbf{Результат} & \textbf{Пояснение} \\ \hline
		Noname & Нет такого ключа & Несуществующий ключ \\ \hline
		Messi & Информация и кол-во сравнений & Первый ключ \\ \hline
		Bonifazi & Информация и кол-во сравнений & Крайний ключ \\ \hline
		Golovin & Информация и кол-во сравнений & Ключ в словаре \\
		\hline
	\end{longtable}
\end{center}

\section{Вывод}

Были представлены листинги всех алгоритмов -- полного перебора, бинарного поиска, сегментного поиска. Также в данном разделе была приведена информации о выбранных средствах для разработки алгоритмов и сведения о модулях программы, проведено функциональное тестирование.
