\chapter*{Введение}
\addcontentsline{toc}{chapter}{Введение}

Во все времена важной задачей была оптимизация, которая позволяла ускорять работу существующих алгоритмов или предлагать совершенно новые пути решения для увеличения скорости выполнения той или иной задачи. Одной из важных задач является логистическая задача поисков оптимальных маршрутов. Чаще всего они решаются полным перебором, что крайне неэффективно при большом количестве вершин в графе (каждую задачу поиска оптимального маршрута можно представить в виде графа -- набора вершин и ребер).


\textbf{Целью данной работы} является изучение задачи коммивояжера, которая решается муравьиным алгоритмом. 
Для достижения поставленной цели необходимо выполнить следующие задачи:
\begin{itemize}
	\item изучить задачу коммивояжера;
    \item описать алгоритмы решения задачи коммивояжера -- полный перебор и муравьиный алгоритм;
    \item привести схемы полного перебора и муравьиного алгоритмов;
    \item описать используемые типы и структуры данных;
    \item описать структуру разрабатываемого программного обеспечения;
    \item реализовать разработанные алгоритмы;
    \item провести функциональное тестирование разработанного алгоритма;
    \item провести сравнительный анализ по времени для реализованного алгоритма;
    \item подготовить отчет по лабораторной работе.
\end{itemize}
