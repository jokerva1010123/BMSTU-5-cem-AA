\chapter{Исследовательская часть}

В данном разделе будут приведены примеры работы программа, а также проведен сравнительный анализ алгоритмов при различных ситуациях на основе полученных данных.

\section{Технические характеристики}

Технические характеристики устройства, на котором выполнялось тестирование представлены далее:

\begin{itemize}
    \item операционная система: Ubuntu 20.04.3 \cite{ubuntu} Linux \cite{linux} x86\_64;
    \item память: 8 GiB;
    \item процессор: Intel® Core™ i5-7300HQ CPU @ 2.50GHz \cite{intel}.
\end{itemize}

При тестировании ноутбук был включен в сеть электропитания. Во время тестирования ноутбук был нагружен только встроенными приложениями окружения, а также системой тестирования.

\section{Демонстрация работы программы}

На рисунке \ref{img:example} представлен результат работы программы для обоих алгоритмов -- полного перебора и муравьиного.

\imgHeight{200mm}{example}{Пример работы программы}
\clearpage

\section{Время выполнения алгоритмов}

Как было сказано выше, используется функция замера процессорного времени  \textit{process\_time(...)} из библиотеки \textit{time} на \textit{Python}. Функция возвращает процессорное время типа float в секундах.

Использовать функцию приходится дважды, затем из конечного времени нужно вычесть начальное, чтобы получить результат.

Замеры проводились для разного размера матриц, чтобы определить, когда наиболее эффективно использовать муравьиный алгоритм.

Результаты замеров приведены в таблице \ref{tbl:time_mes} (время в с).


\begin{center}
\captionsetup{justification=raggedright,singlelinecheck=off}
\begin{longtable}[c]{|p{4cm}|p{4cm}|p{4cm}|}
\caption{Результаты замеров времени\label{tbl:time_mes}}\\ \hline
    Размер & Полный перебор & Муравьиный \\ \hline
    2 &   0.000130 &   0.019932 \\ \hline
    3 &   0.000138 &   0.031615 \\ \hline
    4 &   0.000104 &   0.044361 \\ \hline
    5 &   0.000420 &   0.089291 \\ \hline
    6 &   0.002390 &   0.152131 \\ \hline
    7 &   0.019703 &   0.254059 \\ \hline
    8 &   0.162850 &   0.398472 \\ \hline
    9 &   1.637611 &   0.594024 \\ \hline
    10 &  18.207853 &   0.857666 \\ \hline
\end{longtable}
\end{center}

\clearpage

Также на рисунке \ref{img:graph_alg} приведены графические результаты замеров.

\imgHeight{100mm}{graph_alg}{Сравнение по времени алгоритмов полного перебора путей и муравьиного на разных размерах матриц}

\clearpage


\section{Постановка эксперимента}

Автоматическая параметризация была проведена на двух классах данных -- \ref{eq:kd1} и \ref{eq:kd2}. Алгоритм будет запущен для всех значений $\alpha, \rho \in (0, 1)$.

Итоговая таблица значений параметризации будет состоять из следующих колонок:
\begin{itemize}
    \item $\alpha$ - изменяющийся параметр;
    \item $\rho$ - изменяющийся параметр;
    \item \textit{days} - кол-во дней, изменяющийся параметр;
    \item \textit{Result} - эталонный результат;
    \item \textit{Mistake} - разность получившегося значения и эталонного значения на данных значениях параметров.
\end{itemize}

\textit{Цель эксперимента} -- определить комбинацию параметров, которые решают задачу наилучшим образом. Качество зависит от двух факторов:
\begin{itemize}
    \item количества дней;
    \item погрешности измерений.
\end{itemize}

\subsection{Класс данных 1}

Класс данных 1 представляет собой матрицу смежности размером 9 элементов (небольшой разброс значений - от 1 до 2), которая представлена далее.

\begin{equation}
    \label{eq:kd1}
	K_{1} = \begin{pmatrix}
		0 & 1 & 1 & 2 & 2 & 1 & 1 & 1 & 2 \\ 
        1 & 0 & 1 & 2 & 1 & 1 & 2 & 1 & 1 \\ 
        1 & 1 & 0 & 2 & 2 & 1 & 1 & 2 & 2 \\ 
        2 & 2 & 2 & 0 & 1 & 2 & 1 & 2 & 2 \\ 
        2 & 1 & 2 & 1 & 0 & 2 & 2 & 1 & 1 \\ 
        1 & 1 & 1 & 2 & 2 & 0 & 1 & 1 & 2 \\ 
        1 & 2 & 1 & 1 & 2 & 1 & 0 & 2 & 2 \\ 
        1 & 1 & 2 & 2 & 1 & 1 & 2 & 0 & 2 \\ 
        2 & 1 & 2 & 2 & 1 & 2 & 2 & 2 & 0 
	\end{pmatrix}
\end{equation}

Для данного класса данных приведена таблица с выборкой параметров, которые наилучшим образом решают поставленную задачу.

\begin{center}
    \captionsetup{justification=raggedright,singlelinecheck=off}
    \begin{longtable}[c]{|c|c|c|c|c|}
    \caption{Параметры для класса данных 1\label{tbl:table_kd1}}\\ \hline
        $\alpha$ & $\rho$ & Days & Result & Mistake \\ \hline
        0.1 & 0.3 &  10 &    9 &    0 \\
        0.1 & 0.3 &  50 &    9 &    0 \\
        0.1 & 0.3 & 100 &    9 &    0 \\
        0.1 & 0.3 & 300 &    9 &    0 \\
        0.1 & 0.3 & 500 &    9 &    0 \\ \hline
        0.1 & 0.4 &  10 &    9 &    0 \\
        0.1 & 0.4 &  50 &    9 &    0 \\
        0.1 & 0.4 & 100 &    9 &    0 \\
        0.1 & 0.4 & 300 &    9 &    0 \\
        0.1 & 0.4 & 500 &    9 &    0 \\ \hline
        0.1 & 0.7 &  10 &    9 &    0 \\
        0.1 & 0.7 &  50 &    9 &    0 \\
        0.1 & 0.7 & 100 &    9 &    0 \\
        0.1 & 0.7 & 300 &    9 &    0 \\
        0.1 & 0.7 & 500 &    9 &    0 \\ \hline
        0.2 & 0.5 &  10 &    9 &    0 \\
        0.2 & 0.5 &  50 &    9 &    0 \\
        0.2 & 0.5 & 100 &    9 &    0 \\
        0.2 & 0.5 & 300 &    9 &    0 \\
        0.2 & 0.5 & 500 &    9 &    0 \\ \hline
        0.2 & 0.7 &  10 &    9 &    0 \\
        0.2 & 0.7 &  50 &    9 &    0 \\
        0.2 & 0.7 & 100 &    9 &    0 \\
        0.2 & 0.7 & 300 &    9 &    0 \\
        0.2 & 0.7 & 500 &    9 &    0 \\ \hline
        0.3 & 0.4 &  10 &    9 &    0 \\
        0.3 & 0.4 &  50 &    9 &    0 \\
        0.3 & 0.4 & 100 &    9 &    0 \\
        0.3 & 0.4 & 300 &    9 &    0 \\
        0.3 & 0.4 & 500 &    9 &    0 \\ \hline
        0.3 & 0.5 &  10 &    9 &    0 \\
        0.3 & 0.5 & 100 &    9 &    0 \\
        0.3 & 0.5 & 300 &    9 &    0 \\
        0.3 & 0.5 & 500 &    9 &    0 \\ \hline
        0.4 & 0.5 &  10 &    9 &    0 \\
        0.4 & 0.5 &  50 &    9 &    0 \\
        0.4 & 0.5 & 100 &    9 &    0 \\
        0.4 & 0.5 & 300 &    9 &    0 \\
        0.4 & 0.5 & 500 &    9 &    0 \\ \hline
        0.6 & 0.1 &  10 &    9 &    0 \\
        0.6 & 0.1 &  50 &    9 &    0 \\
        0.6 & 0.1 & 100 &    9 &    0 \\
        0.6 & 0.1 & 300 &    9 &    0 \\
        0.6 & 0.1 & 500 &    9 &    0 \\ \hline
\end{longtable}
\end{center}


\subsection{Класс данных 2}

Класс данных 2 представляет собой матрицу смежности размером 9 элементов (большой разброс значений - от 1000 до 9999), которая представлена далее.

\begin{equation}
    \label{eq:kd2}
	K_{1} = \begin{pmatrix}
		0 & 9271 & 8511 & 2010 & 1983 & 7296 & 7289 & 3024 & 1011 \\
        9271 & 0 & 7731 & 4865 & 5494 & 6812 & 4755 & 7780 & 7641 \\
        8511 & 7731 & 0 & 1515 & 9297 & 7506 & 5781 & 5804 & 7334 \\
        2010 & 4865 & 1515 & 0 & 3662 & 9597 & 2876 & 8188 & 9227 \\
        1983 & 5494 & 9297 & 3662 & 0 & 8700 & 4754 & 7445 & 3834 \\
        7296 & 6812 & 7506 & 9597 & 8700 & 0 & 4216 & 5553 & 8215 \\
        7289 & 4755 & 5781 & 2876 & 4754 & 4216 & 0 & 4001 & 4715 \\
        3024 & 7780 & 5804 & 8188 & 7445 & 5553 & 4001 & 0 & 9522 \\
        1011 & 7641 & 7334 & 9227 & 3834 & 8215 & 4715 & 9522 & 0 
	\end{pmatrix}
\end{equation}


Для данного класса данных приведена таблица с выборкой параметров, которые наилучшим образом решают поставленную задачу.

\begin{center}
    \captionsetup{justification=raggedright,singlelinecheck=off}
    \begin{longtable}[c]{|c|c|c|c|c|}
    \caption{Параметры для класса данных 2\label{tbl:table_kd2}}\\ \hline
        $\alpha$ & $\rho$ & Days & Result & Mistake \\ \hline
        0.1 &  0.3 &  100 & 34192 &     0 \\
        0.1 &  0.3 &  300 & 34192 &     0 \\
        0.1 &  0.3 &  500 & 34192 &     0 \\ \hline
        0.1 &  0.7 &  100 & 34192 &     0 \\
        0.1 &  0.7 &  300 & 34192 &     0 \\
        0.1 &  0.7 &  500 & 34192 &     0 \\ \hline
        0.2 &  0.1 &  100 & 34192 &     0 \\
        0.2 &  0.1 &  300 & 34192 &     0 \\
        0.2 &  0.1 &  500 & 34192 &     0 \\ \hline
        0.2 &  0.2 &  100 & 34192 &     0 \\
        0.2 &  0.2 &  300 & 34192 &     0 \\
        0.2 &  0.2 &  500 & 34192 &     0 \\ \hline
        0.2 &  0.5 &  100 & 34192 &     0 \\
        0.2 &  0.5 &  300 & 34192 &     0 \\
        0.2 &  0.5 &  500 & 34192 &     0 \\ \hline
        0.2 &  0.8 &  100 & 34192 &     0 \\
        0.2 &  0.8 &  300 & 34192 &     0 \\
        0.2 &  0.8 &  500 & 34192 &     0 \\ \hline
        0.3 &  0.1 &  100 & 34192 &     0 \\
        0.3 &  0.1 &  300 & 34192 &     0 \\
        0.3 &  0.1 &  500 & 34192 &     0 \\ \hline
        0.3 &  0.2 &    5 & 34192 &     0 \\
        0.3 &  0.2 &   50 & 34192 &     0 \\
        0.3 &  0.2 &  100 & 34192 &     0 \\
        0.3 &  0.2 &  300 & 34192 &     0 \\
        0.3 &  0.2 &  500 & 34192 &     0 \\ \hline
        0.4 &  0.5 &   50 & 34192 &     0 \\
        0.4 &  0.5 &  300 & 34192 &     0 \\
        0.4 &  0.5 &  500 & 34192 &     0 \\ \hline
        0.5 &  0.2 &  100 & 34192 &     0 \\
        0.5 &  0.2 &  300 & 34192 &     0 \\
        0.5 &  0.2 &  500 & 34192 &     0 \\ \hline
        0.6 &  0.2 &  100 & 34192 &     0 \\
        0.6 &  0.2 &  300 & 34192 &     0 \\
        0.6 &  0.2 &  500 & 34192 &     0 \\ \hline
        0.6 &  0.3 &  300 & 34192 &     0 \\
        0.6 &  0.3 &  500 & 34192 &     0 \\ \hline
        0.6 &  0.4 &  100 & 34192 &     0 \\
        0.6 &  0.4 &  500 & 34192 &     0 \\ \hline
        0.6 &  0.5 &  100 & 34192 &     0 \\
        0.6 &  0.5 &  300 & 34192 &     0 \\
        0.6 &  0.5 &  500 & 34192 &     0 \\ \hline
\end{longtable}
\end{center}

\section{Вывод}

В результате эксперимента было получно, что использование муравьиного алгоритма наиболее эффективно при больших размерах матриц. Так при размере матрицы, равном 2, муравьиный алгоритм медленее алгоритма полного перебора в 153 раза, а при размере матрицы, равном 9, муравьиный алгоритм быстрее алгоритма полного перебора в раз, а при размере в 10 -- уже в 21 раз. Следовательно, при рамзерах матриц больше 8 следует использовать муравьиный алгоритм, но стоит учитывать, что он может давать погрешности вычислений.

Также при проведении эксперимента с классами данных было получено, что на первом классе данных (\ref{eq:kd1}) муравьиный алгоритм лучше всего показывает себя при параметрах:
\begin{itemize}
    \item $\alpha = 0.1, \rho = 0.3, 0.4, 0.7$;
    \item $\alpha = 0.2, \rho = 0.5, 0.7$;
    \item $\alpha = 0.3, \rho = 0.4, 0.5$;
    \item $\alpha = 0.4, \rho = 0.5$;
    \item $\alpha = 0.6, \rho = 0.1$.
\end{itemize}  
Следовательно, для класса данных 1 (\ref{eq:kd1}) стоит использовать данные параметры. 

Для класса данных 2 (\ref{eq:kd2}) было получено, что наилучшим образом алгоритм работает на значениях параметров, которые представлены далее:
\begin{itemize}
    \item $\alpha = 0.1, \rho = 0.3, 0.7$;
    \item $\alpha = 0.2, \rho = 0.1, 0.2, 0.5, 0.8$;
    \item $\alpha = 0.3, \rho = 0.1, 0.2$;
    \item $\alpha = 0.4, \rho = 0.5$;
    \item $\alpha = 0.5, \rho = 0.2$;
    \item $\alpha = 0.6, \rho = 0.2, 0.3, 0.4$.
\end{itemize} 
То есть, для второго класса данных (\ref{eq:kd2}) стоит использовать данные параметры.

Также во время ислледования было замечено -- чем меньше $\alpha$, тем меньше погрешностей возникает. При этом кол-во дней сильно влияет на отсутствие погрешностей: чем значение параметра $Days$ больше, тем меньше погрешностей.
