\chapter{Конструкторская часть}
В этом разделе будут представлено описание используемых типов данных, а также схемы алгоритма полного перебора и муравьиного алгоритма.

\section{Описание используемых типов данных}

При реализации алгоритмов будут использованы следующие типы данных:

\begin{itemize}
	\item размер матрицы смежности - целое число типа \textit{int};
	\item название файла - строка типа \textit{str};
	\item коэффициент $\alpha$, $\beta$, \textit{evaporation\_koef} - числа типа \textit{float};
	\item матрица смежностей - матрица типа \textit{int} для хранения длины путей между городами.
\end{itemize}


\section{Структура разрабатываемого ПО}

В данном ПО будет реализован метод структурного программирования. Для взаимодействия с пользователем будет разработано меню, которое будет предоставлять возможность выбрать нужный алгоритм - полного перебора или муравьиный, замерить время, провести параметризацию муравьиного алгоритма, а также построить графики для сравнения времени выполнения алгоритмов. 

Для работы будут разработаны следующие процедуры:

\begin{itemize}
	\item процедура, реализующая генерацию квадратной матрицы значениями в определенном диапазоне по ее размеру, входные данные - размер матрицы, диапазон значений (start\_num, end\_num), выходные - полученная матрица;
	\item процедура вывода полученной матрицы на экран (для отладки), входные данные - матрица, выходные - выведенная на экран матрица;
	\item процедура, реализующая алгоритм полного перебора путей, входные данные - матрица, количество городов (размер матрицы), выходные - длина найденного минимального пути и сам путь;
	\item процедура, реализующая муравьиный алгоритм перебора путей, входные данные - матрица, количество городов (размер матрицы), коэффициенты $\alpha$, $\beta$, \textit{evaporation\_koef}, кол-во дней, выходные - длина найденного минимального пути и сам путь;
	\item процедуры замера времени алгоритма полного перебора и муравьиного алгоритма на различных размерах матриц, входные данные - начальный размер матрицы, конечный размер матрицы, выходные - результаты замеров времени;
	\item процедура для построения графиков по полученным временным замерам, входные данные - замеры времени, выходные - график. 
\end{itemize}


\section{Схемы алгоритмов}
На рисунке \ref{img:full_comb} представлена схема алгоритма полного перебора путей, а на рисунках \ref{img:ant_alg_part1}--\ref{img:ant_alg_part2} схема муравьиного алгоритма поиска путей. Также на рисунках \ref{img:find_pos}--\ref{img:update_phero} представлены схемы вспомогательных функций для муравьиного алгоритма.

\imgScale{0.5}{full_comb}{Схема алгоритма полного перебора путей}
\imgScale{0.5}{ant_alg_part1}{Схема муравьиного алгоритма (часть 1)}
\imgScale{0.5}{ant_alg_part2}{Схема муравьиного алгоритма (часть 2)}
\imgScale{0.5}{find_pos}{Схема алгоритма нахождения массива вероятностных переходов в непосещенные города}
\imgScale{0.5}{rand_choice}{Схема алгоритма нахождения следующего города на основании рандома}
\imgScale{0.5}{update_phero}{Схема алгоритма обновления матрицы феромонов}

\clearpage


\section{Классы эквивалентности при тестировании}

Для тестирования выделены классы эквивалентности, представленные ниже.

\begin{enumerate}
	\item Неверно выбран пункт меню - не число или число, меньшее 0 или большее 8.
	\item Неверно введены коэффициенты $\alpha$, $\beta$, \textit{evaporation\_koef} - не число или число, меньшее 0.
	\item Неверно введено кол-во дней - не число или число, меньшее 0.
	\item Неверно введен размер матрицы - не число или число, меньшее 2.
	\item Корректный ввод всех параметров.
\end{enumerate}


\section{Вывод}

В данном разделе были построены схемы алгоритмов, рассматриваемых в лабораторной работе, были описаны классы эквивалентности для тестирования, структура программы.
