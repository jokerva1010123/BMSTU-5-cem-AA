\chapter*{Заключение}
\addcontentsline{toc}{chapter}{Заключение}

В результате исследования было определено, что муравьиный алгоритм стоит использовать при больших рамзерах матрицы (больше 8), при этом алгоритм полного перебора в 21 раз меньше при значении размера, равном 10, и чем больше размер -- тем муравьиный алгоритм быстрее. Но при этом стоит учитывать, что муравьиный алгоритм имеет погрешности в вычислениях, которые можно устранить путем выбора правильного набора параметров. Так, было установлено то, что чем меньше значение параметра $\alpha$, тем точнее работает муравьиный алгоритм. Также при точность увеличивается при увеличении параметра \textit{Days}, но это влияет на время работы алгоритма.

Цель, которая была поставлена в начале лабораторной работы была достигнута, а также в ходе выполнения лабораторной работы были решены следующие задачи:

\begin{itemize}
	\item были изучены алгоритмы решения задачи коммивояжера -- полного перебора путей и муравьиный;
    \item были реализованы алгоритмы полного перебора, а также муравьиный алгоритм для поиска путей;
	\item проведен сравнительный анализ по времени алгоритмов на разном размере матриц;
	\item проведена параметризация муравьиного алгоритма, которая показала лучшие наборы параметров для работы алгоритма на определенных классах данных;
	\item подготовлен отчёт о лабораторной работе.
\end{itemize}