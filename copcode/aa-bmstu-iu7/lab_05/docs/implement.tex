\chapter{Технологическая часть}

В данном разделе будут рассмотрены средства реализации, а также представлены листинги линейной и конвейрной орбаботок матрицы.

\section{Средства реализации}
В данной работе для реализации был выбран язык программирования $C++$ \cite{cpp-lang}. В текущей лабораторной работе требуется замерить процессорное время для выполняемой программы, а также реализовать принципы многопоточного алгоритма. Все эти инструменты присутствуют в выбранном языке программирования. 

Функции построения графиков были реализованы с использованием языка программирования $Python$ \cite{python-lang}.

Время замерено с помощью \textit{std::chrono::system\_clock::now(...)} - функции из библиотеки $chrono$ \cite{cpp-lang-chrono}.


\section{Листинги кода}

В листинге \ref{lst:bres_alg} представлены реализация алгоритма Брезенехема для построения отрезка, а в листингах \ref{lst:par_alg}-\ref{lst:no_par_alg} - алгоритмы для построения спектра отрезков по Брезенехему с многопоточностью и без нее соответственно.

\clearpage

\begin{center}
    \captionsetup{justification=raggedright,singlelinecheck=off}
    \begin{lstlisting}[label=lst:bres_alg,caption=Алгоритм линейной обработки матрицы]
void parse_linear(int count, size_t size, bool is_print)
{
	time_now = 0;

	std::queue<matrix_t> q1;
	std::queue<matrix_t> q2;
	std::queue<matrix_t> q3;

	queues_t queues = {.q1 = q1, .q2 = q2, .q3 = q3};

	for (int i = 0; i < count; i++)
	{
		matrix_t res = generate_matrix(size);
		
		queues.q1.push(res);
	}

	for (int i = 0; i < count; i++)
	{
		matrix_t matrix = queues.q1.front();
		stage1_linear(matrix, i + 1, is_print);
		queues.q1.pop();
		queues.q2.push(matrix);

		matrix = queues.q2.front();
		stage2_linear(matrix, i + 1, is_print); // Stage 2
		queues.q2.pop();
		queues.q3.push(matrix);

		matrix = queues.q3.front();
		stage3_linear(matrix, i + 1, is_print); // Stage 3
		queues.q3.pop();
	}
}
\end{lstlisting}
\end{center}

\clearpage

\begin{center}
    \captionsetup{justification=raggedright,singlelinecheck=off}
    \begin{lstlisting}[label=lst:par_alg,caption=Алгоритм конвейерной обработки матрицы]
void parse_parallel(int count, size_t size, bool is_print)
{
	time_now = 0;

	std::queue<matrix_t> q1;
	std::queue<matrix_t> q2;
	std::queue<matrix_t> q3;

	queues_t queues = {.q1 = q1, .q2 = q2, .q3 = q3};

	
	for (int i = 0; i < count; i++)
	{
		matrix_t res = generate_matrix(size);
		
		q1.push(res);
	}

	std::thread threads[THREADS];

	threads[0] = std::thread(stage1_parallel, std::ref(q1), std::ref(q2), std::ref(q3), is_print);
	threads[1] = std::thread(stage2_parallel, std::ref(q1), std::ref(q2), std::ref(q3), is_print);
	threads[2] = std::thread(stage3_parallel, std::ref(q1), std::ref(q2), std::ref(q3), is_print);

	for (int i = 0; i < THREADS; i++)
	{
		threads[i].join();
	}
}
\end{lstlisting}
\end{center}


\clearpage


\begin{center}
    \captionsetup{justification=raggedright,singlelinecheck=off}
    \begin{lstlisting}[label=lst:no_par_alg,caption=Алгоритм запуска 1 потока для нахождения среднего арифметического элементов матрицы]
void stage1_parallel(std::queue<matrix_t> &q1, std::queue<matrix_t> &q2, std::queue<matrix_t> &q3, bool is_print)
{
	int task_num = 1;

	std::mutex m;

	while(!q1.empty())
	{      
		m.lock();
		matrix_t matrix = q1.front();
		m.unlock();

		log(matrix, task_num++, 1, get_avg, is_print);

		m.lock();
		q2.push(matrix);
		q1.pop();
		m.unlock();
	}
}
\end{lstlisting}
\end{center}

\clearpage

\begin{center}
    \captionsetup{justification=raggedright,singlelinecheck=off}
    \begin{lstlisting}[label=lst:no_par_alg,caption=Алгоритм запуска 2 потока для нахождения максимального элемента матрицы]
void stage2_parallel(std::queue<matrix_t> &q1, std::queue<matrix_t> &q2, std::queue<matrix_t> &q3, bool is_print)
{
	int task_num = 1;

	std::mutex m;

	do
	{   
		m.lock();
		bool is_q2empty = q2.empty();
		m.unlock();

		if (!is_q2empty)
		{   
			m.lock();
			matrix_t matrix = q2.front();
			m.unlock();

			log(matrix, task_num++, 2, get_max, is_print);

			m.lock();
			q3.push(matrix);
			q2.pop();
			m.unlock();
		}
	} while (!q1.empty() || !q2.empty());
}
\end{lstlisting}
\end{center}

\clearpage

\begin{center}
    \captionsetup{justification=raggedright,singlelinecheck=off}
    \begin{lstlisting}[label=lst:no_par_alg,caption=Алгоритм запуска 3 потока для заполнения матрицы вычисленными значениями]
void stage3_parallel(std::queue<matrix_t> &q1, std::queue<matrix_t> &q2, std::queue<matrix_t> &q3, bool is_print)
{
	int task_num = 1;

	std::mutex m;

	do
	{
		m.lock();
		bool is_q3empty = q3.empty();
		m.unlock();

		if (!is_q3empty)
		{
			m.lock();
            matrix_t matrix = q3.front(); 
            q3.pop();
            m.unlock();

			log(matrix, task_num++, 3, fill_matrix, is_print);
		}

	} while (!q1.empty() || !q2.empty() || !q3.empty());
}
\end{lstlisting}
\end{center}

\section{Сведения о модулях программы}
Программа состоит из двух модулей:
\begin{itemize}
	\item \textit{main.cpp} - файл, содержащий меню программы;
    \item \textit{matrix.h} и \textit{matrix.cpp} - файлы, содержащие код работы с матрицей (генерация, печать и алгоритмы обработки матрицы);
	\item \textit{conveyor.h} и \textit{conveyor.cpp} - файлы, содержащие код реализации линейной и конвейерной обработок, а также функции замера времени;
	\item \textit{graph\_build.py} - файл, содержащий функции построения графиков для замеров по времени.
\end{itemize}


\section{Функциональные тесты}

В таблице \ref{tbl:functional_test} приведены тесты для функций программы. Тесты \textit{для всех функций} пройдены успешно.

\begin{table}[h]
	\begin{center}
        \begin{threeparttable}
        \captionsetup{justification=raggedright,singlelinecheck=off}
		\caption{\label{tbl:functional_test} Функциональные тесты}
		\begin{tabular}{|c|c|c|c|}
			\hline
			Алгоритм & Кол-во матриц & Размер матриц & Ожидаемый результат \\
			\hline
            Конвейрная & -5 & 500 & Сообщение об ошибке \\
            \hline
            Конвейрная & 10 & -23 & Сообщение об ошибке \\
            \hline
			Линейная & 50 & 1500 & Вывод лога работы программы \\
            \hline
			Конвейрная & 10 & 100 & Вывод лога работы программы \\
			\hline
			Конвейрная & 5 & 10 & Вывод лога работы программы \\
			\hline
		\end{tabular}
        \end{threeparttable}
	\end{center}
\end{table}

\section{Вывод}

Были представлены листинги всех алгоритмов линейной и конвейерной обработки матриц. Также в данном разделе была приведена информации о выбранных средствах для разработки алгоритмов и сведения о модулях программы, проведено функциональное тестирование.
