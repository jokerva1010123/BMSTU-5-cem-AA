\chapter{Аналитическая часть}
В этом разделе будет представлена информация по поводу сути конвейерной обработки данных.


\section{Конвейерной обработка данных}

\textbf{Конвейер} \cite{conveyer} -- организация вычислений, при которой увеличивается количество выполняемых инструкций за единицу времени за счет использования принципов параллельности.

Конвейеризация (или конвейерная обработка) в общем случае основана на разделении подлежащей исполнению функции на более мелкие части, называемые ступенями, и выделении для каждой из них отдельного блока аппаратуры. Так, обработку любой машинной команды можно разделить на несколько этапов (несколько ступеней), организовав передачу данных от одного этапа к следующему. При этом конвейерную обработку можно использовать для совмещения этапов выполнения разных команд. Производительность при этом возрастает, благодаря тому что одновременно на различных ступенях конвейера выполняется несколько команд. Конвейерная обработка такого рода широко применяется во всех современных быстродействующих процессорах.

Конвейеризация увеличивает пропускную способность процессора (количество команд, завершающихся в единицу времени), но она не сокращает время выполнения отдельной команды. В действительности она даже несколько увеличивает время выполнения каждой команды из-за накладных расходов, связанных с хранением промежуточных результатов. Однако увеличение пропускной способности означает, что программа будет выполняться быстрее по сравнению с простой, неконвейерной схемой.


\section{Описание алгоритмов}

В качестве примера для конвейерной обработки будет обрабатываться матрица. Всего будет использовано три ленты, которые делают следующее.

\begin{enumerate}
	\item Находится среднее арифмитическое значений матрицы.
	\item Находится максимальный элемент матрицы.
	\item Нечетный элемент матрицы заменяется на среднее арифметическое матрицы, а четные - на макимальный элемент.
\end{enumerate}


\section{Вывод}

В данном разделе было рассмотрено понятие конвейерной обработки, а также выбраны этапы для обработки матрицы, которые будут обрабатывать ленты конвейера.

Программа будет получать на вход количество задач (количество матриц), размер матрицы (используются только квадратные матрицы), а также выбор алгоритма -- линейный или конвейерный. При неверном вводе какого-то из значений будет выдаваться сообщение об ошибке.

Реализуемое ПО дает возможность получить лог программы для N задач при линейной и конвейрной обработке. Также имеется возможность провести тестирование по времени для разного количества задач (матриц) и разных размеров самих матриц.
