\chapter*{Введение}
\addcontentsline{toc}{chapter}{Введение}

В математике и программировании часто приходится прибегать к использованию матриц. Существует огромное количество областей их применения в этих сферах. Например, матрицы активно используются при выводе различных формул в физике:

\begin{itemize}
    \item градиент;
    \item дивергенция;
    \item ротор.
\end{itemize}

Также часто применяются и операции над матрицами - сложение, возведение в степень, умножение. При различных задачах размеры матрицы могут достигать больших значений. Поэтому оптимизация операций работы над матрицами является важной задачей в программировании. Об оптимизации операции умножения пойдет речь в данной лабораторной работе.


\textbf{Целью данной работы} является изучение, реализация и исследование алгоритмов умножения матриц - классический алгоритм, алгоритм Винограда и оптимизированный алгоритм Винограда. 
Для достижения поставленной цели необходимо выполнить следующие задачи:
\begin{itemize}
	\item изучить и реализовать алгоритмы - классический, Винограда и его оптимизацию;
    \item провести тестирование по времени и по памяти для алгоритмов лабораторной работы;
    \item провести сравнительный анализ по времени классического алгоритма и алгоритма Винограда;
    \item провести сравнительный анализ по времени алгоритма Винограда и его оптимизации;
	\item описать и обосновать полученные результаты в отчете о выполненной лабораторной работе, выполненного как расчётно-пояснительная записка к работе.
\end{itemize}
