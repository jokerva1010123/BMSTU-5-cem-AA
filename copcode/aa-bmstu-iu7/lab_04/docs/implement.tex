\chapter{Технологическая часть}

В данном разделе будут рассмотрены средства реализации, а также представлены листинги алгоритма Брезенехема для построенияотрезка отрезка, а также листинги алгоритмов построения спектра таких отрезков с использованием многопоточности и без.

\section{Средства реализации}
В данной работе для реализации был выбран язык программирования $C++$ \cite{cpp-lang}. В текущей лабораторной работе требуется замерить процессорное время для выполняемой программы, а также реализовать принципы многопоточного алгоритма. Все эти инструменты присутствуют в выбранном языке программирования. 

Функции построения графиков были реализованы с использованием языка программирования $Python$ \cite{python-lang}.

Время замерено с помощью \textit{std::chrono::system\_clock::now(...)} - функции из библиотеки $chrono$ \cite{cpp-lang-chrono}.


\section{Листинги кода}

В листинге \ref{lst:bres_alg} представлены реализация алгоритма Брезенехема для построения отрезка, а в листингах \ref{lst:par_alg}-\ref{lst:no_par_alg} - алгоритмы для построения спектра отрезков по Брезенехему с многопоточностью и без нее соответственно.

\clearpage

\begin{center}
    \captionsetup{justification=raggedright,singlelinecheck=off}
    \begin{lstlisting}[label=lst:bres_alg,caption=Алгоритм Брезенехема для построения отрезка]
void breshenham_float(const request_t &request, const point_t &pt1, const point_t &pt2){
	int dx = abs(pt2.x - pt1.x);
	int dy = abs(pt2.y - pt1.y);
	int sx = sign(pt2.x - pt1.x);
	int sy = sign(pt2.y - pt1.y);
	int change = 0;

	if (dy > dx){
		SWAP(int, dx, dy);
		change = 1;
	}
	double m = (double)dy / (double)dx;
	double e = m - 0.5;
	int x = pt1.x;
	int y = pt1.y;

	for (int i = 0; i < dx; i++){
		draw_pixel(request, x, y);
		if (e >= 0){
			if (change == 0){
				y += sy;
			}
			else{
				x += sx;
			}
			e -= 1;
		}
		if (e < 0){
			if (change == 0){
				x += sx;
			}
			else{
				y += sy;
			}
			e += m;
		}
	}
}
\end{lstlisting}
\end{center}


\begin{center}
    \captionsetup{justification=raggedright,singlelinecheck=off}
    \begin{lstlisting}[label=lst:par_alg,caption=Алгоритм построения спектра отрезков по Брезенехему (c многопоточностью)]
void calculate_part_beam(const request_t &request, const beam_stgs_t &settings, int thread_ind){   
    int spektr = 360;
    int sector_in_thread = spektr / settings.threads_count;
    int part_ind = thread_ind * sector_in_thread;

    if (thread_ind + 1 == settings.threads_count){
        sector_in_thread += (spektr - sector_in_thread * settings.threads_count);
    }
    point_t pt1, pt2;
    pt1.x = WIN_X / 2;
    pt1.y = WIN_Y / 2;

    for (int i = part_ind; i < part_ind + sector_in_thread; i += 1){
        pt2.x = cos(i * PI / 180) * settings.d + WIN_X / 2;
        pt2.y = sin(i * PI / 180) * settings.d + WIN_Y / 2;

        // Round for Bresenham float
        pt1.x = round(pt1.x);
        pt1.y = round(pt1.y);
        pt2.x = round(pt2.x);
        pt2.y = round(pt2.y); 
        bresenham_line(request, pt1, pt2);
    }
}

void calculate_beam_parallel(const request_t &request, const beam_stgs_t &settings){
    std::vector<std::thread> threads(settings.threads_count);

    for (int i = 0; i < settings.threads_count; i++){
        threads[i] = std::thread(calculate_part_beam, request, settings, i);
    }
    for (int i = 0; i < settings.threads_count; i++){
        threads[i].join();
    }
}
\end{lstlisting}
\end{center}


\begin{center}
    \captionsetup{justification=raggedright,singlelinecheck=off}
    \begin{lstlisting}[label=lst:no_par_alg,caption=Алгоритм построения спектра отрезков по Брезенехему (без многопоточности)]
void calculate_beam_no_parallel(const request_t &request, const beam_stgs_t &settings){
	int spektr = 360;

	point_t pt1, pt2;
	pt1.x = WIN_X / 2;
	pt1.y = WIN_Y / 2;

	for (int i = 0; i < spektr; i += 1){
		pt2.x = cos(i * PI / 180) * settings.d + WIN_X / 2;
		pt2.y = sin(i * PI / 180) * settings.d + WIN_Y / 2;

		// Round for Bresenham float
		pt1.x = round(pt1.x);
		pt1.y = round(pt1.y);
		pt2.x = round(pt2.x);
		pt2.y = round(pt2.y);

		bresenham_line(request, pt1, pt2);
	}
}
\end{lstlisting}
\end{center}

\section{Сведения о модулях программы}
Программа состоит из двух модулей:
\begin{itemize}
	\item \textit{main.cpp} - файл, содержащий функцию, вызывающую интерфейс программы;
    \item \textit{mainwindow.h} и \textit{mainwindow.cpp} - файлы, содержащие код всех методов, реализующих интерфейс программы и взаимодействующие с ним;
	\item \textit{building.h} и \textit{building.cpp} - файлы, содержащие код алгоритма Брезенехема для построения отрезка, а также алгоритмов построения спектра отрезков с использованием многопоточности и без нее, а также функции замера времени;
	\item \textit{graph\_build.py} - файл, содержащий функции построения графиков для замеров по времени.
\end{itemize}


\section{Функциональные тесты}

В таблице \ref{tbl:functional_test} приведены тесты для функций программы. Тесты \textit{для всех функций} пройдены успешно.

\begin{table}[h]
	\begin{center}
        \begin{threeparttable}
        \captionsetup{justification=raggedright,singlelinecheck=off}
		\caption{\label{tbl:functional_test} Функциональные тесты}
		\begin{tabular}{|c|c|c|}
			\hline
			Длина отрезка & Кол-во потоков & Ожидаемый результат \\
			\hline
            -5 & 4 & Сообщение об ошибке \\
            \hline
            350 & 0 & Сообщение об ошибке \\
            \hline
            350 & 4 & Вывод спектра отрезков на канвас \\
            \hline
            10 & 32 & Вывод спектра отрезков на канвас \\
			\hline
			228 & 1 & Вывод спектра отрезков на канвас \\
			\hline
		\end{tabular}
        \end{threeparttable}
	\end{center}
\end{table}

\section{Вывод}

Были представлены листинги всех алгоритмов построения спектра отрезков по Брезенехему с распараллеливанием и без него, а также сам алгоритм Брезенехема для генерации отрезка. Также в данном разделе была приведена информации о выбранных средствах для разработки алгоритмов и сведения о модулях программы, проведено функциональное тестирование.
