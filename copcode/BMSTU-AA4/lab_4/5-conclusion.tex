\chapter{Заключение}
В ходе работы были рассмотрены два алгоритма поиска наиболее коррелирующих столбцов: классический алгоритм
и многопоточный.
Были составлены схемы алгоритмов.
Было реализовано работоспособное ПО, удалось провести анализ зависимости затрат по времени от размера матрицы и количества потоков.

Многопоточный алгоритм показал более высокие результаты, начиная с $N=50$.
16-поточный алгоритм показал наилучшие результаты на матрицах размером $N >= 200$. 
Классический алгоритм на небольших размерах матрицы оказался быстрее, не требуя при этом
поддержки поточности.
В случае превышения количеством потоков количества логических ядер,
выигрыша по времени не наблюдается, что заметно по поведению графика для $M_1 = 16$ и $M_2 = 32$.
При таком превышении растёт нагрузка на процессор, но выигрыша не наблюдается, поэтому можно сделать вывод,
что такое число потоков не является эффективным.
Оптимальный размер количества потоков -- 16, что соответствует количеству логических ядер.