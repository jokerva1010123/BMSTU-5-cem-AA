\chapter{Технологическая часть}
В этом разделе будут приведены листинги кода и результаты функционального тестирования.

\section{Средства реализации программного обеспечения}
В качестве языка программирования выбран C++, так как имеется опыт разработки проектов на этом языке, 
а также есть возможность использования нативных потоков \cite{thread}.
Для замеры общего времени используется библиотека chrono \cite{time}.

\section{Листинг кода}
\FloatBarrier
На листингах 4.1-4.4 представлена реализация фрагментов разработанного алгоритма.

\begin{lstinputlisting}[language=C++, caption=Реализация классического алгоритма, linerange={16-55}, 
	basicstyle=\footnotesize\ttfamily, frame=single,breaklines=true]{src/main.cpp}
\end{lstinputlisting}
\FloatBarrier

\FloatBarrier
\begin{lstinputlisting}[language=C++, caption=Реализация схемы подсчёта среднего арифметического для каждого потока, 
	linerange={64-74}, basicstyle=\footnotesize\ttfamily, frame=single,breaklines=true]{src/main.cpp}
\end{lstinputlisting}
\FloatBarrier

\FloatBarrier
\begin{lstinputlisting}[language=C++, caption=Реализация схемы подсчёта корреляции Пирсона, linerange={90-125}, 
	basicstyle=\footnotesize\ttfamily, frame=single, breaklines=true]{src/main.cpp}
\end{lstinputlisting}
\FloatBarrier

\FloatBarrier
\begin{lstinputlisting}[language=C++, caption= Общая реализация алгоритма с использованием потоков, linerange={127-167}, 
	basicstyle=\footnotesize\ttfamily, frame=single, breaklines=true]{src/main.cpp}
\end{lstinputlisting}
\FloatBarrier

\section{Функциональные тесты}
В таблице \ref{tabular:functional_test} приведены результаты функциональных тестов. 
В первом столбце -- исходная матрица.
Во втором столбце -- ожидаемый результат: наиболее коррелирующих признаков и само значение корреляции.

\FloatBarrier
\begin{table}[h]
		\caption{Функциональные тесты}
		\label{tabular:functional_test}
		\begin{tabular}{| c | c |}
			\hline
		Матрица & Ожидаемый результат \\ \hline
		\begin{tabular}{c c c} 
			1 & 0.5 & 2 \\
			15 & 3 & 9 \\
			3 & 11 & 12 \\
		\end{tabular}
		&
		2, 3, 0.867502 \\
		\hline
		
		\begin{tabular}{c c c c} 
			1 & 0.5 & 2 & 2\\
			2 & 3 & 9 & 4\\
			3 & 11 & 12 & 6\\
			4 & 5 & 3 & 8 \\
		\end{tabular}
		&
		1, 4, 1 \\
		\hline
		
		\begin{tabular}{c c} 
			1 & 0.5 \\
			2 & 3 \\
			3 & 11 \\
			5 & 6 \\
		\end{tabular}
		&
		1, 2, 0.567236 \\
		\hline
		
		\begin{tabular}{c c} 
			1 & 0.5 \\
			2 & 3 \\
		\end{tabular}
		&
		1, 2, 1 \\
		\hline
		
		\begin{tabular}{c c c c c} 
			5 & 9 & 1 & 9 & 2  \\
			1 & 3 & 8 & 5 & 0 \\
			5 & 5 & 6 & 0 & 6 \\
			7 & 8 & 0 & 1 & 5 \\
			8 & 9 & 1 & 7 & 6 \\
			2 & 3 & 3 & 3 & 9 \\
		\end{tabular}
		&
		1, 2, 0.879347 \\
		\hline
		\end{tabular}
\end{table}
\FloatBarrier

Все тесты алгоритмами были пройдены успешно.

\section{Вывод}
Были сформированы требования к ПО, приведены листинги кода.
Были проведены функциональные тесты.
Все алгоритмы справились с тестированием.