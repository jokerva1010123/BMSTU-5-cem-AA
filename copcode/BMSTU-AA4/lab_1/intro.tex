\newpage
\addcontentsline{toc}{chapter}{Введение}

\chapter*{Введение}
Расстояние Левенштейна\cite{Levenshtein} - мера, которая определяет, насколько отличаются друг от друга две строки. Фактически величина показывает, сколько нужно произвести односимвольных изменений, чтобы преобразовать одно слово к другому.

Расстояние Левенштейна получило широкое применение в компьютерной лингвистике. Активно его использует Яндекс\cite{yandex} для поисковых систем. Мера используется для:
\begin{itemize}
    \item Автоматического исправления ошибок в тексте
    \item Поиска возможных ошибок в поисковых запросах
    \item Расчёта изменений в различных версиях текста (утилита diff)
\end{itemize}

Также расстояние Левенштейна нашло применение в биоинформатике для нахождения разности последовательностей генов.\cite{bio}

Актуальность работы заключается в том, что нахождение расстояния Левенштейна должно выполняться за максимально короткое время.
Целью данной работы является разработка программы, которая реализует четыре алгоритма поиска редакционного расстояния. 
Для достижения поставленной цели необходимо выполнить следующее:
\begin{itemize}
\item рассмотреть существующие алгоритмы поиска редакционного расстояния;
\item привести схемы реализации рассматриваемых алгоритмов;
\item определить средства реализации алгоритмов;
\item реализовать рассматриваемые алгоритмы;
\item провести модульное тестирование всех реализаций алгоритмов;
\item оценить реализацию алгоритмов по времени и памяти.
\end{itemize}