\newpage
\addcontentsline{toc}{chapter}{Введение}

\chapter*{Введение}
Словарь - структура данных, позволяющая идентифицировать её элементы не по числовому индексу, а по произвольному.
В жизни встречаются книжные бумажные словари и телефонные справочники - они являются словарями.
Особенностью словаря является его динамичность: можно добавлять новые элементы с произвольными ключами и удалять уже существующие элементы.

Поиск в словаре -- одна из основных операций над структурой данных, и перебор ключей не всегда отвечает требованиям к скорости работы программы.
Поэтому существует несколько оптимизаций поиска в словаре.

Цель лабораторной работы -- разработать ПО, которое реализует три алгоритма поиска в словаре: перебор ключей, бинарный поиск и поиск с использованием сегментации.

Для достижения поставленной цели необходимо выполнить следующее:
\begin{itemize}
	\item рассмотреть понятие словаря;
	\item рассмотреть три основных алгоритма поиска в словаре;
	\item привести схемы реализации алгоритмов;
	\item определить средства программной реализации;
	\item реализовать три алгоритма для решения задачи;
	\item протестировать разработанное ПО;
	\item оценить реализацию алгоритмов по времени и памяти.
\end{itemize}