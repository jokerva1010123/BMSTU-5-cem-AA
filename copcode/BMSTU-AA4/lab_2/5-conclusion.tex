\chapter{Заключение}

В ходе работы были рассмотрены три алгоритма умножения матриц: классический алгоритм,
алгоритм Винограда и оптимизированный алгоритм Винограда.
Были составлены схемы алгоритмов, подсчитана их трудоёмкость.
Было реализовано работоспособное ПО, удалось провести анализ зависимости затрат по времени от размера матрицы..

В случае чётных размеров лучший результат показал оптимизированный алгоритм Винограда, опередив при размере $N=401$
на 2.5\% обычного Винограда. Классический алгоритм сработал в два раза медленнее.

В случае нечётных размеров лучший результат показал оптимизированный алгоритм Винограда, опередив при размере $N=401$
на 7\% обычного Винограда и 11\% -- классический алгоритм.

При этом при значения $N<6$ классический алгоритм работает быстрее, чем алгоритм Винограда,
и для маленьких матриц такие преобразования не имеют смысл.

Дополнительных затрат на память для классического алгоритма не потребовалось, в то время как
алгоритмы Винограда имеют зависимость потребления памяти $O(N)$.