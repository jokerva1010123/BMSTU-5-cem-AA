\chapter{Технологический раздел}
В этом разделе будут приведены листинги кода и результаты функционального тестирования.
Также будет произведена оценка трудоёмкости алгоритмов.

\section{Средства реализации программного обеспечения}
В качестве языка программирования выбран Python 3.9, так как имеется опыт разработки проектов на этом языке.
Для замера процессорного времени используется функция process\_time\_ns из библиотеки time. 
В её результат не включается время, когда процессор не выполняет задачу \cite{python}. 

\section{Листинг кода}
\FloatBarrier
На листингах 4.1, 4.2 и 4.3 приведены реализации различных алгоритмов умножения матриц.
\begin{lstinputlisting}[language=Python, caption=Реализация алгоритма классического умножения матриц, linerange={11-25}, 
	basicstyle=\small\sffamily, frame=single]{src/algs.py}
\end{lstinputlisting}
\FloatBarrier

\FloatBarrier
\begin{lstinputlisting}[language=Python, caption=Реализация алгоритма Винограда, linerange={27-66}, 
	basicstyle=\small\sffamily, frame=single]{src/algs.py}
\end{lstinputlisting}
\FloatBarrier

\FloatBarrier
\begin{lstinputlisting}[language=Python, caption=Реализация оптимизированного алгоритма Винограда, linerange={68-105}, 
	basicstyle=\small\sffamily, frame=single]{src/algs.py}
\end{lstinputlisting}
\FloatBarrier

\section{Функциональные тесты}
В таблице \ref{tabular:functional_test} приведены результаты функциональных тестов. 
В первом столбце -- первый операнд умножения матриц.
Во втором столбце -- второй операнд умножения матриц.
В третьем столбце -- ожидаемый результат.

\begin{table}[h]
	\begin{center}
		\caption{Функциональные тесты}
		\label{tabular:functional_test}
		\begin{tabular}{| c | c | c |}
			\hline
			Матрица 1 & Матрица 2 & Ожидаемый результат \\ \hline
			\begin{tabular}{c c c} 
				1 & 2 & 3 \\
				4 & 5 & 6 \\
				7 & 8 & 9 \\
			\end{tabular}
			&
			\begin{tabular}{c c c} 
				9 & 8 & 7 \\
				6 & 5 & 4 \\
				9 & 8 & 7 \\
			\end{tabular}
			&
			\begin{tabular}{c c c} 
				30 & 24 & 18 \\
				84 & 69 & 54 \\
				138 & 114 & 90 \\
			\end{tabular}\\
			\hline
			\begin{tabular}{c c c c} 
				1 & 2 & 3 & 4\\
				5 & 6 & 7 & 8 \\
				9 & 10 & 11 & 12 \\
				13 & 14 & 15 & 16
			\end{tabular}
			&
			\begin{tabular}{c c c c} 
				16 & 15 & 14 & 13 \\
				12 & 11 & 10 & 9 \\
				8 & 7 & 6 & 5 \\
				4 & 3 & 2 & 1 \\
			\end{tabular}
			&
			\begin{tabular}{c c c c} 
				80 & 70 & 60 & 50 \\
				240 & 214 & 188 & 162 \\
				400 & 358 & 316 & 274 \\
				560 & 502 & 444 & 386 \\
			\end{tabular}\\
			\hline
			\begin{tabular}{c c c} 
				1 & 2 & 3 \\
				4 & 5 & 6 \\
			\end{tabular}
			&
			\begin{tabular}{c c} 
				6 & 5 \\
				4 & 3 \\
				2 & 1 \\
			\end{tabular}
			&
			\begin{tabular}{c c} 
				20 & 14 \\
				56 & 41 \\
			\end{tabular} \\
			\hline
			\begin{tabular}{c c c c} 
				1 & 0 & 0 & 0\\
				0 & 1 & 0 & 0 \\
				0 & 0 & 1 & 0 \\
				0 & 0 & 0 & 1 \\
			\end{tabular}
			&
			\begin{tabular}{c c c c} 
				1 & 2 & 3 & 4 \\
				5 & 6 & 7 & 8 \\
				9 & 10 & 11 & 12 \\
				13 & 14 & 15 & 16 \\
			\end{tabular}
			&
			\begin{tabular}{c c c c} 
				1 & 2 & 3 & 4 \\
				5 & 6 & 7 & 8 \\
				9 & 10 & 11 & 12 \\
				13 & 14 & 15 & 16 \\
			\end{tabular}\\
			\hline
			\begin{tabular}{c c c c} 
				0 & 0 & 0 & 0\\
				0 & 0 & 0 & 0 \\
				0 & 0 & 0 & 0 \\
				0 & 0 & 0 & 0 \\
			\end{tabular}
			&
			\begin{tabular}{c c c c} 
				1 & 2 & 3 & 4 \\
				5 & 6 & 7 & 8 \\
				9 & 10 & 11 & 12 \\
				13 & 14 & 15 & 16 \\
			\end{tabular}
			&
			\begin{tabular}{c c c c} 
			0 & 0 & 0 & 0\\
			0 & 0 & 0 & 0 \\
			0 & 0 & 0 & 0 \\
			0 & 0 & 0 & 0 \\
			\end{tabular}\\
			\hline
			\begin{tabular}{c c c} 
				1 & 2 & 3 \\
				4 & 5 & 6 \\
			\end{tabular}
			&
			\begin{tabular}{c c c} 
				9 & 8 & 7 \\
				6 & 5 & 4\\
			\end{tabular}
			&
			- \\
			\hline
			\begin{tabular}{c c c} 
				1 & 2 & 3 \\
				4 & 5 & 6 \\
			\end{tabular} 
			&
			* 
			&
			- \\
			\hline
		\end{tabular}
	\end{center}
\end{table}

* -- пустая матрица.

- -- данные введены некорректно.

Все тесты алгоритмами были пройдены успешно.

\section{Вычисление трудоёмкости дял алгоритмов}
Для каждого из алгоритмов проведём вычисление трудоёмкости.
Для всех случаев размеры первой матрицы -- $n_1 \times m_1$, а второй -- $n_2 \times m_2$.

\subsection{Вычисление трудоёмкости классического алгоритма}
Инициализация матрицы результата: $1 + 1 + n_1(1 + 2 + 1) + 1 = 4n_1 + 3$

Подсчет:\\
$1 + n_1(1 + (1 + m_2(1 + (1 + m_1(1 + (8) + 1) + 1) + 1) + 1) + 1) + 1 = 
n_1(m_2(10m_1 + 4) + 4) + 4) + 2 = 10n_1m_2m_1+ 4n_1m_2 + 4n_1 +2
$

\subsection{Вычисление трудоёмкости алгоритма Винограда}
Первый цикл: $\frac{15}{2}n_1m_1 + 5n_1 + 2$ 

Второй цикл: $\frac{15}{2}m_2n_2+ 5m_2 + 2$

Третий цикл: $13n_1m_2m_1 + 12n_1m_2 + 4n_1 + 2$

Условный переход: $\begin{bmatrix}
	2    &&, \text{невыполнение условия}\\
	15n_1m_2 + 4n_1 + 2 &&, \text{выполнение условия}\\
\end{bmatrix} $ \\

Итого: $  13n_1m_2m_1 + \frac{15}{2}n_1m_1 +\frac{15}{2}m_2n_2 + 12n_1m_2 + 5n_1 + 5m_2 + 4n_1 + 6 + \\
\begin{bmatrix}
	2    &&, \text{невыполнение условия}\\
	15n_1m_2 + 4n_1 + 2 &&, \text{выполнение условия}\\
\end{bmatrix} $ \\

\subsection{Вычисление трудоёмкости оптимизированного алгоритма Винограда}
Первый цикл: $\frac{11}{2}n_1m_1 + 4n_1 + 2$ 

Второй цикл: $\frac{11}{2}m_2n_2+ 4m_2 + 2$

Третий цикл: $\frac{17}{2}n_1m_2m_1 + 9n_1m_2 + 4n_1 + 2$

Условный переход: $\begin{bmatrix}
	1    &&, \text{невыполнение условия}\\
	10n_1m_2 + 4n_1 + 2 &&, \text{выполнение условия}\\
\end{bmatrix} $ \\

Итого: $\frac{17}{2}n_1m_2m_1 + \frac{11}{2}n_1m_1 + \frac{11}{2}m_2n_2 + 9n_1m_2 + 8n_1 + 4m_2 + 6 + \\
\begin{bmatrix}
	1    &&, \text{невыполнение условия}\\
	10n_1m_2 + 4n_1 + 2 &&, \text{выполнение условия}\\
\end{bmatrix} $ \\

\section{Вывод}
Были сформированы требования к ПО, приведены листинги коды.
Были вычислены трудоёмкости алгоритмов и проведены функциональные тесты.
Все алгоритмы справились с тестированием.