\chapter{Технологическая часть}
В этом разделе будут приведены листинги кода и результаты функционального тестирования.

\section{Средства реализации программного обеспечения}
В качестве языка программирования выбран C++, так как имеется опыт разработки проектов на этом языке, 
а также есть возможность использования нативных потоков \cite{thread}.
Для замеры общего времени используется библиотека chrono \cite{time}.

\section{Листинг кода}
\FloatBarrier
На листингах 4.1-4.4 представлена реализация фрагментов разработанного алгоритма.

\begin{lstinputlisting}[language=C++, caption=Реализация линейного алгоритма, linerange={164-219}, 
	basicstyle=\footnotesize\ttfamily, frame=single,breaklines=true]{src/main.cpp}
\end{lstinputlisting}
\FloatBarrier

\FloatBarrier
\begin{lstinputlisting}[language=C++, caption=Реализация этапов вычислительного конвейера, linerange={221-300}, 
	basicstyle=\footnotesize\ttfamily, frame=single, breaklines=true]{src/main.cpp}
\end{lstinputlisting}
\FloatBarrier

\FloatBarrier
\begin{lstinputlisting}[language=C++, caption=Общая реализация вычислительного конвейера, linerange={303-344}, 
	basicstyle=\footnotesize\ttfamily, frame=single, breaklines=true]{src/main.cpp}
\end{lstinputlisting}
\FloatBarrier

\section{Функциональные тесты}
В таблице 3.1 приведены результаты функциональных тестов. 
В первом столбце -- исходная матрица.
Во втором столбце -- ожидаемый результат: наиболее коррелирующих признаков и само значение корреляции.

\FloatBarrier
\begin{table}[h]
		\caption{Функциональные тесты}
		\label{tabular:functional_test}
		\begin{tabular}{| c | c |}
			\hline
		Массив & Ожидаемый результат \\ \hline
		1, 2 & -1, 1 \\
		\hline
		-3, 2, 11, 8, 0
		&
		-1.28, -0.31, 1.43, 0.85, -0.69 \\
		\hline
		-2, -1, 0, 1, 2 & 
		-1.41, -0.71, 0, 0.71, 1.41\\
		\hline
		0.1, 0.12, 0.11, 0.13, 0.1  &
		-1.03, 0.69, -0.17, 1.54, -1.03 \\
		\hline
		8.56, 992.12, 134, 1, 567 &
		-0.86, 1.69, -0.54, -0.88, 0.58
		\\
		\hline
		\end{tabular}
\end{table}
\FloatBarrier

Все тесты алгоритмами были пройдены успешно.

\section{Вывод}
Были приведены листинги кода.
Были проведены функциональные тесты.
Все алгоритмы справились с тестированием.