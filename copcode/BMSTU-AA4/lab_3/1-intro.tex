\newpage
\addcontentsline{toc}{chapter}{Введение}

\chapter*{Введение}
Алгоритмы сортировки используются в широком наборе задач.
Сортировка встречается повсеместно: например, в интернет-магазине нужно упорядочить товары по возрастанию цены.
В компьютерной графике, в базах данных, в файловых системах - везде требуется сортировка.

Сортировки требует больших затрат по времени и памяти, поэтому существуют различные алгоритмы со своими преимуществами.
С распространением BigData технологий выбор оптимального решения стал особенно актуальным, так как требуется быстро
работать с огромными массивами данных.

Актуальность работы заключается в том, что нужно оптимизировать траты системных ресурсов на сортировку.

Целью данной работы является разработка программы, которая реализует три алгоритма сортировки.

Для достижения поставленной цели необходимо выполнить следующее:
\begin{itemize}
	\item рассмотреть существующие алгоритмы сортировки;
	\item привести схемы реализации рассматриваемых алгоритмов;
	\item определить средства программной реализации;
	\item реализовать рассматриваемые алгоритмы;
	\item протестировать разработанное ПО;
	\item провести модульное тестирование всех реализаций алгоритмов;
	\item оценить реализацию алгоритмов по времени и памяти.
\end{itemize}