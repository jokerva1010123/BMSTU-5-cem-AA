\chapter{Исследовательская часть}

В данном разделе приводятся результаты замеров затрат реализаций алгоритмов по процессорному времени.

\section{Технические характеристики}

Технические характеристики устройства, на котором выполнялись замеры:

\begin{itemize}
	\item[---] Операционная система Window 10 Home Single Language;
	\item[---] Память 8 Гб;
	\item[---] Процессор 11th Gen Intel(R) Core(TM) i5-1135G7 2.42 ГГц, 4 ядра.
\end{itemize}

Во время замера устройство было подключено к сети электропитания, нагружено приложениями окружения и самой системой замера.

\section{Время выполнения реализаций алгоритмов}

Процессорное время реализаций алгоритмов замерялось при помощи функции process\_time() из библиотеки time языка Python. Данная функция возвращает количество секунд, прошедших с начала эпохи, типа float.

Контрольная точка возвращаемого значения не определна, поэтому допустима только разница между результатами последовательных вызовов.

Замеры времени для каждого размера матрицы проводились 500 раз. В качестве результата взято среднее время работы алгоритма на данном размере. При каждом запуске алгоритма, на вход подавались случайно сгенерированные матриц. Тестовые пакеты создавались до начала замера времени.

Результаты замеров приведены на рисунках \ref{img:time1}, \ref{img:time2} (в микросекундах). Графики зависимостей времени работы алгоритмов от размеров матриц приведены на рисунках \ref{img:graph1} и \ref{img:graph2}.

\img{50mm}{time1}{Результаты замеров времени алгоритмов при четных размерах матриц (в микросекундах)}

\img{50mm}{time2}{Результаты замеров времени алгоритмов при нечетных размерах матриц (в микросекундах)}

\img{90mm}{graph1}{Зависимость времени работы алгоритма от четного размера квадратной матрицы (микросекунды)}

\img{90mm}{graph2}{Зависимость времени работы алгоритма от нечетного размера квадратной матрицы (микросекунды)}

\section*{Вывод}

В результате эксперимента было получено, что при больших размерах матриц (свыше 10), алгоритм Винограда быстрее стандартного алгоритма более, чем 1.2 раза, а оптимизированный алгоритм Винограда быстрее стандартного алгоритма в 1.3 раза. Также при проведении эксперимента было выявлено, что на четных размерах реализация алгоритма Винограда в 1.2 раза быстрее, чем на нечетных размерах матриц, что обусловлено необходимостью проводить дополнительные вычисления для крайних строк и столбцов.

