\chapter{Конструкторская часть}
В этом разделе были приведены требования к вводу и программе, а также схемы алгоритмов умножения матриц.

\section{Разработка алгоритмов}

Предполагается, что на вход всех алгоритмов поступили матрицы верного размера.

На рисунках \ref{img:schema1}, \ref{img:schema21}, \ref{img:schema22} и \ref{img:schema3} приведены схема стандартного алгоритма умножения матриц, схема алгоритма Винограда и схема оптимизированного алгоритма Винограда соответственно.

\img{140mm}{schema1}{Схема стандартного алгоритма умножения матриц}
\clearpage

\img{230mm}{schema21}{Схема алгоритма Винограда умножения матриц}
\clearpage

\img{230mm}{schema22}{Схема алгоритма Винограда умножения матриц (продолжение)}
\clearpage

\img{230mm}{schema3}{Схема оптимизированного алгоритма Винограда умножения матриц}
\clearpage


\section{Модель вычислений}

Для последующего вычисления трудоемкости необходимо ввести модель вычислений:
\begin{enumerate}
	\item операции из списка (\ref{for:opers}) имеют трудоемкость 1;
	\begin{equation}
		\label{for:opers}
		+, -, *, /, \%, ==, !=, <, >, <=, >=, [], ++, {-}-
	\end{equation}
	\item трудоемкость оператора выбора \code{if условие then A else B} рассчитывается, как (\ref{for:if});
	\begin{equation}
		\label{for:if}
		f_{if} = f_{\text{условия}} +
		\begin{cases}
			f_A, & \text{если условие выполняется,}\\
			f_B, & \text{иначе.}
		\end{cases}
	\end{equation}
	\item трудоемкость цикла рассчитывается, как (\ref{for:for});
	\begin{equation}
		\label{for:for}
		f_{for} = f_{\text{инициализации}} + f_{\text{сравнения}} + N\cdot(f_{\text{тела}} + f_{\text{инкремента}} + f_{\text{сравнения}})
	\end{equation}
	\item трудоемкость вызова функции равна 0.
\end{enumerate}


\section{Трудоемкость алгоритмов}

В следующих частях были рассчитаны трудоемкости алгоритмов умножения матриц.

\subsection{Стандартный алгоритм умножения матриц}

Трудоёмкость внешнего цикла по $i \in [1..N]$:
\begin{equation}
	\label{for:xxxxx}
	f = 2 + N \cdot (2 + f_{body})
\end{equation}

Трудоёмкость цикла по $j \in [1..Q]$:
\begin{equation}
	\label{for:yyyyy}
	f = 2 + Q \cdot (2 + f_{body})
\end{equation}

Трудоёмкость цикла по $k \in [1..M]$:
\begin{equation}
	\label{for:yyyyy}
	f = 2 + 10\cdot M
\end{equation}

Учитывая, что трудоёмкость стандартного алгоритма равна трудоёмкости внешнего цикла, можно вычислить ее по следующей формуле:
\begin{equation}
	\label{for:standard}
	f_{standard} = 2 + N \cdot (4 + Q \cdot (4 + 10M)) \approx 10\cdot M\cdot N\cdot Q
\end{equation}

\subsection{Алгоритм Винограда}

Трудоёмкость создания и инициализации массивов MH и MV:
\begin{equation}
	\label{for:init}
	f_{init} = Q + N
\end{equation}

Трудоёмкость заполнения массива MH:
\begin{equation}
	\label{for:MH}
	f_{MH} = 2 + N\cdot(2 + 11\cdot \frac{M}{2})
\end{equation}

Трудоёмкость заполнения массива MV:
\begin{equation}
	\label{for:MV}
	f_{MV} = 2 + Q\cdot(2 +11\cdot \frac{M}{2})
\end{equation}
	
Трудоёмкость цикла заполнения для чётных размеров:
\begin{equation}
	\label{for:cycle}
	f_{cycle} = 2 + N \cdot (4 + Q \cdot (11 + \frac{M}{2} \cdot 23))
\end{equation}
	
Трудоёмкость цикла, для дополнения умножения суммой последних нечётных строки и столбца, если общий размер нечётный:
\begin{equation}
	\label{for:last}
	f_{last} = \begin{cases}
		2, & \text{чётная,}\\
		4 + N \cdot (4 + 14\cdot Q), & \text{иначе.}
	\end{cases}
\end{equation}

Итого, для худшего случая (нечётный общий размер матриц) имеем:
\begin{equation}
	\label{for:bad}
	f =  f_{MH} + f_{MV} + f_{cycle} + f_{last}\approx 11.5 \cdot M\cdot N\cdot Q
\end{equation}

Для лучшего случая (чётный общий размер матриц) имеем:
\begin{equation}
	\label{for:good}
f =  f_{MH} + f_{MV} + f_{cycle} + f_{last} \approx 11.5 \cdot M\cdot N\cdot Q
\end{equation}


\subsection{Оптимизированный алгоритм Винограда}

Трудоёмкость создания и инициализации массивов MH и MV:
\begin{equation}
	\label{for:impr_init}
	f_{init} = Q + N
\end{equation}
	
Трудоёмкость заполнения массива MH:
\begin{equation}
	\label{for:impr_MH}
	f_{MH} =  2 + N\cdot(2 + \frac{M}{2} \cdot 8)
\end{equation}
	
Трудоёмкость заполнения массива MV:
\begin{equation}
	\label{for:impr_MV}
	f_{MV} =  2 + Q\cdot(2 + \frac{M}{2} \cdot 8)
\end{equation}
	
Трудоёмкость цикла заполнения для чётных размеров:
\begin{equation}
	\label{for:impr_cycle}
	f_{cycle} =2 + N \cdot (4 + Q \cdot (11 + \frac{M}{2} \cdot 18))
\end{equation}
	
Трудоёмкость дополненительных вычислений при нечетном размере матрицы:
\begin{equation}
	\label{for:impr_last}
	f_{last} = 
	\begin{cases}
		1, & \text{чётная,}\\
		4 + Q \cdot (4 + 10\cdot M), & \text{иначе.}
	\end{cases}
\end{equation}

Итого, для худшего случая (нечётный общий размер матриц) имеем:
\begin{equation}
	\label{for:bad_impr}
	f = f_{MH} + f_{MV} + f_{cycle} + f_{last} \approx 9\cdot M\cdot N\cdot Q
\end{equation}

Для лучшего случая (чётный общий размер матриц) имеем:
\begin{equation}
	\label{for:good_impr}
	f = f_{MH} + f_{MV} + f_{cycle} + f_{last} \approx 9\cdot M\cdot N\cdot Q
\end{equation}


\section*{Вывод}

На основе теоретических данных, полученных из аналитического раздела, были построены схемы обоих алгоритмов умножения матриц.  Оценены их трудоёмкости в лучшем и худшем случаях.
