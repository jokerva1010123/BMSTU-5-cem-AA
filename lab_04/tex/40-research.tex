\chapter{Исследовательская часть}

В данном разделе приводятся результаты замеров затрат реализаций алгоритмов по процессорному времени.

\section{Технические характеристики}

Технические характеристики устройства, на котором выполнялось тестирование:

\begin{itemize}
	\item[---] операционная система Window 10 Home Single Language;
	\item[---] память 8 Гб;
	\item[---] процессор 11th Gen Intel(R) Core(TM) i5-1135G7 2.42 ГГц, 4 ядра.
\end{itemize}

Во время замера устройство было подключено к сети электропитания, нагружено приложениями окружения и самой системой замера.

\section{Время выполнения реализации алгоритмов}

Для замера процессорного времени использовалась функция 
\\ \textit{std::chrono::system\_clock::now(...)} из библиотеки $chrono$ \cite{cpp-lang-chrono} на C++. Функция возвращает процессорное время типа float в секундах.

Контрольная точка возвращаемого значения не определна, поэтому допустима только разница между результатами последовательных вызовов.

Замеры времени для каждого размера матрицы проводились 200 раз. В качестве результата взято среднее время работы алгоритма на данном размере. При каждом запуске алгоритма, на вход подавались случайно сгенерированные матриц. Тестовые пакеты создавались до начала замера времени.

В таблице \ref{tbl:cnt_time1} приведены результаты замеров времени работы реализаций последовательного и параллельного алгоритмов при 4 потоков.

\begin{table}[h]
	\begin{center}
		\begin{threeparttable}
		\captionsetup{justification=raggedright,singlelinecheck=off}
		\caption{\label{tbl:cnt_time1} Результат замеров времени (мкс)}
		\begin{tabular}{|c@{\hspace{7mm}}|c@{\hspace{7mm}}|c@{\hspace{7mm}}|}
			\hline
		  Размер  & Без многоточности & 4 потока\\ 
			\hline
			100x100&0.210&0.120\\
            \hline
            200x200&0.195&0.105\\
            \hline
            300x300&0.245&0.230\\
            \hline
            500x500&0.760&0.430\\
            \hline
            750x750&1.805&0.815\\
            \hline
            1000x1000&2.540&1.290\\
            \hline
            2000x2000&9.085&4.020\\
            \hline
            5000x5000&64.065&44.95\\
            \hline
		\end{tabular}
		\end{threeparttable}
	\end{center}
\end{table}
В таблице \ref{tbl:cnt_time2} приведены результаты замеров времени работы реализации параллельного алгоритмов на квадратной матрице размером 2000.
\begin{table}[h]
	\begin{center}
		\begin{threeparttable}
		\captionsetup{justification=raggedright,singlelinecheck=off}
		\caption{\label{tbl:cnt_time2} Результат замеров времени }
		\begin{tabular}{|c@{\hspace{7mm}}|c@{\hspace{7mm}}|c@{\hspace{7mm}}|}
			\hline
		  Количество потоков & Время\\ 
			\hline
			1&9.040\\
            \hline
            2&4.945\\
            \hline
            4&3.445\\
            \hline
            8&3.500\\
            \hline
            16&3.845\\
            \hline
           
		\end{tabular}
		\end{threeparttable}
	\end{center}
\end{table}

\section*{Вывод}

В данном разделе было произведено сравнение времени выполнения реализации алгоритма при последовательной реализации и многопоточности. Резуьтат показал, что выгоднее всего по времени использовать столько потоков, сколько у процессора логических ядер.


