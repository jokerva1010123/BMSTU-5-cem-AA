\chapter{Аналитическая часть}
В этом разделе будет рассмотрено понятие многопоточности и представлен алгоритм варианта задания.

\section{Многопоточность}

\textbf{Многопоточность} \cite{threads} --- способность центрального процессора или одного ядра в многоядерном процессоре одновременно выполнять несколько процессов или потоков, соответствующим образом поддерживаемых операционной системой.

Процесс --- это программа в ходе своего выполнения. Когда мы выполняем программу или приложение, запускается процесс. Каждый процесс состоит из одного или нескольких потоков.

Поток --- это сегмент процесса. Потоки представляют собой исполняемые сущности, которые выполняют задачи, стоящие перед исполняемым приложением. Процесс завершается, когда все потоки заканчивают выполнение.

Каждый поток в процессе --- это задача, которую должен выполнить процессор. Большинство процессоров сегодня умеют выполнять одновременно две задачи на одном ядре, создавая дополнительное виртуальное ядро. Это называется одновременная многопоточность или многопоточность $Hyper-Threading$, если речь о процессоре от Intel. 

Эти процессоры называются многоядерными процессорами. Таким образом, двухъядерный процессор имеет 4 ядра: два физических и два виртуальных. Каждое ядро может одновременно выполнять только один поток.

Как упоминалось выше, один процесс содержит несколько потоков, и одно ядро процессора может выполнять только один поток за единицу времени. Если мы пишем программу, которая запускает потоки последовательно, то есть передает выполнение в очередь одного конкретного ядра процессора, мы не раскрываем весь потенциал многоядерности. Остальные ядра просто стоят без дела, в то время как существуют задачи, которые необходимо выполнить. Если мы напишем программу таким образом, что она создаст несколько потоков для отнимающих много времени независимых функций, то мы сможем использовать другие ядра процессора, которые в противном случае пылились бы без дела. Можно выполнять эти потоки параллельно, тем самым сократив общее время выполнения процесса.

\section{Алгоритм варианта задания}

Задание требует найти первое число в матрице, которое больше среднего геометрического, и заменить его на максимальный элемент матрицы. Если невозможно найти среднее геометрическое число или нет элемент, большего среднего геометрического, сообщить на экране.

Чтобы решить задание, сначала нужно найти среднее геометрическое матрицы. Затем необходимо найти максимальный элемент и положение первого элемента больше среднего геометрического. И, наконец, замените этот элемент на самый максимальный элемент. Процесс нахождения максимального элемента и элемента, большего среднего геометрического, может использовать параллельные вычисления.

\section*{Вывод}

В данном разделе было рассмотрено понятие многопоточности и был представлен алгоритм варианта задания.
