\chapter{Аналитическая часть}

В данном разделе будут разобраны алгоритмы нахождения расстояний Левенштейна и Дамерау-Левенштейна.

\section{Расстояние Левенштейна}

\textbf{Расстояние Левенштейна} между двумя строками, позволяющая определить «схожесть» двух строк — минимальное количество операций вставки одного символа, удаления одного символа и замены одного символа на другой, необходимых для превращения одной строки в другую (каждая операция имеет свою цену -- штраф). \newline

\textit{Редакционное предписание} -- последовательность действий, необходимых для получения из первой строки вторую, и минимизирующую суммарную цену (и является расстоянием Левенштейна).\newline
Пусть $S_{1}$ и $S_{2}$ -- две строки, длиной \textit{N} и \textit{M} соответственно. 
Введены следующие обозначения:
\begin{itemize}
        \item I (aнгл. Insert) -- вставка символа в произвольной позиции ($w(\lambda,b)=1$);
        \item D (aнгл. Delete) -- удаление символа в произвольной позиции ($w(\lambda,b)=1$);
        \item R (aнгл. Replace) -- замена символа на другой ($w(a,b)=1, \medspace a \neq b$);
        \item M (aнгл. Match) -- совпадение двух символов ($w(a,a)=0$). \newline
\end{itemize}

С учетом введенных обозначений, расстояние Левенштейна может быть подсчитано по формуле \ref{eq:D}:

\begin{equation}
	\label{eq:D}
	D(i, j) = \begin{cases}

		0 &\text{i = 0, j = 0}\\
		i &\text{j = 0, i > 0}\\
		j &\text{i = 0, j > 0}\\
		\min \lbrace \\
		\qquad D(i, j-1) + 1\\
		\qquad D(i-1, j) + 1 &\text{i > 0, j > 0}\\
		\qquad D(i-1, j-1) + m(a[i], b[j]) \\
		\rbrace
	\end{cases}
\end{equation}


Функция \ref{eq:m} $m(a, b)$ определена как:
\begin{equation}
	\label{eq:m}
	m(a, b) = \begin{cases}
		0 &\text{если a = b,}\\
		1 &\text{иначе}
	\end{cases}.
\end{equation}

\subsection{Матричный алгоритм нахождения расстояния}

Рекурсивный алгоритм вычисления расстояния Левенштейна может быть не эффективен при больших $i$ и $j$, так как множество промежуточных значений $D(i, j)$ вычисляются не один раз, что сильно замедляет время выполнения программы.

В качестве структуры данных для хранения промежуточных значений можно использовать \textit{матрицу}, имеющую  размеры:

\begin{equation}
	(length(S1)+ 1) \times ((length(S2) + 1),
\end{equation}
где $length(S)$ -- длина строки $S$

Значение в ячейке $[i, j]$ равно значению $D(S1[1...i], S2[1...j])$. Первая строка и первый столбец заполнены нулями.

Всю таблицу (за исключением первого столбца и первой строки) заполняем в соответствии с формулой \ref{eq:mat1}:
\begin{equation}
	\label{eq:mat1}
	A[i][j] = min \begin{cases}
		A[i-1][j] + 1\\
		 A[i][j-1] + 1\\
		 A[i-1][j-1] + m(S1[i], S2[j]))\\
	 \end{cases}.
 \end{equation}

Функция $m(S1[i], S2[j])$ определена как:
\begin{equation}
\label{eq:m2}
m(S1[i], S2[j]) = \begin{cases}
0, &\text{если $S1[i - 1] = S2[j - 1]$,}\\
1, &\text{иначе}
\end{cases}.
\end{equation}

Результат вычисления расстояния Левенштейна будет ячейка матрицы с индексами $i = length(S1$) и $j = length(S2)$.

\section{Расстояние Дамерау-Левенштейна}

\textbf{Расстояние Дамерау-Левенштейна} между двумя строками, состоящими из конечного числа символов — это минимальное число операций вставки, удаления, замены одного символа и транспозиции двух соседних символов, необходимых для перевода одной строки в другую.

Является модификацией расстояния Левенштейна -- добавлена операции \textit{транспозиции}, то есть перестановки, двух символов.

Расстояние Дамерау -- Левенштейна может быть найдено по формуле \ref{eq:d}, которая задана как


\begin{equation}
	\label{eq:d}
	d_{a,b}(i, j) = \begin{cases}
		\max(i, j), &\text{если }\min(i, j) = 0,\\
		\min \lbrace \\
			\qquad d_{a,b}(i, j-1) + 1,\\
			\qquad d_{a,b}(i-1, j) + 1,\\
			\qquad d_{a,b}(i-1, j-1) + m(a[i], b[j]), &\text{иначе}\\
			\qquad \left[ \begin{array}{cc}d_{a,b}(i-2, j-2) + 1, &\text{если }i,j > 1;\\
			\qquad &\text{}a[i] = b[j-1]; \\
			\qquad &\text{}b[j] = a[i-1]\\
			\qquad \infty, & \text{иначе}\end{array}\right.\\
		\rbrace
		\end{cases},
\end{equation}

Формула выводится по тем же соображениям, что и формула (\ref{eq:D}).

\subsection{Рекурсивный алгоритм нахождения расстояния}
Рекурсивный алгоритм вычисления расстояния Дамерау-Левенштейна реализует формулу \ref{eq:d}

Минимальная цена преобразования -- минимальное значение приведенных вариантов.

Если полагать, что \textit{a', b'} -- строки $a$ и $b$ без последнего символа соответственно, а \textit{a'', b''} -- строки $a$ и $b$ без двух последних символов, то цена преобразования из строки $a$ в $b$ выражается из элементов, представленных ниже:
\begin{enumerate}[label=\arabic*)]
	\item сумма цены преобразования строки $a'$ в $b$ и цены проведения операции удаления, которая необходима для преобразования $a'$ в $a$;
	\item сумма цены преобразования строки $a$ в $b'$  и цены проведения операции вставки, которая необходима для преобразования $b'$ в $b$;
	\item сумма цены преобразования из $a'$ в $b'$ и операции замены, предполагая, что $a$ и $b$ оканчиваются на разные символы;
	\item сумма цены преобразования из $a''$ в $b''$ и операции перестановки, предполагая, что длины $a''$ и $b''$ больше 1 и последние два символа $a''$, поменянные местами, совпадут с двумя последними символами $b''$;
	\item цена преобразования из $a'$ в $b'$, предполагая, что $a$ и $b$ оканчиваются на один и тот же символ.
\end{enumerate}

\subsection{Матричный алгоритм нахождения расстояния}

Рекурсивный алгоритм вычисления расстояния Дамерау-Левенштейна может быть не эффективен при больших $i$ и $j$, так как множество промежуточных значений $D(i, j)$ вычисляются не один раз, что сильно замедляет время выполнения программы.

В качестве структуры данных для хранения промежуточных значений можно использовать \textit{матрицу}, имеющую  размеры:

\begin{equation}
	(length(S1)+ 1)\\\times((length(S2) + 1),
\end{equation}
где $length(S)$ -- длина строки $S$

Значение в ячейке $[i, j]$ равно значению $D(S1[1...i], S2[1...j])$. Первая строка и первый столбец тривиальны.

Всю таблицу (за исключением первого столбца и первой строки) заполняем в соответствии с формулой \ref{eq:mat}.
\begin{equation}
	\label{eq:mat}
	A[i][j] = min \begin{cases}
		A[i-1][j] + 1\\
		 A[i][j-1] + 1\\
		 A[i-1][j-1] + m(S1[i], S2[j]))\\
		 A[i-2][j-2] + 1, \text{если $i > 1$, $j > 1$ и} \\
		 	\text{\quad \quad $S1[i - 2] = S2[j - 1]$, $S2[i - 1] = S2[j - 2]$}
	 \end{cases}
 \end{equation}

Функция $m(S1[i], S2[j])$ определена как:
\begin{equation}
\label{eq:m2}
m(S1[i], S2[j]) = \begin{cases}
0, &\text{если $S1[i - 1] = S2[j - 1]$,}\\
1, &\text{иначе}
\end{cases}
\end{equation}

Результат вычисления расстояния Дамерау-Левенштейна будет ячейка матрицы с индексами $i = length(S1$) и $j = length(S2)$.


\subsection{Рекурсивный алгоритм нахождения расстояния с использованием кеша}

Чтобы уменьшить время работы рекурсивного алгоритма заполнения можно использовать \textit{кеш}, который будет представлять собой матрицу.

Ускорение достигается за счет использования матрицы для предотвращения повторной обработки уже обработанных данных.

Если данные ещё не были обработаны, то результат работы рекурсивного алгоритма заносится в матрицу. 
В случае, если обработанные данные встречаются снова, то для них расстояние не находится и выполняется следующий шаг.


\section{Вывод}
В данном разделе были рассмотрены алгоритмы поиска расстояния Левенштейна и расстояния Дамерау-Левенштейна. В частности были приведены рекурентные формулы работы алгоритмов, объяснена разница между расстоянием Левенштейна и расстоянием Дамерау-Левенштейна.
