\chapter{Исследовательская часть}

В данном разделе будут приведены примеры работы программы, а также проведен сравнительный анализ алгоритмов при различных ситуациях на основе полученных данных.

\section{Технические характеристики}

Технические характеристики устройства, на котором выполнялось тестирование представлены далее:

\begin{itemize}
    \item операционная система: Window 10 Home Single Language;
    \item память: 8 Гб;
    \item процессор: 11th Gen Intel(R) Core(TM) i5-1135G7 @ 2.40GHz   2.42 GHz .
\end{itemize}

Во время тестирования устройство было подключено к сети электропитания, нагружено приложениями окружения и самой системой тестирования

\section{Демонстрация работы программы}

На рисунке \ref{img:example} представлен результат работы программы.

\imgScale{0.75}{example}{Пример работы программы}
\clearpage

\section{Время выполнения алгоритмов}

Для замера времени используется функция замера процессорного времени process\_time(...) из библиотеки time на Python. Она возвращает пользовательское процессорное время типа float.

Использовать функцию приходится дважды, затем из конечного времени нужно вычесть начальное, чтобы получить результат.

Замеры проводились для длины слова от 0 до 7 по 300 раз на различных входных данных.

На рисунках \ref{img:graph_lev}, \ref{img:graph_dam_lev}, \ref{img:graph_dam_lev_two} приведены графические результаты замеров.

\imgHeight{100mm}{graph_lev}{Результат работы алгоритма нахождения расстояния Левештейна (матричного)}

\imgHeight{100mm}{graph_dam_lev}{Сравнение алгоритмов нахождения расстояния Дамерау-Левенштейна (матричного, рекурсивного и рекурсивного c использованием кеша)}

\imgHeight{100mm}{graph_dam_lev_two}{Сравнение алгоритмов нахождения расстояния Дамерау-Левенштейна (матричного и рекурсивного c использованием кеша)}
\clearpage

Сложность матричного алгоритма нахождения расстояния Левенштейна составляет $O(n^2)$ (рисунок \ref{img:graph_lev}).

В общем случае рекурсивный алгоритм алгоритм нахождения расстояния Дамерау-Левенштейна медленнее, чем его реализация с кешем или матричная реализация (рисунок \ref{img:graph_dam_lev} ), а также что матричная реализация нескольлко быстрее рекурсивного алгоритма с использованием кеша (рисунок \ref{img:graph_dam_lev_two} ).


\section{Вывод}

В результате замеров можно прийти к выводу, что матричная реализация алгоритмов нахождения расстояний заметно выигрывает по времени при росте строк, но проигрывает по количеству затрачиваемой памяти.