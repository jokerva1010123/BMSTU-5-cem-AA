\chapter{Аналитическая часть}
В этом разделе будут представлены описания алгоритмов гномьей сортировки, сортировки вставками и сортировки двоичным деревом поиска.

\section{Гномья сортировка}

\textbf{Гномья сортировка \cite{Knut}} ---  алгоритм сортировки, который использует только один цикл, что является редкостью. В этой сортировке массив просматривается слева-направо, при этом сравниваются и, если нужно, меняются соседние элементы. Если происходит обмен элементов, то происходит возвращение на один шаг назад. Если обмена не было --- алгоритм продолжает просмотр массива в поисках неупорядоченных пар.


\section{Сортировка вставками}

\textbf{Сортировка вставками \cite{insert}} --- алгоритм сортировки, котором элементы входной последовательности просматриваются по одному, и каждый новый поступивший элемент размещается в подходящее место среди ранее упорядоченных элементов.

В начальный момент отсортированная последовательность пуста. На каждом шаге алгоритма выбирается один из элементов входных данных и помешается на нужную позицию в уже отсортированной последовательности до тех пор, пока набор входных данных не будет исчерпан. В любой момент времени в отсортированной последовательности элементы удовлетворяют требованиям к входным данным алгоритма.

\section{Сортировка бинарным деревом}

\textbf{Сортировка бинарным деревом \cite{Knut}} --- универсальный алгоритм сортировки, заключающийся в построении двоичного дерева поиска по ключам массива, с последующей сборкой результирующего массива путём обхода узлов построенного дерева в необходимом порядке следования ключей.

Шаги алгоритма:
\begin{enumerate}[label=\arabic*)]
	\item построить двоичное дерево поиска по ключам массива;
	\item собрать результирующий массив путём обхода узлов дерева поиска в необходимом порядке следования ключей;
	\item вернуть, в качестве результата, отсортированный массив.
\end{enumerate}

\section*{Вывод}

В данной работе стоит задача реализации 3 алгоритмов сортировки, а именно: гномьей сортировки, сортировки вставками и сортировки двоичным деревом поиска. Необходимо оценить теоретическую оценку алгоритмов и проверить ее экспериментально.


