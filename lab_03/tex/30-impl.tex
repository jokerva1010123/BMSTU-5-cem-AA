\chapter{Технологическая часть}

В данном разделе будут приведены требования к программному обеспечению (ПО), средства реализации и листинги кода.

\section{Требования к ПО}

К программа принимает на вход массив сравнимых элементов. В качестве результата возвращается массив в отсортированном порядке. Программа не должна аварийно завершаться при некорректном вводе.

\section{Средства реализации}

В качестве языка программирования (ЯП) для реализации данной лабораторной работы был выбран ЯП Python \cite{pythonlang}. 

Данный язык имеет все небходимые инструменты для решения поставленной задачи.

Процессорное время работы реализаций алгоритмов было замерено с помощью функции process\_time() из библиотеки time \cite{pythonlangtime}.

\section{Сведения о модулях программы}
Программа состоит из четырех модулей:
\begin{enumerate}[label=\arabic*)]
	\item algorithms.py - хранит реализацию алгоритмов сортировок;
	\item unit\_tests.py - хранит реализацию тестирующей системы и тесты;
	\item time\_memory.py - хранит реализацию системы замера памяти и времени;
	\item tools.py - хранит реализацию вспомогательных функций.
\end{enumerate}


\section{Реализация алгоритмов}
В листингах \ref{lst:heap}, \ref{lst:smooth}, \ref{lst:beenary} представлены реализации алгоритмов сортировок (гномьей, вставками и бинарным деревом).

\begin{lstlisting}[label=lst:heap,caption=Алгоритм гномьей сортировки]
def gnome_sort(data):
    i, j, size = 1, 2, len(data)
    while i < size:
        if data[i - 1] <= data[i]:
            i, j = j, j + 1
        else:
            data[i - 1], data[i] = data[i], data[i - 1]
            i -= 1
            if i == 0:
                i, j = j, j + 1
    return data
\end{lstlisting}

\begin{lstlisting}[label=lst:smooth,caption= Алгоритм сортировки вставками]
def insertion_sort(arr):
	for i in range(1, len(arr)):
		j = i - 1
		key = arr[i]
		
		while j >= 0 and arr[j] > key:
			arr[j + 1] = arr[j]
			j -= 1
		arr[j + 1] = key
	return arr	
\end{lstlisting}

\begin{lstlisting}[label=lst:beenary,caption=Алгоритм сортировки бинарным деревом поиска]
class Node:
    def __init__(self, val, left=None, right=None):
        self.val = val
        self.left = left
        self.right = right

    def add(self, val):
        if self.val > val:
            if self.left is None:
                self.left = Node(val)
            else:
                self.left.add(val)
        else:
            if self.right is None:
                self.right = Node(val)
            else:
                self.right.add(val)


def _one_node_sort(node, dst_list):
    if node.left is not None:
        _one_node_sort(node.left, dst_list)

    dst_list.append(node.val)

    if node.right is not None:
        _one_node_sort(node.right, dst_list)


def binary_tree_sort(arr):
    if len(arr) == 0:
        return arr

    head = Node(arr[0])
    for i in range(1, len(arr)):
        head.add(arr[i])

    dst = []
    _one_node_sort(head, dst)
    return dst
\end{lstlisting}

\section{Функциональные тесты}

В таблице \ref{tbl:functional_test} приведены тесты для функций, реализующих алгоритмы сортировки. Тесты пройдены успешно.


\begin{table}[h]
	\begin{center}
	\begin{threeparttable}
	\captionsetup{justification=raggedright,singlelinecheck=off}
	\caption{\label{tbl:functional_test} Функциональные тесты}
		\begin{tabular}{|l|l|l|}
			\hline
			Входной массив & Ожидаемый результат & Результат \\ 
			\hline
			$[]$  & $[]$  & $[]$\\
			$[1]$  & $[1]$  & $[1]$\\
			$[0, 1]$  & $[0, 1]$  & $[0, 1]$\\
			$[1, 0]$  & $[0, 1]$  & $[0, 1]$\\
			$[2, -2]$  & $[-2, 2]$  & $[-2, 2]$\\
			$[2, -2, 2]$  & $[-2, 2, 2]$  & $[-2, 2, 2]$\\
			$[0, 0, 1, 2]$  & $[0, 0, 1, 2]$  & $[0, 0, 1, 2]$\\
			$[5, 2, 1, 8, 9, 10]$  & $[1, 2, 5, 8, 9, 10]$  & $[1, 2, 5, 8, 9, 10]$\\
			$[1, 2, 3, 4, 5, 6, 7]$  & $[1, 2, 3, 4, 5, 6, 7]$  & $[1, 2, 3, 4, 5, 6, 7]$\\
			$[9, 2, 1]$  & $[1, 2, 9]$  & $[1, 2, 9]$\\
			$[9, -10000, -20000]$  & $[-20000, -10000, 9]$  & $[-20000, -10000, 9]$\\
			$[5, 4, 4, 3]$  & $[3, 4, 4, 5]$  & $[3, 4, 4, 5]$\\
			\hline
		\end{tabular}
	\end{threeparttable}
	\end{center}
\end{table}

\section*{Вывод}

В этом разделе была представлена реализация алгоритмов сортировок. Тестирование показало, что алгоритмы реализованы правильно и работают корректно.
