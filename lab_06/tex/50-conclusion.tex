\chapter*{Заключение}
\addcontentsline{toc}{chapter}{Заключение}

Было экспериментально подтверждено различие во временной эффективности муравьиного алгоритма и алгоритма полного перебора решения задачи коммивояжера. В результате исследований можно сделать вывод о том, что при матрицах большого размера (больше 9) стоит использовать муравьиный алгоритм решения задачи коммивояжера, а не алгоритм полного перебора (на матрице размером 10x10 он работает в 15.4 раза быстрее). Также было установлено по результатам параметризации на экспериментальных класса данных, что при коэффиценте $\alpha$ = 0.1, 0.2, 0.3 муравьиный алгоритм работает наилучшим образом.
\vspace{5mm}

Цель, поставленная перед началом работы, была достигнута. В ходе лабораторной работы были решены все задачи:

\begin{itemize}[label = ---]
	\item изучены основы алгоритма полного перебора;
	\item применены изученные основы алгоритма полного перебора для реализации;
	\item изучены основы муравьиного алгоритма;
	\item применены изученные основы муравьиного алгоритма для реализации;
	\item проведена параметризация муравьиного алгоритма;
	\item проведен сравнительный анализ времени работы реализованных
	алгоритмов;
	\item подготовлен отчет о лабораторной работе.
\end{itemize}
