\chapter*{Введение}
\addcontentsline{toc}{chapter}{Введение}

Задача поиска оптимальных маршрутов является одной из важных. Муравьиный алгоритм – один из эффективных полиномиальных алгоритмов для нахождения приближённых решений задачи коммивояжёра, а также решения аналогичных задач поиска маршрутов на графах. Суть подхода заключается в анализе и использовании модели поведения муравьёв, ищущих пути от колонии к источнику питания, и представляет собой метаэвристическую оптимизацию.

Целью данной лабораторной работы является реализация муравьиного алгоритма и приобретение навыков параметризации методов на примере реализованного алгоритма, примененного к задаче коммивояжера.

Для достижения данной цели необходимо решить следующие задачи.

\begin{enumerate}
	\item Изучить алгоритм полного перебора для решения задачи коммивояжера.
	\item Реализовать алгоритм полного перебора для решения задачи коммивояжера.
	\item Изучить муравьиный алгоритм для решения задачи коммивояжера.
	\item Реализовать муравьиный алгоритм для решения задачи коммивояжера.
	\item Провести параметризацию муравьиного алгоритма на трех классах данных.
	\item Провести сравнительный анализ скорости работы реализованных алгоритмов.
	\item Подготовить отчет о выполненной лабораторной работе.
\end{enumerate}