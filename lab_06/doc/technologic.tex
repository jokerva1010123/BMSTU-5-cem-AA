\chapter{Технологическая часть}

В данном разделе будут рассмотрены средства реализации лабораторной работы, представлены листинги алгоритмов сортировки, а также тестовые данные, которые использовались для проверки корректности работы алгоритмов.

\section{Выбор языка программирования и среды разработки}
Для реализации алгоритмов был выбран язык программирования \textit{Python}, так как он предоставляет необходимую функциональность, позволяющую решить поставленные задачи на реализацию алгоритмов и выполнение замеров процессорного времени работы их реализаций, а в качестве среды разработки -- \textit{PyCharm}.
Для замеров процессорного времени использовалась функция \textit{process\_time()} \cite{time} из библиотеки \textit{time}, а для построения графиков -- библиотека \textit{matplotlib} \cite{mpl}.

\section{Реализация алгоритмов}

В листингах \ref{lst:ant_alg} - \ref{lst:ant_alg} представлены функции для конвейерного и ленейного алгоритмов обработки матриц.

\begin{center}
\captionsetup{justification=raggedright,singlelinecheck=off}
\begin{lstlisting}[label=lst:full_combinations_alg,caption=Алгоритм полного перебора]
def full_combinations_alg(matrix, size):

	cities = np.arange(size)
	cities_combs = []

	for combination in it.permutations(cities):
		cities_combs.append(list(combination))

	best_way = []
	min_length = float("inf")

	for i in range(len(cities_combs)):
		cities_combs[i].append(cities_combs[i][0])

		length = calc_length(matrix, size, cities_combs[i])

		if length < min_length:
			min_length = length
			best_way = cities_combs[i]

	return min_length, best_way
\end{lstlisting}
\end{center}	

\clearpage

\begin{center}
\captionsetup{justification=raggedright,singlelinecheck=off}
\begin{lstlisting}[label=lst:ant_alg,caption=Муравьиный алгоритм]
def ant_alg(matrix, size, alpha, beta, evaporation, days):

	pherom_matr = get_pherom_matr(size)
	visib_matr = get_visib_matr(matrix, size)

	q = calc_q(matrix, size)

	best_way = []
	min_length = float("inf")

	for _ in range(days):
		visited_arr = get_visited_places(np.arange(size), size)

		for i in range(size):
			while (len(visited_arr[i]) != size):
				pk = search_probability(pherom_matr, visib_matr, visited_arr, size, i, alpha, beta)  
				next_place = choose_next_place(pk)

				visited_arr[i].append(next_place - 1)

			visited_arr[i].append(visited_arr[i][0])
			
			length = calc_length(matrix, size, visited_arr[i]) 

			if length < min_length:
				min_length = length
				best_way = visited_arr[i]

		pherom_matr = update_pherom_matr(matrix, size, visited_arr, pherom_matr, q, evaporation)

	return min_length, best_way
\end{lstlisting}
\end{center}

\clearpage

\begin{center}
\captionsetup{justification=raggedright,singlelinecheck=off}
\begin{lstlisting}[label=lst:update_pherom_matr,caption=Функция для обновления феромонов]
def update_pherom_matr(matrix, size, visited_arr, pherom_matr, q, evaporation):

	ants = size

	for i in range(size):
		for j in range(size):
			delta_pherom_matr = 0

			for ant in range(ants):
				length = calc_length(matrix, size, visited_arr[ant])
				delta_pherom_matr += q / length

			pherom_matr[i][j] *= (1 - evaporation)
			pherom_matr[i][j] += delta_pherom_matr

			if pherom_matr[i][j] < MIN_PHEROMONE:
			pherom_matr[i][j] = MIN_PHEROMONE

	return pherom_matr
\end{lstlisting}
\end{center}	

\begin{center}
\captionsetup{justification=raggedright,singlelinecheck=off}
\begin{lstlisting}[label=lst:search_probability,caption=Функция для нахождения вероятней перехода в каждый из городов]
def search_probability(pherom_matr, visib_matr, visited_arr, size, ant, alpha, beta):
    pk = [0] * size

    for i in range(size):
        if i not in visited_arr[ant]:
            ant_i = visited_arr[ant][-1]

            pk[i] = pow(pherom_matr[ant_i][i], alpha) * \
                    pow(visib_matr[ant_i][i], beta)
        else:
            pk[i] = 0

    pk_sum = sum(pk)

    for i in range(size):
        pk[i] /= pk_sum  

    return pk
\end{lstlisting}
\end{center}

\clearpage

\begin{center}
\captionsetup{justification=raggedright,singlelinecheck=off}
\begin{lstlisting}[label=lst:choose_next_place,caption=Функция выбора следующего города]
def choose_next_place(pk):

    size = len(pk)
    numb = 0
    i = 0

    probability = random()

    while numb < probability and i < size:
        numb += pk[i]
        i += 1

    return i
\end{lstlisting}
\end{center}	

\begin{center}
\captionsetup{justification=raggedright,singlelinecheck=off}
\begin{lstlisting}[label=lst:calc_length,caption= Функция нахождения длины пути]
def calc_length(matrix, size, way):

    length = 0

    for i in range(size):
        beg_city = way[i]
        end_city = way[i + 1]

        length += matrix[beg_city][end_city]

    return length
\end{lstlisting}
\end{center}


\section{Тестирование}

В таблице \ref{tbl:functional_test} приведены тесты для алгоритмов решения задачи коммивояжера (муравьиного алгоритма и алгоритма полного перебора). Все тесты пройдены успешно.

\begin{table}[h]
	\begin{center}
        \begin{threeparttable}
        \captionsetup{justification=raggedright,singlelinecheck=off}
		\caption{\label{tbl:functional_test} Функциональные тесты}
		\begin{tabular}{|c|c|c|}
            \hline
        	Входные данные& \multicolumn{2}{c|}{Выходные данные}
        	\\ \hline
        	Матрица смежности & Результат & Ожидаемый результат 
        	\\ \hline
        	$\begin{pmatrix}
        		0 & 1 & 2\\
        		1 & 0 & 3\\
        		2 & 3 & 0
        	\end{pmatrix}$ & 6, [0, 1, 2, 0] & 6, [0, 1, 2, 0]
        	\\ \hline
        	$\begin{pmatrix}
        		0 & 3 & 4 & 1\\
        		3 & 0 & 1 & 1\\
        		4 & 1 & 0 & 2\\
        		1 & 1 & 2 & 0
        	\end{pmatrix}$ & 7, [0, 1, 2, 3, 0] & 7, [0, 1, 2, 3, 0]
        	\\ \hline
        	$\begin{pmatrix}
        		0 & 11 & 12 & 14 & 13\\
        		11 & 0 & 15 & 10 & 10\\
        		12 & 15 & 0 & 14 & 13\\
        		14 & 10 & 14 & 0 & 10\\
        		13 & 10 & 13 & 10 & 0\\
        	\end{pmatrix}$ & 56, [0, 1, 3, 4, 2, 0] & 56, [0, 1, 3, 4, 2, 0]
        	\\ \hline
		\end{tabular}
        \end{threeparttable}
	\end{center}
\end{table}

% \begin{center}
% \captionsetup{justification=raggedright,singlelinecheck=off}
% \begin{longtable}[c]{|c|c|c|}
% \caption{Функциональные тесты\label{tbl:functional_test}}
% 	\\ \hline
% 	Матрица смежности & Результат & Ожидаемый результат 
% 	\\ \hline
% 	$\begin{pmatrix}
% 		0 & 1 & 2\\
% 		1 & 0 & 3\\
% 		2 & 3 & 0
% 	\end{pmatrix}$ & 6, [0, 1, 2, 0] & 6, [0, 1, 2, 0]
% 	\\ \hline
% 	$\begin{pmatrix}
% 		0 & 3 & 4 & 1\\
% 		3 & 0 & 1 & 1\\
% 		4 & 1 & 0 & 2\\
% 		1 & 1 & 2 & 0
% 	\end{pmatrix}$ & 7, [0, 1, 2, 3, 0] & 7, [0, 1, 2, 3, 0]
% 	\\ \hline
% 	$\begin{pmatrix}
% 		0 & 11 & 12 & 14 & 13\\
% 		11 & 0 & 15 & 10 & 10\\
% 		12 & 15 & 0 & 14 & 13\\
% 		14 & 10 & 14 & 0 & 10\\
% 		13 & 10 & 13 & 10 & 0\\
% 	\end{pmatrix}$ & 56, [0, 1, 3, 4, 2, 0] & 56, [0, 1, 3, 4, 2, 0]
% 	\\ \hline

% \end{longtable}
% \end{center}

